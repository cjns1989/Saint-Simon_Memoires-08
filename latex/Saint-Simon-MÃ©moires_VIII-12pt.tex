\PassOptionsToPackage{unicode=true}{hyperref} % options for packages loaded elsewhere
\PassOptionsToPackage{hyphens}{url}
%
\documentclass[oneside,12pt,french,]{extbook} % cjns1989 - 27112019 - added the oneside option: so that the text jumps left & right when reading on a tablet/ereader
\usepackage{lmodern}
\usepackage{amssymb,amsmath}
\usepackage{ifxetex,ifluatex}
\usepackage{fixltx2e} % provides \textsubscript
\ifnum 0\ifxetex 1\fi\ifluatex 1\fi=0 % if pdftex
  \usepackage[T1]{fontenc}
  \usepackage[utf8]{inputenc}
  \usepackage{textcomp} % provides euro and other symbols
\else % if luatex or xelatex
  \usepackage{unicode-math}
  \defaultfontfeatures{Ligatures=TeX,Scale=MatchLowercase}
%   \setmainfont[]{EBGaramond-Regular}
    \setmainfont[Numbers={OldStyle,Proportional}]{EBGaramond-Regular}      % cjns1989 - 20191129 - old style numbers 
\fi
% use upquote if available, for straight quotes in verbatim environments
\IfFileExists{upquote.sty}{\usepackage{upquote}}{}
% use microtype if available
\IfFileExists{microtype.sty}{%
\usepackage[]{microtype}
\UseMicrotypeSet[protrusion]{basicmath} % disable protrusion for tt fonts
}{}
\usepackage{hyperref}
\hypersetup{
            pdftitle={SAINT-SIMON},
            pdfauthor={Mémoires\_VIII},
            pdfborder={0 0 0},
            breaklinks=true}
\urlstyle{same}  % don't use monospace font for urls
\usepackage[papersize={4.80 in, 6.40  in},left=.5 in,right=.5 in]{geometry}
\setlength{\emergencystretch}{3em}  % prevent overfull lines
\providecommand{\tightlist}{%
  \setlength{\itemsep}{0pt}\setlength{\parskip}{0pt}}
\setcounter{secnumdepth}{0}

% set default figure placement to htbp
\makeatletter
\def\fps@figure{htbp}
\makeatother

\usepackage{ragged2e}
\usepackage{epigraph}
\renewcommand{\textflush}{flushepinormal}

\usepackage{indentfirst}
\usepackage{relsize}

\usepackage{fancyhdr}
\pagestyle{fancy}
\fancyhf{}
\fancyhead[R]{\thepage}
\renewcommand{\headrulewidth}{0pt}
\usepackage{quoting}
\usepackage{ragged2e}

\newlength\mylen
\settowidth\mylen{...................}

\usepackage{stackengine}
\usepackage{graphicx}
\def\asterism{\par\vspace{1em}{\centering\scalebox{.9}{%
  \stackon[-0.6pt]{\bfseries*~*}{\bfseries*}}\par}\vspace{.8em}\par}

\usepackage{titlesec}
\titleformat{\chapter}[display]
  {\normalfont\bfseries\filcenter}{}{0pt}{\Large}
\titleformat{\section}[display]
  {\normalfont\bfseries\filcenter}{}{0pt}{\Large}
\titleformat{\subsection}[display]
  {\normalfont\bfseries\filcenter}{}{0pt}{\Large}

\setcounter{secnumdepth}{1}
\ifnum 0\ifxetex 1\fi\ifluatex 1\fi=0 % if pdftex
  \usepackage[shorthands=off,main=french]{babel}
\else
  % load polyglossia as late as possible as it *could* call bidi if RTL lang (e.g. Hebrew or Arabic)
%   \usepackage{polyglossia}
%   \setmainlanguage[]{french}
%   \usepackage[french]{babel} % cjns1989 - 1.43 version of polyglossia on this system does not allow disabling the autospacing feature
\fi

\title{SAINT-SIMON}
\author{Mémoires\_VIII}
\date{}

\begin{document}
\maketitle

\hypertarget{chapitre-premier.}{%
\chapter{CHAPITRE PREMIER.}\label{chapitre-premier.}}

1710

~

{\textsc{Première conversation tête à tête avec M. le duc d'Orléans, à
qui je propose de rompre avec M\textsuperscript{me} d'Argenton.}}
{\textsc{- Cérémonial du premier jour de l'an des fils et petits-fils de
France.}} {\textsc{- Continuation de la même conversation.}} {\textsc{-
J'écris à Besons sur le bureau du chancelier, à qui cela m'oblige de
faire confidence du projet, et qui l'approuve.}} {\textsc{- Concert pris
entre Besons et moi.}} {\textsc{- Deuxième conversation avec M. le duc
d'Orléans, le maréchal de Besons en tiers.}}

~

Les quatre premiers jours de l'année 1710 se passèrent en choses qui
méritent une espèce de journal, parce que, outre la part que j'y eus,
elles servirent de fondement à une suite d'événements considérables. Le
premier jour de cette année, qui fut un mercredi, rappela M. le duc
d'Orléans pour les cérémonies et les visites de cette journée. Je le vis
après les vêpres du roi, il m'emmena aussitôt dans son arrière-cabinet
obscur, sur la galerie, où la conversation fut d'abord coupée et
tumultueuse, comme il arrive d'ordinaire après une longue absence, après
quoi je lui demandai de ses nouvelles avec le roi, Monseigneur et les
personnes royales. Il me répondit assez en l'air, ni bien ni mal, et sur
ce que je lui répliquai que ce n'était pas assez, il me dit qu'il avait
donné à Saint-Cloud une fête à l'électeur de Bavière, où il y avait eu
quantité de dames, entre autres M\textsuperscript{me} d'Arco, mère du
chevalier de Bavière, où il n'avait pas cru mal faire de faire trouver
M\textsuperscript{me} d'Argenton\,; que le roi néanmoins l'avait trouvé
mauvais, et le lui avait dit après quelques jours de bouderie\,; que
cela s'était passé ensuite et qu'il était avec lui à l'ordinaire. Je lui
demandai ce qu'il entendait par cette expression \emph{à l'ordinaire},
qui ne m'expliquait rien au bout de quatre mois d'absence, sur quoi il
se mit à battre la campagne comme un homme qui craint d'approfondir. Je
le pressai, et, comme il vit que j'en savais davantage, il me demanda ce
qu'on m'en avait dit. Je ne crus pas devoir lui taire ce que j'en avais
appris. Je lui dis franchement que j'étais bien informé qu'il était fort
mal avec le roi, et si mal qu'il était difficile d'y être pis\,; que le
roi était outré contre lui de tout point\,; que Monseigneur l'était
infiniment davantage, et le montrait aussi avec beaucoup moins de
ménagements\,; qu'à leur exemple, le gros du monde s'éloignait de lui,
et que j'avais appris sur tout cela tant de fâcheux détails, que je lui
avouais que j'en étais au désespoir. Il m'écouta attentivement, et,
après avoir laissé quelque temps la parole tombée, il convint de tout ce
que je venais de lui dire. Il ajouta qu'il sentait bien que c'était là
les effets de l'impression de son affaire d'Espagne, qui nonobstant sa
simplicité avait été empoisonnée par des fripons\,; que le malheur était
qu'il n'y pouvait que faire, et qu'il fallait bien que le temps
raccommodât tout. Je le regardai avec fermeté, et lui répondis qu'il y
avait des choses que le temps effaçait, et d'autres que le temps
imprimait de plus en plus\,; que son affaire d'Espagne était
malheureusement de cette dernière sorte par sa nature, et par
l'expérience, qui lui montrait très-sensiblement qu'il était plus
éloigné du roi et de Monseigneur qu'au premier jour de la fin publique
de cette affaire\,; qu'il n'avait pas besoin de réflexions pour s'en
apercevoir, et que cette triste vérité ne pouvait être contestée.

À ce propos, il rentra fort en lui-même, et me l'avoua. Il convint de
son embarras avec eux, et de leur peine avec lui qui redoublait la
sienne, et qui le retirait de plus en plus d'auprès d'eux. J'en pris
occasion de tirer de lui le même aveu sur l'abandon si entier de tout le
monde, qui après l'autre ne fut pas difficile. Il s'en plaignit à moi
avec assez d'amertume, et, sur ce qu'il y mêla quelque aigreur, je lui
représentai qu'en un temps aussi despotique que ce règne, toute la cour,
et par elle, tout le monde réglait ses démarches sur les mouvements
qu'on ne cessait de chercher dans le roi, premier mobile de toutes
choses\,; que souvent c'était bassesse, ordinairement flatterie, mais
qu'ici c'était juste terreur, puisque chacun n'était que trop informé de
la cause des manières du roi à son égard, si différentes maintenant de
ce qu'elles avaient toujours été, et que quelque dure, quelque étrange,
quelque inouïe que fût la solitude qu'il éprouvait, il ne pouvait avec
raison le trouver mauvais de personne, ni en espérer la fin que par le
changement du roi à son égard, qui entraînerait au moins pour
l'extérieur celui de Monseigneur, et celui de tout le monde. Cette vive
repartie jeta ce prince dans une consternation qui m'émut et qui
m'encouragea. J'étais entré chez lui en résolution de le mettre en voie
de s'ouvrir avec moi pour le sonder et lui jeter de loin des propos
qu'il pût entendre, mais non dans le dessein de rompre la glace. En ce
moment, je me dépouillai de toute crainte et de toute considération
précédente, et je me déterminai à saisir l'occasion si elle se
présentait à moi de bonne grâce, comme je prévoyais qu'il pouvait
arriver et comme en effet elle se présenta peu de moments après.

M. le duc d'Orléans, pénétré de la peinture que je venais de lui faire
de sa situation, et qu'il ne pouvait alors se dissimuler à lui-même, se
leva après un profond silence de quelque temps et se mit à faire
quelques tours de chambre. Je me levai aussi, et appuyé à la muraille,
je l'examinais attentivement lorsque, levant la tête et soupirant, il me
demanda\,: «\,Que faire donc\,?» comme un homme qui, après avoir
profondément pensé, croit répondre sur-le-champ. Alors, voyant
l'occasion si belle et si naturelle, je la saisis sans balancer. «\,Que
faire, répondis-je, que faire\,? d'un ton ferme et significatif, je le
sais bien, mais je ne vous le dirai jamais, et c'est pourtant l'unique
chose à faire. Ah\,! je vous entends bien,\,» répliqua-t-il comme frappé
de la foudre, et, redoublant «\,je vous entends bien,\,» il s'alla jeter
sur un siège à l'autre bout du cabinet. Sûr à l'instant qu'il m'avait en
effet entendu, étourdi moi-même du grand coup que je venais de frapper,
je me retournai un peu vers la muraille pour m'en remettre moi-même, et
pour lui épargner l'embarras d'être regardé dans ces premiers moments.
Le silence fut long\,; je l'entendais se remuer impétueusement sur sa
chaise, et j'attendais en peine par où la conversation reprendrait.
Cependant les soupirs se mêlèrent à l'agitation du corps, et jugeant de
là que les réflexions cuisantes avaient plus de part à toute cette
agitation qu'une colère sèche, je me tournai vers lui, et, les yeux
baissés avec embarras, je rompis le silence qui devenait trop long, et
lui dis, pour presser le combat dont je me doutais en lui, que ce qui
m'était échappé était l'effet d'un concert pris entre Besons et moi, que
je croyais être les deux hommes qui lui fussent plus étroitement
attachés, et qui par ceci même lui en donnaient une preuve bien
signalée\,; que, pénétré de ce que j'avais appris en sortant de mon
carrosse, venant de la Ferté, et de ce qui m'avait été répété de tous
les lieux les plus sûrs et les plus considérables où je pouvais
atteindre, je m'étais tourné de toutes parts pour chercher une sortie à
son état funeste et enseveli\,; que je n'en avais pu découvrir nul
autre\,; que, accablé de sa difficulté, je m'étais ouvert de ma pensée
au maréchal de Besons, qui l'avait ardemment embrassée comme une
ressource assurée, mais unique\,; que nous avions résolu de la lui venir
proposer ensemble, et pris rendez-vous chez moi à Paris pour convenir de
tout, mais que la difficulté de l'entreprise nous ayant effrayés l'un et
l'autre par les réflexions que nous avions faites dans l'entre-deux,
nous étions demeurés d'accord que nous chercherions séparément à le
faire parler, à profiter de son ouverture pour aller aussi avant que
nous le jugerions convenable sur-le-champ, et que, si l'occasion se
présentait, elle serait saisie, et que celui des deux à qui cela
arriverait, décèlerait le complot et son compagnon. Je me tus après ce
court récit\,; il n'augmenta pas l'agitation corporelle, mais les
soupirs, et prolongea son silence. Je me retournai un peu pour lui
laisser plus de liberté, et de temps en temps je disais en monosyllabes,
comme m'encourageant moi-même\,: «\,Il n'y a que cela à faire, c'est
l'unique porte,\,» et d'autres mots semblables. Enfin, après longtemps,
M. le duc d'Orléans se leva, vint à moi, et avec une amertume qui ne se
peut rendre\,: «\,Que me proposez-vous là\,? me dit-il. --- Votre
grandeur, lui dis-je, et le seul moyen de vous remettre comme vous devez
être, et mieux que vous n'avez jamais été.\,» Quelques moments après,
j'ajoutai\,: «\,Oh\,! que je voudrais que Besons fût ici\,!» Il fut
quelque temps sans répondre, puis me dit, mais d'un ton fort concentré
en lui-même\,: «\,Mais il est ici. --- Quoi, dis-je, à Versailles\,? ---
Oui, me dit-il, il me semble que je l'ai vu ce matin chez le roi. --- Eh
bien\,! monsieur, repartis-je, voulez-vous l'envoyer chercher\,?» Il fut
un moment sans répondre\,; je le pressai, il y consentit. Aussitôt je
sortis, et je dis à ses gens qu'il demandait le maréchal de Besons.

Comme nous attendions la réponse on vint annoncer Mgr le duc de
Bourgogne\,: c'est l'usage du premier jour de l'an que les fils de
France rendent aux petit-fils de France, non à aucun prince du sang, la
visite qu'ils en ont reçue le matin pour la bonne année. Nous sortîmes
des cabinets pour l'aller recevoir. La visite se passa debout dans la
chambre du lit, et dura moins d'un quart d'heure. M. le duc d'Orléans
s'y posséda si bien, que je ne me fusse jamais douté de rien si j'avais
ignoré ce qui venait de se passer. La visite achevée, ils entrèrent par
le cabinet de M. le duc d'Orléans dans celui de M\textsuperscript{me} la
duchesse d'Orléans pour la même visite\,; de la porte j'entrevis la
duchesse de Villeroy, que j'appelai pour me tenir compagnie dans ce
cabinet de M. le duc d'Orléans où j'étais demeuré seul. Elle y vint à
demi rechignée, disant qu'elle aimait trop M\textsuperscript{me} la
duchesse d'Orléans pour pouvoir se souffrir dans ce cabinet-là. Je
répondis par des plaisanteries. Comme elle entendit que la visite
finissait, elle me proposa d'aller souper chez elle avec son mari et le
duc de La Rocheguyon, pour causer. Je voulus m'excuser, parce que
j'étais engagé chez Pontchartrain\,; mais elle le trouva mauvais, et ne
voulut point rentrer que je ne lui eusse promis d'aller chez elle.

J'avais entendu, lorsque M. le duc d'Orléans alla recevoir Mgr le duc de
Bourgogne, qu'on lui avait rendu réponse qu'on n'avait point trouvé le
maréchal de Besons, et qu'il avait dit qu'on allât chez Voysin, où il
était souvent. J'attendis son retour de la conduite de Mgr le duc de
Bourgogne, résolu de pousser doucement ma pointe, et de l'abandonner peu
à lui-même. Il ne tarda pas à revenir. Je lui demandai s'il avait
réponse de Besons\,: il me dit qu'il était retourné à Paris\,; et sur ce
que j'en parus chagrin comme d'un contre-temps fâcheux, il me répondit
comme un peu moins en malaise, que cela se retrouverait toujours bien.
J'eus d'abord envie de lui proposer de l'envoyer chercher à Paris\,;
mais à l'air et à la réponse, je craignis qu'il ne me dît de n'en rien
faire, et je pris mon parti de ne plus parler du maréchal, mais de lui
écrire le soir même, dont bien me prit. Je remis doucement M. le duc
d'Orléans sur le propos qu'avait interrompu la visite, moins pour le
presser que pour l'y accoutumer. Je lui représentai que ces sortes
d'engagements ne pouvaient être aussi longs que la vie\,; qu'il était
arrivé en un âge où cela devenait très-messéant\,; que le nombre
d'années et l'éclat avec lequel celui-ci se soutenait, ne lui permettait
plus de le pousser plus loin\,; que la situation où il se trouvait
fixait le moment de le finir\,; qu'il pouvait se souvenir qu'il ne
m'était guère arrivé de lui donner là-dessus d'atteintes, et que les
deux ou trois seules fois que je m'y étais échappé ç'avait été bien
délicatement\,; que je n'aurais jamais pensé à lui proposer positivement
une rupture sans le besoin pressant que j'y voyais, qui avait enfin
surmonté toutes mes craintes et mes répugnances, qui étaient telles
qu'il devait regarder la violence que je me faisais comme le plus grand
effort par lequel je lui pusse marquer mon attachement. Il écouta tout
sans m'interrompre que par de profonds soupirs, et quand j'eus cessé de
parler, il me dit qu'il comprenait bien qu'on ne prenait pas plaisir à
faire des propositions pareilles\,; qu'il sentait bien ce qui m'y avait
déterminé, et l'obligation qu'il m'en devait avoir. Alors content d'en
être venu là dès la première fois, je ne voulus pas trop presser les
choses de peur de nuire à mon dessein, en rebutant peut-être. Je laissai
languir la conversation pour donner lieu aux réflexions intérieures, et
pressé par l'heure, je pris congé\,; il voulut me retenir\,; mais, comme
j'avais mon dessein, je lui dis que j'avais un peu affaire n'ayant fait
presque que passer par Versailles en revenant par la Ferté\,; que aussi
bien il était tantôt l'heure qu'il allât voir Monseigneur chez
M\textsuperscript{me} la princesse de Conti, où, malgré l'attachement
pour M\textsuperscript{me} la Duchesse, Monseigneur allait tous les
soirs par un reste d'habitude et de considération.

Faute de mieux, l'asile offert chez le chancelier n'étant pas encore
prêt, j'allai dans son cabinet où, le trouvant seul, je lui demandai
permission d'écrire un mot pressé sur son bureau. J'y mandai en deux
mots à Besons que l'affaire venait d'être entamée, que je le priais de
se trouver le lendemain à la messe du roi, que je lui conterais tout, et
que nous prendrions nos mesures ensemble pour achever une oeuvre si
nécessaire. Comme j'achevais d'écrire, le duc de Tresmes et le maréchal
de Tessé entrèrent dans le cabinet ensemble, devant qui le chancelier
sonna pour faire fermer le billet. J'y mis le dessus et je l'allai
porter à un de mes gens pour partir sur-le-champ. Ces messieurs qui
venaient d'entrer virent bien que j'avais affaire, ne doutèrent pas que
ce ne fût au chancelier, et sortirent un moment après que je fus rentré.
Ils me laissèrent seul avec lui, et, par là, dans la nécessité de la
confidence. Sa surprise fut grande, il loua fort ma pensée, mon courage,
mon dessein\,; blâma les craintes quoiqu'à son avis même fondées, de ma
mère et de ma femme, par l'excellence de l'oeuvre et l'importance dont
elle était à la situation de M. le duc d'Orléans, telle enfin que nulle
considération ne devait arrêter, sans qu'il se flattât trop du succès,
nonobstant celui de cette première journée. J'allai de là souper chez la
duchesse de Villeroy, qui, en sortant de table, dans une autre pièce,
tandis que son mari et son beau-frère n'étaient pas encore rentrés où
nous étions, me dit encore un mot de son aversion du lieu où elle
m'avait convié. Je me mis à rire, et à répondre que cette disposition ne
lui durerait peut-être pas encore longtemps\,; que ce qui l'en éloignait
ne me déplaisait pas moins, et que peut-être n'étais-je pas inutilement
où elle m'avait vu. «\,Bon, reprit-elle avec impétuosité, voilà de
belles espérances, pouvez-vous me dire cela\,?» Là-dessus les deux ducs
entrèrent, et nous nous mîmes à causer de toutes autres choses.

Le lendemain jeudi 2, comme je m'habillais, je reçus la réponse du
maréchal de Besons. La vue d'une lettre me déplut, dans la pensée que
c'était une excuse\,; en l'ouvrant je fus plus content. Il me mandait
que j'étais le meilleur ami qui fût au monde, et qu'il se trouverait au
rendez-vous. Je m'en allai à la messe du roi, et je rencontrai Besons
dans la galerie, qui m'attendait. Je le surpris beaucoup par le récit de
ce qui s'était passé la veille. Il se récria fort sur ma hardiesse, et
quoique les choses lui parussent bien plus avancées qu'il n'eût osé
l'espérer, il ne se promit encore nul succès\,; mais il convint qu'il
fallait pousser vigoureusement ce que j'avais si fortement commencé, et
surtout tâcher d'emporter ce que nous nous étions proposé sans lâcher
prise, ni quitter de vue M. le duc d'Orléans, jusqu'à ce que nous
l'eussions obligé à faire ce grand effort sur lui-même, ou que nous
pussions juger que nous n'en viendrions pas à bout. Le roi rentré chez
lui, Besons et moi allâmes chez M. le duc d'Orléans. L'usage du
renouvellement de l'année y avait attiré quelque peu de monde qu'il
expédia bientôt et s'enferma avec nous dans ce même arrière-cabinet, où
je l'avais entretenu la veille. Comme nous y allions entrer, quelqu'un
demanda à dire un mot à Besons, et cependant M. le duc d'Orléans me
regardant en souriant\,: «\,Avouez, me dit-il, que vous avez envoyé
querir Besons\,?» Je souris aussi et le lui avouai\,; j'ajoutai que
j'avais eu envie de le lui proposer la veille, mais qu'ayant fait
réflexion qu'il me dirait peut-être de n'en rien faire, j'avais mieux
aimé lui taire mon dessein, et que ce qui m'avait hâté de sortir de chez
lui, était pour écrire au maréchal à temps qu'il reçût mon billet avant
d'être retiré. Le prince convint que je l'avais pénétré, et que, s'il
eût su ce dessein, il m'eût prié de n'en rien faire. Dans ce moment,
Besons revint. Nous entrâmes dans l'arrière-cabinet, et nous nous
assîmes.

Alors je pris la parole, et l'adressant au maréchal, je lui fis une
seconde fois le récit de ce qu'il s'était passé la veille, non pour
l'instruire de ce qu'il savait déjà, mais pour entrer en matière, et
l'exposer ainsi au tiers sans avoir l'air de la rebattre à M. le duc
d'Orléans, à qui pourtant je la voulais de nouveau faire entendre.
Besons regarda le prince, lui demanda ce qu'il lui semblait d'un ami tel
que je me montrais l'être, lui dit la résolution qu'à mon instigation
lui et moi avions prise ensemble, puisque nous avions molli, enfin,
qu'il louait et admirait mon courage, de l'avoir exécutée. Il ajouta
ensuite un raisonnement court, mais juste et fort pour le déterminer, et
se tut après pour le laisser parler. Les propos de M. le duc d'Orléans
ne furent rien de suivi, mais les élans d'un homme qui souffre une
violence étrange, et qui s'en fait même pour la souffrir. Après l'avoir
laissé quelque temps rêver, soupirer, se plaindre, je lui dis que je
souffrais moi-même autant que lui, d'avoir à l'attaquer sur un chapitre
aussi sensible\,; que de cela même il devait juger à quel point de
nécessité à tous égards indispensables il se trouvait réduit à se
vaincre\,; que j'étais parti pour ma campagne, très en peine de la
situation en laquelle je le laissais, et à la cour et dans le monde,
mais qu'à mon retour j'avais été navré de douleur d'apprendre quel
progrès en mal ces quatre mois avaient produit\,; qu'il n'était plus
question de se flatter, qu'il fallait qu'il considérât son état devenu
intolérable\,; qu'il en fallait sortir par quelque voie que ce fût, et
que toute voie lui était fermée, hors celle que je lui avais
présentée\,; qu'elle était dure, cruelle, mais unique\,; qu'après tout
il fallait bien qu'il se séparât un jour de celle qui le tenait sous son
joug\,; qu'un engagement si long, si éclatant, l'avait précipité dans un
abîme sans fond\,; que le jour de s'en arracher était venu, et qu'il ne
tenait qu'à lui de se faire de cet abîme un degré d'honneur, de faveur
et de gloire, qui le porterait en un instant plus haut qu'il n'avait
jamais été. Le maréchal répéta ces dernières paroles en les assurant et
y applaudissant, et nous demeurâmes ainsi quelque temps, nous renvoyant
la balle l'un à l'autre pour n'irriter pas en pressant trop fort, et
donner lieu à digérer ce qui avait été dit, et ce que nous continuions
de pousser en nous parlant ainsi l'un à l'autre, mais aussi sans nous
parler trop longtemps.

Après assez de silence, M. le duc d'Orléans nous demanda, mais en me
regardant, comment nous l'entendions, et par où nous prétendions le
porter si haut, par une démarche qu'il comprenait assez qui pourrait
plaire au roi jusqu'à un certain point, mais qui n'ayant rien de commun
avec les choses qui l'avaient jeté dans une disgrâce sensible, puisque
depuis cet engagement et avant ces autres choses, il s'était longtemps
soutenu à merveille avec lui, et une démarche encore qui ne faisait rien
à personne\,? comment donc nous prétendions le tirer par là de tout ce
dont on l'avait accablé, et du côté de la cour et par rapport au
monde\,? Comme j'en avais ouvert le premier propos, et que M. le duc
d'Orléans semblait m'adresser sa question plus particulièrement qu'au
maréchal, je crus que c'était à moi à répondre, et à mettre cet argument
dans tout son jour, de la force duquel je sentis bien par la question
même que je pourrais tirer un grand secours. Je pris donc la parole, et
je dis qu'en quittant une vie qui scandalisait depuis si longtemps ceux
même qui, peu attentifs à leur conscience, ne l'étaient qu'à l'honneur
du monde, il se déchargerait du blâme qu'il avait encouru en la menant,
et de tout celui encore qui lui avait été imputé pendant sa durée\,;
qu'une violente passion ne réfléchit à rien et se laisse entraîner à
tout ce qui en est la suite\,; que ses curiosités sur l'avenir, qu'il
avait cru avoir peu frappé, et être depuis longtemps effacées, s'étaient
renouvelées et grossies depuis quelque temps à tel point, qu'elles
étaient regardées comme un crime du premier ordre, comme une impiété
détestable, et par les yeux les plus favorables, comme une faiblesse qui
faisait un tort extrême à tout ce qu'on avait pensé de lui de grand et
de solide\,; qu'il était considéré comme un homme tourmenté d'une soif
ardente de régner, née à la vérité de son ambition, mais inspirée par
les choses qui lui avaient été montrées dans les exercices de ces
curiosités, reçues avec terreur des uns, avec dédain des autres, mais de
tous, comme ce qui lui avait fait monter dans l'esprit ces superbes
pensées qui ne pouvaient s'accorder avec l'homme sage, moins encore avec
le bon sujet\,; que de là se tiraient les sources de son affaire
d'Espagne, avec les raisonnements et les conséquences les plus
sinistres, et bien d'autres choses encore que je ne pouvais prendre sur
moi de lui déployer\,; il m'en pressa, c'était ce que je voulais.

Après m'en être défendu assez longtemps pour exciter sa curiosité
davantage, et pour le préparer à entendre d'affreuses énormités, je lui
dis que, puisqu'il me le commandait, et puisqu'il était encore en tel
état, qu'il était besoin qu'il sût tout, et ce que personne n'osait lui
dire, il apprît donc qu'il s'était débité, et {[}avait été{]} trop reçu
par les fripons et par ceux qui, trop éloignés, n'avaient aucune
connaissance de lui, qu'il avait un concert avec la cour de Vienne pour
épouser la reine douairière d'Espagne, dont le grand amas d'argent et de
pierreries lui serviraient à se frayer un chemin au trône d'Espagne sans
trop fouler les alliés\,; que, pour y parvenir, il répudierait sa
femme\,; que, par l'autorité de l'empereur tout-puissant à Rome par la
terreur qu'il avait imprimée au pape, il ferait casser son mariage comme
étant honteux, et fait par oppression violente, conséquemment déclarer
ses enfants bâtards\,; que, n'en pouvant point espérer de la reine
douairière d'Espagne, il attendrait sa mort du bénéfice du temps et de
l'âge pour épouser M\textsuperscript{me} d'Argenton à qui les génies
avaient promis une couronne\,; que, pour ne lui rien celer, il était
doublement heureux d'avoir conservé M\textsuperscript{me} la duchesse
d'Orléans à travers les infirmités et les dangers de la grossesse et de
la couche dont elle venait de se tirer, parce que, outre sa
conservation, le recouvrement de sa santé faisait honteusement taire les
scélérats qui n'avaient pas craint de répandre qu'elle était
empoisonnée\,; qu'il n'était pas fils de Monsieur pour rien, et qu'il
allait épouser sa maîtresse.

À ce terrible récit, M. le duc d'Orléans fut saisi d'une horreur qui ne
se peut décrire, et en même temps d'une douleur qui ne se peut exprimer
d'être déchiré d'une manière si âprement et si singulièrement cruelle.
Il s'écria plusieurs fois, et moi qui voulais avaler ce calice tout d'un
trait, sans être obligé d'y replonger mes lèvres, j'avais toujours
étouffé sa voix dans sa naissance, pour avoir le temps de tout dire de
suite. Quand j'eus fini je me tus, et M. le duc d'Orléans aussi, qui
était tout hors de lui-même. Besons, éperdu de ce qu'il venait
d'entendre, avait les yeux fichés sur le parquet qu'il m'a dit depuis
qu'il avait cru s'enfoncer, et n'osait les remuer d'épouvante. Ce n'est
pas qu'il ignorât rien de ce que je venais de dire, dont nous avions
raisonné ensemble, et dont lui-même m'avait appris le plus horrible,
mais de me l'entendre exposer nettement à ce prince, il ne savait plus
où il en était. Après quelques moments de silence, M, le duc d'Orléans
le rompit par les plaintes les plus amères de ce comble d'iniquités de
gens capables d'imaginer de tels forfaits, des desseins également
insensés et barbares, pour oser l'en accuser, et de la malice insigne,
ou de la brute stupidité de ceux qui prêtaient l'oreille à ces horreurs,
et leur langue pour les répandre. Je crus devoir laisser quelques
moments à de si justes plaintes, et au maréchal, éperdu d'ouïr faire de
tels récits en face, de reprendre un peu ses esprits. Revenu un peu à
lui, il mêla ses plaintes, mais il confirma en peu de mots la publicité
de ces terribles bruits. Enfin je repris mon discours, dont je n'avais
fait qu'une partie.

Je dis à M. le duc d'Orléans, que maintenant qu'il voyait à découvert
les causes de l'abandon du monde et de l'éloignement prodigieux du roi
pour lui et de sa famille, il apercevait du même coup d'oeil la
connexité de la rupture de cet attachement funeste, avec le
rétablissement de tout lui-même, que rompant des liens qui, par leur
durée et par les effets qu'on leur attribuait, n'étaient plus regardés
qu'avec une horreur et une indignation générale, et qui seraient au
moins toujours susceptibles de toutes les noirceurs que les scélérats
tâcheraient d'en tirer, il ferait tomber les effets avec leur cause, et
libre de cet arrangement, deviendrait net de tout crime et de tout
soupçon. Je fis encore en cet endroit une pause pour faire entrer peu à
peu, et le moins qu'il se pourrait à dégoût un raisonnement si fâcheux
par sa vérité et sa force, et ne révolter pas en accablant trop coup sur
coup. Le maréchal de Besons, qui jusque-là n'avait pas dit grand'chose,
se mit à parler davantage, et n'ayant qu'à suivre un chemin que j'avais
ouvert à force de bras, il battit à son tour avec force et justesse. M.
le duc d'Orléans, outré et abattu de plus d'une douleur bien vive, ne
disait rien, et c'était beaucoup qu'il écoutât. Nous parlions le
maréchal et moi comme l'un à l'autre, louant chacun quelque mot que son
compagnon avait dit pour l'inculquer par là plus fort, mais d'une façon
plus douce que si nous nous fussions toujours adressés au prince même,
laissant assez tomber la conversation pour fatiguer moins celui pour qui
seul elle se faisait. Alors M. le duc d'Orléans, comme sortant d'un
profond sommeil par une plainte amère, s'écria\,: «\,Mais comment m'y
résoudre, et comment lui dirai-je\,?» Ce mot échappé à l'effort de la
persuasion confirma mon espérance. Je le saisis avidement et je répondis
d'un ton ferme, qu'il avait trop bon esprit pour ne pas sentir à quel
point cette résolution était nécessaire à former promptement et à
exécuter de même\,; que, s'il n'était question que de la manière, chose
que je n'avais osé lui entamer, mais qui n'était pas moins principale à
agiter pour sa grandeur et pour sa gloire, je le suppliais d'avoir
encore la patience de m'entendre là-dessus, que mon sentiment était, et
que je croyais être aussi celui de Besons, qu'il était également inutile
et dangereux qu'il rompît avec M\textsuperscript{me} d'Argenton, si elle
restait dans Paris\,: dangereux en ce qu'il ne se tiendrait jamais de la
revoir, et que la revoir et renouer avec elle serait même chose, inutile
en ce que ne la revoyant même pas, quoique supposition impossible, il le
serait pour le moins autant d'en persuader le roi et le monde, par quoi
tout son effort ne lui servirait à rien.

À ces mots, il me demanda avec impétuosité ce que je prétendais donc
qu'il fît, et de quel front s'empêcher de la voir au moins pour rompre,
puisque, s'il rompait, ce ne serait ni par dégoût ni par mécontentement
d'elle. Je répondis avec un air de froide tranquillité, que, s'il était
résolu à la revoir comme que ce pût être, tout ce combat était
superflu\,; et le maréchal prenant en même temps la parole, la mena
bien, s'échauffa, et conclut que la revoir serait un bail nouveau, plus
certain, plus fort, plus durable que le premier\,; qu'au nom de Dieu il
ne se laissât pas succomber à cette faiblesse, dont il se repentirait à
jamais\,; que, si déterminément il la voulait revoir, il quittât toute
pensée de rompre avec elle, ou que, s'il était assez généreux pour se
surmonter en ce point, qu'il ne se présentât pas à une défaite assurée,
et qu'il se gardât sur toutes choses de l'aller voir, et même de lui
écrire. J'appuyai cet avis si salutaire tout du mieux que je pus, et sur
les difficultés et les raisons qu'il chercha à opposer, j'ajoutai que
cette manière était même celle de toutes la moins désobligeante,
puisqu'elle témaignait un amour si redoutable qu'on n'osait s'exposer à
voir celle qui l'allumait, quand on avait résolu de l'éteindre. De là je
repris mon discours, et je lui dis que n'y ayant que dangers de toutes
parts de laisser M\textsuperscript{me} d'Argenton à Paris en rompant
avec elle, et quelque grand qu'il fût, n'ayant pas le pouvoir de l'en
bannir, ni quand il l'aurait, grâce à l'exercer sur elle, cela même lui
enseignait la route qu'il devait tenir, et lui fournissait en même temps
tout moyen d'un rétablissement complet\,; qu'il fallait qu'il allât
trouver le roi, qu'il lui dît qu'il venait à lui comme à un asile contre
soi-même\,; qu'une passion démesurée, à laquelle il s'était abandonné
tout entier, lui avait trop déplu, et par son propre déréglement et par
toutes les suites funestes et les malheurs qui en avaient été les
fruits\,; qu'il ne pouvait plus vivre ainsi dans sa disgrâce, si
coupable à ses propres yeux\,; qu'il se jetait donc entre ses bras avec
son ancienne confiance en ses anciennes bontés, pour qu'il lui pardonnât
tous les déplaisirs que ses désordres lui avaient causés, et pour qu'il
aidât sa faiblesse à se tirer d'un engagement, qu'il sentait qu'il ne
pouvait rompre et qu'il le suppliait de briser\,; qu'il lui demandait de
profiter de cet instant qu'une lueur de raison et de devoir l'avait
saisi, et de faire ordonner à M\textsuperscript{me} d'Argenton de sortir
de Paris, afin que, secouru par l'absence, il pût soutenir sa résolution
et le pas qu'il faisait pour sortir des abîmes où l'amour l'avait
précipité. J'ajoutai que, parlant de la sorte à un oncle qui l'avait
tendrement aimé, et qui lui en avait donné toutes sortes de marques, à
un beau-père outré du malheur de sa fille dont par là il verrait la fin,
à un roi aisément pris par la confiance, à un homme d'expérience trop
funeste de la puissance et des fruits de l'amour passionné, il le
toucherait tellement par toutes ces choses à la fois, qu'en un instant
il ferait de lui le père de l'enfant prodigue\,; que je savais
d'ailleurs qu'une des choses du monde qui avait le plus outré le roi
contre lui dans l'affaire d'Espagne était la tendre amitié qu'il s'était
toujours sentie pour lui, et qu'il avait espérée réciproque par un air
de liberté avec lui qu'il avait remarquée et sentie infiniment plus que
dans ses enfants, et qui lui avait extrêmement plu\,; que le dépit de se
voir trompé dans une pensée qui lui était douce l'avait horriblement
piqué contre lui\,; qu'il s'en était une fois entre autres expliqué
ainsi à M\textsuperscript{me} de Maintenon, en entrant chez elle plein
de la chose, les lèvres lui tremblant de colère en lui faisant ces
plaintes, et lui parlant de cela comme d'un malheur extrêmement
sensible\,; qu'un recours au roi, tendre, touchant, confiant, avouant
tout sans rien dire, et cachant sous le voile de son embarras tout ce
qu'il n'était ni bon ni à propos d'expliquer, aurait la force de faire
renaître dans le roi ses premiers sentiments pour lui, que cette
conduite lui ferait croire avoir été bien fondés, avec une satisfaction
d'autant plus utile qu'il se trouverait affranchi du reproche qu'il
s'était fait à lui-même d'avoir été la dupe d'une amitié qui n'était
pas, et se trouverait flatté en sa partie sensible de voir son neveu se
jeter entre ses bras pour le délivrer d'un lien qu'il n'avait pas la
force de briser soi-même, tandis qu'il se souviendrait, avec ce retour
satisfaisant d'amour-propre, que ce sacrifice se serait fait uniquement
à lui\,; que de ces favorables dispositions naîtraient aisément en lui
l'opinion que toutes les fautes, les plus graves imputations, les
curiosités condamnables et suspectes, que l'affaire d'Espagne étaient
les suites, les fautes, les effets d'une passion si forte et d'un si
violent amour, dont toute la faiblesse lui était montrée par la manière
de s'en arracher, qui le flatterait encore.

M. le duc d'Orléans n'eut pas la patience d'en entendre davantage sans
m'interrompre. «\,Quoi\,! me dit-il, vous voulez que je la charge
(M\textsuperscript{me} d'Argenton) de toute l'iniquité qu'on m'a
imputée, et que j'en sorte à ses dépens\,? Et n'est-ce pas assez de
rompre, si je m'y résous, sans la livrer encore\,? outre que ce serait
injustement sur les affaires d'Espagne, auxquelles elle n'a eu aucune
part\,; et pardonnez-moi si je vous dis que je m'étonne que ce soit vous
qui m'ouvriez une telle porte. --- L'amour vous aveugle, monsieur, lui
répondis-je, et vous fournit une délicatesse que je vous avoue que je ne
crains pas de combattre, pourvu que, sur un point qui vous est si
capital, vous veuillez bien m'écouter avec défiance de vous-même. Vous
sentez que je veux faire de M\textsuperscript{me} d'Àrgenton le bouc
émissaire de l'ancienne loi, et vous, vous vous en hérissez comme d'une
proposition qui vous flétrirait. Je ne me défends pas que ce ne soit mon
dessein, M. le maréchal sera notre juge.\,» M. le duc d'Orléans s'écria
encore et pressa le maréchal de parler, qui, après plusieurs circuits
pour ne rien dire, prononça enfin que cela lui répugnait. Je ne me
rendis point et voulus me faire entendre, et je dis que ce qui en aucun
cas possible ne devait être fait, c'était de tirer son avantage aux
dépens d'un autre, beaucoup moins par un mensonge infiniment pis quand
cet autre avait été dans notre liaison, mais qu'ici rien de tout cela.
Que la vérité se conservait entière des deux côtés\,; que le dommage
était nul de l'un, l'avantage infini de l'autre, et que les choses étant
exactement ainsi, je n'y voyais nulle matière de scrupule d'honneur, de
probité, ni de délicatesse. Ils convinrent tous deux du principe et m'en
laissèrent faire l'application. Continuant donc, je demandai au prince
s'il pouvait disconvenir que son amour ne l'eût pas entièrement retiré
de tous les devoirs de famille et de tous ceux encore de sujet du roi si
principal et si bien traité, pour le jeter dans une vie obscure,
retirée, avec un tas de petites gens, parmi des amusements indignes de
son rang et de son esprit, dans des profusions qui avaient attaqué les
fondements solides de ce que, dans les particuliers, on appelle leur
fortune. Je lui demandai s'il pouvait nier que ce ne fût pas ce même
amour qui l'avait replongé plus avant, et plus continuellement que
jamais, dans ces curiosités auparavant bannies de chez lui, et si
sinistrement interprétées, et si ce qui s'en était justement et
véritablement débité jusque par lui-même n'avait pas donné lieu aux plus
fâcheuses augmentations et aux plus funestes interprétations qui s'en
étaient faites.

Il m'avoua nettement ces choses, et, sur cet aveu, je pris droit de
conclure qu'il était donc vrai à la lettre que son amour l'avait jeté
dans les plus grands dérèglements, dans des suites funestes, dans des
désordres, dans des malheurs, dans des abîmes\,; que non-seulement cela
était trop vrai, mais trop connu et trop notoire\,; qu'il ne dirait donc
rien au roi de faux ni de nouveau en lui parlant comme je le lui
proposais\,; que, conséquemment, rien à cet égard ne devait l'arrêter\,;
que, pour ce qui était de craindre de faire du mal à
M\textsuperscript{me} d'Argenton, cette appréhension me paraissait
absurde\,; que séparée d'avec lui, et hors de Paris, qui était une seule
et même chose, je ne voyais point ce qui lui pouvait arriver de
fâcheux\,; qu'il pourvoirait sans doute à l'aisance de sa vie, outre ce
qu'il lui avait déjà donné\,; que la cessation du commerce ne devait pas
emporter celle de la protection\,; que le roi même avait été trop
amoureux en sa vie pour n'être pas susceptible de la délicatesse et du
devoir de ce procédé\,; qu'agir contre M\textsuperscript{me} d'Argenton
en quelque sorte que ce fut, quoique séparée de son neveu, serait agir
contre son neveu même, et le flétrir cruellement, chose bien éloignée
d'un rapprochement tendre et sincère, qui était l'unique but que je me
proposais\,; qu'ainsi il était clair que M\textsuperscript{me}
d'Argenton n'avait rien à craindre, mais beaucoup mieux à espérer de la
façon que j'avais proposée de recourir et de parler au roi, et qu'à
l'égard de l'avantage qu'en retirerait M. le duc d'Orléans, je n'en
voulais d'autre juge que lui-même\,; quant à l'affaire d'Espagne, que
n'étant point de nature à pouvoir en reparler, il ne pouvait, avec
aucune bienséance, dire au roi que M\textsuperscript{me} d'Argenton y
avait ou n'y avait point de part\,; que son juste et bienséant embarras
en parlant au roi sur sa rupture, ne lui permettait aucun détail\,;
qu'ainsi lui dire que sa maîtresse était cause de ceci et non de cela
étant chose ridicule et absurde, et l'ayant en effet et de son propre
aveu entraîné dans tout, excepté dans l'affaire d'Espagne, rien n'était
plus utile, plus dans l'ordre, plus à propos, plus hors de toute
atteinte de la moindre blessure de délicatesse et d'honneur, que de
parler au roi dans le vague dont je lui avais donné l'idée\,; que, si
après le roi joignait dans la sienne l'affaire d'Espagne à tout le
reste, comme lui n'exprimait rien, et moins celle-là que nulle autre,
comme il n'en pouvait, en quoi que ce pût être, arriver ni pis ni mieux
à M\textsuperscript{me} d'Argenton, je ne voyais pas quel scrupule il
s'en pouvait faire, ni pourquoi se priver d'un aussi grand bien que
celui de se raccommoder si parfaitement avec le roi, auquel il ne
pouvait s'empêcher de parler comme je le pensais, pour recourir à lui
avec succès certain et infiniment nécessaire, puisque les choses en
étaient venues à ce point que c'était très-peu faire que rompre pour
rompre, si, au plaisir de père, il n'ajoutait au roi l'aise de père de
famille, d'oncle, et surtout ceux de roi et de maître.

Nous disputâmes assez longtemps là-dessus et Besons ne témoignant pas se
rendre entièrement, je conclus que je ne voyais pas quel scrupule
pouvait rester, M\textsuperscript{me} d'Argenton à couvert, le mensonge
banni, la vérité conservée, et tout avantage procuré, sans que la
ténuité du scrupule pût se fonder sur aucune base perceptible dans la
manière pleinement vraie, juste et honnête de se le procurer. M. le duc
d'Orléans soutenait toujours qu'il y avait là un tour de courtisan, et
la droiture du maréchal, une fois hérissée, avait peine à s'accoutumer à
ma manière de penser, sur quoi je m'avisai de leur demander ce qui les
choquait. Besons voulut répondre, mais ne pouvant trouver sous sa main
rien, pour ainsi dire, susceptible d'être empoigné, et y sentant au
contraire sûreté pour M\textsuperscript{me} d'Argenton, vérité effective
dans la chose, l'éblouissement emportant l'affaire d'Espagne, il cessa
d'être peiné\,; et depuis, M. le duc d'Orléans est convenu plus d'une
fois avec moi qu'il n'avait disputé que pour prolonger la dispute, et
détourner cependant l'objet véritable de la conversation. Il cessa alors
de contester sans s'avouer rendu\,; et, après avoir déclaré que cette
contestation ne serait bonne que lorsqu'il se serait déterminé sur le
grand point (de rompre), ce qu'il n'était du tout point, il retomba dans
un silence très-profond que le maréchal n'interrompit pas, et que je ne
voulus pas troubler sitôt après une reprise de conversation si vive.

Cependant je m'aperçus bientôt que non-seulement M. le duc d'Orléans
souffrait beaucoup en se taisant, mais qu'il était agité entre parler et
ne parler pas. Il lui échappa ensuite des commencements de paroles,
qu'un effort retenait à demi prononcées, ce qui, s'étant répété
quelquefois, m'enhardit à lui dire que je voyais bien qu'il voulait se
soulager avec nous de quelque peine qui l'agitait\,; que je ne le
pressais point de le faire, mais que je le suppliais de considérer qu'il
était entre ses deux plus assurés serviteurs, et dans un état qui ne
demandait point de contrainte. Il ne répondit rien, et je me tus. Après
un assez long et vif combat intérieur, il nous dit, comme tout à coup,
qu'après avoir bien balancé, il se sentait pressé d'une chose qui lui
faisait une peine infinie à nous dire, mais que la situation en laquelle
il se trouvait en lui-même, et l'entière confiance qu'il avait en nous
le forçait à la dire\,; que parmi tout ce qui le combattait contre ce
que nous essayions de lui persuader de faire, une des choses qui le
peinaient le plus était son domestique, et la vie en laquelle il
retombait en rompant. Je repris la parole. Je lui dis que j'avais
commencé à sentir ce qui l'agitait entre parler et se taire avant qu'il
eût lâché ce mot\,; que, par respect, je n'avais osé l'en laisser
apercevoir, mais que j'étais ravi qu'il eût enfin pris le parti de
l'ouverture avec de si véritables et de si sincères serviteurs\,; puis,
entrant en matière, je lui dis que je ne m'étonnais pas qu'il eût peine
à s'engager dans une sorte de vie qui lui était tout à fait inconnue, et
dont il n'avait jamais eu le temps de connaître les douceurs.

À ce mot qu'il releva avec une sorte de transport, il nous avoua un
éloignement extrême pour sa femme, et tel qu'il ne se sentait pas
capable de {[}le{]} vaincre jamais. Je regardai le maréchal, et je dis,
en lui adressant la parole, que c'était là la chose à laquelle je
m'étais le plus attendu, et qui aussi m'embarrassait le moins\,; qu'il
était tout naturel que M. le duc d'Orléans, marié contre son gré, excité
au dégoût de son mariage par ceux-là mêmes dont l'autorité l'en devait
défendre contre celle du roi, ou combattre ce dégoût après l'avoir mis
en état de le regarder comme le plus grand malheur de sa vie, tombé
ensuite en de mauvaises mains qui, par intérêt ou par flatterie,
l'avaient non-seulement soutenu dans ce dégoût, mais persuadé que le
marquer était une partie principale de sa dignité et de sa gloire,
plongé ensuite en des déréglements passagers mais continuels, enseveli
enfin dans une passion qui occupait tout son coeur et tout son temps,
qu'il était, dis-je, non-seulement naturel, mais impossible que tout
ayant concouru à former et à fortifier un éloignement si dangereux, il
ne fût devenu tel que M. le duc d'Orléans nous le représentait\,; mais
que c'était au bon esprit, aux sages réflexions, aux considérations
générales et particulières à détruire l'ouvrage pernicieux des passions,
des mauvais conseils, du temps si longuement écoulé dans l'habitude de
ces sentiments pour en prendre d'autres tout contraires, et dans
lesquels seuls il trouverait son repos et sa véritable gloire, avec la
grandeur solide de sa famille particulière.

Besons appuya intiniment ces propos, loua M\textsuperscript{me} la
duchesse d'Orléans, et me donna lieu de la louer aussi\,; mais ces
louanges, bien loin de produire un bon effet, irritèrent M. le duc
d'Orléans, et le replongèrent dans son premier silence d'agitation et
d'embarras. Enfin il débonda (et voici où la confidence et la confiance
fut pleine, entière, nette, ne cachant ni choses ni noms) et nous dit ce
que nous eussions voulu ne point entendre, mais ce qu'il fut pourtant
très-heureux qu'il nous dît. Le maréchal se jeta sur des généralités
très-vraies, mais j'eus le bonheur de trouver par des hasards à moi
très-particuliers, mais tout à fait naturels et justes, des raisons
tellement pertinentes, et des preuves si nettes et si exactes que M. le
duc d'Orléans céda à leur force, ne put s'empêcher de demeurer
convaincu, et ne put me rien opposer par diverses répliques, sinon que
je ne lui dirais pas du mal de sa femme quand j'en saurais. «\,Non,
monsieur, lui répondis-je, le regardant avec feu, très-assurément, je ne
vous en dirais pas\,; mais aussi ne vous parlerais-je pas aussi
positivement que je fais, si je n'étais non-seulement très-persuadé,
mais si j'avais aucun soupçon qu'il s'en pût prendre d'elle, puisque
vous en dire du mal, quelque vrai qu'il fût, serait une noirceur
affreuse, et que, s'efforcer de vous persuader en sa faveur un mensonge,
et par des faits décisifs et positifs qui seraient contre la vérité,
contre ma conscience, serait une autre sorte de trahison. Si je voyais
donc que vous eussiez malheureusement raison, content de n'en pas
convenir, je me tirerais d'embarras comme je pourrais par des verbiages
généraux qui ne manquent jamais, mais je me garderais bien d'avancer des
choses et des preuves positives qui répugneraient également à la vérité,
à l'honneur, et à la conscience, qui chez moi vont et doivent aller
avant tout.\,»

Cette assertion si nette, si ferme, et en même temps si sincère, força
son dernier retranchement\,; il ne put même dissimuler sa joie de
pouvoir sûrement compter qu'il avait été méchamment trompé. Il s'en
dilata davantage sur cet étrange chapitre, et, battu sur le fond des
choses, il nous présenta beau pour l'être encore plus sur les indignes
et scélérats auteurs. Il nous nomma M\textsuperscript{me} la Duchesse,
M\textsuperscript{me} d'Argenton et quelques autres femmes perdues, la
plupart intimes de sa maîtresse, auxquelles nous lui fîmes honte d'avoir
ajouté foi, pour le peu qu'en méritait leur réputation, sur des
personnes même indifférentes, combien moins encore avec l'intérêt si
sensible qu'elles avaient à mettre entre M\textsuperscript{me} sa femme
et lui les derniers éloignements. Besons parla ensuite dignement et
assez longtemps en ce même sens. J'ajoutai après qu'il devait à jamais
bénir cette journée, où le hasard lui avait fourni des réponses et des
preuves sans réplique, et où sa raison forcée se voyait contrainte de
s'avouer ses fautes, et de s'en repentir salutairement. Devenu de là
plus hardi par avoir ôté la cause la plus empoisonnée de l'éloignement,
je repris les louanges de M\textsuperscript{me} sa femme, sur lesquelles
je me rendis éloquent. Je lui fis valoir sa patience sur la conduite
qu'il avait avec elle, la retenue exacte de toute plainte, le vif
intérêt qu'elle prenait à sa gloire, ses déplaisirs et ses mouvements
dans son affaire d'Espagne, l'utilité de la tendrese du roi pour elle en
cette fâcheuse occasion, et je m'étendis sur tous ces points. Besons m'y
seconda très-bien, et M. le duc d'Orléans écouta tout avec beaucoup de
patience. Nous nous mîmes après tous deux à lui vanter les douceurs et
le prix d'un heureux mariage\,; et comme nous en parlions tous deux par
la plus douce expérience, nous lui fîmes beaucoup d'impression.

Ce fut l'état dans lequel nous le laissâmes, pressés par l'heure déjà
fort tardive, et malgré lui, et en vérité bien fatigués d'un travail si
rude et si étrange. Il nous conjura de ne le point abandonner dans le
terrible combat où nous l'avions engagé\,; et nous l'assurâmes que, dès
que nous aurions dîné, nous ne différerions pas à revenir auprès de lui.
En sortant, Besons me dit que j'étais le meilleur et le plus hardi ami
qui se pût imaginer, que la force de ce que j'avais dit l'avait fait
trembler à plusieurs reprises jusqu'à lui ôter la respiration, avouant
que cela était nécessaire pour arracher de force ce qui ne se pouvait
espérer autrement, et ce qu'il n'espérait même guère encore. Il me
promit de s'encourager pour seconder ma force le mieux qu'il pourrait,
me dit qu'il fallait surtout empêcher ce prince d'aller à Paris, où un
moment renverserait tout notre travail, et tâcher même à ne le pas
perdre de vue jusqu'au bout. Nous nous séparâmes promptement pour ne
donner pas aux gens de M. le duc d'Orléans à penser et à raisonner plus
que nous sûmes après qu'ils faisaient déjà d'une séance si longue, après
la mienne de la veille\,; et nous convînmes de retourner aussitôt que
nous aurions dîné, et de passer toute la journée avec M. le duc
d'Orléans.

\hypertarget{chapitre-ii.}{%
\chapter{CHAPITRE II.}\label{chapitre-ii.}}

1710

~

{\textsc{Troisième conversation avec M. le duc d'Orléans, le maréchal de
Besons en tiers.}} {\textsc{- Duc d'Orléans fait demander à
M\textsuperscript{me} de Maintenon à la voir.}} {\textsc{- Propos tête à
tête entre Besons et moi.}} {\textsc{- Singularité surprenante qui
m'engage à un serment, puis à une étrange confidence.}} {\textsc{-
Rupture de M. le duc d'Orléans avec M\textsuperscript{me} d'Argenton.}}
{\textsc{- Colloques entre Besons et moi.}} {\textsc{- Dons de M. le duc
d'Orléans à M\textsuperscript{me} d'Argenton en la quittant.}}
{\textsc{- Surprise et propos de la duchesse de Villeroy avec moi.}}

~

Dans cet intervalle, je fis réflexion que, dans ma conversation tête à
tête de la veille, il m'avait paru que M. le duc d'Orléans s'était trop
appuyé sur sa proximité du roi et des fils de France\,; il m'avait avoué
que, lorsque le roi lui avait parlé de l'affaire d'Espagne pour y mettre
fin, et se donnant pour croire tout ce qu'il lui voulut dire, il l'avait
fait en peu de paroles avec poids et gravité, et lui avait conseillé de
parler aussi à Monseigneur, lequel lui avait répondu mot pour mot comme
avait fait le roi, mais avec bien plus de gravité et de froid encore\,;
que ce concert d'une si semblable réponse la lui avait fait juger
concertée, et de là soupçonner que cette réponse si pareille et si
compassée était de gens non persuadés, et sur ce qu'il avait insisté
avec moi, qu'il avait trouvé Mgr le duc de Bourgogne assez favorable, et
M\textsuperscript{me} la duchesse de Bourgogne entièrement, je lui avais
répondu que ce prince avait été aussi piqué et aussi sévère que le roi
et Monseigneur, mais adouci par son épouse, qui, non moins sensible
qu'eux, avait néanmoins cherché à les apaiser par honneur pour son
oncle, et par amitié pour M\textsuperscript{me} la duchesse d'Orléans\,;
mais qu'il se mécomptait beaucoup, si pour tout cela, il se croyait bien
avec elle\,; qu'il fallait qu'il pensât qu'elle avait fait comme une
mère qui veut tirer son fils des mains de la justice, et qui, bien
qu'elle le sache coupable, dit et fait tout ce qu'elle peut pour le
sortir d'affaire, et elle d'affront, bien résolue après de le châtier en
particulier et de lui faire sentir, sans danger, toute son indignation.
Cette comparaison le fit souvenir qu'il l'avait priée de faire ses
remercîments à la reine d'Espagne sur la modération de ses lettres en
cette occasion, que la réponse en devait être arrivée depuis plusieurs
mois sans qu'il en eût ouï parler, et qu'il trouvait en effet
M\textsuperscript{me} la duchesse de Bourgogne bien plus réservée avec
lui depuis la fin de cette affaire d'Espagne, que pendant qu'elle avait
duré. Ces réflexions qui me revinrent me résolurent à lui rompre tout
reste de retranchements sur l'amitié dont il s'était voulu flatter.

Plein de ces pensées je retournai chez M. le duc d'Orléans un peu avant
trois heures\,; je le trouvai dans son entre-sol, et déjà Besons avec
lui. Il me vit arriver avec plaisir, et me fit asseoir entre lui et le
maréchal, que je complimentai sur sa diligence, et lui demandai sur quoi
ils en étaient. «\, Toujours sur la même chose, me dit-il, et dans le
même combat.\,» Je répondis que si était-il enfin temps de mettre fin à
ces incertitudes et de prendre une bonne résolution pour sortir du plus
fâcheux et dangereux état où prince de ce rang se pût jamais trouver, et
tout de suite je mis sur le tapis son peu de ressource, puisque celle-là
même qui avait le mieux fait pour lui dans son affaire d'Espagne lui
manquait depuis dans tout le reste de propos délibéré. Je m'étendis
beaucoup là-dessus sans que M. le duc d'Orléans m'interrompît que par
des soupirs et des changements de postures dans sa chaise, d'un homme
fort en malaise avec lui-même.

Vers ce temps-là entra Mademoiselle, suivie de M\textsuperscript{me} de
Maré, sa gouvernante\,; elle embrassa M. son père, qui l'aimait avec
passion dès sa plus tendre enfance, et se mit à causer avec lui, et moi
avec M\textsuperscript{me} de Maré. Elle était ma parente et fort mon
amie. Je lui dis tout bas d'emmener sa princesse, parce qu'elle
interrompait quelque chose qui voulait être suivi. Elle n'en eut pas la
peine, parce qu'un moment après M. le duc d'Orléans la renvoya, et
aussitôt nous nous rassîmes.

Cette visite me donna occasion de prendre de nouvelles armes, et de me
servir de la tendresse paternelle. Je savais, par M. le duc d'Orléans,
qu'il y avait près de deux ans que le roi, de lui-même, lui avait parlé
de Mademoiselle comme d'un parti qui pouvait être convenable pour M. le
duc de Berry. Je demandai à M. le duc d'Orléans ce qu'il prétendait en
faire, qu'ayant plus de quatorze ans et la figure d'une jeunesse plus
avancée, il me semblait qu'elle devait commencer à lui peser\,; qu'après
les grandes espérances que le roi lui avait fait naître si naturellement
pour un établissement si solide pour sa grandeur personnelle, et celle
de M. son fils, si agréable encore en ne la séparant {[}point{]} de lui
par un mariage étranger, tout autre gendre que M. le duc de Berry lui
devait paraître une chute\,; qu'il s'était mis en état néanmoins de
faire évanouir toutes ces pensées, et que je ne voyais aucun moyen de
les faire renaître que la rupture, et la manière de la faire que je lui
avais proposée. M. le duc d'Orléans ne se récria plus sur la manière,
mais seulement sur la rupture, et avec plus d'angoisse que de
sécheresse, ce qui me donna tout courage d'aller plus en avant. Je lui
demandai donc si, se résolvant enfin d'y venir, il n'en parlerait pas à
M\textsuperscript{me} de Maintenon. Il demeura quelques moments sans me
répondre, puis dit que s'il y venait il faudrait bien qu'il lui en
parlât. Alors j'insistai à ce qu'il s'en expliquât avec elle de la même
manière que je lui avais conseillé de faire avec le roi, mais de
s'étendre davantage avec elle d'un air de confiance sur sa douleur de
l'état auquel il se sentait avec le roi, se répandre en tendresse et en
reconnaissance pour lui, bien inculquer que cette tendresse seule lui
arrachait ce sacrifice, et l'espérance de rentrer par un effort si
douloureux dans ses bonnes grâces et sa familiarité premières, appuyer
que nulle autre considération n'eût pu l'obtenir de lui.

Il entra très-bien dans ce raisonnement, et le maréchal aussi. J'en pris
occasion de m'étendre sur l'inutilité de la vie suivie et d'une conduite
unie et sage avec le roi et avec elle, que leur goût était constant pour
les prosélytes et les pénitents du monde, que tout était plein de gens
irréprochables, même dans les choses de leur gré, qui n'avaient jamais
pu rien faire, et de fortunes agréables, de plusieurs solides, de
quelques-unes même éclatantes de gens qu'ils avaient haïs et méprisés,
de gens perdus par tout ce qu'une conduite peut entasser de plus
misérable et de plus honteux, du retour desquels leur amour-propre
s'était trouvé flatté, qu'ils avaient récompensé en ces personnes\,;
qu'une dévotion ignorante y aidait encore par la considération mal
appliquée de la miséricorde de Dieu sur les pécheurs, qui les rendait
dupes de l'effort de l'ambition, qui souvent prenait la place de l'amour
des plaisirs, et changeait le libertinage en une assiduité dont la
constance eût langui sans être regardée, et dont le retour était au
contraire presque toujours salarié\,; que, égaré au point où il l'était,
cette imitation lui restait pour toute ressource, que je le conjurais de
songer avec fruit qu'il ne lui restait plus un seul instant à perdre
pour y recourir, qui tous lui étaient infiniment précieux.

Le maréchal appuya de son côté, mais je vis distinctement et avec
frayeur que M. le duc d'Orléans était moins réduit que lorsque nous
l'avions quitté le matin, et qu'il avait funestement repris haleine
pendant notre courte absence. Je le pressai donc, et lui demandai s'il
commencerait par le roi ou par M\textsuperscript{me} de Maintenon. Il me
répondit avec une fermeté que je n'avais point sentie dans les deux
précédentes conversations, qu'il n'était point encore question qu'il pût
prendre un parti, mais que, s'il avait à le prendre, il parlerait
d'abord à M\textsuperscript{me} de Maintenon\,; que cela lui marquerait
plus d'amitié et de confiance, et l'engagerait à mieux faire valoir la
chose au roi que s'il ne lui en parlait qu'après\,; qu'il pourrait même,
en lui confiant sa résolution, recevoir d'elle des conseils utiles pour
la manière de s'en déclarer au roi, et plus encore d'appui, parce que,
engagée par la confiance et par la déférence à suivre ses avis, elle se
ferait un honneur de les lui rendre les plus avantageux qu'elle
pourrait, et de former pour la suite une sorte de liaison avec lui dont
il pourrait tirer beaucoup d'avantages. Nous pesâmes ses raisons Besons
et moi, et nous les trouvâmes très-sages et très-judicieuses, mais en
même temps un raisonnement si libre, dans un homme que nous avions
laissé si peu en état d'en former aucun, me fit peur.

Je compris fort clairement que M. le duc d'Orléans avait repris des
forces contre nous pendant l'intervalle de notre absence, et je sentis
par là que, si nous n'emportions la rupture à ce coup comme d'assaut, il
ne la fallait plus espérer après le long espace de la nuit jusqu'au
lendemain que la conversation se pourrait reprendre\,; que peut-être
nous échapperait-il tout à fait, ou par s'être déterminé pendant la nuit
à n'écouter que l'amour, et nous fermerait la bouche quand au matin nous
penserions retourner à la charge, ou que, prenant peut-être un parti
plus assuré, nous le trouverions allé à Paris quand nous viendrions le
chercher. Cette réflexion, qui me frappa tout à coup, et que je pesai de
toute l'application de mon esprit, tandis que Besons discourait sur les
raisons de parler à M\textsuperscript{me} de Maintenon avant de parler
au roi, me détermina à ramasser toutes mes forces pour embler d'effort
une sanglante victoire sans plus rien ménager. Je laissai donc parler
Besons tant qu'il voulut, et, après qu'il eut fini, je demeurai dans un
profond silence. Je rêvais cependant à ce que j'avais à dire, et la
vérité est que j'en tremblais.

Enfin, après un assez long temps que personne ne disait mot, je regardai
tristement M. le duc d'Orléans, et je lui dis que, quelque peine qu'il
ressentît du combat auquel nous l'avions engagé, je le suppliais de se
bien fortement persuader que le nôtre était pour le moins aussi
terrible\,; que pour lui il n'avait à combattre que l'amour, et que je
convenais que cela était effroyable pour un homme aussi passionnément
épris, mais qu'il ne nous refusât pas de réfléchir sur l'horrible peine
qu'un ami véritable ressentait d'affliger un ami, de lui flétrir le
coeur aux parties les plus sensibles, de lui dire des choses dures,
fâcheuses, poignantes, de le déchirer, de le désespérer par une violence
extrême, et par des raisons de cette violence plus solidement et presque
aussi sensiblement cruelles que la violence même\,; combien plus quand
cela ne se passait non plus entre amis égaux, mais entre gens aussi
disproportionnés que nous l'étions de lui, aussi accoutumés par là au
respect, à la complaisance, à toute déférence, à éviter avec le soin le
plus exact jusqu'aux moindres choses qui pourraient non pas formellement
déplaire, mais plaire moins, surtout quand à ce respect profond du rang
en était joint un autre bien plus intime dans l'âme, et qui retenait
infiniment plus que l'autre, parce qu'il naissait de l'estime et de
l'admiration de l'esprit, des lumières et de plusieurs vertus de cet
ami, qui augmentait l'honneur, la douceur, la reconnaissance d'une telle
amitié\,; que de là il devait mesurer la grandeur de notre combat, et
sur la grandeur de notre combat la grandeur de la nécessité de ce qui
nous avait fait résoudre à l'entreprendre, et qui nous le faisait
soutenir avec une sorte d'honneur qui ne se pouvait rendre\,; qu'au nom
de Dieu il daignât y réfléchir et ne nous accabler point du poids
immense de la douleur d'avoir si longuement et si cruellement combattu
en vain\,; qu'il se pouvait souvenir qu'à deux fois différentes je
m'étais hasardé de lui jeter quelques propos sur cette rupture avec
grande circonspection et presque en monosyllabes\,; qu'une troisième
fois j'avais pris confiance de pousser jusqu'à une seconde période, et
que sur l'air qu'il prit tant soit peu moins ouvert, je m'étais arrêté
tout court et avais changé de discours\,; qu'il devait donc comparer ces
extrêmes réserves d'alors avec tout l'opposé de maintenant, et en
conclure qu'il n'y avait donc que la plus âpre et la plus pressante
nécessité qui m'avait forcé et soutenu\,; qu'encore une fois il y fît
des réflexions salutaires, qu'il ne s'abandonnât pas lui-même dans un
abîme sans fond pour n'avoir pas la force de s'en tirer, et nous au
désespoir de l'y voir périr sans aucune espérance de ressource. Je me
tournai ensuite au maréchal, pour l'exhorter à presser et à ne laisser
pas sur ma seule insuffisance le poids d'une affaire si capitale. Je me
tus après pour reprendre haleine et courage, et pour observer, dans la
réponse et dans la contenance du prince, ce qu'opérait un discours si
touchant. Besons, ému par ce que je venais de dire, voulut parler aussi
en même sens. Il fit des représentations pleines de justesse, mais trop
mesurées pour l'état auquel nous nous trouvions. L'esprit de M. le duc
d'Orléans était désormais convaincu, ou hors de moyen de l'être, après
tout ce que nous lui avions démontré. Il n'était plus question que de
déterminer une volonté arrêtée par une passion qui la tyrannisait, et
cette opération violente avait un extrême besoin de force et de
véhémence. Il échappa à M. le duc d'Orléans de témoigner en s'adressant
à Besons que, s'il se séparait de sa maîtresse, ce ne serait qu'à
condition de la voir et de l'y préparer lui-même, et là-dessus Besons
s'écria qu'avec cette résolution, non-seulement il ne romprait pas
présentement avec elle, mais qu'il ne la quitterait jamais\,; que, s'il
avait tant de peine à prendre en son absence un parti salutaire et
forcé, que deviendraient les réflexions en sa présence\,? que l'amour
les détruirait en un instant, que ses efforts ne lui serviraient que de
trophées et à la douceur de s'y livrer sans réserve et tout de
nouveau\,; qu'il était absurde d'imaginer qu'il pût résister aux larmes
et aux caresses, et que la fin de tout ceci serait un nouveau bail plus
honteux, plus durable, plus dangereux encore que celui qu'il s'agissait
de rompre, également cruel pour ses amis, et funeste pour lui.

Un grand silence succéda à ces vives reprises. Elles firent sur M. le
duc d'Orléans une impression dont je ne tardai pas à m'apercevoir, à un
abattement et à une sorte d'amértume que j'avais regrettée en lui,
tandis que je l'avais ouï raisonner si librement sur parler à
M\textsuperscript{me} de Maintenon avant d'aller au roi. Je remarquai
même une espèce de déconcertement, d'où je compris que c'était l'instant
favorable de profiter de son trouble par les plus grands efforts\,;
ainsi me ranimant moi-même, je rompis le silence, après l'avoir laissé
durer quelque temps, par des louanges que je crus nécessaires pour
préparer la voie à ce que j'avais dessein de leur faire succéder, et
lorsque je crus qu'il était temps d'amener un autre langage, je lui dis
qu'il était également étrange et déplorable qu'il laissât perdre de si
grands talents, et par le seul homme du sang royal qui, par ses
conseils, s'il se mettait à portée d'être consulté, et par sa capacité à
la guerre, s'il se remettait en état d'en faire usage, pouvait sauver le
royaume de ses pères, {[}et qu'il{]} voulût s'ensevelir tout vivant dans
un désordre et dans une obscurité qui seuls enfonceraient le plus simple
particulier dans des ténèbres infâmes et sans retour, combien plus un
prince de son rang, qui outre les débauches avait tant d'autres malheurs
à réparer\,; que je ne pouvais plus me retenir enfin de lui faire faire
attention à quelques considérations que je n'avais pu jusque-là faire
sortir de moi-même, mais que l'aimant et l'estimant au point que je
faisais, je me croirais aussi trop coupable, si après les avoir ménagées
jusqu'au bout et, ne voyant point de fruit de tout ce que je lui avais
dit et de tout ce que je lui avais tu, je ne lui disais tout enfin au
péril de lui déplaire, et de lui paraître trop hardi, puisque je ne
pourrais jamais espérer de repos avec moi-même, si je me laissais ce
reproche de ne lui avoir pas tout dit, et par ce faux respect de l'avoir
abandonné dans un abîme, d'où la juste opinion que j'avais de lui me
devait persuader que je l'eusse enfin retiré, si je n'avais eu pour lui
ces ménagements perfides.

Après cette préface, je me levai brusquement en pied, et, me tournant
avec action vers M, le duc d'Orléans, je lui dis que je ne pouvais donc
plus lui taire la juste indignation du public, qui, après avoir conçu de
lui les plus hautes espérances, et avoir eu pour lui la plus grande et
la plus longue indulgence, tournait les unes en mépris, l'autre en une
sorte de rage qui produisait le déchaînement universel et inouï contre
lui, aussi vif dans les plus libertins que dans les hommes dont les
moeurs étaient les plus austères\,; qu'il y avait temps et manières pour
tout\,; que son libertinage avait été supporté par égards pour son âge
et pour ce qu'il valait d'ailleurs\,; mais que le monde, las enfin de
voir que ce libertinage devenu abandon depuis tant d'années
s'approfondissait de plus en plus\,; que ni l'âge, ni l'esprit, ni les
lumières, ni les grands emplois n'avaient pu le changer\,; qu'il était
devenu non-seulement concubinage, mais ménage public\,; personne ne
pouvait plus souffrir dans un petit-fils de France de trente-cinq ans ce
que le magistrat et la police eût châtié il y a longtemps dans quiconque
n'eût pas été d'un rang à couvert de ces sortes de voies de remettre les
gens dans l'ordre, au moins hors d'état d'insulter à tout un royaume par
le scandale affreux de sa vie\,; qu'à une conduite si honteusement
suivie, il avait ajouté des imprudences de nature si délicate, si
jalouse, tellement unies à la licence effrénée de la vie, que le comble
de toute horreur en était retombé sur lui, et retombé de façon si
naturelle, que, quelque innocent qu'il fût du fond de ces imprudences,
il était pourtant vrai qu'il fallait en être bien au fait et bien porté
à l'en croire pour n'en concevoir pas l'opinion la plus sinistre, qui, à
commencer par le roi, par Monseigneur, par les personnes royales et les
autres les plus principales, avait trouvé entrée dans l'esprit de tout
le monde, et avait produit une aliénation générale qui tenait de la
fureur\,; que le public, outré de s'être trompé dans les espérances
qu'il avait conçues de lui, aigri d'ailleurs de ne trouver personne en
qui les mettre, en un temps si déplorable, était par là également porté
à ne garder plus aucune sorte de mesure pour lui s'il continuait par
l'opiniâtreté de son débordement à n'en mériter nulles, et à revenir
aussi à lui avec rapidité, s'il le voyait capable de retour en rompant
de si honteux liens, en avouant tacitement ses fautes par un digne
changement de conduite et de vie, et en méritant par un attachement
sincère et assidu à ses devoirs\,; que de cette sorte et non autrement,
il se laverait des souillures qui l'avaient défiguré\,; que le courage
qu'il aurait de le faire surprendrait l'acclamation publique, qui
relèverait avec joie le mérite de sa nouvelle vie, par celle de voir
renaître ses espérances.

Tandis que je parlais de la sorte, j'étais infiniment attentif à percer
M. le duc d'Orléans de mes regards, et je m'aperçus que mon impétuosité
faisait sur lui une impression profonde. Je m'arrêtai néanmoins pour
donner lieu à Besons de m'aider à le pousser, et je me rassis comme un
homme qui a tout dit. Ce n'était pourtant pas mon dessein d'en demeurer
là, et ce le fut bien moins encore lorsque je sentis la faiblesse du
maréchal, qui, me regardant de la tête aux pieds, n'avait de réflexion
que la peur qu'il prenait de la force de mon discours, et de courage que
pour une approbation tremblante en monosyllabes. L'extrême crainte que
je lui remarquai me força de suppléer au défaut de son secours. Je
demandai à M. le duc d'Orléans avec un air d'angoisse s'il ne prendrait
point de parti, et s'il ne voulait point envoyer demander à
M\textsuperscript{me} de Maintenon audience pour le lendemain matin. Il
tarda un peu à répondre, puis me dit qu'il ne pouvait encore s'y
résoudre. Ce mot \emph{encore} me donna une grande espérance. Je me
tournai au maréchal\,; je le pressai de presser à son tour pour ne me
pas rendre odieux à la fin par une importunité trop vive. Il parla, mais
avec faiblesse, et conclut promptement qu'il n'y avait rien à ajouter à
mes propos, soit pour leur force, leur justesse ou leur vérité, et dans
le désordre où je vis bien que l'effroi l'avait jeté, je trouvai qu'il
avait beaucoup fait de m'avoir approuvé, quoique si laconiquement, d'une
manière si précise.

J'insistai donc encore sur le message, et sentant le prince mollir et
ployer sous le faix de ma véhémence, je crus la devoir pousser, et, me
levant de nouveau, je lui dis qu'il fallait qu'il me permît encore ce
mot\,: qu'il avait vu de tout temps et qu'il voyait encore le brillant,
le lustre, la splendeur qui accompagnait les ministres, les généraux
d'armée, ceux pour qui le roi montrait une estime et une amitié solide
par sa confiance et par ses bienfaits\,; que leur état radieux était
l'objet de l'envie des uns, de l'émulation des autres, des désirs de
tous\,; que sa naissance grossissait naturellement sa cour de ces grands
personnages, et de leur cour particulière\,; que la faveur et la
confiance du roi l'avaient mis souvent au-dessus d'eux en crédit, et
toujours en autorité, et avaient fait de ces distributeurs, si souvent
arrogants, des grâces, ses courtisans et ses complaisants, avec respect,
crainte et soumission\,; que d'autre part il voyait aussi des seigneurs
que leur naissance, leurs familles, leurs établissements, leurs dignités
portaient si naturellement aux distinctions de leur état, avilis par
leurs débauches, inconnus à la cour par leur obscurité, abandonnés à
leur propre honte et à leur misère, rejetés des plus chétives
compagnies, objets de la censure et du mépris du roi et du public,
réduits à ce degré bizarre d'être au dessous des coups qu'on dédaignait
de frapper sur eux\,; je lui nommai quelques-uns de ceux-là qu'il voyait
malgré tant d'avantages ensevelis dans la fange, et après ces peintures
que je fis les plus vives que je pus, je demandai à M. le duc d'Orléans
auquel des premiers ou des seconds il aimait mieux ressembler. J'ajoutai
qu'il ne fallait pas se tromper par une illusion grossière\,; que plus
sa proximité du trône l'élevait avec éclat, et lui donnait de facilité
de joindre à cette naturelle splendeur la splendeur empruntée par les
autres de l'estime, de la faveur, de la confiance du maître commun de
tous, des grands emplois, du crédit, de l'autorité, plus aussi le
dénûment de ces choses, et le dénûment produit par le déréglement et la
saleté de sa vie le ferait tomber plus bas que ces seigneurs pris sous
les ruines de leur obscurité débordée, dans un mépris d'autant plus
cruellement profond, qu'il serait inouï et justement invoqué\,: que
c'était désormais à lui, dont les deux mains touchaient à ces deux
différents états, d'en choisir un pour toute sa vie, puisque, après
avoir tant perdu d'années et nouvellement depuis l'affaire d'Espagne,
meule nouvelle qui l'avait nouvellement suraccablé, un dernier
affaissement aurait scellé la pierre du sépulcre où il se serait enfermé
tout vivant, duquel, après, nul secours humain, ni sien ni de personne,
ne le pourrait tirer. Je terminai un discours si nerveux par des excuses
et des louanges et par la considération du prodigieux dommage de la
perte civile d'un prince de son rang, de son âge et de ses talents, puis
me tournant brusquement au maréchal, je tombai sur lui de ce qu'il me
laissait tout faire, et seul en proie à tout le mauvais gré.

Alors M. le duc d'Orléans me remercia d'un ton de gémissement auquel je
connus l'impression profonde que j'avais faite en son âme, et bien plus
encore lorsque, se levant de sa chaise, il se mit à reprocher à Besons
sa mollesse à lui parler. Le maréchal s'excusa sur ce que je ne lui
laissais rien à dire, et je lui répondis vivement exprès que, mon zèle
me faisait tout dire, parce que je voyais que pensant tout comme moi, il
n'osait néanmoins parler. Cette bizarre dispute nous donna lieu à tous
deux de placer encore dans le discours de nouveaux raisonnements forts,
et des considérations vives. Cependant M. le duc d'Orléans s'était
rassis. Je lui proposai encore, tandis que je le voyais ébranlé,
d'envoyer chez M\textsuperscript{me} de Maintenon\,; Besons lui demanda
s'il voulait qu'il appelât quelqu'un de ses gens\,; je me mis à louer
l'action comme ne doutant pas qu'il ne la fît, et pour l'y exciter
davantage, je parlai de la douceur qu'on sentait après un pénible et
généreux effort. Tandis que nous dissertions ainsi, Besons et moi, l'un
avec l'autre, n'osant plus l'attaquer directement après l'étrange assaut
qu'il venait d'essuyer, nous fûmes bien étonnés qu'il se levât tout à
coup de sa chaise, qu'il courût avec impétuosité à sa porte, l'ouvrît et
criât fortement pour se faire entendre de ses gens.

Il en accourut un à qui il ordonna tout bas d'aller chez
M\textsuperscript{me} de Maintenon savoir si et à quelle heure il
pourrait lui parler le lendemain matin. Il revint aussitôt se jeter dans
sa chaise comme un homme à qui les forces manquent et qui est à bout.
Incertain de ce qu'il venait de faire, je lui demandai aussitôt s'il
avait envoyé chez M\textsuperscript{me} de Maintenon. «\,Eh\,! oui,
monsieur,\,» me dit-il avec un air désespéré. À l'instant je me jetai à
lui, et le remerciai avec tout le contentement et toute la joie
imaginables. Il me dit qu'il n'était pas bien sûr qu'il parlât à
M\textsuperscript{me} de Maintenon\,; sur quoi Besons, qui lui avait
aussi témoigné son extrême satisfaction, l'exhorta à ne pas reculer
après avoir pris une résolution si pénible, mais si salutaire. Je me
contentai de l'y soutenir en rassurant Besons. Je lui dis que M. le duc
d'Orléans, convaincu par la force de nos raisons, et résolu à se faire
cette violence si nécessaire, n'avait pas fait un pas qui l'engageait si
fort pour reculer après\,; que son coeur palpitait encore, mais que
j'ouvrais enfin le mien aux plus douces espérances. Lui et moi menâmes
quelque temps la parole, louant la résolution, admirant le courage,
plaignant les douleurs, compatissant à tout, incitant à la gloire,
réfléchissant sur la solidité, la sûreté, la douceur du repos et du
calme après l'orage, fortifiant indirectement ainsi ce prince sans lui
adresser la parole pour ne le pas rebuter. Il entra peu dans notre
conversation, mais sur la fin il dit encore à Besons qu'il l'avait trop
ménagé, qu'il sentait bien l'extrême besoin qu'il avait d'être vivement
poussé, et me remercia encore de ma force et de ma liberté à lui parler.
Cela m'encouragea à bien espérer, et à l'exciter encore, mais vaguement.
Je lui dis seulement que j'avais cru devoir lui représenter nûment
toutes les vérités qui m'avaient paru indispensables à lui faire
connaître. Peu après il demanda quelle heure il était, et il était neuf
heures du soir. Il voulut à son ordinaire aller voir Monseigneur chez
M\textsuperscript{me} la princesse de Conti. Je lui demandai permission
de demeurer dans son entre-sol avec Besons, lui et moi n'ayant point de
logement ni où nous entretenir sur pareille matière\,; il s'en alla. Je
fermai la porte, et le maréchal et moi nous nous rassîmes.

Je lui demandai ce qu'il lui semblait du succès de notre terrible
après-dînée\,; il me dit franchement que, nonobstant l'audience
demandée, il ne se tenait sûr de rien, et moi je l'assurai que, encore
qu'absolument parlant je n'osasse m'engager à répondre de rien, je
trouvais les choses avancées. Nous raisonnâmes sur les étonnants
obstacles que nous avions trouvés\,; il m'avoua qu'il ne le croyait
presque plus amoureux\,; je convins qu'encore que je fusse bien persuadé
qu'il l'était encore beaucoup, je ne pensais pas que ce combat dût en
rien approcher de ce que j'en éprouvais. Je me plaignis à lui en amitié,
mais en amertume, du peu de secours qu'il m'avait donné, et de m'être
trouvé dans la nécessité de parler presque seul, et seul de dire les
choses les plus dures. Il m'en fit excuse, et m'avoua ingénument qu'il
admirait la force et la hardiesse que j'avais eues, qu'il en avait bien
senti la nécessité, que le succès lui montrait encore que de cela seul
il avait dépendu\,; mais que pour rien il n'eût dit à cent lieues près
aucunes des choses qu'il avait entendues avec terreur\,; que je l'avais
épouvanté à ne savoir où se fourrer\,; que je l'avais souvent mis hors
de lui-même, quelque assaisonnement que j'eusse mis avant et après les
vérités que j'avais si rudement assenées. Nous admirâmes ensuite l'excès
de la puissance des égarements qui avaient jeté ce prince dans un si
profond abîme, et qui lui coûtaient un si furieux combat, plus encore la
bonté, la douceur, la patience incomparables, avec lesquelles il avait
écouté tant de choses énormes par leur dureté, et nous convînmes
aisément de l'horrible dommage qu'un prince de tant de grands et
d'aimables talents, et capable d'où il s'était plongé d'écouter la voix
si âpre et si étonnante des vérités que nous lui avions fait entendre,
se fût précipité dans les abîmes où nous le déplorions\,; nous convînmes
que moins qu'en aucun temps précédent, il ne devait être abandonné à
lui-même un seul instant possible. Nous ne laissâmes pas de nous
plaindre réciproquement de notre excessive fatigue de corps et d'esprit,
et nous nous donnâmes rendez-vous dans la galerie pendant le souper du
roi pour convenir de ce qu'il nous restait à faire. Nous y fûmes exacts.

Je demandai au maréchal s'il ne savait point quelle réponse il y avait
eu de M\textsuperscript{me} de Maintenon. Il me dit qu'il n'avait vu ni
M. le duc d'Orléans ni pas un de ses gens, depuis que nous l'avions
quitté. Je lui remontrai l'importance d'en être instruit, et le priai de
vouloir bien s'en aller informer chez ce prince, tandis que je
l'attendrais au même lieu où je lui parlais. Besons y fut et me revint
dire aussitôt que M\textsuperscript{me} de Maintenon mandait à M. le duc
d'Orléans, qu'elle l'attendrait le lendemain toute la matinée, mais que
M. le duc d'Orléans n'avait encore pu apprendre cette réponse. Là-dessus
je proposai à Besons, sur notre même principe, d'accompagner M. le duc
d'Orléans chez lui, au sortir de chez le roi, d'être présent lorsque la
réponse lui serait rendue, d'en prendre thèse, pour l'exhorter encore
d'exécuter courageusement son salutaire dessein, et dans le sens dont
nous étions convenus, de l'obséder jusqu'à ce qu'il se mît au lit, de se
trouver le lendemain à son lever, de lui bien parler encore, de tâcher
de le mener chez M\textsuperscript{me} de Maintenon, et de venir après à
la messe du roi, où nous nous trouverions pour régler ce que nous
aurions à faire. Il me promit de faire exactement tout cela, et
là-dessus nous nous séparâmes.

Le lendemain vendredi 3 janvier, je ne trouvai point Besons dans la
galerie, ni dans l'appartement\,; le roi sortit pour la messe, et M. le
duc d'Orléans à huit ou dix pas devant lui. Dans l'impatience de savoir
s'il avait vu M\textsuperscript{me} de Maintenon, je m'approchai de lui,
et quoique je lui parlasse bas, n'osant rien nommer, je lui demandai
s'il avait vu cette femme. Il me répondit un \emph{oui} si mourant, que
je fus saisi de la crainte qu'il l'eût vue pour rien, tellement que je
lui demandai s'il lui avait parlé. Sur un autre \emph{oui} pareil à
l'autre, je redoublai d'émotion. «\,Mais lui avez-vous tout dit\,? ---
Eh oui, répondit-il, je lui ai tout dit. --- Et en êtes-vous content\,?
repris-je. --- On ne peut pas davantage, me dit-il. J'ai été près d'une
heure avec elle, elle a été très-surprise et ravie.\,» Il fit là une
assez longue pause à proportion du chemin qui s'avançait toujours, puis
après avoir à deux ou trois fois voulu, puis s'être retenu de me parler,
il me regarda tristement comme exprès, et tout à coup me dit qu'il avait
quelque chose qui le peinait sur moi, qu'il fallait qu'il me le dît,
mais qu'il me demandait d'amitié de lui répondre sincèrement et avec
vérité. Cela me surprit.

Je l'assurai que je ne lui déguiserais rien. «\,C'est, me dit-il
toujours bas, que cette femme m'a parlé tout comme vous\,; mais ce qui
m'a frappé, c'est qu'elle m'a dit les mêmes choses, les mêmes phrases,
jusqu'au même arrangement et aux mêmes mots que vous. Ne vous
aurait-elle point parlé, et n'avez-vous eu aucune charge d'agir auprès
de moi\,? --- Monsieur, lui dis-je, je n'ai pas accoutumé à faire des
serments, mais je vous jure par celui de chez lequel nous approchons (et
c'était de la chapelle), et par tout ce qu'il y a de plus saint, que je
vous ai parlé de moi-même\,; que qui que ce soit, ni directement, ni
indirectement, ni en aucune manière quelconque, n'y a eu aucune part, et
que, pour cette femme ni le roi, non-seulement ils ne m'ont point parlé
ni rien fait dire, mais ils ne peuvent pas savoir un mot de ce qui s'est
passé, et après ce grand serment que je vous fais contre ma coutume,
j'ose vous dire que vous devez me connaître assez pour m'en croire sur
ma parole.\,» Il fit un soupir\,; et me prenant la main\,: «\,Voilà qui
est fait, me dit-il, je vous en crois\,; mais vous me faites plaisir de
me parler comme vous faites, car je vous avoue que cette conformité m'a
paru si singulière, qu'elle m'a frappé entre vous et cette femme, à qui
le roi dit tout et qui gouverne l'État. --- Monsieur, encore un coup,
repris-je, soyez rassuré, car je vous répète que je vous dis la vérité
la plus exacte et la plus nette. --- Voilà qui est fait, me répondit-il
encore, je n'ai pas le moindre scrupule.\,»

La tribune où nous étions déjà avancés quelques pas nous sépara. Il
était fête de sainte Geneviève, ce qui m'obligea à demeurer à entendre
la messe du roi pour être libre après. La fin du motet et la prière pour
le roi après le dernier évangile me donna lieu de sortir de la tribune
avant le roi pour chercher Besons, que je trouvai à deux pièces de là.
Quoique je susse des nouvelles, je lui en demandai, dans l'espérance
qu'au sortir de chez M\textsuperscript{me} de Maintenon, M. le duc
d'Orléans lui aurait conté sa conversation. Mais il me dit qu'il ne
savait rien\,; que la veille au soir, il l'avait mené de chez le roi
chez lui, que ce matin il s'était trouvé à son lever, l'avait toujours
exhorté suivant ce que nous en étions convenus, l'avait accompagné
jusqu'à la porte de M\textsuperscript{me} de Maintenon, qu'il l'y avait
laissé, et appris depuis qu'il y était demeuré longtemps avec elle, et
qu'il n'en savait pas davantage. Je lui dis que je le ferais donc plus
savant, et, en lui contant ce que M. le duc d'Orléans m'avait dit en
allant à la messe du roi, j'ajoutai qu'il m'avait dit la chose du monde
la plus surprenante, qui m'avait engagé à lui faire un serment, et je
lui rendis le fait.

Le maréchal n'en fut pas ému un moment, et me dit avec cette sorte de
brusquerie, que la conviction produit quelquefois, qu'il n'y avait qu'à
répondre à M. le duc d'Orléans une seule chose bien simple et bien
vraie, savoir que la vérité est une, et que par là elle s'était trouvée
dans la bouche de M\textsuperscript{me} de Maintenon précisément comme
dans la mienne. Comme nous en étions là, le roi passa retournant de la
chapelle chez lui, et ne nous laissa que le temps de nous donner
rendez-vous chez M. le duc d'Orléans sur-le-champ, pour éviter d'y aller
ensemble. Quoique je ne me fusse amusé qu'un moment dans la galerie, je
trouvai déjà Besons dans la chambre de M. le duc d'Orléans qui n'était
pas rentré, et qui ne vint qu'une bonne demi-heure après. Je proposai à
Besons d'entrer dans le cabinet, nous en fermâmes la porte, et là tous
deux, nous nous mîmes à raisonner. M. le duc d'Orléans nous avait dit la
veille, que, s'il parlait au roi, ce ne serait qu'immédiatement avant
son dîner, parce que, outre que c'était son heure à lui la plus
ordinaire, c'était aussi la plus naturelle d'être seul avec lui dans ses
cabinets\,; ainsi nous convînmes de demeurer jusqu'à cette heure-là avec
M. le duc d'Orléans pour le soutenir, le fortifier, et lui faire achever
ce qu'il avait commencé. Le maréchal convenait que l'affaire était
avancée au delà d'espérance, mais il ne la pouvait déterminément pousser
jusqu'à compter sur sa consommation avec le roi. Pour moi, je n'osois en
répondre d'une manière positive\,; mais je ne pouvais aussi m'imaginer
qu'elle nous échappât, après ce grand pas fait chez
M\textsuperscript{me} de Maintenon.

Pendant que nous causions ainsi, je songeais à part moi à la bizarre
justesse de la conjoncture où je me trouvais d'attendre à tous moments
une audience particulière du roi, dans une circonstance si propre à
confirmer le soupçon que M. le duc d'Orléans venait de me témoigner.
Après y avoir bien pensé, la délicatesse d'honneur et de probité
l'emporta en moi sur l'orgueil et la politique de courtisan, si
difficile à se ployer à montrer sa disgrâce et ses démarches pour la
finir, tellement que, bien que je n'eusse avant cette affaire-ci ni
liaison ni même le plus léger commerce avec Besons, et qui n'avait pas
plus de douze jours de date, je crus devoir lui confier mon secret pour
le consulter si je le révélerais à M. le duc d'Orléans\,; je lui dis
tout mon fait, et comme à tous moments j'attendais mon audience, mais
sans lui apprendre comment je l'avais obtenue. La rondeur de ce procédé
le surprit et le toucha. Il me conseilla d'en faire la confidence à M.
le duc d'Orléans, et il m'assura que, quoi qu'il eût soupçonné, il me
connaissoit trop bien pour, après ce que je lui avais dit et juré,
penser un moment qu'entre le roi et moi il dût être en rien question de
lui. Sur son avis je me déterminai à le faire. Après avoir été une
demi-heure ensemble, quelqu'un vint demander Besons, qui sortit et me
laissa seul dans le cabinet.

Fort peu après, comme j'étais seul encore, M. le duc d'Orléans entra,
qui venait de chez Madame, et qui tout de suite m'emmena dans son
arrière-cabinet. Il se mit le dos à la cheminée sans proférer un mot,
comme un homme hors de soi. Après l'avoir considéré un moment, je crus
qu'il valait mieux l'importuner par des questions que de le laisser
ainsi à lui-même dans des moments critiques, qui avaient si grand besoin
de soutien, puisque deux heures après arrivait le moment qu'il devait
parler au roi pour se séparer de sa maîtresse. Je lui demandai donc s'il
était bien content de M\textsuperscript{me} de Maintenon, et si elle
était entrée véritablement dans ce qu'il lui avait dit\,; il me répondit
un \emph{oui} si bref, que je me hâtai de lui demander s'il n'était pas
bien résolu d'aller chez le roi un peu avant son dîner\,; il m'effraya
beaucoup par sa réponse. Il me dit de ce même ton qu'il n'irait pas.
«\,Comment\,! monsieur, m'écriai-je d'un air ferme, vous n'irez pas\,?
Eh\,! non, monsieur, répliqua-t-il avec un soupir effroyable, tout est
fait. Tout est fait\,? repris-je vivement, comment l'entendez-vous\,?
tout est fait pour avoir parlé à M\textsuperscript{me} de Maintenon\,?
Eh\,! non, dit-il, j'ai parlé au roi. --- Au roi\,! m'écriai-je, et lui
avez-vous dit ce que vous lui vouliez dire tantôt\,? --- Oui,
répondit-il, je lui {[}ai{]} tout dit. --- Ah\,! monsieur, m'écriai-je
encore avec transport, cela est fait, que je vous aime\,! et, me jetant
à lui, que je suis aise de vous voir enfin délivré\,; et comment
avez-vous fait cela\,? --- Je me suis craint moi-même, me répondit-il.
J'ai été si violemment agité depuis que j'ai eu parlé à
M\textsuperscript{me} de Maintenon, que j'ai eu peur de me commettre à
tout le temps de la matinée, et que, mon parti enfin bien pris, je me
suis résolu de me hâter d'achever. Je suis rentré dans le cabinet du roi
après la messe\ldots.\,» Alors vaincu par sa douleur, sa voix s'étouffa,
et il éclata en soupirs, en sanglots et en larmes. Je me retirai en un
coin. Un moment après Besons entra\,; le spectacle et le profond silence
l'étonnèrent. Il baissa les yeux et n'avança que peu. Je lui fis des
signes qu'il ne comprit point\,; puis, se remettant un peu, me demanda
des yeux ce que ce pouvait être. Enfin nous nous approchâmes doucement
l'un de l'autre, et je lui dis que c'en était fait, que M. le duc
d'Orléans avait vaincu, qu'il avait parlé au roi.

Le maréchal fut si étourdi de surprise et de joie, qu'il en demeura
quelques moments interdit et immobile\,; puis se jetant à M. le duc
d'Orléans, il le remercia, le félicita, et se mit à pleurer de joie.
Cependant nous nous tûmes et laissâmes un assez long temps le silence au
trouble de M. le duc d'Orléans, qui s'alla jeter dans un fauteuil, et
qui, tantôt stupide, tantôt cruellement agité, ne s'exprimait que par un
silence farouche ou par un torrent de soupirs, de sanglots et de larmes,
tandis qu'agités nous-mêmes et attendris d'un état si violent, nous
contenions notre joie, nous n'osions nous parler, et à peine
pouvions-nous nous persuader que cette rupture si salutaire fût achevée.
Peu à peu pourtant nous rompîmes le silence entre nous. Le maréchal et
moi nous nous mîmes à plaindre M. le duc d'Orléans, à louer son généreux
effort, à chercher ainsi obliquement à le calmer un peu dans la violence
de ces premiers moments. Ensuite nous nous encourageâmes, pour essayer
un peu de diversion, à lui demander ce que M\textsuperscript{me} de
Maintenon lui avait dit. Il nous répondit que, mot pour mot, elle lui
avait tenu tous mes mêmes propos, et tellement les mêmes, en même ordre
et en mêmes expressions, qu'il avait cru qu'elle m'avait parlé. Je le
fis souvenir de ce que je lui avais dit et protesté là-dessus avec
serment en allant à la messe du roi, et il me réitéra aussi qu'il ne lui
en restait pas le moindre scrupule, mais que cette singularité était si
grande qu'il lui avait été pardonnable de l'avoir pensé\,; là-dessus
Besons lui parla très-bien au même sens de ce qu'il m'en avait dit.

Je crus que cette occasion était celle que je devais prendre pour lui
faire la confidence de l'audience que j'attendais du roi, avec la
franchise que Besons m'avait conseillée. M. le duc d'Orléans la reçut à
merveilles, et me dit même, avec une amitié dont la politesse me surprit
en l'état où il était, qu'il souhaitait d'avoir mis le roi d'assez bonne
humeur, par ce qu'il venait de lui dire, pour qu'il m'en écoutât plus
favorablement. Nous le remîmes sur son audience de M\textsuperscript{me}
de Maintenon. Il nous dit qu'elle avait été extrêmement surprise de sa
résolution et en même temps ravie\,; qu'elle l'avait assuré que cette
démarche le remettrait avec le roi mieux que jamais\,; qu'elle lui avait
conseillé de lui parler lui-même plutôt que de lui faire parler par elle
ni par personne\,; et qu'elle lui avait promis de faire valoir au roi ce
sacrifice, de manière à lui en ôter tout regret, et à faire que
M\textsuperscript{me} d'Àrgenton fut traitée comme il le pouvait
souhaiter, et comme elle-même trouvait juste qu'elle la fût, sans lettre
de cachet ni rien de semblable, et qu'elle pût se retirer, soit dans un
couvent, soit dans une terre, ou dans une ville telle qu'elle la
voudrait choisir, sans même être astreinte à demeurer dans un même lieu.
C'était aussi ce que j'avais dit à M. le duc d'Orléans que je trouvais
raisonnable, pourvu qu'elle n'allât pas dans ses apanages faire la
dominatrice, et ce que lui-même avait aussi approuvé comme moi. Il nous
dit aussi que M\textsuperscript{me} de Maintenon lui avait promis
d'envoyer chercher la duchesse de Ventadour pour concerter tout avec
elle (et quel personnage pour une dame d'honneur de Madame et pour une
gouvernante des enfants de France\,!) et qu'il ferait bien de la voir
là-dessus. De là mon impatience me porta, malgré l'interruption des
larmes et des fréquents élans de douleur, de lui demander comment il
était content du roi. «\,Fort mal,\,» me répondit-il. J'en fus surpris
et touché au dernier point, et je voulus savoir comment cela s'était
passé.

Il nous dit qu'il avait suivi le roi dans son cabinet après la messe, et
que, comme il étouffait de ce qu'il avait à lui dire, il l'avait prié de
passer dans un autre cabinet, afin qu'il pût lui dire un mot seul\,; que
le roi, effarouché de la proposition en un temps où il n'avait pas
accoutumé de le voir dans son cabinet, lui avait demandé d'un air sévère
et rengorgé ce qu'il lui voulait\,; qu'il avait insisté au
tête-à-tête\,; que le roi, encore plus grave et plus refrogné, l'avait
mené dans l'autre cabinet\,; que là il lui avait dit sa résolution
causée par la douleur de lui déplaire, l'avait prié de faire dire à
M\textsuperscript{me} d'Argenton de sortir de Paris, et de lui épargner
la douleur du mauvais traitement, et la honte de l'exil et d'une lettre
de cachet, qui ne pourrait retomber que sur lui-même\,; que le roi avait
paru très-surpris, mais point épanoui\,; qu'il l'avait loué, mais
froidement, et dit qu'il y avait longtemps qu'il aurait dû mettre fin à
une vie si scandaleuse\,; qu'il voulait bien faire sortir
M\textsuperscript{me} d'Argenton de Paris sans ordre par écrit\,; qu'il
verrait ce qu'il pourrait faire là-dessus\,; après quoi le roi l'avait
quitté brusquement comme un homme non préparé à une audience insolite,
et qui avait peur que cette déclaration ne fût suivie de quelque demande
à laquelle il ne voulait pas laisser de loisir. Quoique ce récit me
déplût fort, je ne laissai pas d'espérer que la froideur du roi venait
moins d'un éloignement invincible que d'un temps mal pris et de la
surprise, qui étaient les deux choses du monde qui le rebroussaient le
plus, et j'espérai que la réflexion, venant sur l'effort du sacrifice,
sur son entière gratuité, puisqu'il n'était accompagné d'aucune demande,
ni même d'aucune insinuation de rien, sur la cessation de la cause et
des effets des déréglements de toutes les sortes et des sujets de
douleur de M\textsuperscript{me} la duchesse d'Orléans, ramèneraient ce
prince dans l'état où il devait être avec le roi, avec toutes les
personnes royales, au moins à l'extérieur pour Monseigneur, et
conséquemment avec le monde. Je le désirais d'autant plus que je faisais
moins de fond que je ne lui avais témoigné sur M\textsuperscript{me} de
Maintenon, et que je ne me fiais guère à la bonne réception qu'elle lui
avait faite, ni aux bons offices qu'elle lui avait promis. Il fallait
bien du spécieux, et même quelque réalité apparente, dans une occasion
comme celle-là\,; une autre conduite aurait trop ouvert les yeux. Il
fallait même que le roi y fût trompé pour lui ôter toute défiance, et
demeurer plus entière aux desservices qu'elle voudrait porter en
d'autres temps. Le funeste bon mot d'Espagne n'était pas pour être
pardonné, et M. du Maine lui était trop intimement cher pour contribuer
à augmenter, même à rétablir, l'amitié et la confiance du roi pour M. le
duc d'Orléans si supérieur à l'autre en tout genre, excepté en fourbe,
en adresse et en esprit de ce genre. Je fis donc de mon mieux pour
rassurer M. le duc d'Orléans sur le roi, par les deux raisons que j'ai
alléguées\,; et Besons et moi n'oubliâmes rien pour le rassurer et le
consoler. Le silence et les propos se succédèrent à diverses reprises.

M. le duc d'Orléans nous dit qu'il venait de rendre compte à Madame de
ce qu'il avait fait, qu'elle l'avait fort approuvé, mais qu'elle l'avait
mis au désespoir par le mal qu'elle lui avait dit de
M\textsuperscript{me} d'Argenton. Il s'aigrit même en nous le racontant,
et je m'en aigris avec lui, parce qu'à la misérable façon dont elle
avait toujours traité et ménagé cette maîtresse, ce n'était pas à elle à
en dire du mal, beaucoup moins au moment de la rupture qui sont des
instants à respecter par les plus sévères. Je me hasardai à lui demander
s'il serait incapable de dire à M\textsuperscript{me} sa femme une
nouvelle qui la regardait de si près\,; mais à ce nom il s'emporta, dit
qu'il ne la verrait au moins de toute la journée, qu'elle serait trop
aise, et que sa joie lui serait insupportable. Je lui répondis
modestement que, par tout ce que j'avais ouï dire d'elle, je la croyais
incapable de tomber dans le même inconvénient de Madame, mais au
contraire plus propre à entrer dans sa peine, par rapport à lui, qu'à
lui montrer une joie indiscrète et fort déplacée. Il rejeta cela avec un
si grand éloignement que je n'osai en dire davantage. Néanmoins, après
quelque intervalle, je ramenai doucement ce propos sur le double plaisir
que ce nouvel effort ferait au roi. Je ne réussis pas mieux. Il me ferma
la bouche par me dire que ce chapitre avait été traité le matin entre
lui et M\textsuperscript{me} de Maintenon, qu'elle était entrée dans sa
répugnance, et qu'elle lui avait conseillé de ne voir
M\textsuperscript{me} la duchesse d'Orléans de toute la journée, s'il ne
voulait, pour ne la pas voir à contre-coeur.

Je changeai de discours. Besons parla aussi, et nous ne cherchâmes pour
le bien dire qu'à bavarder pour étourdir une douleur incapable encore de
raison, plutôt par un bruit extérieur que par la solidité des choses.
Quoique la porte fût défendue, il s'y présenta des gens que le
renouvellement de l'année et la vacance de la fête y amenait. Tels
furent le premier président et les gens du roi du parlement et des
autres compagnies supérieures, et quelques autres principaux magistrats,
qui vinrent à diverses reprises, et que le prince fut obligé d'aller
voir dans sa chambre, où ils étaient entrés. On peut juger de l'étrange
contre-temps. Il les vit tous néanmoins sur la porte de son cabinet pour
être plus à l'obscurité, les entretint, les gracieusa, et nous montra
une force dont peu d'hommes sont capables, mais sous laquelle il
succombait après par un cruel renouvellement de douleur. Je saisis un de
ces intervalles pour demander à Besons ce qu'il lui semblait de cette
journée. Il m'avoua avec transport qu'il en était d'autant plus vivement
pénétré de joie qu'il l'avait moins espérée, et si peu qu'à peine se
pouvait-il encore persuader ce qu'il voyait et entendait, et il m'en
félicita comme d'un projet dû à mon imagination, et d'une exécution due
à mon courage, dont lui et moi étions les seuls à portée, mais qu'il
n'aurait pu ni entamer ni moins amener à fin. Dans un autre intervalle,
nous raisonnâmes sur la manière dont le roi avait reçu la rupture qui
nous alarmait justement, et qui nous fit plus fortement conclure combien
il était important et pressé de finir un si pernicieux genre de vie, et
qui avait mené assez loin pour que cette rupture après tant de désirs
eût été reçue avec si peu de satisfaction. Nous convînmes sans peine que
cela demandait de grandes et de continuelles précautions, et une
conduite bien appliquée et bien suivie, qui à la longue ne coûterait pas
moins que la rupture même. Nous comprîmes combien M. le duc d'Orléans
avait à se tenir en garde contre toutes les sortes de piéges qui lui
seraient tendus, surtout de la boutique de M\textsuperscript{me} la
Duchesse, après ce que lui-même nous avait dit d'elle, tandis que
M\textsuperscript{me} la duchesse d'Orléans vivait avec elle avec tous
les ménagements d'amitié possibles et de rang au delà de raison, puisque
la différence de rang, qui avait causé une haine que rien n'avait pu
amortir, s'allait renouveler de plus belle par la noise de la prétention
de M\textsuperscript{me} la duchesse d'Orléans de faire passer ses
filles devant les femmes des princes du sang, dont je parlerai bientôt.
Enfin nous conçûmes que rien ne serait plus utile à M. le duc d'Orléans
qu'une liaison étroite avec M\textsuperscript{me} sa femme, tant pour
lui fournir des amusements et de bons conseils chez lui que pour prendre
le roi par un changement qui lui serait si agréable. Dans un autre
intervalle, nous pensâmes à nous-mêmes pour éviter la rage de la
séquelle de M\textsuperscript{me} d'Argenton, de M\textsuperscript{me}
la Duchesse et de la sienne, et de tous ceux qui seraient outrés de voir
M. le duc d'Orléans rentré dans le bon chemin, dans l'estime du monde,
dans les bonnes grâces du roi, et dans les suites que ces choses
pourraient avoir.

Le maréchal me témoigna qu'il craignait fort que nous ne fussions déjà
découverts par le nombreux domestique qui nous avait vus obséder M. le
duc d'Orléans pendant ces trois jours, moi seul le premier, lui et moi
les deux autres, à qui sans doute le trouble et la douleur de leur
maître n'aurait pas échappé, et qui de cela voyant éclore la rupture, ne
se méprendraient pas à nous l'attribuer, et par eux tout le monde. À
cela il n'y avait point de remède. Nous nous promîmes seulement de ne
rien avouer, de nous taire, et de laisser dire ce que nous ne pourrions
empêcher sans désavouer honteusement, mais gardant le silence. J'avais
en particulier beaucoup d'ennemis à craindre, tous sûrement très-fâchés
de voir revenir M. le duc d'Orléans dans l'état où il devait être,
surtout M. le Duc et M\textsuperscript{me} la Duchesse avec qui j'étais
en rupture ouverte. Je craignais de plus, que si le roi venait à
découvrir la part que j'avais eue à la séparation de M. le duc d'Orléans
d'avec sa maîtresse, un gré infructueux de vingt-quatre heures ne fût
suivi du danger de me voir chargé des fautes qu'il pourrait faire à
l'avenir, et de celles encore qu'on lui pourrait imposer, le
raisonnement des tout-puissants de ce monde étant trop naturellement et
trop coutumièrement celui-ci\,: que quand on a un assez grand crédit sur
quelqu'un pour lui faire faire un grand pas contre son goût et contre
ses habitudes, on en a assez aussi pour le détourner, si on le voulait,
de toutes les autres choses qu'on lui impute. Mais ces dangers, que je
n'étais pas alors à envisager pour la première fois, n'ayant pas eu le
pouvoir sur moi de m'arrêter dans un projet et dans une exécution
vertueuse, n'eurent pas encore celui de m'épouvanter après m'y être
volontairement et sciemment exposé. Faire ce qui est bon et honnête par
des voies bonnes et honnêtes, garder après une conduite sage et mesurée,
ne s'accabler pas de noeuds gordiens de prévoyance et de prudence
indissolubles par leur nature, laisser dire, faire et agir en
s'abandonnant à la Providence, est un axiome qui m'a toujours paru d'un
grand usage à la cour, pourvu qu'on n'en abuse pas et qu'on s'y tienne
en la façon que je le présente.

M. le duc d'Orléans, revenu avec nous, débarrassé des visites dont j'ai
parlé, nous dit qu'il assurait à M\textsuperscript{me} d'Argenton
quarante-cinq mille livres de rente, dont presque tout le fonds
appartiendrait au fils qu'il avait d'elle, qu'il avait reconnu et fait
légitimer, et qui est devenu depuis grand d'Espagne, grand prieur de
France et général des galères, avec l'abbaye d'Auvillé (car le meilleur
de tous les états en France est celui de n'en avoir point et d'être
bâtard)\,; que, outre ce bien, il restait à sa maîtresse pour plus de
quatre cent mille livres de pierreries, d'argenterie ou de meubles\,;
qu'il se chargeait de toutes ses dettes jusqu'au jour de la rupture,
pour qu'elle ne pût être importunée d'aucun créancier, et que tout ce
qu'elle avait lui demeurât libre, ce qui allait encore à de grandes
sommes\,; et qu'il croyait qu'avec ces avantages, elle-même ne pouvait
prétendre à une plus grande libéralité. Elle passait deux millions, et
je la trouvai prodigieuse, mais en la louant\,; il ne s'agissait pas de
pouvoir dire autrement. Quelque puissant prince qu'il fût, une telle
brèche devait le rendre sage.

Avant de le quitter, Besons, poussé par moi qui n'osois plus parler de
M\textsuperscript{me} la duchesse d'Orléans après mes deux tentatives,
en fit une troisième qui réussit. M. le duc d'Orléans lui promit enfin
qu'il la verrait dans la journée, et lui dirait sa rupture. Cette
complaisance me soulagea fort, dans les vues que j'ai expliquées. Il
était midi et demi, nous le quittâmes, lui pour aller chez la duchesse
de Ventadour, comme il en était convenu le matin avec
M\textsuperscript{me} de Maintenon, nous pour prendre enfin haleine.
Besons me dit en sortant qu'il n'en pouvait plus, et qu'il s'en allait à
Paris se cacher au fond de sa maison pendant le premier éclat de la
rupture, et se mettre à l'abri de toutes questions et de tous propos.

En le quittant dans la galerie de M. le duc d'Orléans, je m'en allai
chez la duchesse de Villeroy, que je trouvai à sa toilette seule avec
ses femmes. Dès en entrant je la priai de les renvoyer, liberté que je
prenais souvent avec elle. Dès qu'elles furent sorties, je lui dis que
l'affaire était faite. «\, Bon\,; faite\,!» me répondit-elle avec
dédain, comprenant bien ce que je lui voulais dire, car je ne l'avais
pas vue depuis notre souper l'avant-veille, je ne le croirai point qu'il
n'ait parlé au roi. Il vous promettra, il n'en fera rien. Croyez-moi,
ajouta-t-elle, vous êtes son ami, mais je le connais mieux que vous. ---
Avez-vous tout dit\,? repris-je en souriant\,; c'est qu'il a parlé ce
matin à M\textsuperscript{me} de Maintenon et au roi, et que la rupture
est bâclée. --- Bon, monsieur\,! me répondit-elle avec vivacité, il vous
a peut-être dit qu'il le fera, et n'en fera rien. --- Mais,
répliquai-je, je vous dis encore un coup qu'il l'a fait, et que je sors
d'avec lui. --- Quoi, cela est fait\,? dit-elle avec transport\,; mais
fait, achevé, rompu sans retour\,? --- Eh oui\,! répliquai-je, madame,
fait et archifait. Je ne vous dis ni conjectures ni contes, je vous dis
nettement que cela est fait.\,» Je ne vis jamais femme si aise, ni qui
de joie eût plus de peine à se persuader ce qu'elle entendait. Après
cette sorte de désordre, elle me demanda fort comment cela s'était fait.
Je lui contai le précis et le plus nécessaire de ce que je viens de
rapporter, et des noms et des détails que j'ai cru devoir omettre ici,
que j'estimai être importants à l'union que je désirais établir entre le
mari et la femme que celle-ci n'ignora pas. Le duc de Villeroy, qui vint
en tiers, le jugea de même. Le récit fut souvent interrompu par les
surprises de la duchesse de Villeroy, et par des exclamations.

À son tour, elle me conta après que M\textsuperscript{me} la duchesse
d'Orléans lui avait dit la veille l'inquiète curiosité où elle était de
découvrir ce qui se passait chez M. son mari, dont elle avait appris
l'angoisse, les larmes et l'obsession où nous l'avions tenu Besons et
moi\,; que sur ce qu'elle (duchesse de Villeroy) lui avait conté, mais
sans en faire cas, le mot que je lui avais dit en sortant de souper avec
elle, M\textsuperscript{me} la duchesse d'Orléans lui avait dit que, si
quelqu'un était en état de faire rompre M. son mari avec sa maîtresse,
c'était moi\,; qu'elle avait souvent essayé par des recherches de
m'approcher d'elle et de m'apprivoiser, sans y avoir pu réussir, et cela
était vrai, et jamais je n'allais chez elle que pour des occasions
indispensables de compliments, tellement qu'elle en était demeurée là
bien aise toutefois qu'un homme d'honneur et d'esprit, duquel, malgré
mon éloignement d'elle, elle ne croyait pas avoir rien à craindre, fût
intimement avec M. le duc d'Orléans. Épanouie de sa propre joie, elle
m'apprit que celle de M\textsuperscript{me} la duchesse d'Orléans serait
d'autant plus vive qu'elle était plus que jamais accablée d'ennui et de
douleur de l'empire insolent de M\textsuperscript{me} d'Argenton, et des
traitements qui en étaient les suites, et plus que jamais hors
d'espérance de les voir finir\,; que dans le désespoir d'une situation
si triste, elle avait épuisée toutes les voies possibles à tenter de
crédit, de conscience, de compassion pour faire chasser
M\textsuperscript{me} d'Argenton, sans que le roi ni
M\textsuperscript{me} de Maintenon s'y fussent laissés entamer le moins
du monde\,; qu'il ne lui restait plus aucune espérance de ce côté-là, ni
de celui de M. le duc d'Orléans, qui, quelquefois refroidi pour sa
maîtresse, n'en devenait que plus passionné et plus abandonné à elle, de
sorte que le désespoir de la princesse n'avait jamais été plus vif, plus
complet, plus sans nulle ressource qu'au moment de cette délivrance. Je
répondis à cette confidence qu'il était fort heureux pour
M\textsuperscript{me} la duchesse d'Orléans qu'elle n'eût pas réussi, et
que la tendresse du roi eût trouvé sa sagesse à l'épreuve\,; que
M\textsuperscript{me} d'Argenton arrachée par autorité à M. le duc
d'Orléans, l'eût, et par amour et peut-être autant par orgueil, irrité
jusqu'à le jeter dans les dernières extrémités\,; que bien difficilement
en eût-il cru M\textsuperscript{me} sa femme innocente\,; que ce
soupçon, une fois monté dans son esprit, eût fait la ruine de sa
famille, et de M\textsuperscript{me} la duchesse d'Orléans la plus
malheureuse princesse de l'Europe. De là, la duchesse de Villeroy me
vanta M\textsuperscript{me} la duchesse d'Orléans, son esprit, sa
prudence, sa solidité, la sûreté de son amitié, la reconnaissance
qu'elle me devait et qu'elle sentirait tout entière, et m'invita fort à
une grande liaison avec elle.

Je répondis à tout cela par tous les compliments qui étaient lors de
saison. Je la priai de lui dire que, dans le désir où j'étais de
parvenir à séparer M. le duc d'Orléans de M\textsuperscript{me}
d'Argenton, j'aurais cru diminuer beaucoup les forces dont j'avais
besoin si, en répondant aux avances qu'elle avait bien voulu faire,
j'avais eu l'honneur de la voir, que cette prudence était devenue un
double bonheur par celui que j'avais eu de détromper à son égard M. le
duc d'Orléans sur les choses secrètes (que je ne rapporte pas ici, et
que j'avais confiées à la duchesse de Villeroy), lequel, malgré mes
preuves, soupçonneux comme il était, n'aurait pu se rendre à la même
confiance en moi, si j'avais été en mesure avec M\textsuperscript{me} sa
femme, comme il avait fait parce que je n'y étais en aucune\,; que
présentement qu'il n'y avait plus d'équilibre à garder avec lui, comme
j'avais fait jusqu'alors ne voyant ni M\textsuperscript{me} sa femme ni
sa maîtresse, je ferais volontiers ma cour à la première et mettrais
tous mes soins à continuer à travailler à une entière réunion\,; mais
que je croyais qu'il fallait aussi continuer d'user de la même prudence,
qu'il n'était pas temps encore que j'eusse l'honneur de la voir, qu'il
fallait un intervalle après ce qu'il venait de se passer pour amener les
choses\,; mais qu'en attendant, je la priais (la duchesse de Villeroy)
de dire à M\textsuperscript{me} la duchesse d'Orléans, etc.,
c'est-à-dire force compliments, et surtout d'exiger d'elle le plus
profond secret, chose dont je n'étais pas en peine, et par son intérêt
et par la matière. Je lui contai après combien je m'étais diverti, la
veille au soir, chez M\textsuperscript{me} de Saint-Géran, des doléances
extrêmes que M\textsuperscript{me} de Saint-Pierre y avait faites des
malheurs de M\textsuperscript{me} la duchesse d'Orléans par cette
tyrannie de M\textsuperscript{me} d'Argenton, à laquelle il n'y avait
plus nul espoir de fin, que je savais résolue et qui éclaterait bien
avant qu'il fût vingt-quatre heures de là.

\hypertarget{chapitre-iii.}{%
\chapter{CHAPITRE III.}\label{chapitre-iii.}}

1710

~

{\textsc{Le roi me donne l'heure de mon audience.}} {\textsc{- Besons,
mandé par M\textsuperscript{me} la duchesse d'Orléans, me fait de sa
part ses premiers remercîments.}} {\textsc{- Mesures pour apprendre la
rupture à M\textsuperscript{me} d'Argenton.}} {\textsc{- Naissance,
fortune et caractère de M\textsuperscript{lle} de Chausseraye.}}
{\textsc{- Audience que j'eus du roi.}} {\textsc{- Succès de mon
audience.}} {\textsc{- M\textsuperscript{me} d'Argenton apprend que M.
le duc d'Orléans la quitte.}} {\textsc{- Vacarme à la cour et dans le
monde à l'occasion de la rupture.}} {\textsc{- Joie du roi de la
rupture, avec qui M. le duc d'Orléans se rétablit, point avec
Monseigneur.}} {\textsc{- Je passe pour avoir fait la rupture, et, par
une aventure singulière, je suis pleinement révélé.}} {\textsc{- Liaison
intime entre M\textsuperscript{me} la duchesse d'Orléans et moi.}}
{\textsc{- Ma première conversation avec elle.}} {\textsc{- Politique du
duc de Noailles, difficile à ramener à M. le duc d'Orléans.}} {\textsc{-
Nancré\,; son caractère.}}

~

L'heure du dîner du roi arrivait, je sortis de chez la duchesse de
Villeroy pour y aller, et pour la laisser habiller pour aller chez
M\textsuperscript{me} la duchesse d'Orléans où elle avait impatience de
s'épanouir avec elle à leur aise. C'était, comme je l'ai dit, un
vendredi, 3 janvier, et le quatrième {[}jour{]} que je me présentais
devant le roi dans l'attente de l'audience qu'il avait promis à Maréchal
de me donner, et je commençais à être en peine de ce qu'elle ne venait
point. Je trouvai le dîner avancé, je me mis le dos au balustre, et vers
la fin du fruit, je m'avançai à un coin du fauteuil du roi, et lui dis
que je le suppliais de se vouloir bien souvenir qu'il m'avait fait
espérer la grâce de m'entendre. Le roi se tourna à moi et d'un air
honnête me répondit\,: «\, Quand vous voudrez. Je le pourrais bien à
cette heure, mais j'ai des affaires, et cela serait trop court,\,» et un
moment après, il se retourna encore, et me dit\,: «\,Mais demain matin
si vous voulez.\,» Je répondis que j'étais fait pour attendre ses
moments et ses grâces, et que j'aurais l'honneur de me présenter le
lendemain matin devant lui. Cette façon de me répondre me sembla de bon
augure, un air affable et point importuné, et envie de m'écouter à
loisir. Maréchal, le chancelier et M\textsuperscript{me} de Saint-Simon
en furent persuadés comme moi.

Sortant du dîner du roi, et passant auprès de l'appartement de
M\textsuperscript{me} la duchesse d'Orléans, je fus surpris de
rencontrer le maréchal de Besons qui sortait de chez elle, et que je
croyais déjà à Paris ou bien près d'y arriver. Il était en usage de la
voir quelquefois. Il me dit qu'inquiète de tout ce qu'il lui était
revenu par le domestique, elle l'avait envoyé chercher. À elle il avoua
tout le fait, et redoubla la joie que quelques bruits avaient fait
naître, et que Madame avait confirmés, qui en revenant de la messe avait
passé chez elle, et lui avait appris la rupture. Le maréchal me dit
qu'il lui avait grossièrement raconté les faits principaux, et me la
représenta transportée de la plus vive joie, et de reconnaissance pour
moi dont elle l'avait prié de m'assurer. Besons était si peiné de
l'éclat qui allait suivre, et si pressé de s'aller mettre à couvert chez
lui, qu'il n'osa demeurer que peu de moments avec moi, de peur qu'on ne
nous vît ensemble, comme si nous avions fait tous deux quelque mauvais
coup. Comme l'affaire principale était faite, je ne voulus pas le
contraindre, et je le laissai s'enfuir.

Je passai toute l'après-dînée avec M. le duc d'Orléans, qui n'était pas
moins vivement touché que le matin même. Il me dit que
M\textsuperscript{me} de Maintenon avait envoyé chercher la duchesse de
Ventadour aussitôt qu'il fut sorti de chez elle\,; qu'elle l'avait
chargée de faire entendre à M\textsuperscript{me} d'Argenton ce dont
était question, sur quoi lui et la duchesse étaient convenus d'envoyer
chercher Chausseraye, à qui il avait envoyé sa chaise de poste à Madrid
où elle avait une petite maison où elle était, et qui ne tarda pas à
venir. La commission lui parut fort dure, mais les prières et les larmes
de la duchesse de Ventadour, son amie intime, la persuadèrent enfin
d'aller apprendre à leur bonne amie commune le changement de son sort.

Chausseraye était une grande et grosse fille, qui avait infiniment
d'esprit, de sens et de vues, et dont tout l'esprit était tourné à
l'intrigue, au manège, à la fortune. Elle n'était rien du tout. Son nom
était Le Petit de Verno. Son père avait une méchante petite terre en
Poitou qui s'appelait Chausseraye. C'était apparemment un compagnon bien
fait, et qui n'était jamais sorti de son petit État ni de son voisinage.
La marquise de La Porte-Vezins, veuve, et qui demeurait dans ces
terres-là, auprès, s'en amouracha et l'épousa. Elle mourut en 1687 et en
laissa cette fille. Elle avait un fils de son premier lit, mort
lieutenant général des armées navales en grande réputation, et fort
honnête homme. Le duc de Brissac, père de la maréchale de Villeroy, la
maréchale de La Meilleraye, M\textsuperscript{me} de Biron, mère du
maréchal-duc de Biron, frère et soeurs de M\textsuperscript{me} de
Vezins, indignés de ce second mariage, ne voulurent jamais la voir ni le
mari encore moins, tellement que M\textsuperscript{lle} de Chausseraye
demeura longtemps dans l'angoisse, l'obscurité et la misère. M. de La
Porte-Vezins, son frère de mère, qui en devait être plus choqué qu'aucun
de la parenté, en prit pitié, et parvint à leur faire voir cette étrange
cousine. Sa figure et son esprit les gagna bientôt\,; jamais créature si
adroite, si insinuante, si flatteuse sans fadeur, si fine ni si fausse,
et qui en moins de temps reconnût ses gens et par où il les fallait
prendre. N'en sachant que faire, et pour la recrépir et lui donner du
pain, le maréchal de Villeroy qui, comme on l'a vu ici plus d'une fois,
pouvait tout et à bonne cause sur la duchesse de Ventadour, la fit par
elle entrer fille d'honneur de Madame qu'on éblouit du cousinage. Là,
sous la protection de M\textsuperscript{me} de Ventadour, elle la gagna
si bien qu'elle fut toute sa vie son amie la plus intime, et comme leurs
moeurs étaient plus semblables que leurs esprits, elle fut son conseil
en quantité de choses, dont elle ne lui en cacha toute sa vie aucune.

La galanterie, et après l'intrigue et l'intimité de
M\textsuperscript{me} de Ventadour, lui acquirent des amis et de la
considération, jusque-là que l'on comptait avec elle dans le monde. Elle
fit toujours tout ce qu'elle voulut des ministres. Barbezieux, le
chancelier de Pontchartrain, dès le temps qu'il avait les finances,
Chamillart ne lui refusaient rien. Elle sut apprivoiser jusqu'à
Desmarets et Voysin, et s'enrichit par eux. Mais ce fut tout autre chose
pendant la régence, qu'elle eut plusieurs millions. Elle était amie
intime de M\textsuperscript{me} d'Argenton, qu'elle avait fort connue
chez M\textsuperscript{me} de Ventadour, et amie de toute cette
séquelle, dont elle tirait du plaisir, et de l'argent de M. le duc
d'Orléans. Elle avait quitté Madame il y avait longtemps comme surannée,
mais elle était demeurée si bien avec elle qu'elle la voyait toujours en
particulier à Versailles, et que Madame l'allait voir aussi quelquefois.
Comme M\textsuperscript{me} de Ventadour elle était devenue dévote, mais
elle n'en intriguait pas moins. Il est incroyable de combien de choses
elle se mêlait. Elle joua toute sa vie tant qu'elle put, et y perdit
littéralement des millions. Le roi la traitait bien, et lui a plus d'une
fois donné des sommes considérables. Elle avait tout crédit sur Bloin et
sur les principaux valets, et voyait même quelquefois
M\textsuperscript{me} de Maintenon. Je la connaissois extrêmement\,; je
l'avais connue chez M\textsuperscript{me}s de Nogaret et d'Urfé, ses
cousines germaines, de chez qui elle ne bougeait à Versailles les
matins. Elle était d'excellente compagnie, et savait mille choses de
l'histoire de chaque jour par ses amis considérables. J'étais avec elle
sur un pied d'amitié et de recherche\,; mais je m'aperçus que la rupture
de M. le duc d'Orléans avec M\textsuperscript{me} d'Argenton m'avait
fort gâté avec elle, et quand elle le put dans les suites, je l'éprouvai
dangereuse ennemie. J'aurai occasion d'en parler ailleurs.

Le lendemain samedi, 4 janvier, le dernier des quatre, si principaux
pour moi par leurs suites, qui commencèrent cette année 1710, j'allai à
l'issue du lever du roi, et le vis passer de son prie-Dieu dans son
cabinet, sans qu'il me dît rien. C'était une heure de cour qui ne
m'était pas ordinaire. Je me contentais de le voir aller et revenir de
la messe\,; parce que depuis une longue attaque de goutte, il
s'habillait presque entièrement sur son lit, où le service ne laissait
guère de place. L'ordre donné, les entrées du cabinet sortaient, tout le
monde allait causer dans la galerie jusqu'à sa messe. Il ne restait
guère dans sa chambre que le capitaine des gardes en quartier, qu'un
garçon bleu avertissait quand le roi allait sortir par la porte de son
cabinet qui donne dans la galerie pour aller à la messe, lequel entrait
alors dans le cabinet pour le suivre. Je demeurai après l'ordre donné,
et le monde écoulé, seul avec le cabitaine des gardes dans la chambre.
C'était Harcourt, qui fut assez étonné de me voir là persévérant, et qui
me demanda ce que j'y faisais. Comme il allait me voir appeler dans le
cabinet, je ne fis point de difficulté de lui dire que j'avais un mot à
dire au roi, et que je croyais qu'il me ferait entrer dans son cabinet
avant la messe. Le P. Tellier, dont le vrai travail se faisait le
vendredi, était demeuré avec le roi\,; il sortit bientôt après, et
presque aussitôt Nyert, premier valet de chambre en quartier, sortit du
cabinet, chercha des yeux et me dit que le roi me demandait.

J'entrai aussitôt dans le cabinet. J'y trouvai le roi seul et assis sur
le bas bout de la table du conseil, qui était sa façon de faire, quand
il voulait parler à quelqu'un à son aise et à loisir. Je le remerciai en
l'abordant de la grâce qu'il voulait bien me faire, et je prolongeai un
peu mon compliment pour observer mieux son air et son attention, qui me
parurent l'un sévère, l'autre entière. De là, sans qu'il me répondît un
mot, j'entrai en matière. Je lui dis que je n'avais pu vivre davantage
dans sa disgrâce (terme que j'évitais toujours par quelque
circonlocution pour ne le pas effaroucher, mais dont je me servirai ici
pour abréger) sans me hasarder de chercher à apprendre par où j'y étais
tombé\,; qu'il me demanderait peut-être par quoi j'avais jugé du
changement de ses bontés pour moi\,; que je répondrais que, ayant été
quatre ans durant de tous les voyages de Marly, la privation m'en avait
paru une marque qui m'avait été très-sensible, et par la disgrâce, et
par la privation de ces temps longs de l'honneur de lui faire ma cour.
Le roi, qui jusque-là n'avait rien dit, me répondit, d'un air haut et
rengorgé, que cela ne faisait rien et ne marquait rien de sa part. Quand
je n'eusse pas su à quoi m'en tenir sur cette privation, l'air et le ton
de la réponse m'eût bien appris qu'elle n'était pas sincère\,; mais il
la fallut prendre pour ce qu'il me la donnait\,: ainsi je lui dis que ce
qu'il me faisait l'honneur de me dire me causait un grand soulagement,
mais que, puisqu'il m'accordait l'honneur de m'écouter, je le suppliais
de trouver bon que je me déchargeasse le coeur en sa présence, ce fut
mon terme, et que je lui disse diverses choses qui me peinaient
infiniment, et dont je savais qu'on m'avait rendu auprès de lui de fort
mauvais offices, depuis que des bruits, que mon âge et mon insuffisance
m'empêchaient de croire fondés, mais qui avaient fort couru, qu'il avait
jeté les yeux sur moi pour l'ambassade de Rome (ils étaient très-réels
comme on l'a vu ailleurs, mais il fallait parler ainsi, parce qu'il ne
me l'avait pas fait proposer dans l'incertitude de la promotion du
cardinal de La Trémoille\,;et que, dès qu'elle fut faite, il cessa d'y
vouloir envoyer un ambassadeur), l'envie et la jalousie s'étaient
tellement allumées contre moi, comme contre un homme qui pouvait devenir
quelque chose et qu'il fallait arrêter de bonne heure\,; que depuis ce
temps-là je n'avais pu dire ni faire rien d'innocent\,; que jusqu'à mon
silence même ne l'avait pas été, et que M. d'Antin n'avait pas cessé de
m'attaquer. «\,D'Antin\,! interrompit le roi, mais d'un air plus doux,
jamais il ne m'a nommé votre nom.\,» Je répondis que ce témoignage me
faisait un plaisir sensible, mais que d'Antin m'avait si attentivement
poursuivi dans le monde en toutes occasions que je n'avais pu ne pas
craindre ses mauvais offices auprès de lui.

En cet endroit le roi, qui avait déjà commencé à se rasséréner, prenant
un visage encore plus ouvert, et montrant une sorte de bonté et presque
de satisfaction à m'entendre, me coupa la parole comme je commençais un
autre discours par ces mots\,: «\,Il y a encore un autre homme\ldots.\,»
et me dit\,: «\,Mais aussi, monsieur, c'est que vous parlez et que vous
blâmez, voilà ce qui fait qu'on parle contre vous.\,» Je répondis que
j'avais grand soin de ne parler mal de personne\,; que, pour {[}parler
mal{]} de Sa Majesté, j'aimerais mieux être mort, en le regardant avec
feu entre deux yeux\,; qu'à l'égard des autres, encore que je me
mesurasse beaucoup, il était difficile que des occasions ne donnassent
pas lieu à parler quelquefois un peu naturellement. «\,Mais, me dit le
roi, vous parlez sur tout, sur les affaires, je dis sur ces méchantes
affaires, avec aigreur\ldots\,» Alors à mon tour j'interrompis le roi,
observant qu'il me parlait de plus en plus avec bonté\,; je lui dis que
des affaires j'en parlais ordinairement fort peu et avec de grandes
mesures\,; mais qu'il était vrai que, piqué quelquefois par de fâcheux
succès, il m'échappait d'abondance de coeur des raisonnements et des
blâmes\,; qu'il m'était arrivé une aventure qui, ayant fait un grand
bruit contre mon attente, m'avait aussi fait le plus de mal\,; que
j'allais l'en rendre juge, afin de lui en demander un très-humble pardon
si elle lui avait déplu, ou que, s'il en jugeait plus favorablement, il
vît que je n'étais pas coupable.

Je savais à n'en pas douter qu'on avait fait un prodigieux et pernicieux
usage de mon pari à Lille\,; j'avais résolu de le conter au roi, et j'en
saisis ici l'occasion qu'il me donna belle, mais avec la légèreté qu'il
convenait sur les acteurs avec lui. Je continuai donc à lui dire que,
lors du siége de Lille, touché de l'importance de sa conservation, au
désespoir de voir avec quelle diligence les ennemis s'y fortifiaient,
avec quelle lenteur son armée se mettait en mouvement, après trois
courriers dépêchés coup sur coup portant ordre de marcher au secours,
impatienté d'entendre continuellement assurer une levée de siége si
glorieuse et si nécessaire, laquelle je voyais impossible par le temps
que ces lenteurs donnaient aux ennemis de se mettre tout à fait à
couvert de cette crainte, il m'était échappé, dans le dépit d'une de ces
disputes, de parier quatre pistoles que Lille ne serait pas secouru et
qu'il serait pris. «\,Mais, dit le roi, si vous n'avez parlé et parié
que par intérêt de la chose, et par dépit de voir qu'elle ne réussissait
pas, il n'y a point de mal, et au contraire, cela n'est que bien\,; mais
quel est cet autre homme dont vous me vouliez parler\,?» Je lui dis que
c'était M. le Duc, sur lequel il garda le silence, et ne me dit point,
comme il avait fait sur d'Antin, qu'il ne lui avait point parlé de moi,
et je lui racontai en peu de mots autant que je pus, sans rien omettre
d'utile, le fait et le procédé de M\textsuperscript{me} de Lussan\,; et
comme sur le pari de Lille j'avais soigneusement évité de lui nommer les
noms de Chamillart, de Vendôme et de Mgr le duc de Bourgogne, j'évitai
ici avec le même soin de lui nommer M\textsuperscript{me} la Duchesse sa
fille, pour en mieux tomber sur M. le Duc. Je dis donc au roi que je
n'entrais point dans le fond de l'affaire de M\textsuperscript{me} de
Lussan pour ne l'en pas importuner, mais que M. le chancelier et tout le
conseil, M. le premier président et tout le parlement où elle avait été
portée, en avaient été indignés jusqu'à lui en avoir fait de fâcheuses
réprimandes\,; que cette femme m'ayant attaqué partout et par toutes
sortes de mensonges, j'avais été contraint de me défendre par des
vérités poignantes à la vérité, mais justes et nécessaires\,; qu'avant
de les publier j'avais supplié M. le Prince d'en entendre la lecture\,;
que je la lui avais faite, et qu'il avait trouvé très-bon que je les
publiasse\,; que je n'avais jamais pu approcher de M\textsuperscript{me}
la Princesse ni de M. le Duc\,; qu'il était étrange qu'il s'intéressât
plus dans l'affaire de la dame d'honneur de M\textsuperscript{me} la
Princesse que M. le Prince même, lequel avait fort gourmandé
M\textsuperscript{me} de Lussan là-dessus\,; qu'enfin Sa Majesté
trouvait bon que ses sujets eussent tous les jours des procès contre
elle, et qu'il serait étrange qu'on n'osât se défendre des mensonges de
M\textsuperscript{me} de Lussan, dont la place serait plus que la
première du royaume, si elle lui donnait le droit de plaider et de
mentir sans réplique. J'ajoutai que M. le Duc ne me l'avait jamais
pardonné depuis, qu'il n'y avait point d'occasion où je ne m'en fusse
aperçu, et que c'était une chose horrible que moi, absent naturellement
et à la Ferté, comme j'avais accoutumé à Pâques, et sans savoir M. le
Prince en état de mourir, M. le Duc eût dit à Sa Majesté, sur l'affaire
des manteaux, que c'était dommage que je n'y fusse et que je me
donnerais bien du mouvement.

Le roi, qui m'avait laissé tout dire, et sur qui je remarquai que
j'avais fait impression, me répondit avec l'air et la façon d'un homme
qui veut instruire, qu'aussi je passais pour être vif sur les rangs, que
je m'y étais mêlé de beaucoup de choses, que je poussais les autres, et
me mettais à leur tête. Je répondis qu'à la vérité cela m'était arrivé
quelquefois, et qu'en cela même je n'avais pas cru rien faire qui lui
pût déplaire, mais que je le suppliais de se souvenir que, depuis
l'affaire de la quête dont je lui avais rendu compte, il y avait quatre
ans, je n'étais entré en aucune sorte d'affaire. Je lui remis en deux
mots le fait de celle-là, et de celle de la princesse d'Harcourt\,; et
sur ce que je lui dis que j'avais eu lieu de croire qu'il en avait été
content, il en convint, et m'en dit des choses de lui-même, qui me
montrèrent qu'il s'en souvenait parfaitement, sur quoi je ne manquai pas
de lui dire que la maison de Lorraine ne l'avait pas oublié, et n'avait
cessé de me le témoigner depuis. Revenant tout de suite d'où je m'étais
écarté, j'ajoutai que c'était bien assez de ne m'être mêlé de rien
depuis quatre ans, pour que M. le Duc, à qui je n'avais jamais rien
fait, ne fit pas souvenir de moi dans un temps d'absence où je ne
pensais à rien moins. L'air de familiarité que j'avais usurpé dans la
parenthèse des Lorrains, et en retombant sur M. le Duc, et celui
d'attention, d'ouverture et de bonté non ennuyée que je vis dans le roi,
me fit ajouter que j'avais beau d'entrer en rien, puisque, dans ma
dernière absence dont j'arrivais, il m'avait été mandé de beaucoup
d'endroits qu'on avait extrêmement parlé de moi sur ce qui était arrivé
entre les carrosses de M\textsuperscript{me}s de Mantoue et de
Montbazon, et que j'osois lui demander ce que je pouvais faire pour
éviter ces méchancetés, et des propos qui se tenaient gratuitement, moi
absent depuis longtemps, et dans la parfaite ignorance de l'aventure de
ces dames. «\,Cela vous fait voir, me dit le roi en prenant un vrai air
de père, sur quel pied vous êtes dans le monde, et il faut que vous
conveniez que cette réputation, vous la méritez un peu. Si vous n'aviez
jamais eu d'affaires de rangs, au moins que vous n'y eussiez pas paru si
vif sur celles qui sont arrivées, et sur les rangs mêmes, on n'aurait
point cela à dire. Cela vous doit montrer aussi combien vous devez
éviter tout cela, pour laisser tomber ce qu'on en peut dire, et faire
tomber cette réputation par une conduite sage là-dessus, et suivie, pour
ne point donner prise sur vous.\,» Je répondis que c'était aussi ce que
j'avais continuellement fait depuis quatre ans, comme je venais d'avoir
l'honneur de le lui dire, et ce que je ferais continuellement à
l'avenir, mais qu'au moins le suppliais-je de voir combien peu de part
j'avais eu en ces dernières choses, desquelles néanmoins je ne me
trouvais pas quitte à meilleur marché\,; que j'avais une telle crainte
de me trouver en tracasseries et en discussions, surtout devant lui,
qu'il fallait donc que je lui disse maintenant la véritable raison qui
m'avait fait rompre le voyage de Guyenne qu'il m'avait permis de
faire\,; que cette raison était celle des usurpations, étranges du
maréchal de Montrevel sur mon gouvernement, qui étaient telles que je
n'y pouvais aller qu'elles ne fussent décidées\,; que M. le maréchal de
Boufflers, qui avait commandé en chef en Guyenne, à qui j'avais exposé
mes raisons, avait jugé en ma faveur, et cru que M. de Montrevel l'en
voudrait bien croire\,; mais que ce dernier s'étant opiniâtré à vouloir
que Sa Majesté décidât, j'avais mieux aimé perdre mes affaires qui
avaient grand besoin de ma présence, et laisser encore le maréchal de
Montrevel usurper tout ce que bon lui semblait et semblerait, que d'en
importuner Sa Majesté, tant j'étais éloigné de toutes querelles, et
surtout de l'en fatiguer.

Le roi goûta tellement ce propos qu'il l'interrompit plusieurs fois par
des monosyllabes de louanges pour ne pas troubler le fil de mon
discours, à la fin duquel il me loua davantage et m'applaudit plus à son
aise, sans pourtant entrer en rien sur ces différends de Guyenne, tant
il abhorrait toute discussion, et aimait mieux que tout s'usurpât et se
confondit, souvent même au préjudice connu de ses affaires, que d'ouïr
parler de cette matière, et surtout de décision. Je lui parlai aussi de
la longue absence que j'avais faite de douleur de me croire mal avec
lui, d'où je pris occasion de me répandre moins en respects, qu'en
choses affectueuses sur mon attachement à sa personne, et mon désir de
lui plaire en tout, que je poussai avec une sorte de familiarité et
d'épanchement, parce que je sentis à son air, à ses discours, à son ton
et à ses manières, que je m'en étais mis à portée. Aussi furent-ils
reçus avec une ouverture qui me surprit, et qui ne me laissa pas douter
que je ne me fusse remis parfaitement auprès de lui. Je le suppliai même
de daigner me faire avertir, s'il lui revenait quelque chose de moi qui
pût lui déplaire, qu'il en saurait aussitôt la vérité, ou pour pardonner
à mon ignorance, ou pour mon instruction, ou pour voir que je n'étais
point en faute. Comme il vit qu'il n'y ayait plus de points à traiter,
il se leva de dessus sa table. Alors je le suppliai de se souvenir de
moi pour un logement, dans le désir que j'avais de continuer à lui faire
une cour assidue\,; il me répondit qu'il n'y en avait point de vacant,
et avec une demi-révérence riante et gracieuse, s'achemina vers ses
autres cabinets, et moi après une profonde révérence je sortis en même
temps par où j'étais entré, après plus d'une demi-heure d'audience la
plus favorable\,; et fort au delà de ce que j'avais pu espérer.

J'allai tout droit chez Maréchal, par un juste tribut, lui raconter tout
ce qui se venait de passer, et que je lui devais uniquement, dont il fut
ravi et en augura au mieux\,; de là chez le chancelier à qui la messe du
roi me donna loisir de tout conter. Il pesa attentivement chaque chose,
et fut tellement surpris de la façon dont le roi était descendu dans
tous les détails, de ses réponses, de ses interruptions, et puis de ses
reprises, qu'il me protesta qu'il ne connaissoit pas encore quatre
hommes à la cour, de quelque sorte qu'ils fussent, avec qui le roi en
eût usé ainsi. Il m'exhorta à une grande circonspection, à une grande
assiduité, à bien espérer, et m'assura que, connaissant le roi comme il
faisait, pour ainsi dire à revers, je pouvais compter, non-seulement
qu'il ne lui restait aucune impression contre moi, mais qu'il était bien
aise qu'il ne lui en restât aucune, et que j'étais très-bien avec lui.
Ce qui me surprit le plus et qui me donna encore plus de confiance, fut
la conformité de l'avis de M. de Beauvilliers, et même de ses paroles,
qu'il ne connaissoit pas un autre homme avec qui le roi se fût ouvert,
et fût entré de la sorte.

On ne peut exprimer la joie de ces amis, et combien le chancelier traita
avec élargissement le chapitre de ma retraite que son adresse avait
arrêtée, et combien je sentis et lui témoignai l'obligation que je lui
en avais. J'allai ensuite tirer M\textsuperscript{me} de Saint-Simon
d'inquiétude que je changeai en une grande joie. C'était elle qui
m'avait aposté le chancelier et tous mes amis, et qui par là m'avait
forcé, comme je l'ai dit, à ce dernier remède, dont le succès fut tel
que le roi m'a toujours depuis, non-seulement bien traité, mais avec une
distinction marquée pour mon âge, jusqu'à sa mort, et sans lacune\,; je
dis pour mon âge quoiqu'à trente-cinq ans que j'allais avoir ce ne fût
plus jeunesse, mais à son égard, c'était encore au-dessous, surtout pour
un homme sans charge, et sans occasion de familiarité avec lui, et voilà
quel trésor est une femme sensée et vertueuse. Elle m'avoua alors
l'extrême éloignement du roi qu'elle avait su de M\textsuperscript{me}
la duchesse de Bourgogne, et qu'elle m'avait prudemment caché pour ne me
pas éloigner moi-même davantage. Elle crut sagement aussi qu'ayant eu
recours à celte princesse qui l'avait si bien reçue, elle lui devait
rendre compte de ce qui venait de se passer, sur quoi elle lui témoigna
beaucoup de joie et toutes sortes de bontés. Comme rien n'était plus
rare qu'une audience du roi à ceux qui n'avaient point de particulier
naturel avec lui, celle que je venais d'avoir, et surtout sa longueur,
fit plus de bruit que je ne désirais. Je laissai dire et me tins en
silence, parce qu'on n'est point obligé de rendre compte de ses
affaires. Maréchal me dit deux jours après que le roi m'avait fort loué
à lui, et {[}avait{]} témoigné toutes sortes de satisfaction de mon
audience. Retournons maintenant à M. le duc d'Orléans avec qui je passai
encore toute cette après-dînée.

Chausseraye était allée la veille tout droit chez la duchesse de
Ventadour à Versailles, chez M\textsuperscript{me} d'Argenton à Paris,
où elle ne la trouva point, et sut qu'elle était allée jouer et souper
chez la princesse de Rohan, d'où elle ne reviendrait que fort tard, sur
quoi elle lui manda qu'elle avait à lui parler et qu'elle l'attendait
chez elle. M\textsuperscript{me} d'Argenton ne se pressant point de
revenir, M\textsuperscript{lle} de Chausseraye renvoya et la fit
arriver. Elle lui dit que ce qu'elle avait à lui apprendre était si
sérieux qu'elle eût bien voulu qu'une autre en fût chargée\,; et avec
ces détours comme pour annoncer la mort de quelqu'un, elle fut longtemps
sans être entendue. Enfin elle la fut. Les larmes, les cris, les
hurlements firent retentir la maison, et annoncèrent au nombreux
domestique la fin de sa félicité, lequel ne fut pas plus ferme que la
maîtresse. Après un long silence de Chausseraye, elle se mit à parler de
son mieux, à faire valoir les largesses, la délicatesse sur tout ordre
par écrit, la liberté dans tout le royaume excepté Paris et les
apanages. M\textsuperscript{me} d'Argenton au désespoir, mais peu à peu
devenue plus traitable, demanda à se retirer pour les premiers temps
dans l'abbaye de Gomerfontaine en Picardie, où elle avait été élevée et
y avait une soeur religieuse. L'Abbé de Thesut, secrétaire des
commandements de M. le duc d'Orléans, ami intime de toute cette séquelle
et dont j'aurai occasion de parler dans la suite, fut mandé, puis envoyé
à Versailles, chargé d'une lettre de M\textsuperscript{me} d'Argenton
pour M. le duc d'Orléans, et d'une autre pour la duchesse de Ventadour,
priée de voir M\textsuperscript{me} de Maintenon sur cette retraite.

Tandis que j'étais chez M. le duc d'Orléans, avec deux ou trois de ses
premiers officiers, à causer pour l'amuser comme nous pouvions, l'abbé
de Thesut entra, qui lui vint dire un mot à l'oreille. À l'instant je
vis une grande altération sur son visage. Il rêva un moment, se leva,
alla à l'autre bout de l'entre-sol avec l'abbé, puis m'appela, ce qui
fit sortir les autres. Demeurés seuls tous trois, M. le duc d'Orléans me
demanda avec angoisse si j'avais jamais vu une dureté pareille,
m'expliqua la demande de Gomerfontaine et sa cause, et à peine m'en
eût-il dit le refus, qu'il entra en une espèce de rage et de fureur, et
s'abandonna au repentir de ne s'en être pas fui de Besons et de moi dans
le sein de sa maîtresse la nuit qui précéda la rupture, comme il en
avait été mille fois tenté. Après avoir laissé quelque cours à cette
tempête, je lui représentai qu'avant de s'abandonner ainsi au
déchaînement, il fallait voir un peu mieux de quoi il s'agissait\,; que,
si la chose était crue ainsi qu'on la lui disait, je ne pouvais
disconvenir qu'il n'eût lieu d'être en colère, et que j'y étais autant
que lui, mais que je le suppliais que nous puissions raisonner un
moment. Je demandai à l'abbé de Thesut ce qu'on prétendait que
M\textsuperscript{me} d'Argenton devînt, et pourquoi on ne voulait pas
la laisser se retirer en un lieu si naturel, et où elle pourrait trouver
de la consolation, de l'instruction et des exemples. Il me répondit que
M\textsuperscript{me} de Maintenon aimait l'abbesse et la maison de
Gomerfontaine, où elle avait envoyé des demoiselles de Saint-Cyr,
qu'elle avait des desseins dessus, et qu'elle ne voulait pas que
M\textsuperscript{me} d'Argenton la gâtât. Je dis à M. le duc d'Orléans,
qui cependant tempêtait de toutes ses forces, qu'il aurait regret de
s'être tant tourmenté pour si peu de chose, que je ne voyais que deux
choses qui pussent lui faire de la peine et intéresser
M\textsuperscript{me} d'Argenton\,: un ordre par écrit qu'il était sûr
qu'elle n'aurait pas, une contrainte sur sa liberté que je ne voyais pas
ici\,; et que, s'il voulait m'en croire, je parierais toutes choses
qu'il aurait contentement.

J'eus peine à lui faire entendre raison. À la fin il consentit à la
proposition que je lui fis d'écrire à M\textsuperscript{me} de
Maintenon. Après avoir écrit les deux premiers mots, il se renversa dans
sa chaise, me dit qu'il ne pouvait penser, encore moins écrire, et qu'il
me priait de faire la lettre. J'en fis le compliment à l'abbé de Thesut,
puis je la fis. Ils la trouvèrent bien tous deux, l'abbé la lui dicta,
il l'écrivit, et mit le dessus de sa main, et l'envoya par Imbert, son
premier valet de chambre, comme le roi était déjà chez
M\textsuperscript{me} de Maintenon, qui était ce que je voulais pour
qu'il la vît. Imbert la donna à l'officier des gardes qui demeurait là
de garde. Celui-ci la porta à M\textsuperscript{me} de Maintenon\,; mais
le roi ayant demandé et su de qui était la lettre, la prit, et c'était
ce que nous désirions. J'essuyai tout le soir des regrets cuisants
demeuré tête à tête, et pour la première fois de ma vie je vis des
lettres de M\textsuperscript{me} d'Argenton. M. le duc d'Orléans lui
écrivit, et j'eus peine à obtenir qu'il s'en abstiendrait tout à fait à
l'avenir. Après le souper, le roi dit à M. le duc d'Orléans qu'il avait
vu sa lettre, que Gomerfontaine ne se pouvait, parce que
M\textsuperscript{me} de Maintenon ne le désirait pas, par les raisons
que nous savions, qu'il lui répéta\,; mais qu'à l'exception de ce lieu,
il n'y en avait aucun où sa maîtresse n'eût liberté d'aller et de
demeurer, tant et si peu qu'il lui plairait. Tout cela fut accompagné
d'amitiés, et d'un air fort différent de celui que le temps mal pris et
la surprise avaient causé lors de la déclaration de la rupture.

M\textsuperscript{me} d'Argenton ne demeura que quatre jours à Paris,
depuis que Chausseraye la lui était allée dire. Elle s'en alla chez son
père qui vivait chez lui près de Pont-Sainte-Maxence, et le chevalier
d'Orléans, son fils, demeura au Palais-Royal. Cette retraite excita
toutes les langues. Les amies de M\textsuperscript{me} d'Argenton s'en
irritèrent comme d'un outrage, n'osant crier contre la rupture même. La
duchesse de Ventadour, naturellement douce, et d'ailleurs retenue par la
cour, se contenta de pleurer. La duchesse douairière d'Aumont, sa soeur,
ne se contraignit pas tant. Dévote outrée, joueuse démesurée par accès,
et souvent tous les deux ensemble, et toujours méchante, elle était la
meilleure amie de M\textsuperscript{me} d'Argenton, et força la duchesse
d'Humières, sa belle-fille, de la venir voir partir avec elle. La
duchesse de La Ferté et M\textsuperscript{me} de Bouillon s'emportèrent
fort aussi, et toute la lie de Paris et du Palais-Royal sans mesure. Les
ennemis de M. le duc d'Orléans, particulièrement M\textsuperscript{me}
la Duchesse, et tout ce qui tenait à elle, prirent un autre tour. Ils
semèrent que le roi était sa dupe\,; qu'à bout du joug, dur, cher et
capricieux de sa maîtresse, il s'était fait avec lui un faux mérite et
un honteux honneur de sa rupture\,; que le procédé de l'y avoir fait
entrer était d'un bas courtisan, raffiné\,; que la victime était bien à
plaindre, mais que bientôt M. le duc d'Orléans, lassé d'une vie
raisonnable, prendrait quelque nouvel engagement. Les indifférents et
les raisonnables qui firent le plus grand nombre, ne purent refuser
leurs louanges à la rupture, leur approbation à la manière. Deux
millions leur parurent une libéralité excessive. De laisser
M\textsuperscript{me} d'Argenton dans Paris aux risques de renouer avec
elle, au moins de donner lieu tous les jours à le dire et à le croire,
leur sembla contre tout bon sens, et impossible de l'en faire sortir par
l'autorité du roi, par conséquent de nécessité absolue de lui confier
d'abord la rupture, et quant à la manière de l'en faire retirer, ils y
trouvèrent tous les ménagements possibles.

Le roi, comme je viens de le dire, revenu de la surprise d'un temps mal
pris, se livra à la plus grande joie, et la témoigna dès le lendemain à
M. le duc d'Orléans\,; il le traita depuis toujours de bien en mieux.
M\textsuperscript{me} de Maintenon n'osa pas n'y point contribuer un peu
dans ces commencements, où les jésuites servirent très-bien ce prince,
qui se les était attachés. M\textsuperscript{me} la duchesse de
Bourgogne y fit des merveilles par elle-même\,; et Mgr le duc de
Bourgogne, poussé par le duc de Beauvilliers. Monseigneur seul demeura
le même qu'il était à son égard, continuellement aigri sur l'affaire
d'Espagne par M\textsuperscript{me} la Duchesse et par tout ce qui
l'obsédait avec art et empire. L'espérance de marier la fille aînée de
M\textsuperscript{me} la Duchesse à M. le duc de Berry redoublait encore
leur application à tenir Monseigneur dans cet extrême éloignement.

Plusieurs jours se passèrent sans qu'on parlât d'autre chose que de
cette rupture, qui passa publiquement pour mon ouvrage, sans qu'on y
donnât presque aucune part à Besons. Je m'en défendis constamment
jusques avec mes amis particuliers, tant pour en laisser tout l'honneur
à M. le duc d'Orléans, que pour éviter la rage de tous ceux qui par
intérêt en étaient fâchés, et par une juste crainte de montrer mon
crédit sur l'esprit d'un prince qu'il n'était pas certain de porter
toujours où on voulait, ni qui demeurât toujours exempt de fautes.
Toutefois je ne gagnai rien par cette conduite, sinon de n'avouer
jamais. Chacun demeura persuadé de la vérité du fait, et je crus que le
domestique de M. le duc d'Orléans en fut cause, en racontant ce qu'ils
avaient vu de mes longs et continuels particuliers avec lui
immédiatement auparavant. Mais il m'arriva un autre inconvénient que je
n'avais garde de prévoir et qui mit au fait de la chose ceux-là mêmes
auxquels il m'était le plus important de le tenir caché. J'avais fort
conseillé à M. le duc d'Orléans de rechercher les principaux personnages
en estime et en considération dans le monde et aussi en crédit. Dans
cette vue il se rallia un peu le maréchal de Boufflers, et pour se
l'attacher davantage, il lui parla franchement sur ses torts, il en
convint avec lui, raisonna confidemment de la conduite qu'il avait
résolue à l'avenir, enfin s'ouvrit au point de lui conter tout ce qui
s'était passé sur sa rupture avec sa maîtresse. De tout cela il lui en
demanda le secret, excepté pour moi et pour le duc de Noailles, qui
arrivait de Roussillon dans ces premiers jours de janvier.

Le maréchal, mon ami intime, ravi de me savoir l'auteur et l'exécuteur
d'une oeuvre si bonne, si difficile, et qu'il savait si fort tenir au
coeur du roi et de M\textsuperscript{me} de Maintenon par elle-même, qui
souvent lui en avait parlé avec fureur, ne douta pas qu'il ne me rendît
un excellent office en lui confiant que c'était moi seul qui avais fait
chasser M\textsuperscript{me} d'Argenton. Il me surprit étrangement
lorsqu'il me conta l'aveu que lui en avait fait M. le duc d'Orléans, et
bien davantage qu'il l'avait dit à M\textsuperscript{me} de Maintenon. À
son tour il ne le fut pas moins de ma froideur à ce récit, et m'en
demanda la cause. Je la lui dis\,; mais comme il avait plus de droiture
que d'esprit et de vraie connaissance de cour, où il n'était venu qu'âgé
et déjà dans les grands emplois de guerre, il ne goûta point mes raisons
et se récria sur l'injustice qu'il y avait de prendre thèse sur ce que
j'avais fait faire de bon à M. le duc d'Orléans, pour m'imputer de
n'empêcher pas ce qu'à l'avenir il pourrait faire de mal. Ce qu'il avait
dit était lâché et lâché par principe d'amitié\,; ainsi voyant la chose
sans remède, je ne voulus pas contester vainement, et je le remerciai du
mieux que je pus. Le roi ni M\textsuperscript{me} de Maintenon, laquelle
je ne voyais jamais, ne m'en ont jamais parlé ni rien fait dire\,; mais
par un trait du roi, qui se trouvera dans la suite, je ne puis presque
douter qu'il ne l'ait su.

La rupture ainsi achevée et terminée, je songeai à en faire tirer à M.
le duc d'Orléans tous les plus avantageux partis qu'il me fût possible,
et je n'en crus aucun meilleur, à tous égards, que celui de le lier
étroitement à M\textsuperscript{me} sa femme dans une si favorable
jointure. Il avait été infiniment content de la manière dont elle avait
pris la rupture. Elle contint sa joie avec une modération et une sagesse
qui ne se démentit point, et qui eut une grande force pour ramener M. le
duc d'Orléans vers elle. Comme il me l'avoua dès les premiers jours, et
que je sentis ses froncements mollis, je me hâtai de me servir de ces
ouvertures récentes, et de sa désoccupation ennuyeuse et pénible dans ce
subit changement de vie, pour l'attacher à M\textsuperscript{me} la
duchesse d'Orléans. Jugeant ensuite que je pourrais ne leur être pas
inutile, je lui dis que jusqu'à présent j'avais fait une sorte de
profession publique de ne la jamais voir non plus que les autres
princesses, chez qui je n'allais jamais qu'un instant aux occasions\,;
que maintenant que rien ne les séparait plus, c'était à lui à me
prescrire ma conduite à cet égard, et à mon attachement pour lui à m'y
conformer. À l'instant il me pria de la voir avec un empressement qui me
surprit. Il me dit que c'était une chose qu'il avait résolu de me
demander\,; il ajouta qu'il serait extrêmement aise que la liaison qui
était entre lui et moi s'étendît à elle\,; il s'étendit là-dessus en
raisons et en désirs.

J'étais cependant extrêmement pressé par elle de la voir. Elle avait
chargé la duchesse de Villeroy de m'en témoigner son impatience, et cela
plusieurs fois, c'est-à-dire tous les jours, et de me dire à quel point
elle ressentait ce que j'avais fait pour elle. Elle en avait dit autant
aussi à M\textsuperscript{me} de Saint-Simon avec de grandes effusions
de coeur, qui la voyait souvent\,; mais, sans rien de particulier, lui
avait parlé dans les termes de la plus vive reconnaissance. Ainsi, après
avoir laissé passer quelques jours, pendant lesquels M. le duc d'Orléans
me pressait toujours de la voir, je convins avec la duchesse de Villeroy
de l'heure d'y aller, parce qu'elle me voulait voir en particulier.
Comme je fus annoncé un soir après son jeu, le peu de familières qui
étaient restées s'en allèrent. Elle était dans son cabinet dans un petit
lit de jour, en convalescence de sa couche de la reine d'Espagne. On
m'apporta un siége auprès d'elle où je m'assis. Là, tête à tête, tout ce
qu'elle me dit de gracieux ne se peut rendre. La joie et la
reconnaissance s'exprimaient avec un choix de paroles si juste, si
précis et si fort que j'en fus surpris. Elle eut l'art de me faire
entendre tout ce qu'elle sentait à mon égard sur ce que j'avais fait
pour elle, et qui n'est pas écrit ici, sans qu'il lui échappât rien
d'embarrassant ni pour elle ni pour moi\,; et je me sauvai par des
respects et des compliments vagues. Surtout elle me remercia de l'avoir
si bien servie sans l'avoir jamais auparavant connue, et se récria sur
la générosité, car ce fut le terme qu'elle employa, de ne l'avoir évitée
que pour la mieux délivrer. Il n'y eut protestations qu'elle ne me fit
d'une amitié, d'un souvenir, d'une reconnaissance éternels, et termes
obligeants et forts dont elle ne se servît pour me demander
personnellement mon amitié. Ensuite elle me dit, un peu en continuant de
rougir, car cela lui était arrivé plus d'une fois et avec grâce dans le
cours de ses remercîments, que je serais peut-être surpris qu'elle, qui
avec raison n'avait pas la réputation d'être confiante, me parlât avec
une entière ouverture dès la première entrevue, mais que mon intimité
avec M. le duc d'Orléans, et ce que je venais de faire, le permettait et
l'exigeait même ainsi. Après cette petite préface, elle entra en effet
avec moi en des raisonnements les plus pleins de confiance sur la
conduite que M. le duc d'Orléans avait à tenir pour se tirer de l'état
auquel il s'était mis.

Je fus extrêmement surpris de sentir tant d'esprit, de sens et de
justesse, dont je conclus en moi-même encore plus fortement de
n'épargner aucun soin pour unir le mari et la femme le plus étroitement
que je le pourrais, fermement persuadé, outre la foule des autres
raisons, qu'il ne trouverait nulle part un meilleur conseil qu'en elle.
Nous concertâmes donc, dès cette première fois, diverses choses, bien
résolus de marcher ensemble pour remettre M. le duc d'Orléans au monde,
en quoi néanmoins nous trouvâmes plus de difficulté que nous n'avions
pensé\,; mais au moins je parvins assez aisément à l'unir et à le faire
vivre avec elle aussi agréablement et même aussi intimement qu'il était
en lui, à la grande surprise de la cour, et au grand dépit de
M\textsuperscript{me} la Duchesse et de ses autres ennemis, qui ne
purent même le dissimuler. Devenu ainsi l'auteur de cette union, j'en
devins aussi l'instrument continuel, dans laquelle je fus en tiers dans
une confiance et une intimité égale avec chacun des deux. Leurs ennemis
commencèrent à en craindre les effets, et les miens à publier que je
gouvernais cette barque.

Une des choses à laquelle je crus devoir le plus travailler, fut à faire
que M. le duc d'Orléans se ramenât le monde. Je fis ce que je pus pour
l'engager aux démarches qui y étaient nécessaires, aidé par
M\textsuperscript{me} la duchesse d'Orléans, et favorisé par le grand
changement et public en bien du roi pour lui\,; mais il était encore si
effarouché, qu'il craignait également la solitude et la compagnie, et ne
se pouvait résoudre à donner les moyens et les facilités propres à se
faire rentourer. Le duc de Noailles avait été dans leur plus étroite
confidence à tous deux\,; il s'en était fort retiré depuis l'affaire
d'Espagne, surtout de M. le duc d'Orléans. C'était lui, comme je l'ai
dit ailleurs, qui lui avait donné Flotte\,; il prétendit l'avoir
toujours parfaitement ignorée\,; il craignit de s'y trouver pour quelque
chose, à cause de Flotte, s'il continuait dans la même liaison\,; il
s'éloigna sous prétexte que ce prince s'était trop avantagé dans l'éclat
de cette affaire\,; que c'était lui qui lui avait donné cet homme\,; il
se passa entre eux encore quelque autre chose\,; bref, je n'ai jamais su
le fond de tout cela, ni par le prince, ni par le duc, avec qui j'ai
vécu longtemps en liaison la plus étroite, mais qui ne commença que plus
tard. La prétention des filles de M\textsuperscript{me} la duchesse
d'Orléans sur les femmes des princes du sang était déjà née\,; le duc de
Noailles y était entré fort avant dans les premières, et quoi qu'il eût
pu faire pour se cacher, il ne put éviter que M\textsuperscript{me} la
Duchesse, avec qui il était fort bien, n'en fût informée et piquée
jusqu'aux reproches, et puis à la froideur. Le désir de se raccommoder
avec elle eut peut-être part au procédé qu'il eut avec M. le duc
d'Orléans. Il était déjà personnage à la cour par l'amitié et la
confiance de M\textsuperscript{me} de Maintenon, et par ses emplois, et
M\textsuperscript{me} la Duchesse ne fut pas fâchée de se raccommoder
avec lui. Ces mêmes raisons nous firent désirer à M\textsuperscript{me}
la duchesse d'Orléans et à moi de le ramener. Il était toujours demeuré
fort en mesure avec elle, et elle croyait que M. le duc d'Orléans avait
tort avec lui\,; elle-même en était embarrassée, et désirait fort de
finir tout cela.

Nancré était fort lié avec M\textsuperscript{me} d'Argenton, et fort mal
avec M\textsuperscript{me} la duchesse d'Orléans, qui avait grand lieu
d'en être plus que mécontente. C'était un drôle de beaucoup d'esprit\,;
de manége et de monde, aimable dans le commerce et dans la société, mais
dangereux fripon, pour ne pas dire scélérat, dont il ne s'éloignait
guère, qui aimait à se mêler de tout, dont l'intrigue était la vie, et
qui, n'ayant ni âme ni sentiment, que simulés, voulait cheminer et être
compté, à quoi tous moyens lui étaient bons. La rupture, et M. le duc
d'Orléans raccommodé au mieux avec M\textsuperscript{me} sa femme et se
tournant au sérieux, l'embarrassaient fort. Il était des amis du duc de
Noailles\,; il lui parla de cette brouillerie, et lui promit ce qu'il ne
put tenir.

M. le duc d'Orléans, qui ne comptait pas sur la sûreté de Nancré, sut du
maréchal de Besons que le duc de Noailles lui en avait parlé, et en
saisit l'occasion pour lui remettre cette espèce de négociation. Besons
agit, et trouva Noailles dans des réserves de respect fort sèches.
M\textsuperscript{me} la duchesse d'Orléans le vit chez elle avec une
retenue qui ne put se réchauffer. Il était fort lié avec le maréchal de
Boufflers et aussi avec Besons\,; apparemment qu'il sut d'eux la part
que j'avais eue à la rupture. Il crut ou sut aussi que je n'ignorais pas
le louche qui s'était mis entre M. le duc d'Orléans et lui, tellement
que, encore que je n'eusse avec lui aucune sorte d'habitude ni de
liaison, quoique fort bien de tout temps avec sa mère, je remarquai
qu'il me tournait, et à la fin il me parla en homme plein qui veut
s'épancher et montrer qu'il a raison. Je ne laissai pas d'en être
surpris\,; mais comme tout ce qui me revenait de lui depuis longtemps me
plaisait, je m'approchai à mesure qu'il s'approchait. Il me parla en
général de son fait avec M. le duc d'Orléans, et me pria qu'il pût me le
conter à loisir. Moi qui n'avais que faire de tout cela, sinon en gros,
par le désir de les voir rapprocher, j'évitai doucement cette
conversation demandée. Néanmoins, il se forma un peu plus de commerce
entre eux, mais fort mesuré, {[}M. de Noailles{]} avouant même ses
ménagements renouvelés par M\textsuperscript{me} la Duchesse, tellement
qu'il ne fut pas jugé à propos de le presser davantage, mais bien
d'attendre mieux du bénéfice du temps et d'en profiter quand il serait
possible.

\hypertarget{chapitre-iv.}{%
\chapter{CHAPITRE IV.}\label{chapitre-iv.}}

1710

~

{\textsc{Manége de M\textsuperscript{me} de Maintenon auprès du roi.}}
{\textsc{- Mesures pour faire le maréchal de Besons gouverneur de M. le
duc de Chartres avortées.}} {\textsc{- Inquisition des jésuites.}}
{\textsc{- Division éclatante dans la famille de M. le Prince sur le
testament, qui est porté en justice.}} {\textsc{- Enrôlement forcé par
M. le Duc.}} {\textsc{- Le roi défend aux enfants de M. le Prince tout
accompagnement au palais.}} {\textsc{- Efforts de M\textsuperscript{me}
la duchesse d'Orléans pour me lier avec M. le duc du Maine.}} {\textsc{-
Situation de M\textsuperscript{me} de Saint-Simon, de la duchesse de
Lauzun et de moi, avec M. {[}le duc{]} et M\textsuperscript{me} la
duchesse du Maine.}} {\textsc{- Étrange aventure qui brouille
M\textsuperscript{me} du Maine avec la duchesse de Lauzun, et ses
suites.}} {\textsc{- Mariage du jeune duc de Brancas avec
M\textsuperscript{lle} de Moras.}} {\textsc{- Point d'étrennes au roi ni
du roi cette année.}}

~

Comme je me suis étendu en détail, sur mon audience du roi, pour le
faire mieux connaître par des faits et des choses particulières, aussi
en ajouterai-je une ici qui entre fort dans ce dessein, et que le duc de
Noailles, malgré ses réserves avec M. le duc d'Orléans, nous conta. Se
trouvant en ces mêmes jours en tiers entre lui et moi, dans le cabinet
de ce prince, la conversation se tourna sur M\textsuperscript{me} de
Maintenon. Je pense que son neveu voulut nous faire sentir son intime
situation avec elle, par ce fait qu'il nous raconta, et qui caractérise
bien le roi et le genre de crédit de ses plus intrinsèques. Il nous dit
que, encore qu'il fût vrai dans l'usage que M\textsuperscript{me} de
Maintenon pût tout sur son esprit, il ne l'était pas moins que ce
n'était presque jamais en droiture, et qu'elle n'était jamais sûre de
rien\,; que, pour réussir à ce qu'elle voulait, elle était
très-attentive à le faire proposer d'ailleurs, se réservait à l'appuyer
quand le roi lui en parlait, qui lui parlait toujours de tout, et avec
ce détour, qui dérobait au roi la connaissance de son désir, ne manquait
pas de l'obtenir, en sorte qu'il demeurait dans la parfaite ignorance
que les choses qui passaient ainsi venaient originairement d'elle, et
lui étaient portées par d'autres canaux. C'est ce qui la mettait en
besoin d'avoir des ministres dans son entière dépendance pour lui aider
à ce jeu, qu'elle pratiquait avec encore plus de précautions pour les
siens, à l'égard desquels le roi était en garde infinie, sans que sa
défiance eût d'autre effet qu'une circonvention plus cauteleuse. Il nous
le confirma par ce qui lui était arrivé, il n'y avait pas encore
longtemps.

Il avait eu en se mariant les survivances des gouvernements de
Roussillon, de son père, et de Berry, de son beau-père, mais ce dernier
à condition de le vendre dès qu'il lui serait tombé, et d'en placer le
prix comme partie de la dot de sa femme. Le cas arrivé, il ne put
trouver marchand. L'inquiétude d'en répondre sur son bien en cas de
mort, et que ce gouvernement fût donné gratuitement, le fit songer à un
brevet de retenue qui le tirât de cet embarras. Il en parla à
M\textsuperscript{me} de Maintenon qui goûta ses raisons, mais refusa
d'en parler au roi. Pressée de le faire, elle dit franchement au duc de
Noailles que ce qu'il voulait exiger d'elle était le véritable moyen de
gâter son affaire, mais qu'il fallait que lui-même demandât cette grâce
au roi, qu'il ne manquerait pas de le lui dire\,; qu'alors elle
appuierait bien, et que de cette façon elle répondait du succès, et il
l'eut de la sorte. Ce n'était pas ici le lieu de s'étendre en réflexions
qui pourront mieux se trouver dans la suite. M. le duc d'Orléans songea
en ces premiers jours à exécuter un projet qu'il m'avait confié dès sa
naissance et que j'avais fort approuvé, et ceci commencera à
caractériser ce prince par les faits. On a vu, en plus d'un endroit,
combien Besons lui était attaché, et combien il en avait tiré de
protection et de services, même pour son bâton de maréchal de France. Le
mérite et l'attachement de Besons l'avait également fait désirer à M. et
M\textsuperscript{me} la duchesse d'Orléans pour gouverneur de M. le duc
de Chartres, avant qu'il fût maréchal de France, et cette élévation le
leur augmentait encore beaucoup. Besons, pauvre, sans naissance, âgé,
marié tard et chargé de famille, d'ailleurs modeste et reconnaissant,
n'était pas en terme de lui rien refuser\,; il lui en parla, et Besons
lui répondit avec toute la sagesse et plus d'esprit qu'on n'en pouvait
attendre, laissant une si juste balance qu'il conserva toute sa liberté.
Aussitôt après, il consulta séparément le chancelier, dont il était
parent proche et ami, et moi.

Le chancelier, toujours peu prévenu pour M. le duc d'Orléans, et payé
pour l'être en faveur des officiers de la couronne, fut d'avis du refus.
Moi, au contraire, j'inclinai à l'acceptation, quoique en garde contre
mon penchant à l'intérêt de M. le duc d'Orléans, dans une affaire qui
exigeait de moi un conseil sincère à un homme qui se fiait en moi et qui
me le demandait. Je lui dis donc que cette place était en effet fort
au-dessous du rang où son mérite l'avait porté\,; que néanmoins il
devait considérer que le marquis de Chevrières, homme de qualité
distinguée (Mitte de Miollens), qui avait souvent commandé des corps en
chef, en qualité de lieutenant général, grade alors fort rare, qui avait
passé avec réputation par les premières ambassades, et chevalier du
Saint-Esprit, ce qui distinguait bien plus en ces temps-là, avait été
gouverneur du jeune prince de Condé, père du héros, choisi par Henri
IV\,; que si on objectait qu'alors ce prince était l'héritier de la
couronne, on répondait aussi qu'Henri IV était si bien en état d'avoir
des enfants qu'il en eut six ans après, que nous voyons sur le trône, et
dont M. le duc de Chartres est issu de si près\,; qu'il fallait s'avouer
que Chevrières valait bien de son temps nos nombreux maréchaux de France
d'aujourd'hui\,; que les ducs exerçaient maintenant des charges que les
simples maréchaux de France dédaignaient au commencement de ce règne,
témoin le maréchal d'Aumont, qui, du moment qu'il le fut\footnote{Le
  maréchal d'Aumont ne fut duc qu'en décembre 1665. (\emph{Note de
  Saint-Simon}.)}, n'exerça plus sa charge de capitaine des gardes, et
n'en reprit passagèrement la fonction, qu'il avait laissée à son fils de
treize ans, qu'à la prière de la reine mère, à l'occasion des
troubles\,; témoin MM. d'Estrades, Navailles et La Vieuville, ducs à
brevet ou maréchaux de France, et le second tous les deux, qui avaient
successivement été gouverneurs de M. le duc d'Orléans d'aujourd'hui\,;
qu'il ne fallait donc pas s'en tenir à l'ancien poids\,; qu'il avait une
nombreuse famille, peu de biens, une femme de mérite à qui cette place
en pouvait frayer d'autres pour soutenir sa famille après lui, que, tout
considéré, j'estimais que, le roi parlant, et non autrement, cette place
lui était désirable.

Besons, modeste à m'embarrasser, me dit franchement que le bâton de
maréchal de France ne lui avait point tourné la tête ni fait oublier ce
qu'il était né\,; qu'il avait déjà senti tout ce que je lui disais par
rapport à sa famille\,; qu'il se souvenait de tout ce qu'il devait à M.
le duc d'Orléans\,; que ce choix le devait flatter par l'estime et par
la confiance\,; qu'il m'avouait qu'il ne serait point fâché que le roi
l'y engageât, mais qu'il ne croyait pas aussi devoir rien accepter que
de sa main après l'honneur auquel il l'avait élevé, ce qui lui servirait
même d'excuse auprès de ses confrères s'ils le trouvaient mauvais,
auxquels encore il devait trop de considération, par l'honneur qu'il
avait d'être monté jusqu'à eux, pour ne pas devoir désirer de les
ménager avec toute l'attention possible. Il m'avoua aussi l'avis
contraire du chancelier, que je savais déjà du chancelier même, auquel,
malgré sa déférence, il ne me parut pas résolu de s'arrêter.

Les choses en cet état, il fut question d'en parler au roi, et
auparavant, d'en faire préparer les voies par M\textsuperscript{me} de
Maintenon et par les jésuites. Ceux-ci, attachés comme je l'ai dit à M.
le duc d'Orléans, ne s'y refusèrent pas. Mais, depuis que le P. Tellier
était en place, ils n'entraient en quoi que ce fût qu'après s'être bien
assurés contre tout soupçon de jansénisme. Tout ignorant, tout
militaire, tout homme du monde que fût Besons, il n'était pas net à leur
égard, parce qu'il avait élevé tous ses enfants chez lui, et les y
tenait encore sans en avoir mis aucun en leurs colléges, et que son
frère, l'archevêque de Bordeaux, n'était pas leur valet à tout faire,
quoique sans démêlé jamais avec eux, et même bien avec eux, d'une
doctrine qu'ils n'avaient pu reprendre, et dont le fort portait moins
sur la théologie que sur les matières temporelles et de juridiction du
clergé où il était fort capable, et s'était acquis de l'autorité par là
dans ses assemblées, aussi liant d'ailleurs que son frère l'était peu.
Les perquisitions se trouvèrent telles que le P. Tellier se prêta à tout
ce qu'on voulut. Mais ces menées ne purent être si secrètes, parce
qu'elles durèrent quelque temps, que par un peu de lenteur et
d'indiscrétion de M. le duc d'Orléans, elles ne fussent découvertes, et
l'affaire ébruitée avant d'être entamée avec le roi.

Feu M. le Prince et M. le Duc avaient sondé diverses personnes qui
passaient pour gens de qualité, et d'autres qui s'élevaient à la guerre,
pour l'emploi de gouverneur du jeune duc d'Enghien, quoique eux-mêmes ni
M. le Prince le héros n'en eussent point eu de ce genre, mais de simples
gentilshommes de leurs maisons. Éconduits de tous, ils s'étaient vus
réduits à publier qu'ils voulaient être eux-mêmes les gouverneurs de ce
jeune prince, et mettre sous eux auprès de lui un de leurs gentilshommes
sans titre, ce qu'ils exécutèrent en effet. Ils y en mirent un sage,
sensé, connaissant bien le monde, fort honnête homme et d'une grande
valeur, qui s'appelait La Noue. Ce fut dommage que ce gouverneur ne fût
pas si heureux en pupille que le pupille le fut vainement en gouverneur.
M. le Duc et M\textsuperscript{me} la Duchesse, alarmés d'une nouvelle
et si grande distinction sur eux, les maréchaux de France, jaloux de
leur office, firent un mouvement qui prévint le roi, lequel, journalier
à l'égard de ces derniers, tantôt les élevant au delà de leur juste
portée, tantôt les rabaissant trop, se trouva en tour de les favoriser,
ou plutôt enclin à conserver l'égalité entre deux princes du sang, ses
petits-fils par ses filles bâtardes, qualité qui l'emportait de bien
loin chez lui sur celle de petit-neveu.

Dans une situation si équivoque, M. le duc d'Orléans parla au roi avec
sa négligence trop ordinaire, et il trouva de la résistance qu'il crut
pouvoir vaincre. Si en cet instant il eût aposté Besons à la porte du
cabinet, et qu'il l'y eût fait entrer, ce qui était aisé, je ne crois
pas que le roi eût tenu à l'empressement de l'un, et à la facilité de
l'autre, par la façon même dont il avait résisté. Mais cette précaution
avait été négligée, et M. le duc d'Orléans y ajouta la tranquillité
d'attendre que le roi trouvât Besons et qu'il lui parlât. Le maréchal,
avec qui rien n'était concerté sinon la chose même, était à Paris où M.
le duc d'Orléans ne lui manda rien, quelque chose que je fisse,
tellement qu'y étant allé faire un tour plusieurs jours après, j'allai
chez Besons, lui dis ce qui s'était passé, et le pressai d'aller à
Versailles. Il y fut aussitôt, et dès que le roi l'aperçut, il le fit
entrer dans son cabinet. Là, il lui rendit en conversation, même froide,
ce que M. le duc d'Orléans lui avait dit, y ajouta des propos gracieux
pour le maréchal, mais lui dit bien net qu'il ne voulait pas mortifier
les maréchaux de France, qu'il ne lui commandait rien, et qu'il le
laissait en sa pleine liberté.

Besons, fort surpris, répondit avec une modestie soumise tout ce qu'il
fallait pour s'attirer au moins quelque chose qui sentît un ordre\,;
mais voyant que le roi se rabattait toujours au même point, et qu'il
ajouta de plus qu'il s'abstenait encore de commander par rapport aux
princes du sang, le sage Besons sentit de reste que le roi ne souhaitait
pas qu'il acceptât\,; qu'acceptant de la sorte il s'attirerait sans
garantie et les princes du sang et les maréchaux de France, et se tira
d'affaires à son regret en disant au roi qu'en tout temps, et plus
encore dans l'office auquel il l'avait élevé, il ne pouvait rien
accepter que de Sa Majesté même. Aussitôt après il rendit compte de
cette conversation à M. le duc d'Orléans qui, n'ayant cru d'obstacle
bien véritable que le dessein que le roi pouvait former de se servir de
Besons à la tête de ses armées, croyait avoir tout aplani parce qu'il
avait dit au roi qu'il ne prétendait point que son fils y fût un
obstacle, et qu'il se contenterait des hivers tant qu'il lui plairait
d'employer le maréchal.

M. {[}le duc{]} et M\textsuperscript{me} la duchesse d'Orléans se
trouvèrent également surpris et mortifiés de se voir éconduits d'une
espérance qui avait percé et qui les avait fort flattés. Le roi,
embarrassé avec eux, allégua les maréchaux de France, et se garda bien
de parler des princes du sang, pour n'augmenter pas la haine qui n'était
déjà que trop allumée et trop ouvertement, et pour adoucir la chose, il
s'excusa sur ce qu'il n'y avait point d'exemple. Les réponses à cela
étaient sans nombre\,; et de plus, il y en avait un précis, récent et
domestique. La maréchale de Grancey, après avoir été gouvernante de la
soeur de M. le duc d'Orléans, duchesse de Lorraine, l'avait été auprès
des filles de M. le duc d'Orléans. Elle était morte dans cet emploi, et
M\textsuperscript{me} de Maré sa fille, qui l'était encore, avait été sa
survivancière. Il ne vint dans l'esprit de M. et de
M\textsuperscript{me} la duchesse d'Orléans, ni cette réponse si
décisive, ni aucune autre, et ils demeurèrent courts. Leur parti fut de
ne point donner de gouverneur à M. le duc de Chartres qui n'avait pas
encore six ans et demi. Les princes du sang et les maréchaux de France
en rirent dans leurs barbes assez haut\,; mais le maréchal de Villeroy
ayant su par sa belle-fille que M. le duc et M\textsuperscript{me} la
duchesse {[}d'Orléans{]} se plaignirent fort de ce qu'il s'en était
beaucoup remué, désavoua de s'être mêlé de rien là-dessus, et la chargea
de leur dire qu'ayant l'honneur d'être duc et pair et maréchal de France
aussi, mais d'un temps où on les faisait avec plus de choix, il n'était
point amoureux d'un office qu'il partageait avec les Montesquiou et une
foule de semblables dont trop peu lui importait ce qui arrivait d'eux
pour y faire aucune attention. C'était cacher la bassesse de courtisan
sous une ridicule rodomontade, après l'usage qu'il avait fait de son
bâton si fatal à la France, et dont il était encore alors en disgrâce.
Jamais homme n'en fut plus follement entêté que celui-là, et j'ai
remarqué que ceux qui l'avaient le moins mérité étaient toujours ceux à
qui il avait le plus tourné la tête. On le verra de celui-ci dans la
suite.

La mort de M. le Prince avait mis un grand trouble dans sa famille, dont
il est temps de parler par les grandes et longues suites que ces
divisions ont eues. Il avait fait un testament très-avantageux à M. le
Duc, son fils unique, duquel ses filles crurent avoir de grandes raisons
de se plaindre, dont la discussion est inutile ici.
M\textsuperscript{me} la Princesse, à qui il restait des biens immenses,
même à disposer, fit tout ce qu'elle put en bonne mère commune pour
mettre la paix dans sa famille, mais avec peu d'esprit et de force. Elle
craignait tous ses enfants, et n'osa jamais parler en mère qui a de quoi
donner et ôter, et qui en proposant raison veut être obéie. Le roi y
voulut bien entrer et n'eut pas plus de succès, par la nature des choses
qui fournissait aux parties des défenses apparentes dont aucune ne
voulut se relâcher. Des compliments aux froideurs, des froideurs aux
aigreurs, il y eut peu d'intervalle, et chacun se disposa vigoureusement
à plaider. Les vrais tenants étaient, de chaque côté, M. le Duc et
M\textsuperscript{me} la princesse de Conti, l'aînée de ses soeurs. M.
et M\textsuperscript{me} du Maine gardaient des mesures, mais se
tenaient invinciblement attachés à M\textsuperscript{me} la princesse de
Conti. M\textsuperscript{lle} d'Enghien, dont les droits se trouvaient
conservés par les procédures de ses soeurs, demeura, sans y renoncer,
neutre du reste auprès de M\textsuperscript{me} la Princesse. Le temps
avait coulé depuis la mort de M. le Prince jusqu'à celui-ci en projets
d'accommodement, en allées et venues, en consultations, puis en
assignations et en délais, au bout desquels vint le moment fatal de
plaider tout de bon. Chacun chercha des sollicitations puissantes, et le
duc du Maine, avec toutes ses mesures, non moins soigneusement que les
autres.

M. le Duc, qui redoutait son crédit, se proposa de faire effort de
supériorité de naissance et d'autorité, et contre sa coutume s'avisa de
donner, dix ou douze jours avant la première audience, un grand souper à
Paris à beaucoup de gens de la cour. Dans la chaleur du repas, il but à
eux et voulut qu'ils bussent à lui. Il s'humanisa en compliments
flatteurs qui n'étaient guère de son style\,; et tout de suite leur dit
qu'il avait une telle confiance en leur amitié qu'il se flattait qu'ils
ne l'abandonneraient pas au palais, et qu'ils ne lui refuseraient pas
leur parole de l'y accompagner à toutes les audiences dont il avait
résolu de ne manquer aucune, et de distinguer par ceux qui s'y
trouveraient avec lui ses véritables amis, par ceux qui n'y viendraient
pas les gens qui ne seraient pas ses amis, et par ceux qui y
accompagneraient ses parties ses ennemis. La surprise et l'embarras d'un
compliment si net et si peu attendu, et qui était un enrôlement dans
toutes les formes, produisit un silence profond. Les conviés se
regardèrent, chacun d'eux attendait que quelqu'un prit la parole, aucun
ne l'osa hasarder. M. le Duc, étonné à son tour d'un si éloquent
silence, le laissa durer un peu, puis le rompit par de nouveaux
empressements qui arrachèrent enfin un engagement de toute la compagnie,
qu'elle ne pouvait plus refuser sans lui faire un véritable affront.
Comme la force seule l'avait extorqué, aussi parut-il fort pesant à ceux
qui s'étaient trouvés dans cette nasse.

Personne n'aimait M. le Duc, personne ne voulait s'attirer
M\textsuperscript{me}s ses soeurs et moins M. du Maine encore. Non
content de ce coup de filet d'une nouvelle adresse, M. le Duc se mit
ouvertement à faire des recrues pour l'accompagner, avec des manières
que sa férocité rendait redoutables et qui réveillèrent ses parties. La
princesse de Conti aboyait assez vainement\,; mais le duc et la duchesse
du Maine ramassèrent plus de gens avec politesse et souplesse, et se
surent avantageusement servir avec ménagement de l'opinion commune que
l'affection tacite du roi était de leur côté. Ces mesures de part et
d'autre firent un grand bruit et jetèrent la cour dans un tel embarras,
qu'il n'y eut plus personne qui se pût flatter de pouvoir demeurer
neutre sans offenser les deux partis, ni d'en prendre un sans s'attirer
cruellement l'autre. À la fin le roi, jugeant avec raison que les suites
de tout cela ne pouvaient être bonnes, défendit tout d'un coup aux deux
parties tout engagement au palais.

Le jour même que cette défense fut faite, M\textsuperscript{me} la
duchesse d'Orléans, avec qui je fus assez longtemps seul, me dit que M.
du Maine était en peine de quel parti je prendrais en cette occasion\,;
qu'elle me disait franchement qu'étant maintenant fort ralliée à lui,
elle serait fort touchée que je voulusse être du sien\,; qu'elle ne me
dissimulait point que M. du Maine, qui savait la liaison que j'avais
prise avec elle, l'avait priée de m'en parler\,; et tout de suite, sans
me donner le temps de répondre, elle me fit des compliments infinis de
sa part pour moi et pour M\textsuperscript{me} de Saint-Simon, et
d'autres pareils encore à la duchesse du Maine\,; que tous deux ne se
consolaient point que M\textsuperscript{me} de Saint-Simon, qu'ils
estimaient et qu'ils honoraient infiniment, ce fut son terme, se fût
éloignée d'eux, quoiqu'ils eussent fait tout ce qui avait pu dépendre
d'eux, lors de l'affaire de la duchesse de Lauzun arrivée il y avait
quatre ou cinq ans, pour se la conserver personnellement par toutes les
distinctions et les soins possibles\,; et qu'ils espéraient au moins
que, s'ils ne pouvaient la voir aussi souvent qu'ils avaient
continuellement marqué, et qu'ils ne se lasseraient point de marquer
qu'ils le désiraient, nous serions persuadés de leur désir et ne
voudrions pas nous engager contre eux.

Je répondis à M\textsuperscript{me} la duchesse d'Orléans, après force
compliments, que je lui parlerais avec la même franchise\,; que j'avais
résolu, avant que le roi parlât comme il venait de faire, de tâcher par
tous moyens de conserver la neutralité, persuadé que dans ces sortes de
choix on obligerait peu ceux pour qui on prenait parti, et qu'on se
rendait irréconciliables ceux contre qui on se déclarait\,; mais
qu'advenant impossibilité de demeurer neutre, je ne balancerais pas à
suivre ouvertement le parti de M. du Maine, encore que je n'eusse aucun
commerce avec lui\,; qu'il ne tiendrait qu'à moi de m'en faire un mérite
auprès d'elle, et qu'en effet je serais ravi de me déclarer suivant son
inclination, mais que, pour lui parler avec toute franchise, j'avais un
motif plus fort et plus pressant qui était la manière pleine d'égards,
de mesure et de considération dont M. et M\textsuperscript{me} du Maine
en avaient usé pour moi dans l'affaire de M\textsuperscript{me} de
Lussan, affaire qui avait fait éclater si étrangement contre moi M. le
Duc et M\textsuperscript{me} la Duchesse. Que je n'oubliais point la
différence de ce procédé, et que je la suppliais d'assurer M. et
M\textsuperscript{me} du Maine, si liés alors avec M. le Duc, et qui
avait toujours aimé et protégé M\textsuperscript{me} de Lussan, jusqu'à
avoir marié sa fille, que je leur en témaignerais le souvenir en toute
occasion.

M\textsuperscript{me} la duchesse d'Orléans s'épanouit fort à cette
réponse, à laquelle il me parut qu'elle ne s'attendait pas. Elle me
parla beaucoup de l'estime et de la considération de M. du Maine pour
moi, et surtout de lui et de M\textsuperscript{me} du Maine pour
M\textsuperscript{me} de Saint-Simon, mais avec les expressions les plus
chargées. Elle me demanda pourquoi M\textsuperscript{me} de Saint-Simon
s'était si fort retirée de M\textsuperscript{me} du Maine, avec un
empressement qui me parut d'autant plus de commission qu'elle me pressa
outre mesure de l'en faire rapprocher, et avec des avances si formelles
du mari et de la femme que j'en fus surpris et embarrassé. Je lui dis
qu'après l'affaire de la duchesse de Lauzun, il eût été difficile et
même peu séant dans le monde que sa soeur, avec qui elle était si
intimement unie, eût gardé une autre conduite. Elle me pressa sur tous
les pas qu'ils avaient faits l'un et l'autre vers M\textsuperscript{me}
de Saint-Simon, dont je ne pus disconvenir ni me tirer sans une peine
extrême d'un renouement, que je sentis de reste qu'elle avait charge et
grand désir de procurer, sur lequel je restai honnêtement ferme à n'y
point entendre et à en demeurer, M\textsuperscript{me} de Saint-Simon et
moi, dans les termes où nous en étions avec M. et M\textsuperscript{me}
du Maine, mais avec tous les compliments dont je pus m'aviser.

Il s'est depuis passé tant de choses fortes entre M. du Maine et moi, et
à tant de diverses reprises, et du vivant du roi et après, que je
craindrai moins ici la répétition de quelques traits qui se peuvent
trouver ci-devant, que de ne m'étendre pas suffisamment sur un chapitre
important pour les suites à être bien expliqué. Il faut donc savoir que
M\textsuperscript{me} la duchesse du Maine demeura très-obscure à la
cour les premières années de son mariage. Elle y passait sa vie dans sa
chambre parmi les livres et les savants, par une folle malice de M. le
Prince, qui lui avait fait une peur extrême de la jalousie de M. du
Maine et de son humeur sauvage, en même temps qu'il lui avait fait
accroire que M\textsuperscript{me} sa femme était très-particulière,
adonnée à ce genre de vie, d'étude et qu'il la désespérerait s'il lui
proposait d'en changer. Le temps qui découvre tout, et l'ennui de cette
vie qui devint insupportable à M\textsuperscript{me} du Maine, firent
apercevoir au mari et à la femme qu'ils se désolaient de solitude, l'un
pour l'autre, et que cette étrange et ridicule tromperie était l'ouvrage
de l'extravagante malignité de M. le Prince.

Revenus donc tous deux de leur erreur, et dans la plus grande union du
monde, M\textsuperscript{me} du Maine ne songea plus qu'à se dédommager
du temps perdu, et M. du Maine qu'à lui en fournir tous les moyens
possibles. Aussitôt après, ce ne fut plus chez elle que divertissements
galants, bals singuliers, fêtes et spectacles. Pour décorer sa maison,
elle attira chez elle ce qu'elle put de meilleure compagnie. La duchesse
de Lauzun en fut particulièrement recherchée, et M. du Maine en fit
toutes les avances avec toutes sortes d'empressement. Ils avaient eu, M.
de Lauzun et lui, plus d'une affaire ensemble. M. de Lauzun comptait
toujours que tant de grandes terres qu'il lui avait cédées de
Mademoiselle, pour sortir de Pignerol, l'engageraient au moins à se
servir de son crédit auprès du roi pour l'y remettre, et chercher à le
dédommager. D'ailleurs il était trop courtisan pour ne pas donner dans
ces avances, comme dans une sorte de retour de fortune\,; ainsi
M\textsuperscript{me} de Lauzun fut bientôt de tout à Sceaux, que M. du
Maine venait d'acheter, et qui fut une occasion de redoubler les fêtes
et les plaisirs dans un lieu qui y était si propre, et où
M\textsuperscript{me} du Maine, qui voulait vivre pour elle, se mit à
passer tous les étés, quoique M. du Maine, dont l'abandon aveugle pour
elle fut toujours au comble, n'y osât coucher que très-rarement, par la
prodigieuse assiduité que le roi exigeait de ses enfants naturels,
encore plus que des autres. Le roi, allant et venant de Fontainebleau, y
couchait, et quelquefois deux nuits, et les dames les plus distinguées,
mais en très-petit nombre, de la société de M\textsuperscript{me} du
Maine étaient priées de lui venir aider à faire les honneurs. Cette
liaison de M\textsuperscript{me} de Lauzun y attira
M\textsuperscript{me} de Saint-Simon, qui reçut d'eux les plus grandes
avances, et les empressements les plus marqués\,; et ce fut en ces
passages de Sceaux où M\textsuperscript{me} de Saint-Simon commença à
s'apercevoir des bontés particulières de M\textsuperscript{me} la
duchesse de Bourgogne, et à entrer dans sa familiarité. M. et
M\textsuperscript{me} du Maine ne se bornèrent pas à
M\textsuperscript{me} de Saint-Simon\,; après l'avoir engagée à
plusieurs séjours à Sceaux, ils commencèrent à me faire mille avances, à
moi qui ne les voyais jamais. Ma belle-soeur en fut chargée longtemps.
Lassés de ce que cela ne rendait point, ils pressèrent
M\textsuperscript{me} de Saint-Simon de m'amener à Sceaux. Je m'excusai
longtemps, toujours sans les voir, jusqu'à ce que, les rencontrant par
hasard comme ils montaient tous deux en carrosse à Versailles, sans que
je me pusse détourner, tous deux vinrent à moi, et par leurs reproches
et leurs empressements m'embarrassèrent à l'excès.

Tant de si singulières avances, tant et de si surprenante opiniâtreté
pour s'apprivoiser un homme de nulle ressource pour aucuns de leurs
plaisirs, et de moindre importance encore par le peu de figure
extérieure que je faisais alors dans le monde, me devint enfin suspecte.
J'avais pris les premières avances pour politesse pour ma femme et ma
belle-soeur\,; mais un acharnement semblable, au lieu de la froideur et
du rebut que méritaient mes refuites intarissables, et toujours sans les
voir jamais, me sembla l'effet d'un dessein formé. J'avais toujours
appréhendé de m'initier avec eux, par la crainte du duc du Maine, dont
la réputation n'était pas heureuse, et non moins encore par son rang qui
me donnait un éloignement involontaire que je ne pouvais surmonter. Je
me disais que me forcer pour céder à tant d'avances, et pour vivre en y
cédant avec des gens que je ne pourrais sincèrement aimer, était contre
la probité non moins que contre ma nature. Poussé à bout par leur
constance inouïe, je craignis qu'ils ne cherchassent à me lier à eux
pour découvrir mes sentiments sur bien des choses, et à force de
caresses me mettre dans de pénibles entraves entre l'amitié et le rang,
dans la pensée que les temps ne sont pas toujours les mêmes. Ces
réflexions me déterminérent à ne me laisser point entamer, et à en
demeurer où j'en étais. Les détails jusqu'où je fus poussé très-vivement
et très-longuement sembleraient incroyables à qui a vu ce qu'était M. du
Maine dans ces temps-là, et combien ce qui paraissait de plus
considérable s'empressait inutilement auprès de lui. J'en étais là avec
l'un et l'autre, sans les avoir jamais vu chez eux qu'en ces occasions
rares de compliments où toute la cour y allait par devoirs et par
instants, lors d'une aventure qu'il est nécessaire de rapporter.

J'ai dit ailleurs que, la liste de Marly faite par le roi pour chaque
voyage, il la montrait la veille après son souper dans son cabinet aux
princesses, qui, par rang entre elles, choisissaient les dames qu'elles
voulaient mener, et les envoyaient avertir à la sortie du cabinet, sur
le minuit. Elles prenaient toujours les mêmes. M\textsuperscript{me} de
Saint-Simon, par exemple, allait toujours avec M\textsuperscript{me} la
duchesse d'Orléans\,; M\textsuperscript{me} de Lauzun avec
M\textsuperscript{me} du Maine\,; et au retour à Versailles, les mêmes
revenaient avec elles. Il arriva deux ou trois fois que, les jours qu'on
retournait à Versailles, M\textsuperscript{me} la duchesse de Bourgogne
voulut jouer dans le salon, retint M\textsuperscript{me} de Lauzun qui
était assez dans le gros jeu, et la ramenait à Versailles, parce que
tout le monde était parti avant la fin de son jeu. M\textsuperscript{me}
du Maine, gâtée par la complaisance sans bornes de M. du Maine, était
devenue une manière de divinité fort capricieuse, qui se croyait
tellement tout dû qu'elle ne croyait plus rien devoir à personne. Le
fait était que sa violence était si extrême pour tout ce qu'elle
voulait, que, dans la frayeur continuelle que la tête ne lui tournât, M.
du Maine s'était exécuté sur ses biens et sur toute bienséance. Il se
voyait ruiner en théâtres et en fêtes sans oser dire un seul mot, il en
faisait les honneurs en domestique principal de la maison\,; et il
applaudissait en apparence à ce qui le faisait rougir au dehors, et le
désespérait au dedans. Ainsi, M\textsuperscript{me} du Maine trouva
mauvais qu'ayant amené M\textsuperscript{me} de Lauzun à Marly, elle
s'en retournât avec une autre, quoique cette autre fût
M\textsuperscript{me} la duchesse de Bourgogne. Elle s'en plaignit à la
duchesse de Lauzun, sur le ton de l'amitié qui pourtant laissait sentir
celui du manquement prétendu. M. de Lauzun, qui connaissoit son empire
sur son mari avec qui il ne voulait pas se brouiller, et le peu de
mesure de cette princesse, en eut peur. M\textsuperscript{me} de Lauzun
l'appréhenda de même, tellement qu'elle évita, tant qu'elle put, par
fuite ou par excuse, de rester dans la suite à jouer à Marly avec
M\textsuperscript{me} la duchesse de Bourgogne les jours qu'on
retournait à Versailles.

Il arriva qu'un de ces jours-là M\textsuperscript{me} la duchesse de
Bourgne la voulut si absolument retenir, et s'y prit de si bonne heure
qu'elle ne voulut se payer d'aucune excuse, ni entrer dans l'embarras où
elle allait jeter la duchesse de Lauzun, quoi qu'elle pût lui
représenter. Ma belle-soeur n'eut plus à répliquer, ni d'autre parti à
prendre que d'aller le dire à M\textsuperscript{me} du Maine, Le
compliment fut d'abord fraîchement reçu, incontinent après la marée
monta, et voilà la duchesse du Maine aux reproches d'amitié d'une part,
de manéges de l'autre pour faire sa cour à M\textsuperscript{me} la
duchesse de Bourgogne en lui manquant à elle de respect, à lui dire
qu'elle pouvait désormais chercher qui la mènerait à Marly, si tant
était qu'elle y revînt, et à rompre avec elle en lui tournant le dos de
la manière la plus impérieuse et la plus scandaleuse, ou plutôt la plus
folle. Quelque préparée que ma belle-soeur pût être à être mal reçue,
une femme de sa sorte ne pouvait imaginer d'être exposée à une pareille
sortie. La colère lui ôta la parole et lui fournit des larmes.

En cet état elle revint dans le salon, où elle rendit à
M\textsuperscript{me} la duchesse de Bourgogne tout ce qui lui venait
d'arriver, sagement et modestement, mais aussi sans en oublier une
parole. M\textsuperscript{me} la duchesse de Bourgogne, qui n'aimait pas
la duchesse du Maine, de qui elle recevait peu de devoirs, et par qui,
en cette occasion, elle se sentit peu ménagée, prit l'injure comme faite
à elle-même, se lâcha sur M\textsuperscript{me} du Maine, assura la
duchesse de Lauzun qu'elle en parlerait au roi, et, piquée du reproche
sur Marly, lui dit qu'on verrait si elle y viendrait moins, et lui
promit de l'y mener toujours avec elle\,; et en effet elle n'en manqua
plus de voyages, et toujours avec M\textsuperscript{me} la duchesse de
Bourgogne. L'éclat fut grand. Le soir même M\textsuperscript{me} la
duchesse de Bourgogne parla au roi et à M\textsuperscript{me} de
Maintenon. Le roi lava la tête à M. du Maine sur sa femme, et loua fort
M\textsuperscript{me} de Lauzun. Elle la fut aussi beaucoup de
M\textsuperscript{me} de Maintenon, peu contente d'ailleurs de
M\textsuperscript{me} du Maine, laquelle mal avec M\textsuperscript{me}
la Duchesse, quoique fort liée alors avec M. le Duc, mal encore avec
M\textsuperscript{me} la princesse de Conti, et peu aimée d'ailleurs, se
trouva abandonnée.

Dès le lendemain du retour à Versailles, elle envoya
M\textsuperscript{me} de Chambonas, sa dame d'honneur, chez
M\textsuperscript{me} de Saint-Simon la prier de vouloir bien aller chez
elle, prétextant une incommodité qui l'empêchait de sortir. Cela ne put
se refuser. Dès qu'elle la vit entrer, elle l'emmena dans son cabinet,
où le tête-à-tête dura plus de deux heures. Après la préface la plus
polie, elle lui conta toute l'affaire, mais rhabillée et ajustée pour la
rendre moins intolérable, se condamna en tout et partout, s'excusa
pourtant sur ce que, se croyant blessée dans l'amitié par une amie
qu'elle aimait tendrement, elle ne s'était plus connue elle-même, ni
celle à qui elle parlait, ni la force de ce qu'elle disait, n'oublia
rien pour essayer de raccommoder les choses, sur tout et en toutes les
sortes combla M\textsuperscript{me} de Saint-Simon, la conjura avec les
termes les plus forts et même au delà, que ce malheur ne la refroidît
point pour elle, à quoi elle ajouta tout ce qu'infiniment d'éloquence et
d'esprit peut mettre à la bouche de qui sent tout son tort, et de qui
voit qu'il tombe en entier et très-pesamment sur elle.
M\textsuperscript{me} de Saint-Simon, grave et mesurée, paya de
compliments, ne voulut plus être d'aucune de ses parties, et ne la vit
depuis que très-rarement. Toute la cour s'éleva fort contre
M\textsuperscript{me} du Maine. M. du Maine alla chez le duc de Lauzun,
le trouva, passa ensuite chez M\textsuperscript{me} de Lauzun, y
retourna encore une autre fois, et n'oublia rien de tout ce qu'il
pouvait dire et faire. M\textsuperscript{me} de Lauzun, pour qui il
affecta toujours depuis les plus grands égards, ne revit plus
M\textsuperscript{me} du Maine. Très-longtemps après, elle y fut un
instant à une occasion publique de compliments de toute la cour, et ne
l'a pas revue autrement, encore fut-ce par une espèce de négociation
avec son mari qui le voulut en bas courtisan. Outre que cette aventure
tourna tout à l'avantage de ma belle-soeur, je trouvai que j'y gagnais
beaucoup par la délivrance qu'elle me procura de tout ce à quoi je ne
voulais point entendre. Les égards les plus affectés de M. et de
M\textsuperscript{me} du Maine ne laissèrent pas de continuer à être
extrêmement marqués pour nous, et c'est où nous en étions avec eux lors
de cette conversation de M\textsuperscript{me} la duchesse d'Orléans
avec moi sur le procès de la succession de M. le Prince.

M\textsuperscript{me} du Maine venait de faire l'étrange mariage d'une
créature de rien qui s'était fourrée à Sceaux, je ne sais par où, qui
était assez jolie, mais {[}avec{]} de l'esprit, de la flatterie et de
l'intrigue au dernier point. Elle en avait fait sa favorite. Elle
s'appelait M\textsuperscript{lle} de Moras, et son nom était Fremyn. Son
père, qui avait amassé du bien, s'était recrépi d'une charge de
président à mortier au parlement de Metz. Sa mère, fille de Cadeau,
marchand de drap à Paris, avait un frère conseiller au parlement.
M\textsuperscript{me} du Maine fit accroire au fils du duc de Brancas
qu'il aurait monts et merveilles de ce mariage, tenta le père par de
l'argent, qui au lieu de donner du bien à son fils, reçut gros pour
faire ce beau mariage. Le rare fut que la plus grande partie de la dot
consista en meules de moulins à vendre. Malgré cela, le mariage se fit
chez M\textsuperscript{me} du Maine, qui présenta cette noble duchesse
les premiers jours de cette année.

Le roi ne donna point cette année les étrennes que sa famille recevait
de lui tous les ans\,; et les quarante mille pistoles qu'il prenait pour
les siennes, il les fit distribuer pour les besoins des frontières de
Flandre, ce qui n'était pas encore arrivé\,; aussi toutes sortes de
manquements étaient devenus extrêmes.

\hypertarget{chapitre-v.}{%
\chapter{CHAPITRE V.}\label{chapitre-v.}}

1710

~

{\textsc{Spectacle des maréchaux de Boufflers, Harconrt et Villars.}}
{\textsc{- Éclat du maréchal de Boufflers sur les lettres de pairie de
Villars.}} {\textsc{- Villars fait défendre à Harcourt de se faire
recevoir pair avant loi.}} {\textsc{- Harcourt tombe en apoplexie légère
et va aux eaux.}} {\textsc{- Ambition, manéges, maladie du maréchal
d'Huxelles.}} {\textsc{- Du Bourg fait commandant d'Alsace.}} {\textsc{-
Retour de Rome de l'abbé de Polignac.}} {\textsc{- Secret étrange et
curieux aveu sur lui du duc de Beauvilliers à moi.}} {\textsc{- Maréchal
d'Huxelles et abbé de Polignac plénipotentiaires pour la paix à
Gertruydemberg.}} {\textsc{- Fausseté du maréchal.}} {\textsc{-
Indécence basse sur le maréchal d'Huxelles, plus grande sur l'abbé de
Polignac.}} {\textsc{- Protecteurs des couronnes\,; explication de ce
nom superbe.}} {\textsc{- Cardinal Ottoboni fait peu à propos protecteur
de France\,; ce qui fait rompre Venise avec le roi.}} {\textsc{- Retour
de l'abbé de Pomponne.}} {\textsc{- Caractère d'Ottoboni.}} {\textsc{-
Imposture des Chavignard, dits Chavigny, et ce qu'ils sont devenus.}}
{\textsc{- Naissance du roi Louis XV.}} {\textsc{- Mariage du duc de
Luynes avec M\textsuperscript{lle} de Neuchâtel.}} {\textsc{- Mariage du
duc de Louvigny avec la fille unique du duc d'Humières.}} {\textsc{-
Mariage de Broglio avec une fille de Voysin.}} {\textsc{- Mariage de
Gacé avec la fille du maréchal de Châteaurenauld\,; et a le gouvernement
de son père, sur sa démission.}} {\textsc{- Le duc de Beauvilliers donne
sa charge de premier gentilhomme de la chambre au duc de Mortemart, son
gendre.}}

~

Les maréchaux de Boufflers, d'Harcourt et de Villars furent une partie
de cet hiver en spectacle au monde\,: le premier en exemple du peu de
compte que les rois et leurs ministres tiennent de la vertu et des
services qui ont passé la mesure des récompenses\,; le second attendu
comme l'oracle et le seul sage, appuyé de M\textsuperscript{me} de
Maintenon et de Voysin, couchait en joue les autres ministres pour les
renverser, et ne pouvait plus souffrir de délais pour entrer au conseil
dont il avait si souvent pensé forcer la porte. Il tenait tout le monde
en expectation, et se présentait avec un poids et une autorité qui, avec
tout son esprit, ne s'élaignaient pas de l'audace, quoique applaudi par
le gros de la cour et du monde. Le troisième, dont l'incomparable
fortune avait trouvé les plus singulières ressources pour soi dans la
funeste perte d'une bataille follement donnée, et plus extravagamment
rangée, triomphait du réparateur de ses torts avec la dernière
effronterie, dans l'appartement et les meubles même du prince de Conti
et de la princesse sa mère qui en fut piquée au vif, et M. le Duc aussi
quoique brouillé avec elle, sans que l'orgueil des princes du sang, si
haut porté, osât répliquer une seule parole aux volontés du roi. Qu'eût
dit le prince de Conti grand-père, et le vieux Villars, qui avec raison
se crut au comble de l'honneur et de la fortune quand il se vit son
écuyer, s'ils avaient pu voir la belle-fille et le petit-fils de ce
prince délogés malgré eux pour le fils de Villars, et n'oser ne lui pas
laisser leurs meubles\,?

Là ce fils de la fortune reçut la foule de la cour précisément avec
bonté, et il se peut dire qu'il y tint la sienne\,: jeux continuels,
fêtes, festins, très-souvent la musique du roi les soirs. Le héros
romanesque en soutenait pleinement le personnage. Il ne parlait que par
tirades de pièces de théâtre, et tenait des propos si surprenants qu'il
en embarrassait souvent sa nombreuse compagnie. Ses saillies étaient
continuelles\,; il ne se contraignait d'aucune. Le lit de repos de
dessus lequel il dominait les assistants semblait le théâtre d'un
Tabarin. M\textsuperscript{me} de Maintenon l'allait voir souvent en des
heures particulières. Un jour qu'elle y trouva son fils qui avait lors
huit ans et qu'elle le caressa, le maréchal lui dit qu'à la fin ses
bontés le gâteraient, et prenant un air enjoué qui lui était ordinaire,
ajouta que «\,les héros s'accoutumaient facilement aux bontés des
grandes reines.\,» Cent escapades aussi fortes, mais en autres genres,
mille propos sur la guerre, sur la paix, sur le gouvernement, sur
soi-même à faire trembler, passèrent pour des gaietés et des
gentillesses agréables. En un mot les yeux communs le regardaient comme
un fou échappé de sa cage, tandis que ceux de qui tout dépendait le
considéraient comme l'unique ressource qui n'avait que de légères
imperfections. Voysin portait souvent le portefeuille chez lui,
Desmarets aussi, séparément et quelquefois ensemble. Rien ne lui fut
refusé du personnage de dictateur. Il décidait des projets, des
arrangements. L'oubli et l'avancement des hommes furent dans ses mains.
Ce radieux état pourtant ne l'empêcha pas de songer à ses lettres de
pairie.

Le président de Maisons, son beau-frère, les lui dressa, et il y mit
tout ce qu'il voulut sur ses services. Il eut l'audace d'y faire insérer
que, sans sa blessure, la bataille de Malplaquet était gagnée, et
diverses autres choses à sa louange qui flétrissaient également la
vérité et la gloire du maréchal de Boufflers. Pontchartrain, à qui elles
furent portées pour les expédier, sursit, et en avertit Boufflers qui,
blessé jusqu'au fond de l'âme, devint furieux. Il tomba sur Villars
publiquement jusqu'à l'outrage\,; il en parla à tout le monde et aux
ministres. Cet homme si sage, si mesuré, si craintif à l'égard du roi,
ne se posséda plus. Il déclara tout haut à qui voulut l'entendre qu'il
s'en plaindrait au roi, et que, s'il n'en avait pas justice, il était
résolu de la demander en plein parlement, de s'adresser aux pairs, de
s'opposer aux lettres de Villars et de plaider lui-même sa cause devant
les pairs et tout le parlement assemblé. Il y avait longues années que
propos si hardi n'avait frappé aucune oreille. Aussi fit-il un étrange
fracas. Il fut tel que le roi n'osa refuser à un seigneur si utilement
illustre la justice qu'il lui demanda si haut. Villars épouvanté,
quoique sur les nues, sentit pour lors tout le poids de la vertu et de
la vérité. Il n'osa se commettre avec Boufflers, il désavoua tout ce
qu'il avait attenté dans ses lettres, et, pour voiler l'ordre du roi, il
envoya lui-même ses lettres à Boufflers qui y biffa tout ce qu'il
voulut, et ce qu'il biffa demeura supprimé dans l'expédition qu'en fit
Pontchartrain, et qui lui fut montrée.

Villars pourtant se distilla chez lui publiquement, et tous les jours en
respects pour le maréchal de Boufflers, en soumissions, en louanges, lui
envoya plusieurs messages en hommages et en pardons, et avala cet
affront dans toute son étendue. On négocia et on obtint enfin que
Boufflers après tant de génuflexions irait voir Villars, après avoir
ainsi triomphé de son triomphe. Il fut accueilli avec des respects et
des soumissions profondes qui furent reçues gravement et en maître qui
daigne accepter un tribut. De tous ces procédés se combla une haine que
Boufflers trop naturel exhala même peu décemment quelquefois, et que
Villars resserra en lui-même sous le voile des hommages et des
soumissions\,; toutefois sans rompre, par l'extrême retenue de Villars
qui n'osa plus se commettre, et Boufflers pour ne pas embarrasser le
roi.

Cet éclat fut incontinent après suivi d'un autre, mais qui, à beaucoup
près, ne fut pas porté si loin. Harcourt, duc vérifié cinq ans avant
Villars, et d'une naissance si différente, portait fort impatiemment que
celui-ci eût été fait pair avant lui, et que lui-même n'y fût arrivé
qu'à son occasion. Il n'ignorait pas nos prétentions réciproques de
préséance\,; de M. de La Rochefoucauld et de moi, il voulut adroitement
acquérir les mêmes sur Villars. Il projeta donc de se faire recevoir au
parlement dans la même séance où ses lettres de pairie seraient
enregistrées\,; et pour le faire couler doucement, il ne hasarda pas de
les présenter avant que celles de Villars le fussent, qui étaient
antérieures aux siennes. Mais dès qu'elles le furent, il se prépara à
l'exécution de son projet, comme ne songeant à rien. Malheureusement
pour lui Villars en eut le vent\,; il avait aussi ouï parler de mon
affaire avec M. de La Rochefoucauld, mais sans la savoir. Il me pria de
la lui expliquer, c'était chose qui ne se pouvait refuser. Là-dessus le
voilà aux champs, qui fait grand bruit, qui représente au roi, par un
mémoire qu'il lui envoya, le dessein d'Harcourt et l'impossibilité où sa
blessure le mettait de se faire recevoir, sur quoi, il demanda de ces
deux choses l'une, pour lui éviter un procès pareil au mien\,: ou des
lettres patentes vérifiées au parlement qui lui conservassent son
ancienneté entière sur les pairs postérieurs à lui qui pourvoient être
reçus au parlement avant lui, comme M. de Bouillon les avait obtenues
dans sa minorité\,; ou une défense verbale au maréchal d'Harcourt de se
faire recevoir avant que lui-même le fût. La demande parut au roi
d'autant plus juste qu'elle évitait un procédé qui l'eût embarrassé
entre ces deux hommes, et un procès dont il haïssait les décisions.
Harcourt reçut donc cette défense de la bouche du roi, dont il fut outré
de dépit, et dont Villars ne se contraignit pas de triompher. Fort peu
de jours après, Harcourt tomba en apoplexie, qui mit ses grandes vues et
ses amis en grand désarroi, et qui, au lieu de forcer la porte du
conseil, le fit aller aux eaux de Bourbonne, hors d'état de s'appliquer
à rien, mais retenant toujours sa destination de général de l'armée du
Rhin, comme l'année précédente.

Le maréchal d'Huxelles commandait en chef en Alsace dès l'année 1690, en
avril, à la mort de Montal, et servait de lieutenant général dans
l'armée du Rhin toutes les campagnes, jusqu'en 1703, qu'il fut de la
promotion des maréchaux de France que le roi fit en janvier, à
l'occasion de laquelle je me suis étendu sur lui assez pour n'avoir rien
à y ajouter. Décoré de l'ordre et du bâton, c'était où la profession
militaire le pouvait porter. Son goût ne le tournait point vers le
commandement des armées. Voir aussi de Strasbourg un général d'armée
auquel il fallait obéir dans son commandement s'il était son ancien, et
s'il ne l'était pas, se concerter avec lui de manière fort équivalente à
la subordination, était pour lui une amertume. Depuis 1690, il n'avait
quitté les bords du Rhin ni été ni hiver, que depuis qu'il fut maréchal
de France, et encore y demeura-t-il les premières années. Il petillait
de s'approcher de la cour dans le désir de pousser sa fortune. Il
voulait entrer dans le conseil, au moins être consulté et de quelque
chose. Son grand but était de parvenir à être duc, et celui du premier
écuyer d'être appelé dans ses lettres. Pour cela il fallait être à la
cour et à demeure\,; mais quitter plus de cent mille livres de rente en
abandonnant l'Alsace\,: c'était acheter bien cher des espérances peu
fondées. Il tâta le pavé par quelques voyages à Paris\,; il les
allongea, et fit si bien qu'il lui fut permis de s'y fixer sans se
dépouiller du commandement d'Alsace, qu'on fît exercer par du Bourg,
tellement que cette province eut un gouverneur et deux commandants
payés.

Huxelles établi à Paris tint une excellente table pour avoir compagnie,
sortit peu pour se faire rechercher, se lia au président de Mesmes par
le premier écuyer son ami intime, et par ce président à M. du Maine,
dont il était le commensal. Il fut vanté a M\textsuperscript{lle} Choin
par M\textsuperscript{me} de Beringhen, la cultiva jusqu'à envoyer tous
les jours de sa vie des têtes de lapins et d'autres mangeailles à sa
chienne (et il faut noter qu'il logeait dans la rue
Neuve-Saint-Augustin, vis-à-vis le duc de Tresmes, et
M\textsuperscript{lle} Choin attenant le petit-Saint-Antoine). Il fit sa
cour à Vaudemont et à ses nièces, et s'initia ainsi à Monseigneur, sans
toutefois le voir souvent en particulier, et très-rarement publiquement,
qui le crut la meilleure tête de France et un homme qui ne voulait rien
que son repos. D'autre côté il courtisa Harcourt, qui le produisit à
M\textsuperscript{me} de Caylus pour atteindre à M\textsuperscript{me}
de Maintenon. Harcourt ne le craignait point pour émule, il le
connaissoit trop bien, mais il en voulait faire un écho et un
épouvantail à ministres, contre lesquels tout lui était bon\,;
conséquemment il fut très-bien avec Voysin aussitôt qu'il fut en place.
Tout cela se passait souterrainement. Tant de liaisons importantes ne
rendant rien, il en tomba peu à peu dans un chagrin qui devint noir, qui
attaqua sa santé et qui fit craindre pour sa tête. Il fut près d'un an
chez lui sans vouloir voir personne que le premier écuyer, sa femme, et
un ou deux autres devant qui il ne retenait pas ses faiblesses. Les
médecins furent longtemps sans savoir ce que cela de-viendrait, parce
qu'ils sentirent que ce n'était pas de leur art que dépendait cette
guérison. Ses amis se remuèrent vers les remèdes qu'il lui fallait, le
poulièrent\footnote{Mot déjà employé par Saint-Simon dans le sens de
  \emph{hisser avec une poulie}.} à Marly, et le soulagèrent, mais non
encore entièrement. C'est l'état où il était quand il fut question de
nommer des plénipotentiaires pour les conférences de Gertruydemberg.

Torcy, ami intime de l'abbé de Polignac, l'avait, comme on a vu, tiré
d'un péril imminent et fort dangereux, en le dépaysant\,; et lui en
avoir su tirer grand parti, avec le même appui, pour s'assurer d'un
chapeau. Cela fait, et l'intervalle long de son absence, il eut envie de
se rapprocher. Torcy, qui le destinait à travailler à la paix, pour le
tenir toujours en besogne lui procura la permission de faire un tour à
la cour de quelques mois, sans quitter son auditorat de rote\footnote{Voy.,
  t. II, p.~383, note, en quoi consiste le tribunal de rote.} où il
brillait, et pour avoir où le renvoyer au loin si le cas y échéait. Il
était arrivé sur la fin de l'année précédente, et fut assez bien reçu du
roi, et très-bien de la cour, surtout des dames. Ce retour me procura
une confidence.

Il faut se souvenir de la conversation que j'eus sur lui avec le duc de
Beauvilliers, ci-devant (t. V, p.~97, 98) et de la manière dont il reçut
ce que je lui dis. Oncques depuis nous ne nous étions fait mention de
l'abbé de Polignac l'un à l'autre, ni de rien qui en pût approcher. Mon
retour à Marly fut un des premiers fruits de l'audience que le roi
m'avait accordée. Au premier voyage que j'y fis, étant allé un soir
causer avec le duc de Beauvilliers, et ne parlant de rien moins que de
l'abbé de Polignac, tout d'un coup le duc se mit à me regarder fixement,
à sourire et à me dire qu'il fallait qu'il me fît une confidence\,; et
que c'était une réparation qu'il me devait à laquelle il ne pouvait plus
tenir. Je n'imaginai point ce qu'il me voulait dire. «\,Vous
souvenez-vous bien, me dit-il, de la conversation que nous eûmes
ensemble, dans cette même chambre, il y a quatre ans, sur l'abbé de
Polignac\,? c'est que vous avez été prophète. Il faut que je vous avoue
qu'il m'est arrivé de point en point ce que vous m'aviez prédit, et que
l'abbé de Polignac, initié avec Mgr le duc de Bourgogne par les
sciences, et le voyant souvent seul, m'avait absolument éloigné de
lui.\,» Je m'écriai, il me fit taire. «\,Écoutez tout, me dit-il. Je ne
fus pas longtemps à m'en apercevoir. Je voulus me le rapprocher, je
l'éloignai encore davantage. Plus de consultations, plus même de
raisonnements\,; jusqu'à ma présence lui pesait. M. de Chevreuse se
trouva de même. Je pris le parti de ne lui plus parler de rien, de
répondre en deux mots quand il me parlait, de faire mon service assez
pour que le public ne s'aperçût de rien, et je demeurai dans mes
fonctions comme un étranger le plus mesuré, sans trouver rien à redire
et sans parler que pour répondre. Cela, monsieur, a, s'il vous plaît,
duré plus d'un an. Enfin, il s'est rapproché, il s'est réchauffé, il
s'est trouvé embarrassé de ma réserve, il a tâté le pavé à diverses
reprises. Je le voyais venir toujours respectueusement, sans la moindre
ouverture, jusqu'à ce qu'un beau jour il me prît dans son cabinet et se
déboutonna. Je reçus ce qu'il me dit comme je le devais, et lui dis en
même temps ce que je crus devoir sur l'attachement et la confiance\,;
que je ne tenais à lui que par le coeur, et le désir de son bien et
celui de l'État, et par nulle autre chose\,; et qu'il voyait que je
savais me retirer à proportion de lui, et me tenir dans le respect et
dans la simple fonction de ma charge. Alors dans ce retour d'amitié et
de confiance, il m'avoua que c'était l'abbé de Polignac qui l'avait
éloigné\,; que c'était un enchanteur très-dangereux, une sirène\ldots{}
Eh bien\,! monsieur, interrompis-je, avez-vous eu encore votre cruelle
charité de ne lui pas bien rompre le cou en ce moment que vous l'avez eu
si belle\,? Oh\,! pour cela, me dit-il, ce n'eût pas été charité, c'eût
été abandon de Mgr le duc de Bourgogne, et manquer de charité pour
lui\,; aussi, vous puis-je assurer que je lui ai fait sentir tout ce que
je devais sur cela pour lui-même\,; et que, puisque vous appelez cela
rompre le cou, vous pouvez compter que je l'ai si bien et si
parfaitement rompu à l'abbé de Polignac, qu'il n'en reviendra de sa vie
auprès de Mgr le duc de Bourgogne.\,»

Je l'en louai beaucoup, et comme un homme qui s'est surpassé lui-même\,;
après quoi je me licenciai à le pouiller un peu de ne vouloir ni
connaître les gens ni souffrir qu'on les lui fît connaître. Je le fis
souvenir de notre conversation dans le bas des jardins de Marly, sur le
choix fait et non encore déclaré de Mgr le duc de Bourgogne pour l'armée
de Flandre avec M. de Vendôme, et je lui dis que la prophétie que je lui
en fis alors, qui ne tarda pas à s'accomplir au delà de toute pensée, et
celle-ci dont il m'avouait le plénier effet, le devaient rendre plus
docile à écouter, et à croire et à se garder. Il en convint, et il est
vrai que longtemps avant cet aveu il était moins hérissé à mes discours,
à son gré peu charitables, et me croyait fort volontiers, ce qui ne fit
depuis qu'augmenter de plus en plus à mon égard. Je lui demandai après
où en était le duc de Chevreuse\,; il me dit que le retour était aussi
entier pour lui et de même date que le sien. Le singulier est qu'ils se
conduisirent avec tant de ménagements, que personne, même les valets les
plus intérieurs, ne s'aperçurent jamais de ce changement si grand dans
toute sa longue durée. Il ne servit qu'à mettre ces deux ducs encore
plus intimement avec Mgr le duc de Bourgogne\,; ce qui a duré jusqu'à sa
mort\footnote{Voy., sur l'abbé de Polignac, les notes à la fin du
  volume.}.

L'abbé de Polignac, à son retour de Rome, se trouva bien étourdi de la
froideur marquée de Mgr le duc de Bourgogne, qui ne prit à rien avec lui
en public, et ne le vit point en particulier. Le bon ecclésiastique
craignit pis qu'il n'y avait, et se contint par là dans de pénibles
réserves. Mais bientôt il fut délivré par le choix du maréchal
d'Huxelles et de lui pour aller à Gertruydemberg\,; sur quoi je renvoie
aux Pièces, où les préliminaires de cet envoi et la négociation jusqu'à
sa rupture se trouvent dans tout le détail\footnote{Voy. les Mémoires de
  Torcy.}. Je dirai seulement ici que le maréchal d'Huxelles, qui
mourait de ne rien faire, et que cette nomination guérit, voulut faire
accroire qu'on le faisait aller malgré lui, tandis que Harcourt, Voysin
et M\textsuperscript{me} de Maintenon le préconisaient, et que M. du
Maine le servait. Le jour qu'il fut déclaré au conseil avec l'abbé de
Polignac, Monseigneur dit qu'il ne croyait pas que le maréchal voulût se
charger de cet emploi, et qu'outre qu'il était vieux et infirme (il
n'était point vieux, et n'était malade que de rage de n'être {[}rien{]}
et de ne rien faire), il lui avait dit, il n'y avait pas longtemps,
qu'il aimerait mieux avoir perdu un bras que son nom demeurât à la
postérité souscrit à une paix telle que celle qui terminerait cette
guerre. On verra dans les Pièces et dans les suites que cette
délicatesse ne fut que pour Monseigneur, et pour tâcher de se faire
valoir\,: le renard des mûres, si on ne songeait point à lui,
{[}voulant{]} se faire prier si on y pensait.

Le chancelier, ami intime du premier écuyer et parent d'Huxelles et
Voysin, louèrent sa capacité\,; Desmarets, son ami, aussi. Le roi,
prévenu par M\textsuperscript{me} de Maintenon et M. du Maine, applaudit
ainsi que les deux autres ministres qui firent chorus\,; puis le roi
ajouta en se redressant, qu'il ne croyait pas qu'il refusât, quand il
saurait qu'il ne recevrait aucune excuse, et qu'il voulait bien qu'on le
lui dît, et qu'il ne voulait pas être refusé. Ce même jour le chancelier
s'en allait à Paris. Dès qu'il y fut arrivé, il envoya chercher le
maréchal à qui il conta ce qui s'était passé au conseil, et le détermina
sans peine à accepter. Il avait déjà reçu une lettre de Torcy là-dessus,
qui, ce même jour encore, arriva chez lui avec Desmarets, et
l'entretinrent deux heures. L'abbé de Polignac, qui n'avait avec lui
aucune liaison, le fut voir deux jours après. Le maréchal ne vit le roi
dans son cabinet qu'un demi-quart d'heure, à ce que me conta le premier
écuyer qui en était fort scandalisé, et l'abbé de Polignac point du
tout. Le roi lui dit un mot en passant chez M\textsuperscript{me} de
Maintenon. Je n'entamerai rien ici de ce qui se trouvera dans les
Pièces\,; je dirai seulement que, sur les plaintes que firent les
Hollandais d'une nomination d'éclat par les personnages, lorsqu'ils n'en
voulaient que d'obscurs, on eut recours à une ruse d'enfant, la plus
déshonorante qu'il fût possible. Le maréchal d'Huxelles eut défense de
mettre ses armes à rien, pour ne montrer ni ses bâtons ni son collier de
l'ordre, et l'abbé de Polignac de paraître autrement qu'en habit de
cavalier. Cela ne cachait ni leurs noms ni leur caractère\,; cela avilit
seulement celui que le roi leur donnait pour traiter, et donna fort à
rire aux alliés, qui insultèrent à une complaisance si basse.

Tout ce qui suivit répondit à ce triste début. Si un officier de la
couronne effacé de la sorte devint un spectacle fort nouveau, la
mascarade de l'abbé de Polignac en fut un encore plus étrange. On trouva
même sans cela toutes sortes d'indécences d'employer un ecclésiastique
et un auditeur de rote à consentir, comme il était inévitable, à
beaucoup de choses préjudiciables à la religion catholique dans toutes
les restitutions auxquelles il fallait se livrer\,; un homme qui avait
publiquement la nomination acceptée du roi d'Angleterre au cardinalat
pour signer l'exhérédation et la proscription de ce prince et de sa
postérité en faveur d'un usurpateur protestant et comme tel\,; enfin un
personnage châtié par l'exil en arrivant de son ambassade de Pologne,
exil qui avait duré fort longtemps. Sur tout le reste je renvoie aux
Pièces, qui satisferont pleinement. Tout fut concerté avec Bergheyck,
venu exprès à Versailles, et qui retourna en Flandre vers le départ de
nos deux plénipotentiaires.

On essuya encore en même temps une chose assez désagréable. Le cardinal
de Médicis, en remettant son chapeau pour se marier, comme on l'a dit,
avait fait vaquer la protection des couronnes de France et d'Espagne
qu'il avait. Les couronnes catholiques ont à Rome chacune leur
\emph{protecteur}, étrange nom à l'égard d'une couronne\,; mais les
cardinaux, de longue main en possession d'être des monstres fort à
charge à leurs princes et à leurs nations, et beaucoup plus à l'Église,
après avoir usurpé les choses, ont envahi jusqu'aux noms, et les rois
les ont laissé faire avec une insensibilité nonpareille. Ces messieurs
veulent donc se mêler d'affaires, et ne peuvent le faire que
subordonnément, comme tous les autres qui en seraient chargés\,; ils en
veulent l'honneur, la considération, le profit, mais ils n'en veulent
pas le nom ordinaire\,; il faut leur en voiler les fonctions sous la
majesté d'un nom qui impose, quoique tout le monde en sache la valeur.
Ainsi le cardinal qui est payé pour prendre soin de tout ce qui passe en
consistoire pour une nation s'appelle le protecteur de cette nation\,;
et de là protecteur de la couronne de France, d'Espagne, etc. C'est à
lui que s'adressent les banquiers en cour de Rome pour l'expédition des
bénéfices et des autres choses qui passent en consistoire, où c'est à
lui à proposer et à préconiser les évêchés\,; et il se mêle aussi de
beaucoup de choses qui passent par la chancellerie, par la pénitencerie
et par les signatures. Le roi, ayant donc à choisir un protecteur, jeta
les yeux sur le cardinal Ottobon. Plusieurs raisons l'en devaient
empêcher.

Son oncle que M. de Chaulnes fit pape, et qui avait promis merveilles
sur les franchises et sur d'autres points plus importants qui avaient
brouillé le roi avec Innocent XI, son prédécesseur, qui depuis longtemps
ne donnait aucunes bulles en France, manqua de parole, et se moqua de la
France en pantalon qu'il était\,; en sorte qu'il la fit passer à tout ce
qu'il voulut\,; et à ce qui aurait tout terminé même avec Innocent XI.
Ainsi, ce neveu ne devait pas être un sujet assez agréable pour recevoir
une pareille distinction dans une cour si suivie, et qui ne dompte la
nôtre que par sa suite perpétuelle qu'elle ne rencontre pas dans notre
légèreté. Ce cardinal était un panier percé qui, avec de grands biens,
de grands bénéfices, et les premières charges de la cour de Rome, y
était méprisé par le désordre de ses dépenses, de ses affaires, de sa
conduite et de ses moeurs, quoique avec beaucoup d'esprit, et même
capable d'affaires et aimable dans le commerce. Enfin, il était
Vénitien, et le roi avait tous les sujets du monde de se plaindre de la
con, duite de sa république pendant la guerre d'Italie. De plus, on ne
devait pas ignorer avec quelle jalousie la politique de Venise interdit
à ses sujets tout attachement à quelque prince que ce soit, et combien
elle l'avait montré encore, il n'y avait pas longtemps, à l'occasion de
la nomination du cardinal Grimani par l'empereur, et des emplois qu'il
lui avait donnés, quelque terreur qu'ils eussent de ce prince, et quels
que fussent leurs extrêmes ménagements pour lui. Ce fut à quoi on
s'exposa ici par cette nomination.

Ottobon balança à l'accepter, non qu'il ne la désirât beaucoup, mais par
respect pour ses maîtres, et dans l'espérance de les y faire consentir.
Il y échoua. Ils tinrent ferme, ils refusèrent au roi qui s'abaissa à
les prier. Le roi, qui n'en voulut pas avoir le démenti, pressa Ottobon
de passer outre. Il se trouva embarrassé, et toute cette lutte dura
assez longtemps. Enfin, tenté par de grosses abbayes, il passa le
Rubicon. Les Vénitiens l'effacèrent du livre d'or\footnote{Registre sur
  lequel étaient inscrits les noms des patriciens de Venise.}, le
proscrivirent, défendirent tout commerce avec lui, même à ses plus
proches, et à leur ambassadeur à Rome de le visiter. L'abbé de Pomponne,
ambassadeur à Venise par qui cette négociation avait passé, sortit de
Venise, se retira à Florence, et l'ambassadeur de Venise à Paris eut
ordre de s'en aller, partit sans audience de congé, et ne tarda pas à
arriver à Paris et à Versailles.

Il arriva en même temps une aventure très-singulière, et qui piqua fort
le roi. Un petit procureur du siége de Beaune en Bourgogne s'appelait
Chavignard, et avait deux fils assez bien faits. Ils étudièrent aux
jésuites, qui les prirent sous leur protection. De Chavignard à Chavigny
il n'y a pas loin dans la prononciation. La maison de Chavigny-le-Roi,
ancienne, illustre, grandement alliée, était éteinte depuis longtemps.
Ces deux frères jugèrent à propos de la ressusciter et de s'en dire, et
les jésuites de les produire comme tels. Ils vinrent à Paris sous ce
beau nom comme des cadets de bonne maison, mais qui n'avaient rien, et
qui réclamaient leurs parents, chez qui les jésuites les présentèrent et
les introduisirent parmi leurs amis. M. de Soubise qui croyait ne
pouvoir être dupe que de son gré., et qui avait de bonnes raisons de se
le persuader, le fut tout de bon cette fois-ci\,; il prit pour bon ce
que les jésuites lui dirent, et voulut bien présenter au roi MM. de
Chavigny comme ses parents et leur procurer de l'emploi. La duchesse de
Duras, fille du prince de Bournonville, mort sous-lieutenant des gens
d'armes de la garde, avait eu de la cascade de cette charge un guidon à
vendre dans la même compagnie. M. de Soubise le procura à l'un des deux
frères qui obtint aussi l'agrément d'une petite lieutenance de roi en
Touraine. Il avait, disait-il, épuisé le peu qu'il avait, et boursillé
parmi ses amis pour se faire cet établissement et se mettre en chemin de
faire fortune. Ils allaient voir tout le monde et chacun les recevait
avec plaisir par le nom, la figure et les manières qu'ils présentaient.
L'autre frère eut peu après une abbaye de dix-huit à vingt mille livres
de rente pour aider à son frère à subsister à la cour et à la guerre, où
il avait fait la campagne dernière dans les gens d'armes.

Une si grosse abbaye ne vaquait pas tous les jours. Celle-ci ne l'était
devenue que cet hiver, et causa tant d'envie que les aboyants, outrés de
la voir donner ainsi, se mirent à chercher ce que c'était que cet abbé
de Chavigny, et découvrirent qui il était. Ils en eurent les preuves et
les publièrent avec tant de bruit qu'ils détrompèrent tout le monde. Le
roi, piqué d'une si hardie imposture, dans laquelle il avait si bien
donné, fit arrêter les bulles à Rome, nomma un autre sujet, ordonna à
l'autre frère de se défaire de son guidon en faveur du comte de Pons
pour soixante mille livres, qu'il avait acheté quatre-vingt mille
livres, et de sa lieutenance de roi de Touraine, et fit défendre à tous
deux de se présenter jamais devant lui. On trouva encore la punition
douce. C'étaient deux compagnons de beaucoup d'esprit, d'intrigue et de
manége, de hardiesse, de souplesse, et pour leur âge fort instruits. Ils
disparurent à l'instant et firent le plongeon. Qui ne croirait que ce ne
fût pour toujours après une telle infamie\,? Cet affront ne leur coûta
rien à soutenir. Ils se mirent à faire les espions en Hollande. Torcy se
servit d'eux à l'insu du roi, et comme ils avaient, surtout le guidon,
infiniment d'esprit et d'adresse, il en fut fort content. Ils parurent
même à Utrecht pendant les conférences de la paix. Après la mort du roi
ils continuèrent à s'intriguer.

Dans la suite ils devinrent les instruments de l'abbé Dubois en beaucoup
de choses, puis ses confidents, et ce qu'en langage commun on
appellerait ses âmes damnées. Celui qui avait été abbé voulut du solide.
On n'eut pas honte de lui donner l'agrément d'une charge de président à
mortier au parlement de Besançon, où il s'est comporté avec une audace
et une insolence surprenante, et toujours s'appelant Chavigny. L'autre,
sous le nom de chevalier de Chavigny, plus doux et plus souple en
apparence, continua ses intrigues. L'abbé depuis cardinal Dubois
l'employa en divers lieux, puis en Espagne, à Ratisbonne, en Angleterre,
et maintenant, avec toute honte bue, il est ambassadeur de France en
Portugal à son retour de Danemark, où il était envoyé extraordinaire.
Partout on sait son histoire, partout il en est déshonoré, partout on
est indigné de le voir avec caractère {[}d'ambassadeur{]}, partout on
dit que ceux qui emploient un tel instrument ne le peuvent faire qu'à
dessein de tromper\,; et toutefois il subsiste, on en est content à la
cour, et il y est bien reçu dans les intervalles de ses emplois qu'il y
est venu. N'est-ce point là de ces vérités qui ne sont pas
vraisemblables\,? Pour y mettre le comble, elle était dans le Moréri au
nom de Chavigny-le-Roi, et ils ont eu le crédit de faire défendre qu'on
la mît dans la dernière édition qui en a été faite.

Le samedi 15 février le roi fut réveillé à sept heures, qui était une
heure plus tôt qu'à l'ordinaire, parce que M\textsuperscript{me} la
duchesse de Bourgogne se trouvait mal pour accoucher. Il s'habilla
diligemment pour se rendre auprès d'elle. Elle ne le fit pas attendre
longtemps. À huit heures trois minutes et trois secondes elle mit au
monde un duc d'Anjou, qui est le roi Louis XV, aujourd'hui régnant, ce
qui causa une grande joie. Ce prince fut incontinent ondoyé par le
cardinal de Janson dans la chambre même où il était né, et emporté
ensuite sur les genoux de la duchesse de Ventadour dans la chaise à
porteurs du roi dans son appartement, accompagné par le maréchal de
Boufflers et par des gardes du corps avec des officiers. Un peu après,
La Vrillière lui porta le cordon bleu, et toute la cour l'alla voir,
deux choses qui déplurent fort à M. son frère, qui ne se contraignit pas
de le marquer. M\textsuperscript{me} de Saint-Simon, qui était dans la
chambre de M\textsuperscript{me} la Dauphine, se trouva par hasard une
des premières qui vit ce prince nouveau-né parmi toutes celles qui y
étaient. L'accouchement et ses suites furent fort heureux.

Il se fit en même temps deux mariages auxquels je pris grande part. Le
duc de Chevreuse, avec tout son esprit pénétrant, réglé et métaphysique,
s'était si parfaitement ruiné, à force de vouloir faire ses affaires
lui-même et tendre toujours au mieux, que, sans le gouvernement de
Guyenne, il n'aurait pas eu de quoi vivre. Il avait fait beaucoup de
belles choses à Dampierre. Il avait creusé un canal depuis ses forêts de
Montfort et de Saint-Léger jusqu'à Mantes, avec des frais infinis et des
dédommagements immenses aux riverains, pour porter ses bois jusqu'à la
Seine à bois perdu, dans lequel canal il n'a jamais coulé un muid d'eau.
Ensuite il fit paver toute sa forêt pour en tirer ses bois, sans aucun
usage, et il essuya enfin une grande banqueroute de ses marchands. Il
chercha un riche mariage pour le duc de Luynes, fils du feu duc de
Montfort, son fils aîné, quoiqu'il fût encore fort jeune. Ce bâtard du
dernier comte de Soissons prince du sang, dont j'ai parlé ailleurs, que
M\textsuperscript{me} de Nemours avait choisi pour en faire son
héritier, avait laissé deux filles de la fille du maréchal-duc de
Luxembourg. L'aînée avait quatre-vingt mille livres de rente en belles
terres, et n'avait qu'une soeur qui en devait avoir presque autant,
outre les pierreries et les autres choses qu'elles pouvaient encore
espérer de M\textsuperscript{me} de Nemours, qui n'avait d'yeux que pour
elles ni de volonté que pour ôter tout à ses héritiers naturels. M. de
Luxembourg, leur oncle, gendre en premières noces de M. et de
M\textsuperscript{me} de Chevreuse, sans enfants, avait toujours
conservé avec eux la liaison la plus intime. Il fit ce mariage, dont les
biens, la figure de la jeune femme et le côté maternel étaient à
souhait.

Le duc d'Humières, mon plus ancien et intime ami, maria sa fille unique
au fils aîné du duc de Guiche. En considération de ce noble et riche
mariage, ils obtinrent pour la première fois que le duc de Guiche se
démît de son duché, quoique le duc de Grammont, son père, qui s'en était
démis en sa faveur, vécût encore, et allant et venant par le monde\,;
ainsi ce fut trois générations à la fois ducs et pairs sur le même
duché-pairie.

Il s'en fit quelque temps après deux autres. Voysin maria l'aînée de ses
trois filles au fils aîné de Broglio, qui avait longtemps commandé en
Languedoc, et qui était beau-frère du feu président Lamoignon et du
célèbre Bâville. La veuve de Lamoignon était Voysin, cousine germaine du
ministre qui fit ce mariage.

Gacé, fils du maréchal de Matignon, veuf de sa cousine germaine de même
nom et sans enfants se remaria à la fille du maréchal de Châteaurenauld,
qui fut un très-malheureux mariage. Il eut le gouvernement de la
Rochelle et pays d'Aunis sur la démission du maréchal de Matignon son
père.

M. de Beauvilliers fit en même temps une chose fort contre mon goût, et
dont je fis tout mon possible pour le détourner. Ce fut de donner, avec
l'agrément du roi, sa charge de premier gentilhomme de la chambre au duc
de Mortemart, son gendre, de préférence au duc de Saint-Aignan son
frère. Il crut devoir cette récompense à sa fille, qu'il aimait fort,
des grands biens qu'après avoir perdu ses fils il avait donnés à son
frère\,; ceux que leur mort faisait tomber à la jeune duchesse de
Mortemart, avec la dignité de grand d'Espagne, me paraissaient un
dédommagement bien suffisant. Mais la délicatesse de M. de Beauvilliers
ne put être vaincue par toutes mes raisons. Il savait beaucoup de gré à
son gendre, et à la duchesse de Mortemart sa belle-soeur, de la manière
dont ils s'étaient portés à le presser même de faire beaucoup pour le
duc de Saint-Aignan. Cette duchesse de Mortemart était, après la
duchesse de Béthune, la grande âme de la gnose, et la mieux aimée de
l'archevêque de Cambrai, qui de son diocèse gouvernait toutes ces
consciences. Ce fut par conséquent l'avis aussi du duc de Chevreuse\,;
et la considération de la duchesse de Beauvilliers, qui avec la plus
grande amitié du monde, s'était prêtée à tout ce que le duc de
Beauvilliers avait voulu faire pour son frère, y entra pour beaucoup. Je
vis ce choix avec douleur, qui dans la suite leur en donna beaucoup à
eux-mêmes, et qui ne réussit pas comme ils l'avaient espéré à retirer le
duc de Mortemart de l'obscurité et de la crapule, ni à rendre sa pauvre
femme plus heureuse, qui méritait tant de l'être.

\hypertarget{chapitre-vi.}{%
\chapter{CHAPITRE VI.}\label{chapitre-vi.}}

1710

~

{\textsc{Bouffonneries de Courcillon, à qui on recoupe la cuisse.}}
{\textsc{- Mort de la duchesse de Foix.}} {\textsc{- Mort de Fléchier,
évêque de Nîmes.}} {\textsc{- Mort, caractère et testament de
l'archevêque de Reims Le Tellier.}} {\textsc{- Cardinal de Noailles
proviseur de Sorbonne.}} {\textsc{- Mort de Vassé.}} {\textsc{- Mort de
M\textsuperscript{me} de Lassai.}} {\textsc{- Mort de
M\textsuperscript{me} Vaubecourt.}} {\textsc{- Mort de l'abbé de
Grandpré\,; son sobriquet étrange.}} {\textsc{- Mort de M. le Duc.}}
{\textsc{- Conduite de M\textsuperscript{me} la Duchesse.}} {\textsc{-
Étrange contre-temps arrivé à M. le comte de Toulouse.}} {\textsc{- Nom
et dépouille entière de M. le Duc donnés à M. son fils.}} {\textsc{-
D'Antin chargé du détail de ses charges, puis de ses biens et de sa
conduite.}} {\textsc{- Saintrailles et son caractère.}} {\textsc{-
Caractère de M. le Duc.}} {\textsc{- Orgueil extrême de
M\textsuperscript{me} la duchesse d'Orléans\,; sa prétention de
préséance pour ses filles sur les femmes des princes du sang.}}
{\textsc{- Mesures sur cette dispute, et sa véritable cause.}}
{\textsc{- Adroite prétention de la duchesse du Maine de précéder ses
nièces comme tante.}} {\textsc{- Jugement du roi entre les princesses du
sang mariées et filles en faveur des premières, où il fait d'autres
décisions concernant son sang.}} {\textsc{- Mécanique des après-soupées
du roi.}} {\textsc{- Le roi déclare son jugement aux parties, puis au
conseil, et ne le rend public que quelques jours après, sans le revêtir
d'aucunes formes.}} {\textsc{- Brevet de conservation de rang de
princesse du sang, fille, à la duchesse du Maine.}}

~

J'ai déjà parlé ailleurs de Courcillon, original sans copie, avec
beaucoup d'esprit, et d'ornement dans l'esprit, un fonds de gaieté et de
plaisanterie inépuisable, une débauche effrénée et une effronterie à ne
rougir de rien. Il fit d'étranges farces lorsqu'on lui coupa la cuisse
après la bataille de Malplaquet. Apparemment qu'on fit mal l'opération,
puisqu'il fallut la lui recouper en ce temps-ci à Versailles. Ce fut si
haut que le danger était grand. Dangeau, grand et politique courtisan,
et sa femme que M\textsuperscript{me} de Maintenon aimait fort et qui
était de tous les particuliers du roi, tournèrent leur fils pour
l'amener à la confession. Cela l'importuna. Il connaissoit bien son
père. Pour se délivrer de cette importunité de confession, il feignit
d'entrer dans l'insinuation, lui dit que, puisqu'il en fallait venir là,
il voulait aller au mieux\,; qu'il le priait donc de lui faire venir le
P. de La Tour, général de l'Oratoire, mais de ne lui en proposer aucun
autre, parce qu'il était déterminé à n'aller qu'à celui-là. Dangeau
frémit de la tête aux pieds. Il venait de voir à quel point avait déplu
l'assistance du même père à la mort de M. le prince de Conti et de M. le
Prince\,; il n'osa jamais courir le même risque ni pour soi-même, ni
pour son fils, au cas qu'il vint à réchapper. De ce moment il ne fut
plus de sa part mention de confession, et Courcillon, qui n'en voulait
que cela, n'en parla pas aussi davantage, dont il fit de bons contes
après qu'il fut guéri. Dangeau avait un frère abbé, académicien,
grammairien, pédant, le meilleur homme du monde, mais fort ridicule.
Courcillon, voyant son père fort affligé au chevet de son lit, se prit à
rire comme un fou, le pria d'aller plus loin, parce qu'il faisait en
pleurant une si plaisante grimace qu'il le faisait mourir de rire.

De là passe à dire que, s'il meurt, sûrement l'abbé se mariera pour
soutenir la maison\,; et en fait une telle description en plumet et en
parure cavalière, que tout ce qui était là ne put se tenir d'en rire aux
larmes. Cette gaieté le sauva, et il eut la bizarre permission d'aller
chez le roi et partout sans épée et sans chapeau, parce que l'un et
l'autre l'embarrassait avec presque toute une cuisse de bois, avec
laquelle il ne cessa de faire des pantalonnades.

Il y eut aussi en ce temps-ci plusieurs morts. Celle de la duchesse de
Foix arriva la première, qui fut regrettée de tout le monde, et beaucoup
de M. de Foix. Elle était soeur de Roquelaure, à qui elle fit écrire en
mourant, pour lui demander de pardonner à sa fille et au prince de Léon,
ce qu'il accorda. M\textsuperscript{me} de Foix était la plus jolie
bossue qu'on pût voir, grande, dansant autrefois en perfection, et ayant
tant de grâces qu'on n'eût pas voulu qu'elle n'eût point été bossue\,;
peu de la cour, fort du grand monde et du jeu, extrêmement amusante sans
la moindre méchanceté, et n'ayant jamais eu plus de quinze ans à
cinquante-cinq ans qu'elle mourut sans enfants.

{[}La mort{]} de l'évêque de Nîmes arriva dans son diocèse. C'était
Fléchier qui avait été sous-précepteur de Monseigneur, célèbre par son
savoir, par ses ouvrages, par ses moeurs, par une vie très-épiscopale.
Quoique très-vieux, il fut fort regretté et pleuré de tout le Languedoc,
surtout de son diocèse.

Un bien plus grand prélat mourut en même temps, qui laissa moins de
regrets. Ce fut l'archevêque de Reims de qui j'ai parlé plus d'une fois.
Il avait les abbayes de Saint-Remy de Reims, de Saint-Thierry, près
Reims, qu'il avait fait unir à son archevêché pour le dédommagement de
l'érection de Cambrai en archevêché auparavant suffragant de Reims, qui
n'avait pas été fait, de Saint-Étienne de Caen, de Saint-Bénigne de
Dijon, de Breteuil et quelques autres encore. Il était commandeur de
l'ordre, doyen du conseil, maître de la chapelle du roi, proviseur de
Sorbonne, et le plus ancien archevêque de France. Outre ce que j'ai dit
ailleurs de sa fortune et de son caractère, j'ajouterai que, janséniste
de nom, ennemi des jésuites, savant en tout ce qui était de son état
pour le spirituel et le temporel, c'était avec de l'esprit un composé
fort extraordinaire. Rustre et haut au dernier point, il était humble
sur sa naissance à en embarrasser\,; extrêmement du grand monde,
magnifique et toutefois avare, grand aumônier assez résident chaque
année, gouvernant et visitant lui-même son diocèse qui était le mieux
réglé du royaume, et le mieux pourvu des plus excellents sujets en tout
genre qu'il savait choisir, s'attacher, employer et bien récompenser\,;
avec cela fort de la cour et du plus grand monde, gros joueur, habile en
affaires et fort entendu pour les siennes\,; lié avec les plus doctes et
les plus saints de l'épiscopat, aimé et estimé en Sorbonne qu'il
protégeait et gouvernait très-bien.

C'était un homme fort judicieux et qui avait le talent du gouvernement.
Les ducs d'Aumont et d'Humières, frères de père, et le premier fils
d'une soeur de ce prélat, avaient de grands démêlés d'intérêts qui les
avaient longtemps aigris, et qu'ils remirent enfin à décider à
l'archevêque de Reims dont la brillante santé était un peu tombée depuis
quelque temps. Il mettait la dernière main à cette affaire le samedi 22
février, et y travaillait depuis sept heures du matin, lorsque, vers une
heure après midi, il dit à son secrétaire qu'il se trouvait mal, et
qu'il sentait un grand mal de tête. Un moment après, il s'étendit dans
sa chaise et mourut, à soixante-neuf ans. La marquise de Créqui, sa
nièce, arrivait en même temps pour dîner avec lui, qui parut peu émue,
encore moins attendrie. Son amitié pour elle n'était pourtant pas sans
scandale. Outre des présents gros et continuels, il défrayait sa maison
toute l'année et lui en avait donné une toute meublée. Aussi passait-il
sa vie avec elle quand il était à Paris, à la grande jalousie de tous
ses autres héritiers. Ils furent tous mandés sur l'heure avec des
notaires, et M\textsuperscript{me} de Louvois, sa belle-soeur. Arrivés
qu'ils furent, on voulut chercher le testament. On n'en eut pas la
peine, la marquise de Créqui enseigna où il était. Par la lecture qu'on
fit on trouva qu'il faisait la marquise de Créqui sa légatrice
universelle, et l'abbé de Louvois exécuteur de son testament. Il lui
donnait la magnifique argenterie de sa chapelle et une belle
tapisserie\,; aux religieux de Sainte-Geneviève de Paris, sa
bibliothèque, la plus belle de l'Europe pour un particulier\,; et sa
maison de Paris aux enfants de feu M. de Louvois, son frère. Il avait
dénaturé son patrimoine, en sorte qu'il n'en restait que cette maison\,;
et, comme il n'avait pas douté que son testament ne fût attaqué, pour
peu qu'il pût l'être, il avait si bien fait que, quelque volonté qu'on
eût, cela fut impossible. Ainsi, la marquise de Créqui en eut deux
millions. Ce testament ne contribua pas à lever le scandale, ni le peu
d'affliction de la marquise de Créqui à adoucir l'indignation. Il y eut
des legs pieux et d'honnêtes récompenses au domestique.
M\textsuperscript{me} de Louvois alla le jour même demander au roi la
charge de la chapelle pour l'abbé de Louvois, mais par son oncle et par
lui-même il était écrit en lettres rouges chez les jésuites, et il n'eut
rien de cette grande dépouille. Le cardinal de Noailles fut proviseur de
Sorbonne, et Marillac devint doyen du conseil.

Le premier écuyer, beau-frère de la marquise de Créqui perdit bientôt
après Vassé, son gendre, qui était fort jeune et qui laissa des
enfants\,; et Lassai perdit sa troisième ou quatrième femme, bâtarde de
M. le Prince, dont la tête était un peu dérangée et qui lui laissa une
fille.

M\textsuperscript{me} de Vaubecourt, soeur d'Amelot, l'ambassadeur en
Espagne, etc., mourut aussi en même temps sans enfants, et veuve de
Vaubecourt, lieutenant général, tué en Italie. Elle était encore
belle\,; elle avait fait du bruit et était encore fort du grand monde,
mais jamais de la cour.

Le vieil abbé de Grandpré mourut aussi. Il était frère du feu comte de
Grandpré, lieutenant général et chevalier de l'ordre en 1661, et du
maréchal de Joyeuse. C'était une manière d'imbécile et qui en avait
aussi tout le maintien, mais qui ne laissait pas de sentir sa naissance,
et d'aller partout. Il n'avait qu'une méchante petite abbaye et n'était
point dans les ordres. Son corps n'était pas comme son esprit, les dames
autrefois lui avaient donné le nom \emph{d'abbé Quatorze} qui lui était
demeuré, et ce prodige avait passé en telle notoriété que sa singularité
excuse la honte de le rapporter.

Une autre mort épouvanta le monde et le mit en même temps à son aise. M.
le Duc, tout occupé de son procès, dont la plaidoirie devait commencer
le premier lundi de carême, était attaqué d'un mal bizarre qui lui
causait quelquefois des accidents équivoques d'épilepsie et d'apoplexie
qui duraient peu, et qu'il cachait avec tant de soin qu'il chassa un de
ses gens pour en avoir parlé à d'autres de ses domestiques. Il avait
depuis quelque temps un mal de tête continuel, souvent violent. Cet état
troublait l'aise qu'il sentait de la délivrance d'un père très-fàcheux,
et d'un beau-frère qui, en bien des sortes, avait fait continuellement
le malheur et souvent le désespoir de sa vie. M\textsuperscript{me} la
Princesse, pour qui il avait quelque considération et quelque amitié, le
pressait de penser à Dieu et à sa santé. À force d'exhortations, il lui
promit l'un et l'autre, mais après le carnaval, qu'il voulait donner aux
plaisirs. Il fit venir M\textsuperscript{me} la Duchesse à Paris le
lundi gras, pour les sollicitations et les audiences, et en attendant
pour lui donner deux soupers et à beaucoup de dames, et les mener courre
le bal toute la nuit du lundi et du mardi gras. Sur le soir du lundi, il
alla à l'hôtel de Bouillon, et de là chez le duc de Coislin, son ami de
tout temps, qui était déjà assez malade\,; il n'avait point de flambeaux
et un seul laquais derrière son carrosse. Passant sur le pont Royal,
revenant de l'hôtel de Coislin, il se trouva si mal qu'il tira son
cordon et fit monter son laquais auprès de lui, duquel il voulut savoir
s'il n'avait pas la bouche tournée, et il ne l'avait pas, et par qui il
fit dire à son cocher de l'arrêter au petit degré de sa garde-robe pour
entrer chez lui par-derrière, et n'être point vu de la grande compagnie
qui était à l'hôtel de Condé pour souper. En chemin il perdit la porole
et même la connaissance, il balbutia pourtant quelque chose pour la
dernière fois, lorsque son laquais et un frotteur qui se trouva là le
tirèrent du carrosse et le portèrent à la porte de sa garde-robe qui se
trouva fermée. Ils y frappèrent tant et si fort qu'ils furent entendus
de tout ce qui était à l'hôtel de Condé, qui accourut. On le jeta au
lit. Médecins et prêtres mandés en diligence firent inutilement leurs
fonctions. Il ne donna nul autre signe de vie que d'horribles grimaces,
et mourut de la sorte sur les quatre heures du matin du mardi gras.

M\textsuperscript{me} la Duchesse, au milieu des parures, des habits de
masques et de tout ce grand monde convié, éperdue de surprise et du
spectacle, ne perdit sur rien la présence d'esprit. Quoique mal avec M.
du Maine, elle en sentit le besoin\,; ainsi, fort peu après qu'on eut
mis M. le Duc au lit, elle envoya le chercher à Versailles, M. le comte
de Toulouse et M\textsuperscript{me} la princesse de Conti leur soeur,
et ne manda rien à M. {[}le duc{]} ni à M\textsuperscript{me} la
duchesse d'Orléans, avec qui elle était mal, et du crédit desquels elle
n'avait rien à attendre. On peut juger qu'elle n'oublia pas d'Antin.
Elle ne laissa pas de pleurer un peu en les attendant. Personne ne crut
ses larmes excitées par la tendresse, mais plutôt par un souvenir
douloureux qui l'affligeait en secret depuis un an, et d'une délivrance
trop tardive. M\textsuperscript{me} la princesse de Conti, sa
belle-soeur, avertie de ce qui se passait, alla à l'hôtel de Condé avec
ses enfants, demeura dans les antichambres parmi les laquais assez
longtemps, retourna dans son carrosse sans sortir de la maison, et
revint encore dans les antichambres. La maréchale d'Estrées, douairière,
fort amie de M\textsuperscript{me} la Duchesse, la trouvant là la fit
entrer malgré elle, disant qu'en l'état où elle était avec M. son frère,
elle n'osait se présenter. M\textsuperscript{me} la Duchesse, toujours
fort à elle-même après le premier étonnement, lui fit merveilles.
Bientôt après, l'autre princesse de Conti arriva de Versailles, qui se
mettait au lit lorsque le message de M\textsuperscript{me} la Duchesse
lui vint. Elle demeura peu à l'hôtel de Condé. M. le Duc venait de
mourir\,; elle emmena M\textsuperscript{me} la Duchesse à Versailles.
Vers Chaillot ils trouvèrent M. du Maine qui monta dans leur carrosse,
et vers Chaville M. le comte de Toulouse, qui y monta aussi et s'en
retourna avec eux.

Le contre-temps qui lui arriva fit grand bruit, enfanta des chansons, et
ce fut tout. Le courrier de M\textsuperscript{me} la Duchesse ne le
trouva point chez lui, et pas un de ses gens ne put ou ne voulut dire où
il était, ni l'aller avertir. Il n'était pas loin pourtant, dans un bel
appartement d'emprunt avec une très-belle dame du plus haut parage, dont
le mari était dans le même, qui en faisait deux beaux, où tout le jour
il tenait le plus grand état du monde, mais qui, malgré ses jalousies
quelquefois éclatantes, était hors d'état de les aller surprendre, et la
dame apparemment bien sûre du secret. Ils se reposèrent tous chez
M\textsuperscript{me} la Duchesse, où ses enfants arrivèrent.
M\textsuperscript{me} la princesse de Conti alla éveiller Monseigneur,
et huit heures du matin approchant, M. et M\textsuperscript{me} la
duchesse d'Orléans avertis vinrent chez M\textsuperscript{me} la
Duchesse, où tout se passa entre eux de fort bonne grâce. M. le duc
d'Orléans, M. du Maine et M. le comte de Toulouse allèrent au premier
réveil du roi, où Monseigneur arriva un moment après eux.

Le roi, surpris de les voir à une heure si peu ordinaire, leur demanda
ce qu'il y avait. M. du Maine porta la parole pour tous, et aussitôt le
roi donna à M. le duc d'Enghien le gouvernement, la charge et la pension
de M. son père, et déclara qu'il s'appellerait M. le Duc comme lui. Ils
retournèrent chez M\textsuperscript{me} la Duchesse lui apprendre ces
grâces, et tout de suite menèrent le nouveau M. le Duc attendre le roi
dans ses cabinets, à qui ils le présentèrent. Ce prince, dont la
sensibilité n'avait pas édifié à l'hôtel de Condé, avait plus de
dix-sept ans. Le roi permit qu'il fît auprès de lui le service de grand
maître, mais il ne voulut pas lui commettre l'exercice réel de cette
charge ni du gouvernement de Bourgogne, et, de concert avec
M\textsuperscript{me} la Duchesse, il chargea d'Antin du détail de l'un
et de l'autre, de ses biens et de sa conduite, ce qui se déclara
quelques jours après. M\textsuperscript{me} la Princesse était à
Maubuisson\,; elle avait conservé beaucoup d'affection pour cette
maison, quoiqu'elle eût perdu sa célèbre tante. Elle vint en diligence
et apprit la mort de M. son fils, parce que malgré ses cris elle fut
menée non à l'hôtel de Condé, mais chez elle au Petit-Luxembourg, maison
qu'elle avait superbement bâtie depuis la mort de M. le Prince, et
qu'elle achevait encore alors. Elle envoya aussitôt au roi Saintrailles
le supplier de vouloir bien mettre la paix dans sa famille. Le roi lui
promit d'y travailler, et ordonna à Saintrailles de demeurer auprès de
M. le Duc comme il était auprès du père, dont il commandait l'écurie.
C'était un homme sage avec de l'esprit, fort mêlé dans la meilleure
compagnie, mais qui l'avait gâté en l'élevant au-dessus de son petit
état, et qui l'avait rendu important jusqu'à l'impertinence. C'était un
gentilhomme tout simple et brave, mais qui n'était rien moins que Poton,
qui est le nom du fameux Saintrailles.

La mort du poëte Santeuil aux états de Bourgogne, l'aventure inouïe du
comte de Fiesque à Saint-Maur, et d'autres choses encore qui se trouvent
ci-devant éparses, ont déjà donné un crayon de M. le Duc\,: c'était un
homme très-considérablement plus petit que les plus petits hommes, qui
sans être gras était gros de partout, la tête grosse à surprendre, et un
visage qui faisait peur. On disait qu'un nain de M\textsuperscript{me}
la Princesse en était cause. Il était d'un jaune livide, l'air presque
toujours furieux, mais en tout temps si fier, si audacieux, qu'on avait
peine à s'accoutumer à lui. Il avait de l'esprit, de la lecture, des
restes d'une excellente éducation, de la politesse et des grâces même
quand il voulait, mais il voulait très-rarement\,; il n'avait ni
l'avarice, ni l'injustice, ni la bassesse de ses pères, mais il en avait
toute la valeur, et {[}avait{]} montré de l'application et de
l'intelligence à la guerre. Il en avait aussi toute la malignité et
toutes les adresses pour accroître son rang par des usurpations fines,
et plus d'audace et d'emportement qu'eux encore à embler. Ses moeurs
perverses lui parurent une vertu, et d'étranges vengeances qu'il exerça
plus d'une fois, et dont un particulier se serait bien mal trouvé, un
apanage de sa grandeur. Sa férocité était extrême et se montrait en
tout. C'était une meule toujours en l'air qui faisait fuir devant elle,
et dont ses amis n'étaient jamais en sûreté, tantôt par des insultes
extrêmes, tantôt par des plaisanteries cruelles en face, et des chansons
qu'il savait faire sur-le-champ qui emportaient la pièce et qui ne
s'effaçaient jamais\,; aussi fut-il payé en même monnaie plus
cruellement encore. D'amis il n'en eut point, mais des connaissances
plus familières, la plupart étrangemeut choisies, et la plupart obscures
comme il l'était lui-même autant que le pouvait être un homme de ce
rang. Ces prétendus amis le fuyaient, il courait après eux pour éviter
la solitude, et quand il en découvrait quelque repas, il y tombait comme
par la cheminée, et leur faisait une sortie de s'être cachés de lui.
J'en ai vu quelquefois, M. de Metz, M. de Castries et d'autres, désolés.

Ce naturel farouche le précipita dans un abus continuel de tout et dans
l'applaudissement de cet abus qui le rendait intraitable, et si ce terme
pouvait convenir à un prince du sang, dans cette sorte d'insolence qui a
plus fait détester les tyrans que leur tyrannie même. Les embarras
domestiques, les élans continuels de la plus furieuse jalousie, les vifs
piquants d'en sentir sans cesse l'inutilité, un contraste sans relâche
d'amour et de rage conjugale, le déchirement de l'impuissance dans un
homme si fougueux et si démesuré, le désespoir de la crainte du roi, et
de la préférence de M. le prince de Conti sur lui, dans le coeur, dans
l'esprit, dans les manières même de son propre père, la fureur de
l'amour et de l'applaudissement universel pour ce même prince, tandis
qu'il n'éprouvait que le plus grand éloignement du public, et qu'il se
sentait le fléau de son plus intime domestique, la rage du rang de M. le
duc d'Orléans et de celui des bâtards, quelque profit qu'il en sût
usurper, toutes ces furies le tourmentèrent sans relâche et le rendirent
terrible comme ces animaux qui ne semblent nés que pour dévorer et pour
faire la guerre au genre humain\,; aussi les insultes et les sorties
étaient ses délassements, dont son extrême orgueil s'était fait une
habitude, et dans laquelle il se complaisait. Mais s'il était
redoutable, il était encore plus déchiré. Il se fit un dernier effort
aux états de Bourgogne, qu'il tint après la mort de M. le Prince, d'y
paraître plus accessible. Il y rendit justice avec une apparence de
bonté\,; il s'intéressa avec succès pour la province, et il y donna de
bons ordres de police\,; mais il y traita le parlement avec indignité
sur des prérogatives que M. son père n'avait jamais eues, et qu'il lui
arracha après quantité d'affronts. Quiconque aura connu ce prince n'en
trouvera pas ici le portrait chargé, et il n'y eut personne qui n'ait
regardé sa mort comme le soulagement personnel de tout le monde.

J'appris la mort de M. le Duc à mon réveil à Versailles où j'étais,
j'allai à la messe du roi où je sus ce qui s'était passé là-dessus, et
la disposition de sa dépouille. J'allai ensuite chez M. le duc d'Orléans
qui, après avoir expédié quelques compliments le plus promptement qu'il
put, me mena dans son cabinet où M\textsuperscript{me} la duchesse était
demeurée à l'attendre qu'il eût vidé sa chambre de ceux que les
compliments y avaient amenés. Là, en tiers avec eux, ils me contèrent ce
qui s'était passé entre eux et M\textsuperscript{me} la Duchesse dans la
visite qu'ils lui avaient faite ce même matin, et ensuite entre le roi
et M. le duc d'Orléans sur l'affaire de ses filles avec les princesses
du sang. Comme jusqu'ici je n'en ai dit qu'un mot fort léger et fort en
passant, il en faut parler avec plus d'étendue, sans toutefois entrer
dans le fond que pour le faire entendre, qui se trouvera au long parmi
les Pièces, c'est-à-dire les mémoires donnés au roi de part et d'autre,
et les lettres écrites à lui et à M\textsuperscript{me} de Maintenon, le
jugement rendu par le roi, les considérations et réflexions, toutes
choses qui feraient ici une trop longue digression.

Il faut savoir que M\textsuperscript{me} la duchesse d'Orléans était
peut-être ce qu'il y avait dans le monde de plus orgueilleux, et la
personne aussi qui avait le plus de vues et le plus de suite dans
l'esprit et de ténacité dans ses volontés. Née ce qu'elle était, elle
aurait dû être contente de se voir dans un rang aussi distingué
au-dessus de celui de ses soeurs, mariées pourtant les premières de leur
naissance à des princes du sang. Toutefois ce rang de petite-fille de
France qui se bornait à elle ne lui servait que d'aiguillon à usurper,
comme elle voyait incessamment faire à ses frères et aux princes du sang
sur tout le monde. La pensée que ses enfants ne se-roient que princes du
sang lui était insupportable, et de leur désirer un rang séparé
au-dessus de princes du sang à en former le projet il n'y eut point
d'intervalle. Elle imagina donc un troisième état entre la couronne et
les princes du sang sous le nom d'arrière-petits-fils de France, et se
mit en tête de le former et de le faire passer.

M. le duc d'Orléans, à qui elle en parla, trouva d'abord cela ridicule.
Il était alors comme enterré avec M\textsuperscript{me} d'Argenton, et
comme cela ne regardait ni sa maîtresse ni son genre de vie, sa
négligence et sa facilité naturelle l'entraînèrent peu à peu à laisser
tenter ce qu'il désapprouvait, et à la fin de s'y laisser embarquer
lui-même. L'enfance de M. le duc de Chartres ôtait toute occasion de
montrer des prétentions à son égard, mais leur fille aînée devenait
d'âge et encore plus de figure à être ce qu'on appelle présentée et mise
à la cour et dans le monde. Le premier pas pour arriver à un rang
supérieur aux princes du sang était d'en être distinguée, et pour cela,
il fallait au moins commencer par les précéder. À l'égard des filles
nulle difficulté par l'aînesse de la branche d'Orléans, mais pour les
femmes des princes du sang et de leurs veuves, ce qui était la même
chose, c'est où était l'embarras. Point d'exemple en nulle condition en
France où entre personnes de même rang et de même condition les femmes
ne passassent partout devant les filles, et cet usage s'était toujours
observé parmi les princesses du sang de toutes les branches. Il ne parut
pas prudent de lever tout d'un coup le masque sur la prétention d'un nom
et d'un rang nouveaux et inconnus d'arrière-petit-fils de France.
M\textsuperscript{me} la duchesse d'Orléans eut peur d'effaroucher par
trop\,; mais, voulant le former peu à peu et aller par degrés d'une
prétention à l'autre, elle commença à prétendre que ses filles
précédassent les femmes des princes du sang à titre seulement d'aînesse,
pour, ce point gagné, venir au reste par échelons. Ainsi elle ne
présenta ni ne montra sa fille pour avoir le temps de se tourner.

Elle la fit appeler Mademoiselle tout court au Palais-Royal, n'y en
ayant plus de ce nom depuis le mariage de M\textsuperscript{me} de
Lorraine. Du Palais-Royal, cette dénomination gagna Paris, et le monde
s'y accoutuma\,; les princes du sang plus que les autres ravis qu'une
princesse du sang succédât à un nom qui n'avait jusque-là été usité que
pour deux petites-filles de France. Dans la suite il s'établit tout à
fait\,; le roi n'en dit rien, laissa faire, après quoi
M\textsuperscript{me} la duchesse d'Orléans aurait trouvé fort mauvais
si quelqu'un avait appelé sa fille autrement. Le dédain de la produire
et quelques petites simagrées observées chez elle, quoique dans le plus
petit particulier où on la tenait renfermée, et dont on ne s'accommoda
pas, commença à faire murmurer, et comme cela perça, les princes du sang
se réveillèrent et se tinrent en garde sans mot dire. Enfin il se
présenta des contrats de mariages de particuliers à signer.
Mademoiselle, quoique non présentée ni dans le monde, était d'âge à les
lui faire signer, et ce fut là où la prétention de préséance éclata.
M\textsuperscript{me} la duchesse d'Orléans ne voulut pas qu'elle signât
après les femmes des princes du sang, qui s'en émurent fortement\,;
ainsi Mademoiselle, pour ne leur point céder, ne signa aucun de ces
contrats et la prétention se trouva ainsi formée. Cela fit grand bruit,
et mit une grande aigreur entre M\textsuperscript{me} la duchesse
d'Orléans et M\textsuperscript{me} la Duchesse où leurs amies se
mêlèrent assez mal à propos. La chose éclatée, il la fallut soutenir. Il
se fit des mémoires de part et d'autre, ils doublèrent en réponses et en
répliques avec fort peu de mesure. Les choses en étaient là lorsque M.
le Duc mourut, et le roi différait toujours de décider, par son aversion
naturelle et par la crainte de fâcher ceux qu'il condamnerait. Il y
avait une autre noise dans la maison de Condé.

M\textsuperscript{me} la duchesse du Maine conservait son rang de
princesse du sang, mais elle n'avait point pris de brevet qui le lui
accordât comme avait fait M\textsuperscript{me} de Longueville et les
autres princesses du sang mariées à d'autres qu'à des princes du sang.
Sa raison intérieure était d'appuyer le rang extérieur de prince du sang
dont son mari jouissait et de venir à prétendre qu'il était prince du
sang, et de tourner son rang de princesse du sang fille en celui de
princesse du sang mariée, c'est-à-dire en femme de prince du sang comme
il est le même en tout, excepté les préséances entre elles. Cette
transition était facile à entreprendre. Elle passait sans difficulté
après M\textsuperscript{lle} de Condé, sa soeur aînée, tant qu'elle
vécut\,; avec M\textsuperscript{lle} d'Enghien, sa soeur cadette, point
de difficulté à la précéder. Mais lorsque M\textsuperscript{me} la
Duchesse présenta ses filles et les mit à la cour et dans le monde, il
fallut que la prétention éclatât. Ainsi M\textsuperscript{me} du Maine
évita de se trouver avec elles, et comme elle avait déjà secoué le joug
de la cour, et qu'elle s'était tournée tout aux fêtes, aux plaisirs, à
ne bouger de Sceaux, à ne vivre que pour soi, elle évita assez longtemps
la concurrence sans qu'on s'en aperçût trop\,; mais les contrats de
mariage des particuliers la décelèrent, comme ils avaient fait
M\textsuperscript{me} la duchesse d'Orléans pour Mademoiselle. Néanmoins
elle n'osa parler du rang de M. du Maine\,; mais, laissant à part
qu'elle fût ou non femme d'un prince du sang, elle s'avisa d'alléguer
qu'étant soeur de M. le Duc, elle ne devait pas céder à ses filles, sur
lesquelles elle avait un degré de parenté paternelle, et ne signa plus
aucun contrat de mariage. La prétention était inouïe, et tout cela était
d'autant plus mal cousu, que tant qu'elle avait signé les contrats de
mariage, elle les avait toujours signés au-dessus de son mari, ce qui
n'eût pas été s'il eût été prince du sang, comme M. le prince de Conti
les signait tous au-dessus de M\textsuperscript{me} sa femme, qui était
fille aînée de M. le Prince.

Pour revenir à l'affaire de Mademoiselle, tout ce qui s'était passé
avant la mort de M. le Duc s'était fait avant que j'eusse vu
M\textsuperscript{me} la duchesse d'Orléans, et M. le duc d'Orléans en
était si peu occupé qu'à peine m'en avait-il dit quelque mot en passant,
que j'avais encore moins ramassé. Ce matin-là donc de la mort de M. le
Duc, étant seul avec M. {[}le duc{]} et M\textsuperscript{me} la
duchesse d'Orléans, après m'avoir conté combien leur visite à
M\textsuperscript{me} la Duchesse s'était bien passée, ils me dirent
qu'ils étaient d'avis de se servir de cette occasion pour faire finir la
dispute du rang de leurs filles, qui durait depuis trop longtemps\,; que
dans cet esprit M. le duc d'Orléans avait, dès ce même matin, parlé au
roi et représenté qu'il était de son équité de prononcer, et de sa bonté
de le faire, dans une occasion où toutes les inimitiés suspendues
pouvaient demeurer éteintes si le bois qui entretenait ce feu était
ôté\,; qu'il ne fallait rien espérer entre eux de solide tant que cette
querelle les irriterait\,; que leur état ne comportait aucun autre sujet
de division\,; que ce qu'il venait de se passer entre eux ferait
recevoir avec une soumission douce quelque jugement qui pût
intervenir\,; que le roi, paroissant touché de ses raisons, lui avait
dit qu'il prît garde et qu'il pourrait bien le condamner, à quoi il
n'avait répondu que par une continuation d'instances pour être jugé. Ce
fut la matière de la délibération. Mon avis fut qu'il n'y avait rien de
pis pour eux que de n'être point jugés, parce que la provision était
contre eux fondée sur l'usage de tout temps\,; qu'ainsi, sans être
jugés, ils demeuraient condamnés, puisque Mademoiselle ne pouvait se
trouver nulle part avec les femmes des princes du sang, parce qu'elle ne
pouvait les précéder, et que par la même raison elle ne signait aucun
contrat de mariage. J'ajoutai que, quelque jugement qui intervînt, ils
se retrouveraient toujours sur leurs pieds, parce qu'en perdant même
leur prétention pour leurs filles, ce même jugement déciderait la
préséance de M\textsuperscript{me} la duchesse d'Orléans sur les filles
qu'aurait M. le duc de Berry\,; je crus aussi, en quoi je me trompai
lourdement, que, quoique le roi eût dit à M. le duc d'Orléans qu'il
pourrait bien le condamner, il ne le ferait pas, parce que, s'il avait
eu à le faire, il n'aurait pas résisté à toutes les instances que M. le
Prince et M. le Duc lui avaient faites de juger, dans le temps que M. le
duc d'Orléans était le plus mal avec lui, et ce fut aussi l'avis de M.
{[}le duc{]} et de M\textsuperscript{me} la duchesse d'Orléans\,; nous
convînmes donc, selon que je leur proposai, que M. le duc d'Orléans en
irait dire seulement un mot à M\textsuperscript{me} de Maintenon, pour
se la concilier, et ne la pas fatiguer, et un autre encore au roi avant
qu'il se mît à table. Aussitôt après dîner je retournai chez eux savoir
où ils en étaient.

M\textsuperscript{me} la duchesse d'Orléans s'était mise au lit pour
recevoir les compliments sur la mort de M. le Duc, et M. le duc
d'Orléans et moi, seuls dans sa ruelle, discutâmes avec elle ce qu'il
restait à faire. Il me dit qu'il n'avait pu voir M\textsuperscript{me}
de Maintenon qui ne dînait pas chez elle, et que le roi ne lui avait pas
paru éloigné de juger. Nous conclûmes qu'il fallait concilier et
rafraîchir la mémoire à M\textsuperscript{me} de Maintenon par une
lettre. Nous la fîmes tous trois, moi tenant la plume, et je passai
après avec M. le duc d'Orléans dans son cabinet pour la lui dicter. Il
l'écrivit et l'envoya sur-le-champ, et moi je mis par curiosité le
brouillon dans ma poche, qui se trouvera parmi les Pièces. J'allai de là
rendre l'état des choses à M. de Beauvilliers, qui me promit de parler à
Mgr le duc de Bourgogne, chez lequel M. le duc d'Orléans alla dans
l'après-dînée, et l'entretint longtemps. Ce prince lui dit qu'il était
d'avis de juger, mais qu'il ne pouvait l'assurer s'il serait pour lui.
Après ils se parlèrent avec amitié sur le mariage de M. le duc de Berry
avec Mademoiselle.

Le roi, après sa messe, avait été voir M\textsuperscript{me} la
Duchesse, dolente à merveille dans son lit, et lui avait fort parlé
d'achever d'éteindre toute aigreur entre M\textsuperscript{me} la
duchesse d'Orléans et elle, et d'en saisir cette occasion touchante où
M. {[}le duc{]} et M\textsuperscript{me} la duchesse d'Orléans avaient
si bien fait pour elle et de si bonne grâce. Le roi se trouvait mal à
l'aise de leur division. Son désir de la voir finir lui fit prendre pour
un retour de bonne foi ce que la seule bienséance avait fait dire et
faire des deux côtés en cette journée. Touché d'ailleurs par ce que lui
avait dit M. le duc d'Orléans sur une décision, plus encore de sa lettre
à M\textsuperscript{me} de Maintenon qu'il avait vue, il crut ne pouvoir
trouver de conjoncture plus favorable, puisqu'il fallait bien en venir
un jour à décider, et que, dans ces premiers moments de rapprochement,
les parties seraient plus traitables et recevraient plus doucement sa
décision qu'en aucun autre temps. Rempli de cette pensée, il entra sur
le soir chez M\textsuperscript{me} la duchesse de Bourgogne avant de
passer chez M\textsuperscript{me} de Maintenon, comme il faisait
plusieurs fois tous les jours depuis qu'elle était en couche du roi
d'aujourd'hui, et contre sa coutume, après les premiers moments il en
fit sortir tout le monde. Il ne demeura dans la chambre que
M\textsuperscript{me} de Maintenon, Monseigneur, Mgr le duc de
Bourgogne, et la princesse dans son lit dont tous s'approchèrent, tandis
que le roi envoya querir M. le duc de Berry.

Le roi exposa le fait, ce que M. le duc d'Orléans lui avait dit dans la
journée, M\textsuperscript{me} de Maintenon ce qu'il lui avait écrit\,;
ils convinrent tous qu'il fallait décider. Le roi, qui n'avait pas relu
les mémoires, était plein d'un dernier que feu M. le Duc lui avait donné
depuis peu de jours. Il en avait voulu donner la communication à M. le
duc d'Orléans et la liberté d'y répondre\,; sa paresse et sa négligence
lui persuadèrent que l'un et l'autre était inutile, que ce ne pouvait
être que des redites et qu'il n'avait pas besoin de rien ajouter aux
mémoires qu'il avait donnés. Ainsi il ne vit point ce dernier mémoire
qui pourtant avait persuadé le roi contre la prétention de Mademoiselle.
Il montra un peu ce penchant, mais il laissa toute liberté de discuter
l'affaire et d'opiner, parce que, dans la vérité, il ne se souciait
guère qui de ses deux bâtardes l'emportât. Monseigneur, de longue main
bien instruit et de nouveau recordé, qui haïssait M. le duc d'Orléans à
ne s'en pas contraindre, qui y était sans cesse entretenu, qui aimait
M\textsuperscript{me} la Duchesse, opina de toute sa force pour les
femmes des princes du sang. Mgr le duc de Bourgogne, sur lequel de plus
anciens et de plus solides principes que ceux des mémoires respectifs
faisaient impression, appuya le même avis. On peut ne pas douter que M.
le duc de Berry n'en ouvrit pas un autre. La décision arrêtée, le roi
considéra qu'en ayant fait une pour la préséance de ses filles sur
Madame qu'il ne voulait pas changer, et désirant aussi donner quelque
consolation à M\textsuperscript{me} la duchesse d'Orléans, fit
l'honnêteté à M. le duc de Berry de lui demander s'il n'aurait point de
peine de céder aux filles de Mgr le duc de Bourgogne, qui tout de suite
répondit qu'il n'en aurait point. Ainsi il fut arrêté que les filles de
France non mariées précéderaient, excepté la Dauphine ou la fille de
France directe, les femmes de leurs frères cadets\,; mais que les
petites-filles de France, filles, seraient précédées par les femmes des
fils de France, que par conséquent M\textsuperscript{me} la duchesse
d'Orléans serait assurée de précéder les filles de M. le duc de Berry,
et que les femmes des princes du sang précéderaient toutes les filles
des petits-fils de France et des princes du sang, aînés de leurs maris.

Après cela vint l'article de M\textsuperscript{me} la duchesse du Maine,
que le roi voulut décider en même temps. Pour cela il fut réglé que le
jugement dénoncerait que les princesses du sang, filles, se
précéderaient suivant leur aînesse, ce qui sapait la nouveauté prétendue
par M\textsuperscript{me} du Maine de précéder, comme tante, les filles
de feu M. le Duc son frère, non mariées, parce qu'elle avait un degré
sur elles, et que les petites-filles de France qui épouseraient un
prince du sang, ou un qui ne le serait pas, et les princesses du sang
qui épouseraient un autre qu'un prince du sang, ne conserveraient point
leur rang sans un brevet qui le leur accordât. Ainsi tombait le manége
de M\textsuperscript{me} du Maine en faveur de son mari, qui, avec tout
son extérieur de prince du sang, ne l'était pas, et le roi dit qu'il
ferait expédier un brevet à M\textsuperscript{me} la duchesse du Maine,
en cas qu'elle n'en eût pas déjà un pour conserver son rang. Ainsi elle
fut déclarée ce qu'elle était, c'est-à-dire princesse du sang, fille,
quoique mariée, et marchant au rang de son aînesse après ses nièces.
Tout fut consulté entre eux, excepté l'article des filles de France, que
le roi ne mit pas en délibération, après l'honnêteté faite à M. le duc
de Berry, et la différence qu'il voulut mettre entre les filles et les
petites-filles de France, pour relever d'autant les premières, et
gratifier M\textsuperscript{me} la duchesse d'Orléans, dont M. le duc de
Berry ne s'aperçut pas, et que les autres princes n'osèrent relever.

Tout étant ainsi unanimement convenu et résolu, le roi imposa le secret
jusqu'à la déclaration qu'il en ferait après son souper. Pour mieux
comprendre ce qu'il s'y passa, il faut expliquer en deux mots la
mécanique de l'après-soupée de tous les jours. Le roi sortant de table
s'arrêtait moins d'un demi-quart d'heure, le dos appuyé contre le
balustre de sa chambre. Il trouvait là en cercle toutes les dames qui
avaient été à son souper et qui l'y venaient attendre un peu avant qu'il
sortît de table, excepté les dames assises qui ne sortaient qu'après
lui, et qui, à la suite des princes et princesses qui avaient soupé avec
lui, venaient une à une faire une révérence, et achevaient de former le
cercle debout où les autres dames avaient laissé un grand vide pour
elles, et tous les hommes derrière. Le roi s'amusait à remarquer les
habits, les contenances et la grâce des révérences, disait quelque mot
aux princes et aux princesses qui avaient soupé avec lui et qui
fermaient le cercle auprès de lui des deux côtés, puis faisait la
révérence aux dames à droite et à gauche, qu'il faisait encore une fois
ou deux en s'en allant, avec une grâce et une majesté nonpareilles,
parlait quelquefois, mais fort rarement à quelqu'une en passant, entrait
dans le premier cabinet où il s'arrêtait pour donner l'ordre, et
s'avançait après dans le second cabinet, les portes du premier au second
demeurant toutes ouvertes. Là il se mettait dans un fauteuil, Monsieur,
quand il vivait, dans un autre\,; M\textsuperscript{me} la duchesse de
Bourgogne, Madame, mais seulement depuis la mort de Monsieur,
M\textsuperscript{me} la duchesse de Berry après son mariage, et les
trois bâtardes, M\textsuperscript{me} du Maine quand elle était à
Versailles, sur des tabourets des deux côtés en retour. Monseigneur, Mgr
le duc de Bourgogne, M. le duc de Berry, M. le duc d'Orléans, les deux
bâtards, feu M. le Duc, comme mari de M\textsuperscript{me} la Duchesse
quand il vivait, et, depuis, les deux fils de M. du Maine, quand ils
furent un peu grands, et d'Antin, depuis qu'il eut les bâtiments, tous
debout. M. d'O, comme ayant été gouverneur de M. le comte de Toulouse,
avec les quatre premiers valets de chambre, Chamarande qui en avait
conservé les entrées, les quatre premiers valets de garde-robe, les
premiers valets de chambre de Monseigneur et des deux princes ses fils,
le concierge de Versailles et les garçons bleus étaient dans le cabinet
des Chiens, qui flanquait celui où était le roi, la porte entre-deux
tout ouverte, dans laquelle les principaux se tenaient, dont
quelques-uns demeuraient dans le premier cabinet avec les dames
d'honneur des princesses qui étaient avec le roi, les deux dames du
palais de jour de M\textsuperscript{me} la duchesse de Bourgogne, et les
dames d'atours des filles de France. Ainsi on voyait et on entendait, de
ce premier cabinet et de celui des Chiens, ce qui se disait et se
faisait dans celui où était le roi, qui en arrivant y trouvait les
princes et les princesses qui avaient cette entrée., et qui ne
mangeaient pas avec lui. Le nouveau M. le Duc et M. le prince de Conti,
depuis son mariage, eurent cette entrée\,: l'un comme fils de
M\textsuperscript{me} la Duchesse, l'autre comme son gendre. Partout
cela était de même, suivant la disposition des lieux, sinon qu'à Marly
les dames que M\textsuperscript{me} la duchesse de Bourgogne amenait se
tenaient les après-soupées dans le cabinet du roi avec les dames
d'honneur, et qu'à Fontainebleau il n'y avait qu'un seul cabinet fort
grand, où tout ce qui vient d'être nommé demeurait avec le roi, les
dames d'honneur duchesses assises, joignant les princesses et tout de
suite, les autres debout ou par terre sur le parquet, où même on ne
donnait point de carreau à la maréchale de Rochefort\,; les valets s'y
tenaient peu, et peu à la fois par discrétion.

Cela entendu, le roi, entré dans le second cabinet, appela M. et
M\textsuperscript{me} la duchesse d'Orléans et M. le comte de Toulouse,
et, au lieu de s'asseoir à l'ordinaire, les alla attendre à un coin du
cabinet, où il leur dit ce qu'il avait décidé. M. le duc d'Orléans, peu
capable de prendre les choses à coeur, et qui s'était laissé entraîner
dans cette affaire plutôt qu'il n'y était entré, se contenta aisément,
pour M\textsuperscript{me} la duchesse d'Orléans, elle ne répondit pas
un seul mot. De là le roi, se faisant suivre par le comte de Toulouse,
alla à un autre coin, où il appela M\textsuperscript{me} la princesse de
Conti sa fille, la seule d'entre les princesses du sang qui fût là, et
lui dit aussi le jugement, qui parut surprise et fort aise. Enfin le
roi, toujours avec M. le comte de Toulouse, passa à un autre endroit où
il appela M. {[}le duc{]} et M\textsuperscript{me} la duchesse du Maine,
à qui il dit aussi ce qui les regardait, et qui en parurent fort
mortifiés. Ensuite le roi s'alla asseoir à l'ordinaire, et le temps du
cabinet jusqu'au coucher s'acheva fort sérieusement.

Le lendemain, mercredi des Cendres, le roi déclara son jugement le matin
au conseil, qui y fut fort applaudi, et ensuite du public. Il ajouta
qu'il l'avait tout écrit de sa main, mais qu'il y voulait retoucher
quelque chose. Il le dressa de manière que les enfants en directe,
quoique non enfants des rois, furent déclarés fils et filles de France,
ce qui, par exemple, regardait M. le duc de Berry\,; et il confirma
tacitement le nouvel état et rang de petits-fils et petites-filles de
France. Tout demeura encore comme secret jusqu'au 12 du même mois de
mars, que le roi donna son jugement écrit de sa main, en onze articles,
à Pontchartrain, comme ayant la maison du roi dans son département de
secrétaire d'État, qui l'expédia et le signa seul. Le roi n'y voulut
point d'autres formes ni même sa signature, pour que sa décision, ainsi
toute nue, sans sceau, sans signature des autres secrétaires d'État,
sans vérilication au parlement, tînt plus de sa toute-puissance\,; c'est
au moins toute la raison qu'on en put imaginer. En même temps
Pontchartrain eut ordre d'expédier pour la duchesse du Maine le brevet
de conservation de rang et honneurs de princesse du sang fille, qu'elle
n'avait eu garde de demander, et dont elle se serait si volontiers
passée.

Il ne laissa pas d'être remarquable que le jour de la mort de M. le Duc
eût par cela même fait éclore ce que tout son crédit et celui de M. le
Prince, toute leur ardeur et leur empressement, et toutes les adresses
de M\textsuperscript{me} la Duchesse n'avaient pu obtenir de son vivant.
Elle oublia un peu son état si récent de veuve, dans la sensibilité
très-marquée de ce qu'elle venait de gagner, en quoi
M\textsuperscript{me} la princesse de Conti, sa soeur, parut beaucoup
plus modérée. M\textsuperscript{me} la Duchesse en reçut même les
compliments de ses familiers, ce qui fut imité à Paris par
M\textsuperscript{me} la Princesse et. M\textsuperscript{me} la
princesse de Conti.

\hypertarget{chapitre-vii.}{%
\chapter{CHAPITRE VII.}\label{chapitre-vii.}}

1710

~

{\textsc{Premiers pas directs pour le mariage de Mademoiselle avec M. le
duc de Berry.}} {\textsc{- Désespoir et opiniâtreté de
M\textsuperscript{me} la duchesse d'Orléans, du jugement du rang entre
les princesses du sang, femmes et filles.}} {\textsc{- Obsèques de M. le
Duc.}} {\textsc{- Reformations où d'Antin pousse Livry, premier maître
d'hôtel, sauvé avec hauteur par le duc de Beauvilliers.}} {\textsc{-
Pension de quatre-vingt-dix mille livres à M\textsuperscript{me} la
Duchesse.}} {\textsc{- Visites en cérémonie.}} {\textsc{- Ma conduite
avec M\textsuperscript{me} la Duchesse.}} {\textsc{- Rang pareil à celui
de M. du Maine donné sans forme à ses enfants.}} {\textsc{- Scène
très-singulière de la déclaration du rang des enfants du duc du Maine,
le soir, dans le cabinet du roi.}} {\textsc{- Les deux frères bâtards,
comment ensemble.}} {\textsc{- Triste accueil public à ce rang.}}
{\textsc{- Ma conduite sur ce rang.}} {\textsc{- Conduite du comte de
Toulouse sur ce rang.}} {\textsc{- Repentir du roi, prêt à révoquer ce
rang.}} {\textsc{- Adresse de M. du Maine et de M\textsuperscript{me} de
Maintenon, qui se servent de mon nom, dont M\textsuperscript{me} la
duchesse de Bourgogne me fait demander l'explication.}} {\textsc{-
Survivances des charges de M. du Maine données à ses enfants.}}
{\textsc{- Propos à moi du duc du Maine.}} {\textsc{- Villars reçu pair
au parlement.}}

~

Le lendemain de ce jugement, je vis sortir M. le duc d'Orléans du
cabinet du roi, comme j'entrais dans sa chambre\,; je l'attendis et lui
demandai où il en était. «\,Nous sommes condamnés, me dit-il à
l'oreille,\,» et, me prenant par le bras, «\,venez-vous-en, ajouta-t-il,
voir M\textsuperscript{me} la duchesse d'Orléans.\,» Je la crus outrée,
et n'y voulais point aller, mais il m'y traîna. Nous la trouvâmes dans
la niche de sa petite chambre obscure sur la galerie, une table devant
elle avec du café. Dès que je l'envisageai, ses larmes, qui n'avaient
guère tari, redoublèrent. Je me tins à la porte pour sortir doucement\,;
elle le sentit aussitôt, me rappela, et me força de m'asseoir\,! Là nous
nous lamentâmes à l'aise, puis elle me fit lire une lettre de sa main à
M\textsuperscript{me} de Maintenon par laquelle elle lui exposait ses
peines, et insistait sur le mariage de Mademoiselle avec M. le duc de
Berry, pour être au moins accordé et déclaré, si dès à présent on ne
voulait pas encore passer outre. Je n'ai jamais vu une lettre si forte,
si belle, écrite avec tant de justesse, de délicatesse, de tour, ni dans
son éloquence d'un air plus simple et plus naturel. M. le duc d'Orléans
me conta comment le jugement avait été rendu, puis au cabinet la veille
leur avait été déclaré. Il ajouta à M\textsuperscript{me} la duchesse
d'Orléans et à moi qu'il venait de toucher un mot au roi du mariage de
Mademoiselle qui le consolerait de tout\,; sur quoi, pour toute réponse,
le roi lui avait dit un\,: «\, Je le crois bien,\,» d'un ton sec et avec
un sourire amer et moqueur, ce qui acheva de nous affliger.

M\textsuperscript{me} la duchesse d'Orléans feignit une migraine pour ne
voir personne, pas même Mademoiselle, qu'un moment sur le soir, qu'elle
renvoya aussitôt et qu'elle fit tenir enfermée dans sa chambre. Le
lendemain elle alla fuir le monde à Saint-Cloud et ne vit
M\textsuperscript{me} la Duchesse que le troisième jour. La douleur fut
telle que tout le monde la vit, et qu'elle fut incapable de conseil et
de contrainte. Outre le chagrin d'avoir été condamnée et le dépit de
voir M\textsuperscript{me} la Duchesse l'emporter, elle en sentait un
autre plus intime et dont elle n'osait faire semblant\,: c'était de voir
par ce seul coup avorter tous ces projets de nom et de rang
d'arrière-petit-fils de France, et de voir ses enfants bien et
solidement constitués et déclarés princes du sang, sans nulle
distinction des autres princes du sang, et c'est ce qui la poignait dans
le plus intime de l'âme. Elle résolut de bouder, de s'éloigner du roi,
de tenir plus que jamais Mademoiselle cachée, et de céder en tout au
désespoir qui la possédait, qu'elle couvrait d'un voile de politique
pour embarrasser le roi, disait-elle, et l'obliger à en venir au mariage
qu'elle désirait. M. le duc d'Orléans, infiniment moins fâché et, pour
cette fois, beaucoup plus raisonnable qu'elle, combattait son opinion, à
laquelle il fallut pourtant céder pour quelques temps. On était en
carême, le roi allait trois fois la semaine au sermon, où les princesses
étaient en rang\,; elle s'opiniâtra à ne vouloir point que Mademoiselle
s'y trouvât. Pour achever de suite cette matière, elle voulut faire un
voyage à Paris, tant pour s'éloigner du roi d'une manière plus marquée
et moins accoutumée que pour chercher consolation dans la pleine
jouissance du Palais-Royal, et d'une cour dans Paris, pour la première
fois de sa vie, par la défaite de M\textsuperscript{me} d'Argenton. Le
succès passa ses espérances\,: elle y régna sur la cour de M. le duc
d'Orléans, qui auparavant la regardait à peine, et ses appartements ne
désemplirent point de tout ce qu'il y eut de plus distingué. Transportée
d'un état si brillant et si nouveau pour elle, elle me témoigna souvent
combien elle était sensible à tout ce que j'avais fait. La bienséance
qui, sitôt après la mort de M. le Duc, les empêchait de se montrer à
l'Opéra en public, lui procura un nouveau plaisir. Elle y alla dans la
petite loge faite exprès pour M\textsuperscript{me} d'Argenton, de qui
elle triompha en toutes les façons, et M. le duc d'Orléans et elle
m'obligèrent d'y aller avec eux.

Huit jours se passèrent dans cette pompe, après lesquels il fallut
retourner à Versailles, où ce voyage ne fut pas désapprouvé. Cependant,
M\textsuperscript{me} la duchesse d'Orléans n'en devint pas plus
traitable. La duchesse de Villeroy y échoua\,; et M\textsuperscript{me}
la duchesse de Bourgogne, qui résolut de lui parler et qui le fit avec
beaucoup d'esprit, d'amitié et d'adresse, n'en eut pas plus de
contentement. Elle voyait que cette conduite gâtait tout pour le mariage
de Mademoiselle avec M. le duc de Berry, et elle le désirait pour les
raisons qui s'en verront en leur temps. M\textsuperscript{me} la
duchesse d'Orléans demeurait ferme à gagner Pâques sans montrer
Mademoiselle, temps après lequel il n'y avait plus de lieu public où les
princesses fussent en rang. M. le duc d'Orléans, qui sentait le poids de
cette conduite par rapport à ce mariage, lui en parla un jour en ma
présence plus fortement qu'à l'ordinaire, et peu à peu il s'échauffa,
contre son ordinaire, jusqu'à lui toucher sa naissance d'une manière à
l'affliger et à m'embarrasser beaucoup. Mon parti fut le silence et de
saisir le premier moment que je pus de passer de ce cabinet dans celui
de M. le duc d'Orléans. Il y vint peu après encore tout en colère, et
moi qui y étais aussi j'osai le gronder tout de bon.

Je fus forcé d'aller le lendemain matin chez M\textsuperscript{me} la
duchesse d'Orléans pour raisonner seul avec elle. Elle me fit souvenir
des propos de la veille, je lui avouai tout ce que j'en avais dit à M.
le duc d'Orléans immédiatement après. À peu de jours de là, M. de
Beauvilliers, qui s'intéressait fort aussi au mariage, m'arrêta dans la
galerie pour me représenter combien il importait à cette affaire que
Mademoiselle parût\,; qu'il était bien informé que cette opiniâtreté
retombait avec un grand venin sur M\textsuperscript{me} la duchesse
d'Orléans\,; qu'on se servait de cela pour faire craindre au roi, et
jusqu'à M\textsuperscript{me} la duchesse de Bourgogne, cette même
opiniâtreté et sa hauteur\,; qu'il savait que l'impression en était
commencée\,; qu'il n'y avait pas un moment à perdre pour l'en avertir,
et qu'il me conjurait de le faire à l'heure même sans le nommer. Je lui
racontai à quel point la chose était entrée de travers dans la tête de
M\textsuperscript{me} la duchesse d'Orléans, les tentatives inutiles
même de M\textsuperscript{me} la duchesse de Bourgogne, et que, après ce
que je lui en disais, je croyais tout inutile, et que je ne ferais que
me rendre désagréable. Quoi que je pusse dire, il persista tellement,
que j'obéis à l'heure même. Je trouvai M\textsuperscript{me} la duchesse
d'Orléans seule. Elle me laissa tout dire, me remercia froidement, et
avec un dépit étouffé par la politesse me dit que cela ne l'ébranlerait
pas.

Quatre jours après, M\textsuperscript{me} la duchesse de Bourgogne
envoya chercher Mademoiselle, lui représenta avec une bonté de mère ce
qu'elle risquait pour un vain dépit de sa mère qui ne changerait pas la
décision faite\,; la conjura de se servir de tout son esprit et de tout
son crédit auprès d'elle pour en obtenir de paraître. Ce dernier effort
eut tout son effet. Je fus tout étonné que Mademoiselle allât au premier
sermon d'après cette semonce, habillée en rang. J'allai ce même jour
chez M. le duc d'Orléans, qui me mena chez M\textsuperscript{me} la
duchesse d'Orléans. Nous la trouvâmes au lit tout en larmes, et ne cessa
de pleurer de tout le jour. Elle ne voulut point voir Mademoiselle que
déshabillée, et fut longtemps à s'accoutumer à son grand habit.
Toutefois elle l'alla présenter aux personnes royales, après quoi elle
l'envoya chez les princesses du sang. M\textsuperscript{me} la Duchesse
eut la bonté de la manger de caresses, M\textsuperscript{me} la
princesse de Conti en usa avec elle avec une légèreté très-polie. Depuis
cela Mademoiselle parut quelquefois, pour conserver le mérité de céder
au jugement du roi. Ainsi cette décision, précipitée par des
conjonctures qui persuadèrent le roi qu'elle finirait toute division
entre ses filles, ne fit qu'augmenter l'aigreur entre les deux soeurs,
que leurs prétentions à M. le duc de Berry pour leurs filles porta
bientôt au comble. M\textsuperscript{me} la duchesse d'Orléans reviendra
si souvent dans la suite, par différentes occasions principales, que
j'ai cru me devoir étendre sur des faits qui mieux que des paroles
commencent à la faire connaître.

On trouva à l'ouverture de M. le Duc une espèce d'excroissance ou de
corps étrange dans la tête, qui parvenu à une certaine grosseur le fit
mourir. Le roi ordonna que sa pompe funèbre fût en tout beaucoup moindre
que celle de M. le Prince, qui avait la qualité de premier prince du
sang. M. le prince de Conti, sa queue portée par le marquis d'Hautefort,
et accompagné du duc de La Trémoille, sa queue portée par un
gentilhomme, alla de la part du roi donner l'eau bénite avec les
cérémonies qui ont été décrites ailleurs. Les mêmes parents conviés à
celles de M. le Prince accompagnèrent M. le Duc pour recevoir M. le
prince de Conti, et le coeur aux jésuites. Le corps fut porté à Valery
sans cérémonie, où M. le Duc seul se trouva, et coucha en chemin à
Saint-Ange, belle et singulière maison de Caumartin, où feu M. le Duc
avait couché de même en rendant moins d'un an auparavant le même devoir
à M. son père. De service ni d'oraison funèbre, il n'y en eut point.
Personne ne se soucia assez de M. le Duc pour s'importuner de l'un, et
la matière de l'autre eût été fort difficile. Au retour de ce voyage le
roi mit M. le Duc sous la tutelle de d'Antin pour la gestion de ses
biens, comme il y était déjà pour ses charges, de concert avec
M\textsuperscript{me} la Duchesse, et lui défendit de découcher des
lieux où il serait, sans permission. Il eut l'entrée du cabinet
l'après-soupée comme fils de M\textsuperscript{me} la Duchesse, et
d'Antin fut aussi chargé d'avoir l'oeil sur sa conduite. Ce fut par lui
que M\textsuperscript{me} la Duchesse envoya au roi le portefeuille de
feu M. le Duc, qui regardait la maison du roi où il projetait de
réformer beaucoup d'abus et de pillages. D'Antin, peu ami du duc de
Beauvilliers, y travailla seul avec le roi plusieurs {[}fois{]}, qui
cassa et interdit plusieurs maîtres d'hôtel et régla quantité de
choses\,; ainsi Livry, premier maître d'hôtel, y courut un grand risque,
quoique, pour tout ce qui est de la bouche, sa charge depuis les Guise,
grands maîtres, ne dépendît plus de celle-là\,; mais le duc de
Beauvilliers, dont il avait épousé la soeur pour rien, le prit si haut
et si ferme, contre son ordinaire, qu'il en fut quitte pour quelques
réformations légères, après la peur d'être chassé avec deux maîtres
d'hôtel qui eurent ordre de vendre leur charge.

M\textsuperscript{me} la Duchesse était retombée dans une affliction qui
surprit tout le monde. Elle disait à ses familières que l'humeur de M.
le Duc à son égard était fort changée depuis quelque temps, et à
d'autres moins intimes, elle ne se cachait pas, pour que cela revînt
qu'elle le perdait en des conjonctures si fâcheuses par rapport à son
bien et à l'état de ses affaires et de celui de ses filles, qu'elle ne
savait que devenir, dont elle fit bien sonner la pauvreté. Elle avait eu
un million en mariage, quantité de pierreries et vingt-cinq mille livres
de douaire, etc., avec quoi elle trouvait n'avoir pas de quoi vivre. On
verra dans la suite combien énormément elle et les siens y ont su
pourvoir. Avec ses manières larmoyantes, elle arracha du roi, et assez
malgré lui, tardivement et de mauvaise grâce, trente mille écus de
pension. Monseigneur, transporté de joie, lui en alla apprendre la
nouvelle\,; alors les larmes s'essuyèrent, et la belle humeur revint
tout à fait. Elle vit tout le monde en cérémonie. Elle était sur son lit
en robe de veuve, bordée et doublée d'hermines, pareil à celui des
duchesses veuves, et comme elles ayant le couvre-chef. C'est une
coiffure singulière, basse, de simple toile de Hollande, qui enveloppe
la tête sans rien autre par-dessus, qui tombe amplement sur les épaules
qu'elle enveloppe aussi, et qui est fort longue, mais plus courte de
beaucoup que la queue herminée de la robe, et dont la longueur est
proportionnée sur celle de la queue. Les duchesses sont les dernières
qui aient droit de l'une et de l'autre. La queue de la reine est de onze
aunes, les filles de France en ont neuf, les petites-filles de France
sept, les princesses du sang cinq, les duchesses trois. L'invention du
rang de petites-filles de France a fait croître la queue de la reine et
celle des filles de France chacune de deux aunes. Les queues sans deuil,
au mariage du roi et autres pareilles cérémonies, sont de la même
longueur pour les mêmes, qui alors, au lieu du couvre-chef des mêmes en
veuves, ont une mante qui est une gaze ou un réseau d'or, ou d'argent
attaché au derrière de la tête, qui se rattache sur les épaules, tombe à
terre sur la queue et la dépasse un peu, mais bien plus étroite, et qui
même ne cache pas la taille.

M. le Duc, en manteau, reçut aussi les visites dans l'appartement de feu
M. le Duc. Il y avait à la porte de l'un et de l'autre des piles de
mantes de deuil et de manteaux longs, desquels personne ne fut exempt.
Ceux qui en avaient de chez eux et ceux qui n'en prirent qu'à la porte,
hommes et femmes, en usèrent avec la même affectation d'indécence qu'on
avait marquée aux visites de la mort de M. le Prince. M. le Duc ni
M\textsuperscript{me} la Duchesse ne firent pas semblant de s'en
apercevoir. M. le Duc reçut tout le monde debout, et conduisit
exactement tous les ducs et tous les princes étrangers jusqu'à la
dernière extrémité de son appartement. M. du Maine, de même qu'on vit
aussi en manteau, et M\textsuperscript{me} du Maine en mante, et qui y
furent aussi légers sur l'indécence affectée des accoutrements que M. le
Duc et M\textsuperscript{me} la Duchesse. M\textsuperscript{lle}s ses
filles en mante étaient dans sa chambre, qui conduisirent toutes les
duchesses et les princesses étrangères à la porte de la chambre, et
M\textsuperscript{me} de Laigle, dame d'honneur de M\textsuperscript{me}
la Duchesse, au bout de l'antichambre.

Depuis l'affaire de M\textsuperscript{me} de Lussan, je n'avais eu aucun
lieu de me plaindre de M\textsuperscript{me} la Duchesse. Ce qui lui
était échappé alors, elle l'avait hautement nié. Elle avait fort affecté
de faire toutes sortes d'honnêtetés à M\textsuperscript{me} de
Saint-Simon, lorsque nous ne la voyons point, toutes les fois qu'elle
l'avait rencontrée, et à elle, sur moi. Lorsque nous la vîmes sur la
mort de M. le Prince, elle en avait redoublé. Elle n'avait eu aucune
part à la noirceur de feu M. le Duc sur moi absent, lors de la mort de
M. le Prince et de l'affaire des manteaux, qui nous avait fait cesser
encore une fois de les voir. Nous crûmes donc devoir laisser toute
l'iniquité sur feu M. le Duc seul, et nous priâmes M\textsuperscript{me}
de Laigle de lui dire que ç'avait été sur le compte de feu M. le Duc que
nous ne l'avions point vue à sa dernière couche, avec les propos
convenables. M\textsuperscript{me} de Laigle était fille de M. de Raré
qui, avec toute sa famille, avait toujours {[}eu{]} un grand attachement
d'amitié pour mon père et pour mon oncle. Son mari était de même de tout
temps avec mon père, et son voisin à la Ferté. Elle était extrêmement de
nos amies et avec confiance surtout, et femme de beaucoup de sens et
d'esprit, et qui se faisait fort considérer. Elle fut ravie de la
commission, et de cette façon nous vîmes M\textsuperscript{me} la
Duchesse qui y parut fort sensible. Nous la vîmes toujours aux occasions
depuis, M\textsuperscript{me} de Saint-Simon fort rarement\,; davantage
nous n'eûmes plus nulle occasion de nous plaindre d'elle. J'ai voulu
achever tout de suite ce qui la regardait sur son veuvage et à mon
égard, pour n'avoir plus à y revenir.

M. du Maine, outré du règlement entre les princesses du sang qui
renversait l'échelon que M\textsuperscript{me} sa femme lui préparait
adroitement pour s'élever jusqu'à être prince du sang lui-même, dont ce
règlement et le brevet de conservation de rang à M\textsuperscript{me}
du Maine le faisait tomber, imagina qu'il pouvait profiter de la
faiblesse du roi pour sa douleur. Il trouva l'occasion belle, parce que
le tapis se trouvait nettoyé. La mort de M. le prince de Conti, de M. le
Prince et de M. le Duc ne laissait que des enfants dont le plus vieux
avait dix-sept ans, venait d'être comblé, et se trouvait sous la main de
d'Antin\,; M. le duc d'Orléans, peu soucieux, négligent, mal averti, a
peine raccommodé avec le roi et avec M\textsuperscript{me} sa femme plus
bâtarde de coeur et d'affection que lui-même. Ainsi point d'intérêts
directs et plus grands que lui qui pussent l'embarrasser\,; et à l'égard
des fils de France, ce n'était rien au roi que les sauter à joints
pieds, sans que pas un d'eux, à commencer par Monseigneur, osât dire une
parole. Pour tout le reste du monde, c'était une cour anéantie,
accoutumée à toute sorte de jougs et à se surpasser les uns les autres
en flatteries et en bassesses. Il songea donc à tirer sur le temps, et à
obtenir, tout d'un coup, pour ses enfants tout ce qu'il avait obtenu
d'honneurs et de rang à la longue, par insensibles degrés, et par tant
de degrés entés l'un sur l'autre par des usurpations, des introductions
d'usages, des confirmations verbales, enfin par des réalités existantes,
comme sa séance au parlement telle qu'il l'y avait.

Son grand ressort était M\textsuperscript{me} de Maintenon qui l'avait
élevé, à qui il avait sacrifié M\textsuperscript{me} de Montespan, qu'il
avait toujours depuis ménagée avec tout l'art où il était grand maître,
laquelle aussi l'aimait plus tendrement qu'aucune mie, ni qu'aucune
nourrice, et avec un plus entier abandon. C'était par elle qu'il s'était
poulié du néant à la grandeur en laquelle il se voyait, et qu'une
M\textsuperscript{me} Scarron, devenue reine, trouvait merveilleusement
juste. Par les mêmes motifs, elle entra dans ses désirs pour la grandeur
de ses enfants, et dans la facilité qu'il lui en montra par la nullité
des princes du sang morts ou enfants, et par celle d'une cour
entièrement débellée et asservie. Il n'eut pas de peine à lui persuader
qu'il n'y avait aucune difficulté à craindre de la part des fils de
France, ni de M. le duc d'Orléans, au moindre signe de la volonté du
roi. Quelque faiblesse qu'il eût pour ses bâtards, et pour celui-ci sur
tous les autres, quelque absolu qu'il fût et qu'il se piquât d'être, on
a pu remarquer que, excepté les mariages de ses filles et les
gouvernements et les charges de ses fils, ce qu'il fit d'ailleurs pour
eux ne fut que peu à peu, sans forme, sans rien d'écrit, par une
usurpation d'usages, à reprises, et toujours entraîné au delà de ce
qu'il sentait, jusqu'à ce que le procès de M. de Luxembourg ayant excité
celui de M. de Vendôme, il fut poussé à remettre en vigueur l'édit
mort-né d'Henri IV, comme ne faisant rien de nouveau, et qu'ayant
affermi ses deux fils, par le simple usage, dans tout l'extérieur des
princes du sang au dedans de sa cour, il le leur donna de même dans ses
armées, et voulut enfin y soumettre les ambassadeurs, ce qui ne s'acheva
pas sans une résistance qui subsiste encore dans les nonces qui
deviennent cardinaux, et qui a été enfin vaincue dans tous les autres,
mais toujours sans rien écrire et sans formes. Rien n'est si précis que
la répugnance qu'il eut au mariage de M. du Maine, par la raison qu'il
en allégua, et que ce qu'il dit au maréchal de Tessé allant en Italie,
où il devait trouver M. de Vendôme à la tête d'une armée. Toutes ces
choses se trouvent remarquées ici en leur temps, et de quelle façon, et
combien après il s'écarta dans tous ces faits, comme malgré soi, à des
grandeurs nouvelles en leur faveur, et en celle de M. de Vendôme à cause
d'eux. Ce fut en cette occasion la même chose\,: même résistance, même
vue de l'énormité qui lui était proposée, et pour fin même entraînement,
comme malgré lui, et toujours presque sans forme. Le combat ne fut pas
long, puisqu'il ne fut commencé qu'après le 4 mars, jour de la mort de
M. le Duc et de la décision du rang des princesses du sang entre elles,
et qu'il finit le 16 du même mois par la victoire de M. du Maine.

Quand elle fut résolue entre le roi, M\textsuperscript{me} de Maintenon
et lui, il fut question de la déclarer\,; et cette déclaration produisit
la scène la plus nouvelle et la plus singulière de tout ce long règne,
pour qui a connu le roi, et quelle était l'ivresse de sa
toute-puissance. Entrant le samedi au soir, 15 mars, dans son cabinet,
après souper, à Versailles, et l'ordre donné à l'ordinaire, il s'avança
gravement dans le second cabinet, se rangea vers son fauteuil sans
s'asseoir, passa lentement les yeux sur toute la compagnie, à qui il
dit, sans adresser la parole à personne, qu'il donnait aux enfants de M.
du Maine le même rang et les mêmes honneurs dont M. du Maine
jouissait\,; et sans un moment d'intervalle, marcha vers le bout du
cabinet le plus éloigné, et appela Monseigneur et Mgr le duc de
Bourgogne. Là, pour la première fois de sa vie, ce monarque si fier, ce
père si sévère et si maître s'humilia devant son fils et son petit-fils.
Il leur dit que, devant tous deux régner successivement après lui, il
les priait d'agréer le rang qu'il donnait aux enfants du duc du Maine,
de donner cela à la tendresse qu'il se flattait qu'ils avaient pour lui,
et à celle qu'il se sentait pour ses enfants et pour leur père\,; que,
vieux comme il était, et considérant que sa mort ne pouvait être
éloignée, il les leur recommandait étroitement, et avec toute l'instance
dont il était capable\,; qu'il espérait qu'après lui ils les voudraient
bien protéger par amitié pour sa mémoire. Il prolongea ce discours
touchant assez longtemps, pendant lequel les deux princes un peu
attendris, les yeux fichés à terre, se serrant l'un contre l'autre,
immobiles d'étonnement et de la chose et des discours, ne proférèrent
pas une unique parole. Le roi, qui apparemment s'attendait à mieux et
qui voulait les y forcer, appela M. du Maine qui, arrivant à eux de
l'autre bout du cabinet, où tout était cependant dans le plus profond
silence, le roi le prit par les épaules, et en s'appuyant dessus pour le
faire courber au plus bas devant les deux princes, le leur présenta,
leur répéta en sa présence que c'était d'eux qu'il attendait après sa
mort toute protection pour lui, qu'il la leur demandait avec toute
instance, qu'il espérait cette grâce de leur bon naturel, et de leur
amitié pour lui et pour sa mémoire, et il finit par leur dire qu'il leur
en demandait leur parole.

En cet instant les deux princes se regardèrent l'un l'autre, sans
presque savoir si ce qui se passait était un songe ou une réalité, sans
toutefois répondre un mot jusqu'à ce que, plus vivement pressés encore
par le roi, ils balbutièrent je ne sais quoi qui ne dit rien de précis.
M. du Maine, embarrassé de leur embarras, et fort peiné de ce qu'il ne
sortait rien de net de leur bouche, se mit en posture de leur embrasser
les genoux. En ce moment le roi, les yeux mouillés de larmes, les pria
de le vouloir bien embrasser en sa présence et de l'assurer par cette
marque de leur amitié. Il continua de là à les presser de lui donner
leur parole de n'ôter point ce rang qu'il venait de déclarer, et les
deux princes, de plus en plus étourdis d'une scène si extraordinaire,
bredouillèrent encore ce qu'ils purent, mais sans rien promettre. Je
n'entreprendrai ici pas de commenter une si grande faute, ni le peu de
force d'une parole qu'ils auraient donnée de la sorte. Je me contente
d'écrire ce que je sus mot à mot du duc de Beauvilliers à qui Mgr le duc
de Bourgogne conta tout ce qui s'était passé le lendemain, et que ce duc
me rendit le jour même. On le sut aussi par Monseigneur qui le dit à ses
intimes, en ne se cachant pas d'eux combien il était choqué de ce rang.
Il n'avait jamais aimé le duc du Maine, il avait toujours été blessé de
la différence du coeur du roi et de sa familiarité, et il avait eu des
temps de jeunesse où le duc du Maine, sans de vrais manquements de
respect, avait peu ménagé Monseigneur, tout au contraire du comte de
Toulouse qui s'en était acquis l'amitié. Pour le pauvre Mgr le duc de
Bourgogne, je ne fus pas longtemps sans savoir bien ce qu'il pensait de
cette nouvelle énormité, et l'un et l'autre ne furent point fâchés qu'on
les devinât là-dessus, autre bien étrange faute. Après celle de ce
dernier bredouillement informe de ces deux princes, le roi, à bout d'en
espérer davantage, sans montrer toutefois aucun mécontentement, retourna
vers son fauteuil, et le cabinet reprit aussitôt sa forme accoutumée.
Dès que le roi fut assis, il remarqua promptement le sombre qui y
régnait. Il se hâta de dire encore un mot sur ce rang et d'ajouter qu'il
serait bien aise que chacun lui en marquât sa satisfaction en la
témoignant au duc du Maine, lequel incontinent accueilli de chacun, fut
assez sérieusement félicite jusque par le comte de Toulouse son frère,
que le même honneur regardait à son tour, mais à qui il fut aussi
nouveau qu'à tous les autres. La différence d'âge et d'esprit, qui
donnait au duc du Maine une grande supériorité sur le comte de Toulouse,
n'avait pas contribué à une union intérieure bien grande\,; ils se
voyaient rarement chez eux, les bienséances étaient gardées, mais
l'amitié était froide, la confiance nulle, et M. du Maine avait toujours
fait sa grandeur, et conséquemment la sienne, sans le consulter et même
sans lui en parler. Le bon sens, l'honneur et la droiture de coeur de
celui-ci lui rendaient la conduite de la duchesse du Maine
insupportable. Elle s'en était bien aperçue\,; aussi ne l'aima-t-elle
pas, et ne contribua pas à rapprocher le comte de Toulouse qu'elle
craignait auprès du duc du Maine dont il n'approuvait pas les
complaisances qui pour elle étaient sans bornes, et dont avec cela il
n'évitait pas les hauteurs\,: le reste du cabinet fut court et mal à
l'aise.

La nouvelle éclata le lendemain, et on sut que tout ce qu'il y en aurait
d'écrit était une simple note sur le registre du maître des cérémonies,
en l'absence du grand maître qui servait cet hiver sur la frontière, en
ces mots\,:

«\,Le roi étant à Versailles a réglé que dorénavant les enfants de M. le
duc du Maine auront comme petits-fils de Sa Majesté le même rang, les
mêmes honneurs et les mêmes traitements dont a joui jusqu'à présent
mondit sieur le duc du Maine, et Sa Majesté m'a ordonné d'en faire la
présente mention sur mon registre.\,» Cela dit tout et ne dit rien, et
n'exprime quoi que ce soit, sinon que cela renvoie à l'usage dans lequel
on voyait le duc du Maine, et sans expliquer ni quel ni à quel titre,
mais insinue beaucoup en causant comme petit-fils de Sa Majesté et par
ce terme absolu de petits-fils sans y rien ajouter.

Jamais chose ne fut reçue du public d'une manière si morne\,; personne à
la cour n'osa en dire un mot tout haut, mais chacun s'en parlait à
l'oreille, et chacun la détesta. On n'était pas encore accoutumé au rang
de M. du Maine, qu'on le vit passer à ses enfants. De représentations
là-dessus, on vit bien qu'elles seraient non-seulement inutiles, mais
criminelles\,; et dès que ce qui s'était passé à la déclaration du
cabinet eut percé, et qu'on sut que le roi avait invité à féliciter M.
du Maine, il n'y eut personne qui osât s'en dispenser. On avait éclaté
contre les premiers rangs donnés à M. du Maine\,; à ce comble-ci, qui
que ce soit n'osa dire un seul mot, et la foule courut chez lui avec le
visage triste et une simple révérence, qui sentait plus l'amende
honorable que le compliment.

J'étais tout nouvellement raccommodé avec le roi, et dans l'audience que
j'en avais eue, il m'avait fort exhorté à me mesurer fort sur ce qui
regardait mon rang. Il était cruellement blessé par ce que le roi venait
de faire\,; jamais je n'avais été chez les bâtards sur aucun de ceux
dont le roi les avait accrus. Je vis ducs, princes étrangers et tout
indistinctement y aller\,; je compris que me distinguer en n'y allant
pas ne diminuerait ni leur rang ni leur joie, et me perdrait de nouveau,
bien plus que je ne l'avais été. Je me résolus donc à ce calice, et
j'allai comme les autres, et le plus que je pus parmi beaucoup d'autres,
faire à M. et à M\textsuperscript{me} du Maine une sèche révérence, et
tournai court aussitôt. Tant de gens y étaient à la fois qu'ils ne
savaient à qui entendre, et tandis qu'ils en complimentaient et
conduisaient les premiers sous leur main, les autres s'écoulaient, parmi
lesquels je m'échappai. La bassesse et la terreur firent aviser d'aller
aussi chez le comte de Toulouse\,; et les mêmes réflexions qui m'avaient
mené chez M. et M\textsuperscript{me} du Maine me conduisirent chez lui.
Je ne le trouvai point\,; et comme je traversais en revenant la petite
cour de Marbre, je rencontrai d'O, que je priai de dire à M. le comte de
Toulouse que je venais de chez lui pour les compliments. «\,Sur quoi,
monsieur, des compliments\,?» me répondit d'O avec son froid et son
importance. Je répliquai que ce qui venait d'être fait pour M. du Maine
le regardait d'assez près pour y prendre part. «\,Comment, reprit d'O
avec un air froncé, de ce qu'il passera désormais après les enfants de
M. du Maine\,?» Dans ma surprise je lui dis qu'il me semblait qu'il y
gagnait assez pour les siens, pour passer volontiers après ses neveux.
Alors d'O s'avançant à moi et me regardant fixement comme un homme
pressé de faire une déclaration\,: «\,Monsieur, me dit-il, soyez
persuadé que M. le comte de Toulouse n'a point de part à ce que M. du
Maine a obtenu\,; que M. le comte de Toulouse n'a point d'enfants et ne
prétend rien pour ceux qu'il aura, qu'il est content de son rang, et
qu'il n'en veut pas davantage.\,» Je quittai d'O dans une extrême
surprise.

C'était un homme avec qui je n'avais pas la moindre habitude et que je
ne voyais jamais nulle part\,; je n'en avais pas davantage avec sa femme
ni M\textsuperscript{me} de Saint-Simon non plus. C'était un pharisien
dédaigneux, tout au plus à monosyllabes, et qui m'avait paru saisir avec
empressement l'occasion de s'expliquer à moi de ce que je ne lui
demandais point, et de me dire une chose si étonnante. Je la fus rendre
à l'instant au duc et à la duchesse de Villeroy, amis du mari et de la
femme\,; ce qui comblait ma surprise, c'est que, quelque attachement
personnel et d'emploi qu'eût d'O pour M. le comte de Toulouse, il était
encore plus l'homme de M\textsuperscript{me} de Maintenon et même de M.
du Maine. Le duc et la duchesse de Villeroy m'expliquèrent l'énigme\,;
mais je ne crois pas qu'ils en eussent la véritable leçon\,; je dirai
après ma conjecture. Ils me contèrent que le duc ayant parlé de son
dessein à son frère, il n'avait pu le persuader\,; que le comte de
Toulouse avait même fait ce qu'il avait pu pour le lui faire quitter,
soit par son éloignement présent du mariage et la petitesse de son rang
personnel avec ses neveux, soit qu'il sentît que la chose était si forte
qu'elle pourrait un jour entraîner leur rang à eux-mêmes. Ce qu'il y a
de certain, c'est que cette affaire mit un froid marqué entre eux. Ce
que j'en crus après, car en ce moment je ne le savais pas encore, c'est
que la chose était revenue entre deux fers, par ce qui va être raconté,
ce qui, joint à une déclaration si hors d'oeuvre et si empressée d'un
homme si peu empressé de parler, et à un autre qu'il ne connaissoit que
de nom et de visage et qui ne lui faisait ni question ni raisonnement,
me fit croire que c'était politique, et que le comte de Toulouse voulait
laisser son frère seul dans la nasse sans la partager avec lui. Voici
donc ce qui arriva. De l'un à l'autre, on ne tarda pas à savoir les
sentiments de Monseigneur et Mgr le duc de Bourgogne. Eux-mêmes
comblèrent une si terrible faute en ratifiant ce qu'on en disait,
jusque-là qu'il échappa à la pauvre M\textsuperscript{me} la duchesse de
Bourgogne, que ce rang ne tiendrait pas sous Monseigneur, et moins
encore, s'il se pouvait, sous eux quand ils seraient les maîtres. La
cour, suffoquée du silence qu'elle avait gardé d'abord, sentant un tel
appui, se lâcha au murmure, et en un moment le murmure devint général,
public et fort peu mesuré. Tout fut coupable d'après les deux héritiers
de la couronne. Ainsi, personne ne craignant le châtiment par
l'universalité des complices, la licence alla fort loin.

Le roi était trop appliqué à être informé des moindres choses pour
ignorer ce déluge de discours, beaucoup moins le chagrin de Monseigneur
et de Mgr le duc de Bourgogne, malgré tout ce qu'il avait employé de si
nouveau auprès d'eux\,; le sombre et le repentir le saisirent. M. du
Maine en trembla et M\textsuperscript{me} de Maintenon avec lui, qui le
virent au moment de rétracter ce qu'il venait de faire. Ils se mirent
donc hardiment à faire contre, à vanter au roi l'obéissance même
intérieure qu'il s'était acquise, jamais mieux marquée que par
l'empressement de la foule à lui faire des compliments, par la joie que
tout le monde marquait de la grâce qu'il venait de faire, et les
applaudissements publics qu'elle recevait. Avec cet artifice, il {[}M.
du Maine{]} profita des hommages arrachés à une cour esclave, en
flattant le roi sur ce qui lui était le plus sensible, et le mit à ne
savoir plus que croire.

Le lendemain de mes visites aux bâtards, et trois jours après la
déclaration, j'allai le matin chez M\textsuperscript{me} de Nogaret, qui
m'avait envoyé dire qu'elle avait un mot à me dire dans la matinée. Je
fus bien étonné quand elle me dit que M\textsuperscript{me} la duchesse
de Bourgogne l'avait chargée de savoir de moi, et de sa part, à
découvert, ce qui formait ma liaison si intime avec M. du Maine, et
qu'elle désirait savoir aussi ce qu'il me semblait du rang qui venait
d'être donné à ses enfants. À mon tour je fus curieux où
M\textsuperscript{me} la duchesse de Bourgogne avait pris cette liaison,
et ce qui la pouvait mettre en doute sur ce que je pensais sur ce rang.
M\textsuperscript{me} de Nogaret me dit que, veillant le soir précédent
chez M\textsuperscript{me} la duchesse de Bourgogne encore en reste de
couche du roi, et parlant de ce rang avec le scandale qu'il mérite, elle
lui avait dit que le roi, peiné de sentir combien peu elle goûtait cette
nouveauté, lui avait exagéré l'approbation unanime\,; que le duc du
Maine était comblé des honnêtetés de la cour, et, que prenant ensuite un
air plus ouvert et d'entière complaisance, il avait ajouté qu'enfin
moi-même j'avais visité le duc du Maine et l'avais assuré du plaisir que
je ressentais de sa satisfaction. Je souris avec un dépit amer de la
prostitution de mon nom pour soutenir celle de toute la France. Je
contai à M\textsuperscript{me} de Nogaret ce qui m'était arrivé avec
M\textsuperscript{me} la duchesse d'Orléans et M. du Maine avant la mort
de M. le Duc sur le procès de la succession de M. le Prince, la conduite
de M. et de M\textsuperscript{me} du Maine avec M\textsuperscript{me} de
Saint-Simon, et avec moi, et la nôtre avec eux\,; de là je m'expliquai
avec elle de ce que je pensais et sentais d'un rang que je détestais
dans le père, à plus forte raison continué dans ses enfants\,; je
m'étendis sur ce qu'elle et les deux princes héritiers en marquaient, et
sur les raisons qui m'avaient forcé à aller chez M. et
M\textsuperscript{me} du Maine à cette occasion pour la première fois de
ma vie de cette sorte, où quoi que le roi en crût, M. du Maine n'avait
pu entendre le son de ma voix\,; et je priai M\textsuperscript{me} de
Nogaret de rendre toute cette conversation à M\textsuperscript{me} la
duchesse de Bourgogne, ce qu'elle fit fort exactement.

Cependant la princesse pressée par M\textsuperscript{me} de Maintenon
sur ce rang, demeura ferme et la surprit d'autant plus, qu'elle ne se
doutait pas qu'elle sût rien de ces matières-là, et qu'elle la trouva
instruite de fort bonnes raisons, et qui l'embarrassèrent. Elle voulut
absolument savoir d'elle ce qui s'en disait effectivement dans le monde,
et M\textsuperscript{me} la duchesse de Bourgogne ne la trompa point\,;
le roi et elle demeurèrent donc fort en peine, et tellement que ce rang
fut sur le point d'être rétracté. Mais enfin il était donné, déclaré,
publié\,; le roi ne voulut pas paraître céder, il chercha à se repaître
des artifices flatteurs de M. du Maine, et le rang demeura. La prise de
possession ne tarda pas, et pour que le scandale en fût complet, ce fut
au sermon\,; Le comte de Toulouse, qui avait été voir ses neveux en
cérémonie qui ne lui donnèrent pas la main à la manière des princes du
sang entre eux, s'absenta des sermons pour n'y être pas après eux, et
n'y revint que par une espèce de négociation. M\textsuperscript{me} la
Princesse et M\textsuperscript{me} la princesse de Conti sa fille,
vinrent en ce même temps à Versailles recevoir la visite du roi sur la
mort de M. le Duc. M\textsuperscript{me} la Princesse le remercia des
grâces qu'il avait faites à son petit-fils, et non sans rougir, ajouta
ses remercîments sur celles qu'il venait de faire aux enfants de
M\textsuperscript{me} du Maine, qui les égalait à ceux de son fils.

J'éclaircirai encore d'un mot ce qui me regarde sur cet étrange rang, en
expliquant comment M\textsuperscript{me} la duchesse de Bourgogne
comprit que j'étais si lié avec M. du Maine. Le roi et
M\textsuperscript{me} de Maintenon étant à parler de ce rang dans la
ruelle de cette princesse, tous trois seuls, et M\textsuperscript{me} de
Maintenon employant tout son art pour soutenir son ouvrage contre le
repentir que le roi en avait pris et qu'il lui reprochait, lui
persuadait comme elle pouvait le concours chez M. du Maine et la joie
des compliments, et ajouta que jusqu'à moi j'avais été lui témoigner la
mienne. Le roi le lui fit répéter, et sur ce qu'elle l'assura que M. du
Maine le lui avait dit, ce fut alors que le roi prit cet air de sérénité
et de complaisance, et que se tournant à M\textsuperscript{me} la
duchesse de Bourgogne, lui dit que, puisque celui-là y avait été, il
fallait bien qu'à ce qu'il avait fait il n'y eût pas tant à redire,
comme en se consolant. M\textsuperscript{me} la duchesse de Bourgogne ne
répondit rien, et M\textsuperscript{me} de Maintenon continua ses propos
pour le raffermir. Ce détail, M\textsuperscript{me} de Nogaret me le fit
le lendemain de sa question, en me disant le compte qu'elle en avait
rendu de ma part à M\textsuperscript{me} la duchesse de Bourgogne. Je ne
sais pourquoi elle ne me l'avait pas conté la veille. Je sus d'elle que
M\textsuperscript{me} la duchesse de Bourgogne avait entendu avec
plaisir ce qu'elle lui avait dit de ma part, et qu'elle était bien aise
de ne s'être pas trompée sur le jugement qu'elle avait porté de moi sur
ce rang.

Achevons tout de suite ce qui regarde M. du Maine et ses enfants. Ce
qu'il venait d'obtenir pour eux, beaucoup plus encore la façon si
surprenante dont le roi avait parlé en leur faveur et en la sienne aux
deux princes ses fils et petit-fils, et si étrangement éloignée de son
caractère, lui montrèrent ses forces, et par lui-même et par
M\textsuperscript{me} de Maintenon, au delà de tout ce qu'il aurait pu
croire. Il en profita donc et sut, et par elle et par soi-même, faire
valoir au roi la froideur de ces deux princes, pour n'en rien dire de
plus parmi des discours si touchants et si nouveaux pour eux, et la
juste crainte qu'il en devait concevoir\,; {[}tellement{]} qu'il sut
persuader au roi que, pour montrer qu'il ne se repentait pas de ce qu'il
venait de faire, et pour consolider le rang et les honneurs qu'il
donnait à ses enfants, il était nécessaire de leur donner de l'autorité
et de la puissance. Il obtint cinq semaines après, c'est-à-dire le jeudi
de Pâques, 24 avril, la survivance de sa charge de colonel général des
Suisses et Grisons pour le prince de Bombes son fils aîné\,; âgé de dix
ans, et pour le comte d'Eu qui en avait six, celle de grand maître de
l'artillerie. Ce fut un grand et prompt renouvellement de scandale et de
murmure, mais qui ne diminua rien de la servitude. Toute la cour alla
chez M. et M\textsuperscript{me} du Maine, qui eurent en même temps le
bel appartement du feu archevêque de Reims au château, singularité
encore fort éclatante, aucun prince du sang ni les enfants même de
Monsieur n'en ayant eu que dans un âge bien plus avancé.

Je fus donc comme les autres un matin chez M. du Maine, comptant bien,
comme l'autre fois, n'y faire qu'une apparition, et m'enfuir à la faveur
de la foule. Je le trouvai environné de prélats de l'assemblée du
clergé, et dès que j'eus paru, je me retirai. À l'instant M. du Maine
pria ces prélats de trouver bon qu'il me dît un mot, vint clopinant à
moi de façon que je ne pus éviter ces gens qui me le dirent et me le
montrèrent. Je revins donc à lui, et il me mena à la cheminée au fond de
sa chambre d'où tout le monde sortit, et où nous demeurâmes seuls. Là il
me dit qu'il y avait bien longtemps qu'il cherchait une occasion de me
témoigner toute sa reconnaissance de tout ce qu'il me devait, sur la
manière dont j'avais bien voulu répondre à ce que M\textsuperscript{me}
la duchesse d'Orléans m'avait {[}dit{]} sur le procès de la succession
de M. le Prince, qu'il me suppliait de compter qu'il n'oublierait jamais
cette grâce qu'il avait reçue de moi, et qu'il n'y avait point
d'occasion qu'il ne cherchât avec empressement, pour me témoigner à quel
point il y était sensible\,; que je lui-devais la justice d'être
persuadé qu'il m'avait toujours regardé avec une estime singulière, et
constamment désiré l'honneur de mon amitié\,; que M\textsuperscript{me}
la duchesse du Maine était dans les mêmes sentiments\,; qu'il désirerait
sur toutes choses que nous nous pussions voir quelquefois librement,
puis retombant tout à coup sur M\textsuperscript{me} de Saint-Simon,
pour lui et pour M\textsuperscript{me} sa femme, il n'y eut sortes de
choses qu'il ne me dît, mais avec des termes si pleins, si forts, si
expressifs, et surtout si étrangement polis, que je vis l'heure que je
n'aurais ni le moment ni le moyen d'y répondre. Je le fis néanmoins au
mieux que je le pus, en l'assurant aussi que je n'oublierais point son
procédé et celui de M\textsuperscript{me} la duchesse du Maine lors de
l'affaire de M\textsuperscript{me} de Lussan. Je crus en être quitte en
finissant par là et me voulant retirer. Ce fut de nouvelles louanges sur
M\textsuperscript{me} de Saint-Simon, de nouveaux désirs de
M\textsuperscript{me} du Maine et de lui d'une amitié comme la sienne,
combien ils s'en tiendraient honorés\,; car aucun terme ne fut ménagé ni
pour elle ni pour moi\,; tout ce que M\textsuperscript{me} du Maine
avait fait pour la mériter, et après pour se la conserver touchant
obliquement l'affaire de M\textsuperscript{me} de Lauzun, et qu'il était
si pressé que je susse tous ces sentiments-là, qu'il avait prié
M\textsuperscript{me} la duchesse d'Orléans de me les témoigner en
attendant qu'il pût le faire lui-même. Elle n'en avait rien fait, ou par
oubli ou plutôt parce que tout cela tendait à lier commerce et amitié
avec nous, et que, dès la première fois qu'elle m'en avait parlé, elle
avait bien senti que nous ne voulions ni de l'un ni de l'autre.

Je me tirai à grand'peine d'avec M. du Maine à force de verbiages, de
compliments vagues, et de propos les plus polis que je pus, sans
toutefois rien de précis, sans entrer en quoi que ce fût, encore moins
dans aucun engagement de liaison, sur quoi je me tins fort en garde, et
je sortis enfin accablé des politesses les plus vives et les plus
pressantes. J'évitai celles que j'imaginai que M\textsuperscript{me} la
duchesse du Maine me préparait, qui était environnée de monde, et qui me
voulut faire approcher d'elle, dont je m'excusai pour ne point déranger
les dames, et tout de suite je m'en allai. M\textsuperscript{me} de
Saint-Simon trouva M. et M\textsuperscript{me} du Maine ensemble, qui, à
qui mieux mieux, l'accablèrent à son tour et n'oublièrent rien de
pressant et même d'embarrassant pour lier avec nous. Elle s'en tira
comme j'avais fait avec bien de la peine\,; à ces façons nous n'en eûmes
point à juger que rien leur faisait perdre de vue le dessein et le désir
si extraordinaire et si suivi de lier avec nous, et nous confirma dans
nos anciennes résolutions là-dessus. Je ne les en vis pas davantage,
c'est-à-dire aux occasions de morts, mariages et autres pareils,
indispensables et fort rares, et M\textsuperscript{me} de Saint-Simon
presque pas plus souvent. On verra dans la suite que je ne me suis pas
étendu inutilement sur ces poursuivantes recherches de M. {[}le duc{]}
et de M\textsuperscript{me} la duchesse du Maine, pour lesquels une si
énorme extension d'un rang déjà si odieux ne pouvait guère me donner
d'amitié.

Finissons cette triste matière par une autre aussi peu consolante, qui
est la réception de Villars au parlement, lequel, contre le plus
continuel usage, ne prit aucun pair pour témoin de ses vie et moeurs, et
qui, par cette singularité, donna lieu à cette dissertation publique,
s'il l'avait fait par respect ou par honte, ou par la crainte d'être
refusé. J'eus peine à me résoudre à me trouver à une si humiliante
cérémonie. J'y fus témoin d'une malice du duc de La Meilleraye, qui
poussa M. du Maine de questions pourquoi M. le comte de Toulouse, qui
venait toujours au parlement avec lui, y était venu cette fois
séparément\,; M. du Maine avec tout son esprit en fut embarrassé à
l'excès\,; et l'autre qui s'en amusait, qui n'ignorait pas le froid que
le rang des enfants avait mis entre eux, en donnait aussi le plaisir à
la compagnie. Dès que la réception fut faite et que le parlement alla à
la buvette, je m'en allai et ne pus demeurer à la grande audience.
Villars invita tous les pairs à dîner chez lui. Je le fus comme les
autres et je m'en excusai\,; je sus après que presque aucun n'y avait
été.

\hypertarget{chapitre-viii.}{%
\chapter{CHAPITRE VIII.}\label{chapitre-viii.}}

1710

~

{\textsc{Vendôme, demandé de nouveau pour général par l'Espagne, épouse
tristement M\textsuperscript{lle} d'Enghien.}} {\textsc{- Mort du duc de
Coislin\,; son caractère.}} {\textsc{- Hoquet inouï fait par le roi à
l'évêque de Metz sur sa succession à la dignité de son frère.}}
{\textsc{- Occasion, cause et fin de ce hoquet.}} {\textsc{- Habit et
manière de signer de M. de Metz.}} {\textsc{- Évêques d'Espagne, devenus
grands par succession, ne portent plus le nom de leur évêché.}}
{\textsc{- Mort, aventures, caractère et singularités de la maréchale de
La Meilleraye.}} {\textsc{- Maison de Cossé.}}

~

Ces mois de mars et d'avril furent heureux pour les bâtards. L'Espagne
pressa de nouveau pour obtenir M. de Vendôme qui, se voyant sans
ressource en ce pays-ci, et confiné fort solitairement à Anet, brûlait
d'envie d'obtenir la permission d'y aller, qu'il avait négociée comme on
l'a dit ailleurs avec la princesse des Ursins, et sur laquelle il
faisait insister. En attendant, se voyant délivré de M. le Prince et de
M. le Duc, il espéra qu'il n'y aurait plus d'obstacle à son mariage avec
M\textsuperscript{lle} d'Enghien, à qui M. et M\textsuperscript{me} du
Maine l'avaient mis dans la tête, mais dont ils n'avaient pu venir à
bout tant que M. le Prince et même M. le Duc avaient vécu. Elle avait
trente-trois ans, elle était extrêmement laide\,: sa vie s'était passée
au fond de l'hôtel de Condé dans la plus cruelle gêne, ce qui lui avait
fait désirer, pour en sortir, quelque mariage que ce fût. La gêne avait
fini avec M. le Prince, mais l'ennui subsistait avec
M\textsuperscript{me} la Princesse, de chez qui elle ne pouvait sortir
qu'en se mariant. M. du Maine voulait une princesse du sang pour M. de
Vendôme, et décorer de plus en plus la bâtardise. M. de Vendôme, qui
n'avait jamais voulu se marier, fut touché de l'honneur de devenir
gendre de M. le Prince, piqué de n'avoir pu en être accepté ni même de
M. le Duc pour beau-frère par sa disgrâce. Toutes ces raisons le
pressèrent de faire ce mariage après eux. Ce fut l'ouvrage de M. du
Maine\,; le roi y consentit, et le mariage fut déclaré le 26 avril.

S'il fallait de l'ambition pour se résoudre à épouser
M\textsuperscript{lle} d'Enghien, il fallait un grand courage pour
épouser M. de Vendôme, presque sans nez et manqué deux fois par les plus
experts. Mais tout leur fut bon à l'un et à l'autre, à elle pour avoir
du bien et de la liberté, à l'autre par la vanité de se montrer encore
assez grand dans l'état de santé et de disgrâce où il était, pour
épouser une princesse du sang qu'il acheta de tout son bien qu'il lui
donna par leur contrat de mariage, s'il mourait avant elle sans enfants,
comme toutes les apparences y étaient, et comme cela arriva en effet.
M\textsuperscript{me} la Princesse et M\textsuperscript{me} la Duchesse
n'apprirent ce mariage que par M. du Maine, et comme arrêté et comme le
roi le voulant. M\textsuperscript{me} la Princesse se mit à pleurer,
allégua vainement la mémoire peu comptée de M. le Prince, et ne pouvant
rien empêcher, laissa tout faire sans en vouloir plus ouïr parler.
M\textsuperscript{me} la Duchesse se rengorgea, se fâcha, mais ce fut
tout\,; elle n'avait point d'autorité sur sa belle-soeur. M. du Maine se
chargea de tout, du contrat de mariage, de la publication des bans, de
la noce. La manière dont tout s'y passa montra à quel point M. de
Vendôme était perdu. Il eut peine à obtenir permission d'aller en parler
au roi à Versailles\,; ce fut à condition de se tenir beaucoup dans sa
chambre, de n'y voir personne, et personne presque ne s'y présenta. Sa
conversation avec le roi fut sèche et courte, et il retourna tout
aussitôt à Anet. Il n'eut pas la liberté de venir faire signer son
contrat de mariage. M. du Maine tout seul le présenta à signer sans être
accompagné de personne d'aucun côté, et le roi voulut qu'on prît le
temps d'un voyage à Marly pour faire le mariage à Sceaux, sans fête,
sans bruit, dans la plus grande obscurité, et ne voulut point ouïr
parler de fiançailles dans son cabinet\,; le contrat de mariage fut donc
signé à Marly, le 13 mai, de cette façon clandestine.

M. de Vendôme vint droit d'Anet à Sceaux, le jeudi 15 mai, fut le soir
même fiancé, marié et couché avec M\textsuperscript{lle} d'Enghien\,;
M\textsuperscript{me} la Princesse, M. le Duc, M. le comte de Charolais,
son frère, M\textsuperscript{me} la princesse de Conti, M. son fils et
M\textsuperscript{me}s ses filles, M. et M\textsuperscript{me} du Maine
et MM. leurs enfants présents avec quelques domestiques, et qui que ce
soi autre. Dès que la messe fut dite à minuit, tous les princes et
princesses du sang s'en allèrent et ne revinrent plus. M. de Vendôme
demeura le lendemain vendredi à Sceaux, avec M. et M\textsuperscript{me}
du Maine, leurs enfants et leurs domestiques uniquement et la nouvelle
mariée, et le samedi M. de Vendôme l'y laissa et s'en retourna à Anet.
Ni l'un ni l'autre ne reçurent aucun compliment de la part du roi, ni de
pas une des personnes royales\,; on ne parla pas seulement de ce
mariage\,; ce fut comme chose non avenue. M. du Maine revint dès qu'il
le put à la cour, et M\textsuperscript{me} de Vendôme retourna chez
M\textsuperscript{me} la Princesse jusqu'à ce que la maison du grand
prieur au Temple fût prête, qui était en grand désarroi, et le grand
prieur hors du royaume. Quel eût été l'éclat de cette noce quelques
années plus tôt, et quel contraste avec les retours si radieux de M. de
Vendôme d'Italie\,! On remarqua que M. de Vendôme, qui n'avait point vu
tous ces princes et princesses du sang qui se trouvèrent à son mariage,
ne leur y fit pas le moindre compliment. Il fut là comme à la noce d'un
autre, et depuis à Anet comme s'il avait oublié qu'il était marié.

Le duc de Coislin ne survécut pas longtemps à son ami M. le Duc\,;
c'était le seul homme qui l'eût subjugué, qui ne lui passait rien et qui
lui lâchait quelquefois des bordées effroyables, sans que M. le Duc osât
souffler. C'était un homme de beaucoup d'esprit, extraordinaire au
dernier point, et qui se divertissait à le paraître encore plus qu'il ne
l'était en effet, plaisant en sérieux et sans chercher à l'être,
toujours salé, fort amusant, méchant aussi et dangereux, qui ne se
refusait rien, qui méprisait la guerre qu'il avait quittée il y avait
longtemps, et la cour où il n'allait presque jamais, par conséquent mal
avec le roi, dont il ne se mettait guère en peine\,; fort du grand
monde, qu'il cherchait moins qu'il n'en était recherché et de la
meilleure compagnie. Il se piquait de ne saluer jamais personne le
premier et le disait si plaisamment qu'on ne pouvait qu'en rire. Quand
le roi eut achevé Trianon comme il est aujourd'hui, tout le monde
s'empressa de l'aller voir. Roquelaure demanda au duc de Coislin ce
qu'il lui en semblait\,: il lui dit qu'il ne lui en semblait rien parce
qu'il ne l'avait pas vu. «\,Je sais bien pourquoi, lui répondit
Roquelaure, c'est que Trianon ne t'est pas venu voir le premier.\,» Il
faut encore que je dise ce trait du duc de Coislin. La fantaisie lui
prit, au duc de Sully, son beau-frère, et à M. de Foix, d'aller au
parlement, et ils me pressèrent tant d'y aller avec eux que je ne pus le
refuser, et c'est l'unique fois que j'y aie été sans nécessité. M. de
Foix, qui était paresseux et qui passait les nuits en compagnie, n'y
vint point, de sorte que je m'y trouvai assis entre les deux
beaux-frères.

Le Nain, doyen alors du parlement, et un des plus estimés pour sa
probité, son exactitude et ses lumières, rapporta un procès considérable
où il y avait pour quarante mille francs de dépens qu'il conclut à
compenser\,; les premiers avis furent conformes à celui du rapporteur.
C'était à huis clos, à la petite audience\,; ainsi nous entendions tout
parce qu'on opinait de sa place sans se lever. Le Meusnier, vieux
conseiller, clerc aussi fort habile, mais de réputation plus que louche,
ouvrit l'avis de faire payer les dépens. Plusieurs le suivirent et
d'autres non, car pour le fond du jugement il fut tout d'une voix de
l'avis du rapporteur. Voilà le duc de Coislin qui se met à rire et à me
dire qu'il faut faire un partage, et que cela sera plaisant de voir la
grand'chambre s'aller faire départager à une chambre des enquêtes. Je
crus qu'il plaisantait, mais comme je le vis attentif à suivre et à
compter les voix de part et d'autre et à me presser de partager,
c'est-à-dire de prendre l'opinion la moins nombreuse, je lui demandai
s'il n'avait point de honte de vouloir coûter quarante mille livres à
des gens, pour se divertir\,; qu'ignorants comme nous l'étions, il
fallait aller à l'avis le plus doux, surtout avec la garantie d'un homme
exact, éclairé et intègre comme était Le Nain., qui avait bien examiné
l'affaire. Il se moqua de moi et dit toujours que cela serait plaisant
et qu'il ne le manquerait pas. De pitié pour ces parties, dont nous ne
connaissions aucune, je m'assurai du duc de Sully, qui blâma son
beau-frère et qui convint avec moi qu'il serait pour compenser les
dépens. Nous opinâmes les derniers, et tous trois tînmes parole. Le duc
de Coislin, qui par son calcul avait vu qu'il partagerait en prenant
l'avis de Le Meusnier, en fut. Je me rangeai après à celui de Le Nain,
et après moi le duc de Sully. Le premier président Harlay, qui avait
compté aussi et qui vit le partage, se met à regarder les présidents à
mortier, à leur dire qu'il y a partage, puis à remontrer à la compagnie
l'indécence de cet inconvénient dans un tribunal comme la
grand'chambre\,; qu'il fallait tâcher de se réunir à une opinion\,; que
la sienne était de compenser les dépens\,; et qu'il allait reprendre les
voix. Pendant qu'on opinait, le duc de Coislin crevait de rire, et moi
de l'exhorter à se contenter du plaisir qu'il s'était donné et de ne pas
pousser l'affaire à bout. Jamais il n'y voulut entendre, bien résolu de
changer d'avis ou non, suivant que cela servirait au partage. Il fut
encore de l'avis de Le Meusnier, le duc de Sully et moi de celui du
rapporteur, le premier président aussi\,; et encore partage.

Voilà le premier président fort fâché qui harangua près d'un quart
d'heure, qui tâcha de piquer d'honneur messieurs d'éviter la honte de
s'aller faire départager aux enquêtes, qui dit qu'il va reprendre pour
la troisième fois les avis, et que, pour abréger, parce que les raisons
sont suffisamment entendues, il suffira que chacun opine qu'il est de
l'avis du rapporteur ou de Le Meusnier. Le diable voulut que le partage
subsistât, quoique plusieurs conseillers eussent changé d'avis suivant
qu'ils comptaient jusqu'à eux pour éviter le partage, et toujours M. de
Coislin pour payer les dépens. Le malheur fut qu'avec une voix de plus
pour Le Meusnier il n'y avait plus partage. Harlay, qui l'avait bien
compté et qui regardait noir le duc de Coislin, dont la seule voix fit
en dernier lieu ce désordre, exposa le cas à la compagnie, tâcha de la
toucher en faveur des parties perdantes, à qui une seule voix coûterait
un partage injurieux pour la compagnie, ou quarante mille livres de
plus. Il eut beau dire, personne ne répondit à ses semonces réitérées,
tellement que, comme il vit qu'il fallait enfin prononcer, il préféra
l'honneur prétendu de la grand'chambre à la bourse de ces pauvres
parties, dit que pour éviter le partage, il revenait à l'avis de Le
Meusnier et prononça l'arrêt avec la condamnation aux dépens. Je
pouillai le duc de Coislin tant que je pus, qui était ravi et mourait de
rire.

Il était notoirement impuissant, et pour cela même se ruinait avec une
comédienne, qui le gouverna jusqu'à sa mort, et à qui sa famille, et
tout ce peu de gens qui pouvaient avoir affaire à lui, faisaient leur
cour. Il était veuf depuis longtemps de la soeur d'Alègre, depuis mort
maréchal de France, qu'il avait rendue fort malheureuse\,: M. de Metz et
la duchesse de Sully, son frère et sa soeur, étaient ses héritiers. Il
mourut à Paris, dans le temps du mariage de M. de Vendôme, pendant que
le roi était à Marly, où j'étais ce voyage. On y apprit cette mort entre
midi et une heure. La dignité passait de plein droit à M. de Metz, son
frère unique, et cela fit la conversation.

Le comte de Roucy qui, sans avoir le sens commun, mais beaucoup de
brutalité, d'assiduité et de bassesse, était de tout à la cour de
Monseigneur, et quoique sans estime, depuis Hochstedt surtout, point
trop mal avec le roi, était aussi avec un air de bon homme et sans façon
avec tout le monde, et particulièrement avec les valets, à qui cela
plaisait fort, le plus envieux de tous les hommes, et en dessous le plus
sottement glorieux. {[}Il{]} se trouva choqué que M. de Metz devînt duc
et pair. Il alla chez Monseigneur, à qui il dit que l'évêque de Metz
serait plaisant à voir en épée et en bouquet de plumes\,; et comme il
avait affaire à un aussi habile homme que lui, il l'infatua, par ces
sottises-là, que M. de Metz, étant prêtre et évêque, ne pouvait être duc
et pair\,; comme si, pour l'être, il fallait porter une épée et un
bouquet de plumes, et qu'il n'y eut pas des évêques pairs séant au
parlement avec un habit qui leur est particulier. De là il alla à la fin
du dîner de Mgr et de M\textsuperscript{me} la duchesse de Bourgogne,
avec les mêmes propos, qui ne les persuadèrent pas si facilement. Mgr le
duc de Bourgogne se moqua de lui et de ses fades et malignes
plaisanteries, et voulut bien démontrer, ce qui fut court et aisé, que
M. de Metz pouvait et devait recueillir la dignité de son frère,
puisqu'il en héritait de droit, qu'il était fils de celui pour qui
l'érection avait été faite, et qu'il n'était mort au monde par aucun
crime ni par aucun voeu religieux. Les envieux et les ignorants dont les
cours sont pleines, il s'en trouva en nombre qui firent chorus avec le
comte de Roucy, sans que pas un pût alléguer quoi que ce fût, que ce
ridicule inepte d'épée et de bouquet de plumes qui à peine aurait pu
surprendre les petits enfants. M. de Metz n'était point mal avec le
comte de Roucy, et il n'y avait jamais eu d'ocasions entre eux\,; mais
il avait aussi sa portion de cadet d'extraordinaire, n'était pas bon,
n'était pas aimé de tout le monde, et sa fortune ecclésiastique avait
révolté contre lui beaucoup de gens de cet état, quoique la plupart hors
de portée d'un siège tel que Metz et d'une charge comme la sienne. Toute
la journée se passa dans cette dispute dans les compagnies et dans le
salon\,; mais le soir l'étonnement fut grand, quand on apprit que le roi
y faisait de la difficulté, que Monseigneur l'avait fort appuyée dans le
cabinet après le souper, et que Mgr le duc de Bourgogne y avait aussi
solidement qu'inutilement plaidé pour M. de Metz. Le lendemain il eut
défense du roi, par Pontchartrain, de prendre ni titre, ni marque, ni
rang, ni honneurs de duc jusqu'à ce que le roi se fût fait rendre compte
de son affaire. M. de Metz eut beau presser du moins que quelqu'un en
fût chargé, il n'en put venir à bout\,; et, las d'attendre dans un état
aussi triste, il fit ôter ses armes de sa vaisselle, de ses carrosses,
et de partout où elles étaient parce qu'il n'osait porter le manteau
ducal, et qu'il ne voulait pas s'en abstenir\,; et de dépit il s'en alla
brusquement dans son diocèse. Il n'avait garde d'obtenir que quelqu'un
fût chargé de son affaire pour en rendre compte au roi, encore moins
d'être entendu lui-même. Le roi, quoique peu instruit, savait très-bien
qu'il n'y avait nulle difficulté, et qu'il était duc et pair de plein
droit à l'instant de la mort de son frère\,; mais il était outré contre
M. de Metz, il l'était de façon à ne vouloir pas le montrer, et il fut
ravi de cette sottise du comte de Roucy et du bruit qu'elle fit dans un
peuple ignorant et jaloux de tout. Il la saisit, et ne pouvant faire pis
à M. de Metz, il le châtia cruellement de la sorte, sous prétexte de ne
rien précipiter, et d'un éclaircissement qu'il n'avait garde de prendre,
mais dont il pouvait faire durer le prétexte tant qu'il lui plairait, et
par conséquent le désespoir de M. de Metz, qui en tomba malade, et à
qui, réellement et de fait, la tête en pensa tourner et en fut fort
près. Son fait que voici était double.

Le roi, après avoir fort aimé le cardinal de Coislin et eu pour lui
jusqu'à sa mort une estime déclarée qui allait, et très-justement,
jusqu'à la vénération, se laissa depuis aller au P. Tellier, qui, pour
fourrager à son plaisir le diocèse d'Orléans, de concert en cela avec
Saint-Sulpice, persuada au roi que ce cardinal était janséniste, et
qu'il avait mis en place dans son diocèse tous gens qu'il en fallait
chasser. C'étaient des hommes du premier mérite en tout genre, et connus
et goûtés comme tels, et qui étaient fort attachés au cardinal. Ils
furent chassés et quelques-uns exilés. Tout le diocèse cria. Cela aigrit
les persécuteurs qui avaient Fleuriau, évoque d'Orléans, à leur tête.
Ils firent ôter la tombe du cardinal, parce qu'on s'était accoutumé à y
aller prier\,; et on empêcha avec violence ce pieux usage qui avait
commencé dès sa mort, et qui n'était qu'une suite de la constante
réputation de toute sa vie. M. de Metz qui avait protégé tant qu'il
avait pu ces ecclésiastiques chassés et exilés, perdit toute patience à
l'enlèvement de la tombe de son oncle, surtout après en avoir fortement
et inutilement parlé au roi. Il s'échappa en propos qui furent rapportés
et envenimés, et par ceux qu'ils regardaient le plus, et qui mirent le
roi de part dans leur querelle et dans leur ressentiment. L'autre point
de M. de Metz fut que, s'étant trouvé un jour avec le duc de La
Rocheguyon, le duc de Villeroy et MM. de Castries, qu'on commençait à
découvrir tout à fait la nouvelle chapelle qui était achevée, ils
allèrent la voir et y menèrent Fornaro avec eux.

Ce Fornaro était un prétendu duc sicilien de beaucoup d'esprit, que M.
de La Feuillade avait ramené avec lui de Sicile, où il n'avait osé
retourner depuis l'amnistie, parce qu'il était accusé d'avoir empoisonné
sa femme. Il demeura chez M. de La Feuillade tant qu'il vécut, suivant
son fils dans sa jeunesse comme un gouverneur, et je l'ai vu chez moi
avec lui sur ce pied-là\,; et néanmoins, il tirait quelque chose du roi,
que M. de La Feuillade lui avait fait donner. Après la mort de M. de La
Feuillade, il trouva moyen de se fourrer chez M. de La Rochefoucauld,
mais sans loger chez lui\,; et ce fut là, dont il ne bougea, qu'il
commença à faire l'homme de qualité. Il dessinait en perfection, et il
avait beaucoup de connaissance de l'architecture, et un goût exquis pour
toutes sortes de bâtiments, surtout pour les grands édifices. Il fit un
degré charmant à Liancourt dans un emplacement où on n'en avait jamais
pu mettre, même un vilain. Cela lui donna de la réputation, M. de La
Rochefoucauld s'en engoua et le prôna. Il le fit aller à Marly, et sur
la liste comme les autres courtisans. Le roi lui parlait quelquefois de
ses bâtiments et de ses fontaines, au point que Mansart en prit jalousie
et peur. Il fut accusé de rapporter, et en effet, M. de La Rochefoucauld
le chassa de chez lui pour quelque chose qui avait été dit entre trois
ou quatre personnes, dont aucune autre que Fornaro ne pouvait être
soupçonnée, et que le roi sut et reprocha à M. de La Rochefoucauld, et
tout de suite doubla la pension de Fornaro, qui demeura à Versailles
mieux avec le roi que devant, et allant plus souvent à Marly, mais fui
et méprisé de tout le monde.

M. de Metz, allant donc voir la nouvelle chapelle avec ces messieurs,
comme je l'ai dit, et Fornaro pour voir ce qu'il en jugerait et la mieux
considérer avec lui, aigri des affaires d'Orléans, et frappé de la
quantité, de la magnificence et de l'éclat de l'or, des peintures et des
sculptures, ne put s'empêcher de dire que le roi ferait bien mieux, et
une oeuvre bien plus agréable à Dieu, de payer ses troupes qui mouraient
de faim que d'entasser tant de choses superbes, aux dépens du sang de
ses peuples qui périssaient de misère sous le poids des impôts\,; et il
allait paraphraser encore cette morale sans M. de Castries, aussi
considéré qu'il était imprudent, qui le retint et lui fit peur de
Fornaro\,; mais il en avait bien assez dit, et dès le soir même le roi
le sut mot pour mot. Les lettres que M. de Metz écrivit à ses amis,
étant à Metz, depuis ces affaires d'Orléans, ne furent pas plus
discrètes. Depuis le fatal secret trouvé par M. de Louvois pour violer
la foi politique et celle des lettres, le roi en vit toujours les
extraits, et c'étaient des nouveaux sujets de colère, qui le piquaient
d'autant plus que, retenu par la nature des voies qui l'informaient, il
ne voulait pas la montrer. Aussi se plut-il pendant près d'une année
complète à se venger cruellement de M. de Metz, en suspendant son état
sans en vouloir ouïr parler, et à se moquer de lui après. Quand il crut
enfin que cela ne se pouvait soutenir davantage sans une iniquité trop
déclarée, il fit dire un matin par Pontchartrain à M. de Metz qu'il
n'avait pas besoin d'éclaircissements sur son affaire\,; qu'il n'avait
jamais douté qu'il ne fût duc et pair de plein droit par la mort de son
frère\,; qu'il avait eu des raisons pour en user comme il avait fait\,;
mais qu'il trouvait bon maintenant qu'il prît le titre, les marques, le
rang et les honneurs de duc et pair\,; et qu'il lui permettait aussi de
se faire recevoir au parlement en cette qualité quand il voudrait. Il
était lors à Versailles et moi aussi. À l'instant il me le manda, parce
qu'il me savait grand gré de la manière dont j'avais pris sa défense.
Une heure après il fut remercier le roi, mais il n'en put tirer quoi que
ce fût sur les raisons qu'il avait eues. Il fut reçu honnêtement, et ce
fut tout. Aussitôt il prit tout ce qu'il aurait dû prendre dès l'instant
de la mort de son frère, et se disposa à se faire recevoir au parlement.

Il y trouva un hoquet auquel il n'avait pas lieu de s'attendre. Son
habit fut contesté par les magistrats, et même par des ducs, dont
beaucoup ne savent rien et ne veulent rien apprendre, qui prétendirent
qu'il ne pouvait paraître qu'en rochet et camail, parce qu'il était pair
par soi et non par son siége. Cette difficulté était d'autant plus
absurde que pair ecclésiastique n'est qu'un nom, et n'est pas une chose,
puisque, quant à la dignité, il n'y a différence quelconque entre les
ecclésiastiques et les laïques, et que l'habit des uns et des autres,
par conséquent, ne peut être que le même pour tous, suivant la
profession ecclésiastique ou laïque. Ainsi, après quelques disputes et
quelques jours de délai, la raison à la fin l'emporta, et M. de Metz fut
reçu en habit de pair ecclésiastique, et il n'en a point porté d'autre.
Il signa aussi d'abord «\,le duc de Coislin, évêque de Metz.\,» Bientôt
après il supprima «\,évêque de Metz\,» et ne signa plus que «\,le duc de
Coislin.\,» Les évêques s'en scandalisèrent, il s'en moqua, mais le
bruit qu'ils en firent l'engagea à ajouter «\,évêque de Metz\,» quand il
écrivait à des évêques, ce qu'il ne faisait en aucune autre lettre, et
souvent même il le supprima en leur écrivant, et les y accoutuma. Je ne
sais pourquoi il ne se fit pas appeler «\,le duc de Coislin.\,» Les
évêques d'Espagne n'y manquent pas quand il arrive qu'ils deviennent
grands par héritage, et il n'y en a point par siége, comme je l'ai vu de
l'évêque de Cuença qu'on n'appelait que «\,le duc d'Abrantès.\,» Je
pense que, se sentant mal avec le roi, il n'osa le hasarder, ni, étant
le premier exemple d'un évêque devenu duc par succession, la nouveauté
d'en porter le nom.

La maréchale de La Meilleraye mourut en ce même temps à
quatre-vingt-huit ans. Elle était tante paternelle de la maréchale de
Villeroy et du duc de Brissac, mon beau-frère, à l'occasion de quoi j'ai
parlé (t. I\^{}er, p.~77) de sa folie sur sa maison, et de l'imagination
de ce bonnet qu'elle lui fit prendre, à ses armes, qui a été imité de
quelques-uns, je ne sais pas pourquoi. On peut ignorer aussi la cause de
cette prodigieuse ivresse de sa maison. Elle a fort brillé sous François
I\^{}er et sous ses enfants par les hommes illustres qu'elle a produits,
et les grands emplois qu'ils ont exercés. Mais si on va au delà on
trouvera que le maréchal de Gonnor et son frère aîné le maréchal de
Brissac, si célèbre par les guerres de Piémont, père du comte de
Brissac, si fameux pour son âge, et du premier duc de Brissac, maréchal
de France de la Ligue, puis confirmé tel en recevant Henri IV dans
Paris, ont fait valoir par leurs talents la faveur de leur mère, soeur
du grand maître et du cardinal de Boisy et de l'amiral de Bonnivet qui
pouvaient tout sur François I\^{}er, desquels Anne de Montmorency,
depuis grand maître et connétable de France, était cousin germain. Cette
Gouffier qui avait épousé leur père, si connu sous le nom du gros
Brissac, fut gouvernante des enfants de France, et fit son mari ensuite
leur gouverneur, grand panetier et grand fauconnier, et gouverneur
d'Anjou et du Maine. Tout cela est illustre, mais il ne faut pas
remonter plus haut. Le père et le grand-père de ce gros Brissac, qui
était un gausseur et un homme d'esprit, de manège et de bonne chère,
étaient au bon roi René, l'un gouverneur du château de Beaufort, l'autre
sénéchal de Provence\,; leurs femmes des plus médiocres, leurs terres
rien, et par delà rien de suivi, et dans cela même rien que des écuyers
avec les plus petits emplois, sans filiation connue, et qui ne passe pas
l'an 1386. Cela ne fait pas une grande origine. Les dernières alliances
des ducs de Brissac des deux branches {[}sont{]} pitoyables, et
eux-mêmes, depuis le dernier maréchal, aussi pitoyables qu'elles. Ce mot
de remarque m'échappe, parce que je ne vois autre chose depuis la mort
du roi que des gens qui, par des noms de personnages de ce temps-là,
dont ils sont ou dont ils se font, et de plus anciens encore, mais qui
depuis eux n'ont eu que des lacunes en tout genre, chaussent le
cothurne, éblouissent les sots et prennent des airs tout à fait
ridicules. L'antiquité, la suite, les fiefs, les alliances, les emplois,
au moins avec quelque durée, dans les premiers temps connus, constituent
une grandeur effective, et non des choses modernes, passagères, et, pour
ceux dont je parle, depuis lors sans suite et sans trace de l'homme
illustre dont ils font bouclier, duquel le plus souvent ils ne
descendent même pas. Mais revenons à la maréchale de La Meilleraye. On
parlait devant elle de la mort du chevalier de Savoie, frère du comte de
Soissons et du fameux prince Eugène, mort fort jeune, fort brusquement,
fort débauché et fort plein de bénéfices, et on moralisait là-dessus.
Elle écouta quelque temps, puis, avec un air de conviction et
d'assurance\,: «\,Pour moi, dit-elle, je suis persuadée qu'à un homme de
cette naissance-là, Dieu y regarde à deux fois à le damner.\,» On éclata
de rire, mais on ne la fit pas revenir de son opinion. Sa vanité fut
cruellement punie. Elle faisait volontiers des excuses d'avoir épousé le
maréchal de La Meilleraye, dont elle fut la seconde femme, et n'en eut
point d'enfants. Après sa mort, amourachée, devant ou après, de
Saint-Ruth qu'elle avait vu page de son mari, elle l'épousa et se garda
bien de perdre son tabouret en déclarant son mariage. Saint-Ruth était
un très-simple gentilhomme fort pauvre, grand et bien fait, et que tout
le monde a connu\,; extrêmement laid\,: je ne sais s'il l'était devenu
depuis son mariage. C'était un fort brave homme et qui acquit de la
capacité à la guerre, et qui parvint avec distinction à devenir
lieutenant des gardes du corps, et lieutenant général. Il était aussi
fort brutal, et quand la maréchale de La Meilleraye lui échauffait les
oreilles, il jouait du bâton et la rouait de coups. Tant fut procédé que
la maréchale, n'y pouvant plus durer, demanda une audience du roi, lui
avoua sa faiblesse et sa honte, lui conta sa déconvenue, et implora sa
protection. Le roi avec bonté lui promit d'y mettre ordre. Il lava la
tête à Saint-Ruth dans son cabinet, et lui défendit de maltraiter la
maréchale. Cela fut plus fort que lui. Nouvelles plaintes de la
maréchale. Le roi se fâcha tout de bon et menaça Saint-Ruth. Cela le
contint quelque temps. Mais l'habitude du bâton était si forte en lui
qu'elle prévalut encore. La maréchale retourna au roi qui, voyant
Saint-Ruth incorrigible, eut la bonté de l'envoyer en Guyenne sous
prétexte de commandement, dont il n'y avait aucun besoin que celui de la
maréchale d'en être séparée. De là le roi l'envoya en Irlande où il fut
tué, et il n'eut point d'enfants.

La maréchale de la Meilleraye avait été parfaitement belle et avait
beaucoup d'esprit. Elle tourna la tête au cardinal de Retz, jusqu'à ce
point de folie de vouloir tout mettre sens dessus {[}dessous{]} en
France, à quoi il travailla tant qu'il put, pour réduire le roi en tel
besoin de lui qu'il le forçât d'employer tout à Rome pour obtenir
dispense pour lui, tout prêtre et évêque sacré qu'il était, d'épouser la
maréchale de La Meilleraye dont le mari était vivant, fort bien avec
elle, homme fort dans la confiance de la cour, du premier mérite et dans
les plus grands emplois. Une telle folie est incroyable et ne laisse pas
d'avoir été.

\hypertarget{chapitre-ix.}{%
\chapter{CHAPITRE IX.}\label{chapitre-ix.}}

1710

~

{\textsc{Je retourne à Marly avec le roi.}} {\textsc{- Propos sur Mgr le
duc de Bourgogne, entre le duc de Beauvilliers et moi, qui en exige un
discours par écrit.}}

~

Les couches de M\textsuperscript{me} la duchesse de Bourgogne, suivies
du carême, avaient tenu le roi plusieurs mois à Versailles sans faire de
voyages à Marly. Il y alla le lendemain du dimanche de Quasimodo, 28
avril, jusqu'au samedi 17 mai. J'étais allé faire un tour à la Ferté,
M\textsuperscript{me} de Saint-Simon se présenta pour ce voyage. C'était
le premier que le roi y faisait depuis l'audience qu'il m'avait donnée.
Nous fûmes de ce voyage. J'arrivai à Marly de la Ferté, et depuis je
n'en ai manqué qu'un jusqu'à la mort du roi, même de ceux dont
M\textsuperscript{me} de Saint-Simon ne put être\,; et je remarquai dès
ce premier-là que le roi me parlait, et me distinguait plus qu'il ne
faisait aux gens de mon âge, sans charge ni familiarité avec lui. C'est
dans l'espace de ce voyage que le contrat de mariage de M. de Vendôme
fut signé, qu'il se maria à Sceaux, et que le duc de Coislin et la
maréchale de La Meilleraye moururent, ainsi que je l'ai rapporté.

Rendu ainsi à mon genre de vie accoutumé, je raisonnais souvent avec les
ministres de mes amis, et des courtisans principaux qui en étaient, du
triste état des affaires, qu'ils ne me dissimulaient pas et sur
lesquelles ils pensaient comme je faisais. Quelques jours après le
retour à Versailles, j'allai passer une journée à Vaucresson, ce qui
m'arrivait souvent, où le duc de Beauvilliers s'était ajusté la plus
jolie retraite du monde, où d'ordinaire il passait le jeudi et vendredi
de chaque semaine, et qu'il avait rendue inaccessible à tout le monde,
excepté à sa plus intime famille et à quatre ou cinq amis au plus qui
avaient la liberté d'y aller. Causant tête à tête avec lui dans son
jardin, nous tombâmes insensiblement sur Mgr le duc de Bourgogne, et je
ne lui celai point ce que je pensais de sa conduite. Quoique cette
matière eût été souvent traitée entre le duc de Beauvilliers et moi, le
hasard avait fait que ce n'avait jamais été avec tant d'étendue, ni
qu'il eût été si frappé de mes sentiments là-dessus. La conversation se
tourna ensuite sur autre chose, et nous ne sortîmes du jardin et de ce
long tête-à-tête que lorsque le dîner fut servi. En sortant de table le
duc de Beauvilliers, qui avait réfléchi sur notre conversation, me pria
de faire encore un tour de jardin avec lui, de lui redire encore sur Mgr
le duc de Bourgogne les mêmes choses dont je l'avais entretenu avant le
repas et d'y ajouter ce qui me pourrait venir avec plus de temps et de
loisir que nous n'en avions eu le matin. Je m'en défendis, parce qu'il
ne pouvait pas l'avoir oublié, et que je croyais avoir dit à peu près
tout ce qu'il y avait à dire. Il me pressa et j'obéis. La conversation
fut fort longue et peu contredite. Lorsqu'elle fut épuisée, il me
proposa de mettre par écrit ce qu'il me semblait de la conduite de ce
prince, et ce que j'estimais qu'il y dût corriger et ajouter. La
proposition me surprit\,; il me pressa, je m'en défendis, et je lui
demandai ce qu'il prétendait faire. Il me répondit qu'un discours de
cette nature pourrait faire grand bien à Mgr le duc de Bourgogne ou au
moins lui être utile à lui-même (duc de Beauvilliers) en parlant à ce
prince. Je m'en défendis encore davantage, et je me retranchai sur le
danger de découvrir à ces gens-là qu'on les connaît si bien. Il me
rassura là-dessus tant qu'il put sur la vertu et la manière de penser de
Mgr le duc de Bourgogne\,; et finalement nous capitulâmes, moi que
j'écrirais, lui qu'il ne ferait aucun usage de mon écrit que de mon
consentement. Nous nous séparâmes de la sorte pour rejoindre la
compagnie dans la maison, moi toujours dans la surprise de ce qu'il
exigeait de moi, résolu néanmoins de lui obéir par un discours
ostensible à Mgr le duc de Bourgogne. J'y travaillai peu de jours après.

J'en fis à peu près la moitié dans ce dessein qu'il pût être montré au
prince, mais la plume me tourna après dans les doigts par la nécessité
de n'omettre pas des choses très-nécessaires. Je m'y abandonnai alors,
mais dans la résolution d'en ôter plusieurs traits au cas que M. de
Beauvilliers voulût le lui faire lire, lesquels toutefois me
paraissaient indispensables. J'en gardai un double que, bien qu'un peu
long, je ne renverrai point parmi les Pièces, mais j'insérerai ici,
parce qu'il donne une grande connaissance de Mgr le duc de Bourgogne. Il
est adressé au duc de Beauvilliers. Les premières lignes en marquent
l'occasion\,; et, s'il s'y trouve des raisonnements, des exemples et des
comparaisons du goût de peu de gens, c'est qu'un discours fait pour
persuader Mgr le duc de Bourgogne devait être accommodé à son goût et à
son esprit, à celui encore du duc de Beauvilliers qui, bien plus sûr et
plus libre de scrupules que celui du prince ne l'était encore pour lors,
étaient l'un et l'autre plus susceptibles d'être frappés par cette sorte
de raisonnement que par d'autres plus à la convenance de tout le monde
\textsuperscript{{[}Voy. notes à la fin du volume.{]}}{[}Voy. notes à la
fin du volume.{]}.

DISCOURS SUR MGR LE DUC DE BOURGOGNE, 25 MAI 1710, ADRESSÉ À M. LE DUC
DE BEAUVILLIERS QUI ME L'AVAIT DEMANDÉ.

Puisque notre conversation de Vaucresson vous a paru mériter assez
d'attention pour désirer de la voir étendue au delà des bornes
ordinaires d'un entretien à l'ombre de vos arbres, qui s'efface aisément
en rentrant dans la maison, j'en ferai d'autant moins de difficulté que,
s'agissant d'un prince sur lequel j'ose disputer de respect,
d'attachement tendre et d'admiration pour ses rares vertus intactes au
siècle, avec vous-même, rien de tout ce que je pense ne pourra vous
blesser\,; et l'épanchement secret de mon zèle pour sa personne,
inséparable, par ce qu'il est né, du bien de l'État, se bornant avec
vous seul, je me soulagerai en vous obéissant, en vous représentant
nûment ce que je pense.

Je suis fermement persuadé que peu de siècles ont produit de princes en
qui Dieu ait si libéralement répandu tant de vertus solides et tant de
grands talents qu'on en voit en Mgr le duc de Bourgogne, un esprit vif,
vaste, juste, appliqué, pénétrant, laborieux, naturellement porté aux
sciences difficiles, curieux de tout rechercher et plein de bonne foi en
ses recherches. C'est le riche champ qui vous a été présenté à cultiver,
et duquel, aidé par la plus habile main en tout genre et singulièrement
formée par le ciel pour l'art d'instruire un prince, vous avez
heureusement formé celui-ci à tout ce qu'on en pouvait attendre pour
réparer les profonds malheurs du plus beau royaume de l'Europe, destiné
à lui obéir un jour. La nature, qui se plaît à mille jeux différents,
avait mêlé son tempérament d'une ardeur qui, dans la jeunesse d'un
prince de ce rang, avait paru longtemps redoutable\,; mais la grâce, qui
se plaît aussi à dompter la nature, a tellement opéré en lui, que son
ouvrage peut passer pour un miracle, par l'incroyable changement qu'elle
a fait en si peu de temps, au milieu des plus impétueux bouillons de la
jeunesse\,; et à travers tous les obstacles sans nombre que l'âge, le
rang et la situation particulière qui, raffermie par plusieurs années,
sans qu'aucun de tous ces dangereux obstacles, toujours subsistants,
aient pu l'entamer, ôte toute inquiétude sur sa durée et sa solidité.
Dans cet état il n'y aurait rien à désirer, si tout ce qu'il y a de
grand, de rare, de merveilleux, d'exquis en lui en tout genre, se
montrait aussi à découvert qu'il lui serait aisé de le faire, et si des
bagatelles laissées aux plus grands hommes pour faire souvenir lés
autres qu'ils ne sont que des hommes, et les préserver de l'idolâtrie,
paraissaient moins. Je ne m'arrêterai donc pas à vous faire un portrait
de ce prince, qui surpasserait les forces des meilleurs peintres, et qui
vous est si parfaitement connu. Je me contenterai seulement d'en toucher
quelques traits, lorsque la matière m'y obligera pour la mieux éclaircir
et pour mieux exposer à vos yeux le fond de mes pensées, par rapport aux
choses en elles-mêmes et par rapport aux sentiments du monde dans lequel
la nécessité et la triste oisiveté de mon état me laissent plus répandu
que vous, et plus exposé à ses sottises.

Les devoirs d'un roi étant infinis, il ne semble pas que ce soit un
bonheur, pour ceux que Dieu appelle au trône par le droit de leur
naissance, d'y monter de bien bonne heure, et puisque dans les états,
même de toutes les conditions, la vie privée doit former aux emplois, et
de ne s'occuper que de se rendre dignes de ceux auxquels porte
naturellement la profession où on se trouve engagé, puisqu'il serait
également inutile et trop immense pour la portée de l'esprit humain de
tendre tout à la fois à se rendre capable de tous les emplois possibles,
il paraît qu'un prince que la couronne d'un grand État regarde ne doit
occuper tous les moments qu'il ne la porte pas encore, qu'à se rendre
capable de ce poids par toutes les connaissances qu'il exige, et comme
leur nombre est infini, à faire un juste choix des plus importantes,
certain que leur acquisition suppléera de reste à toutes les autres, et
que le point capital ne consiste qu'en la sagesse de ce discernement, et
après l'avoir fait en une application continuelle à s'instruire de ce
qu'on s'est proposé de savoir parfaitement\,; mais il ne semble pas
moins nécessaire d'ajouter une seconde partie à cette première, et c'est
de faire un tel usage de cette sorte d'étude, qu'un prince ne se
contente pas de se rendre capable de l'autorité souveraine\,; s'il
n'arrive encore à persuader à ceux qui seront un jour ses sujets qu'il
est déjà et qu'il deviendra de plus en plus digne de leur commander.

Rien n'embrasse mieux tout à la fois ces deux points de vue si
principaux que de joindre à la connaissance des sciences qui ouvrent
d'abord l'esprit, qui l'aiguisent dans la suite, et, ce qui est bien
plus important à un prince, celle de l'histoire de son pays, ce qui
renferme bien des choses, d'y joindre, dis-je, la connaissance des
hommes, sans laquelle l'esprit le plus éminent et le plus éclairé, ni
les précautions les plus exactes ni les plus vigilantes, ne peuvent
garantir des ténèbres les plus épaisses qui, répandues dans tout par
l'ignorance des instruments de tout, qui sont les hommes, précipitent en
des erreurs dont rien ne peut préserver, auxquelles nulle autre
connaissance ne peut suppléer, et dont toutes les suites deviennent des
abîmes en tout genre. Ce n'est donc pas un médiocre avantage à un prince
qui doit régner de vivre assez longtemps sujet, en âge de discernement,
pour pouvoir connaître les hommes par une sorte de familiarité et de
communication avec eux, qu'écarte ou qu'obscurcit d'ordinaire l'éclat du
diadème, et de profiter d'un intervalle de temps dont l'incertitude de
la durée ne sert pas peu à lui laisser voir les hommes à peu près tels
qu'ils sont, puisque, ne pouvant guère espérer pour le présent et pour
le futur qu'avec incertitude d'un prince encore éloigné de la
distribution des grâces, et néanmoins approchant souvent et
familièrement de lui, la liberté et l'impatience naturelle des hommes ne
se trouvant point captivée par la vivacité des vues présentes, et se
rencontrant souvent dans l'occasion, résistent difficilement à la longue
à les montrer à découvert tels qu'ils sont, et par ce moyen instruisent
infiniment un prince d'eux-mêmes et des autres. Ce raisonnement mal
expliqué, mais à la vérité duquel il se trouverait, je crois, peu de
contradicteurs, me conduit à me plaindre de deux choses, l'une réelle,
l'autre de l'effet qu'elle produit. C'est que Mgr le duc de Bourgogne ne
peut connaître les hommes à la vie qu'il mène, que conséquemment il ne
peut en être connu et qu'il ne l'est point en effet\,; son temps n'est
partagé qu'en deux sortes d'occupations, dont les unes, conformes à son
goût, le renferment dans le sérieux et la solitude cachée de son
cabinet\,; les autres, présentées par les liens de son état, sont par
lui tournées de manière à ne l'éloigner pas moins que les premières de
cette double connaissance des hommes, si recommandable et base unique du
bon usage de toutes les autres. Il est un temps qui doit être
principalement consacré à l'instruction particulière des livres, et ce
temps ne doit pas être borné à l'âge qui affranchit du joug des
précepteurs et des maîtres\,; il doit s'étendre des années entières plus
loin, afin d'apprendre à user des études qu'on a faites, à s'instruire
par soi-même, à digérer avec loisir les nourritures qu'on a prises, à se
rendre capable de sérieux et de travail, à se former l'esprit au goût du
bon et du solide, à s'en faire un rempart contre l'attrait des plaisirs
et l'habitude de la dissipation, qui ne frappent jamais avec tant de
force que dans les premières années de la liberté. Mais ce second temps
d'étude a déjà été si heureusement rempli, que le pousser au delà de ses
justes bornes est un larcin fait à d'autres sortes d'applications, pour
lesquelles celles-là n'ont dû servir que de préparations. Il est donc un
temps d'amasser et il est un temps de répandre, et c'est ce dernier qui
est déjà arrivé depuis longtemps, sans que Mgr le duc de Bourgogne
semble le reconnaître, et qui lui échappe avec un dommage infini. Si
l'enfance d'un prince était capable de percer les raisons des leçons
diverses qui lui sont successivement données, il reconnaîtroit que
l'intention de ses maîtres n'est que de lui donner une connaissance des
différentes sciences également nécessaires pour lui ouvrir l'esprit, lui
donner de l'application et de la solidité, le former au travail et au
sérieux, le préserver d'une ignorance fâcheuse, mais que leur dessein
n'est rien moins que de le pousser dans la suite à ces sciences, et de
lui faire perdre un temps destiné aux plus grandes fonctions de l'esprit
humain, à devenir un maître lui-même en ces sciences, par elles-mêmes
inutiles à tout ce qu'il doit être et sans contredit nuisibles, si,
porté à les suivre par son goût et par sa facilité, il continuait à les
cultiver dans la suite, puisque les jours étant limités à un certain
nombre d'heures et l'esprit à une certaine mesure d'application, il
pervertirait dangereusement l'ordre de son état et de sa destination en
mettant les sciences à la place des autres choses qui doivent uniquement
l'occuper.

Ce que l'enfance d'un prince n'est pas capable de pénétrer, la maturité
de l'âge le doit faire\,; et dès qu'il a atteint une connaissance
parfaite des sciences, il doit entrer en garde contre leur attrait, et,
pesant désormais leur estime à une juste balance propre à son état,
content de s'en être servi à l'usage pour lequel elles lui ont été
proposées, il ne doit plus regarder la continuation de l'étude que comme
un obstacle aux grandes fonctions où son esprit est appelé, et comme un
amusement peu digne de sa naissance, se réservant d'estimer les sciences
en elles-mêmes et les particuliers qui, étant nés pour elles, y ont fait
d'heureux et utiles progrès, également différent de ceux qui se
dédommagent de leur ignorance par un mépris insensé des sciences et
superbe des savants, et de ceux aussi qui, n'ayant par leur état que
l'oisiveté à combattre, remplissent excellemment la leur par les plus
précieux moyens d'orner et d'occuper leur esprit.

Quelque modestie qu'ait conservée Mgr le duc de Bourgogne parmi un si
grand nombre de connaissances vastes et profondes, dans lesquelles il
surpasse de bien loin tous ceux qui n'en ont pas fait de longues études
particulières, il ne peut néanmoins s'empêcher de reconnaître qu'il en a
acquis infiniment au delà de son besoin, par conséquent qu'il doit
porter sa curiosité et son application à ces autres choses pour
lesquelles il est né et pour lesquelles seules il a dû s'instruire.
C'est un ouvrier qui, ayant un ouvrage de main à exécuter, s'est fait
lui-même tous les outils, tous les instruments dont il peut avoir besoin
pour travailler à son ouvrage, auquel il se doit mettre sans délai,
sitôt qu'il s'est fourni de tout ce dont il avait affaire, et qui
différerait vainement et nuisiblement de travailler, si, ayant achevé
tous ses outils, il voulait encore s'en faire d'autres semblables, sans
qu'il en eût de nécessité.

On peut, ce semble, rapporter à cette comparaison le trop grand
attachement de Mgr le duc de Bourgogne dans son cabinet, et sa trop
grande complaisance pour le goût qu'il conserve de l'étude des sciences,
et pour le plaisir d'en parler. Quelques mots rares dans des occasions
convenables sont bienséants dans la bouche d'un prince qui sait et qui
veut exciter et honorer les sciences et les savants\,; mais il est aisé,
quand on en est plein et qu'on s'y plaît trop, d'excéder en cela et de
donner lieu au murmure d'une cour ignorante, mais instruite pourtant que
ce n'est pas le fait d'un grand prince, et que cela le distrait par trop
de ce qui doit faire son application principale.

Il serait donc à désirer que Mgr le duc de Bourgogne, moins assidu dans
son cabinet après y avoir rempli les devoirs du christianisme, n'occupât
toute sa solitude qu'à la lecture des histoires et des choses qui se
rapportent à ce que les livres peuvent contribuer à la connaissance des
hommes, à la science du gouvernement, et à quelques remarques là-dessus
courtes, mais pleines, et qu'il regardât cette sorte d'occupation comme
son unique affaire, comme la seule pour laquelle il lui est permis de se
dérober à la vue de la cour, et j'ajouterai, sans crainte, comme une
sorte de prière qui, dans un homme de son rang, n'est pas moins
précieuse devant Dieu que la meilleure prière de ceux dont l'état ne les
en distrait point. Rempli de la sorte par cette étude si conforme à
l'humanité, et à laquelle elle se porte plus naturellement qu'à aucune
autre, Mgr le duc de Bourgogne trouverait un remède qui lui est
nécessaire contre les distractions que les sciences abstraites
nourrissent, et que le monde passe si difficilement aux plus grands
hommes, bien moins encore à ceux qui doivent devenir les maîtres de
tous, et dont, par conséquent, le monde et chaque particulier regardent
les distinctions comme un larcin de leurs biens acquis, je veux dire
d'une application à eux, à leur parler, à leur répondre, simplement même
à les remarquer, à les distinguer au moins de l'air, et par les
manières, enfin à s'apercevoir d'eux, monnaie si utile aux princes,
ressort si puissant sur les sujets, espèce de dette que l'amour-propre
exige avec tant de rigueur, et qu'il est si avantageux aux princes qui
soit ainsi exigée, mais que les distractions abolissent en lui ôtant au
moins son cours avec peu de grâce qui s'interprète encore plus mal parmi
le monde qui en est si avide, par le peu qu'il comprend qu'il doit
coûter au prince.

Moins de temps donné au cabinet et plus précieusemeut employé, comme je
viens de le dire, en fournirait beaucoup plus pour la vie publique qui
forme si uniquement les liens réciproques d'un prince et d'une cour
qu'il doit regarder comme un abrégé de l'État, et par là même plus
d'occasions et de moyens de connaître les hommes par eux-mêmes, ce qui
ne s'acquiert que par leur fréquentation. Plus Mgr le duc de Bourgogne a
de devoirs à remplir par la jouissance que Dieu lui accorde encore de la
vie précieuse du roi et de Monseigneur, plus il doit être bon ménager du
temps qu'il doit donner au monde aux dépens de son cabinet, pour pouvoir
fournir à ses devoirs de sujet et de fils, et à ceux où l'engage sa
naissance envers la cour et le monde, puisqu'il doit faire assidûment
deux cours, et cependant en tenir une soigneusement lui-même\,; il a cet
avantage de voir, dans la conduite de Monseigneur envers le roi, ce que
lui-même doit faire envers l'un et l'autre, et il s'y porte si
naturellement à souhait que, s'il voulait ajouter au respect et à
l'assiduité du sujet un peu plus de la liberté du fils et du petit-fils,
il augmenterait la dignité de la bienséance de ses manières avec eux, et
ne leur plairait pas moins en leur donnant lieu à un épanchement plus
doux avec lui qui, sans rien ajouter à l'amitié et à la confiance qui ne
peuvent être désirées plus entières, attirerait peut-être davantage ce
qu'on ne peut bien exprimer que par dire se trouver bien à son aise, et
les flatterait plus sensiblement par cette sorte de respect plein
d'onction qui n'est permis qu'aux enfants des rois. C'est un remède
délicat et doux contre une timidité dont cette naissance et la tendresse
des traitements doivent défendre, et à laquelle l'entrée dans les
conseils, et ce qui les suit d'intime pour la communication des affaires
n'aurait pas dû laisser de ressources, il y a longtemps. Mgr le duc de
Bourgogne vient d'en faire un essai en la dernière promotion d'officiers
généraux\footnote{Ancenis, qui est aujourd'hui le duc de Béthune, alors
  mestre de camp du régiment de Bourgogne, fait brigadier à sa seule
  prière, Monseigneur n'en a de sa vie tant obtenu. (\emph{Notes de
  Saint-Simon}.)}, qui n'a pas été moins douce pour le roi que pour
lui-même, qui lui a fait un honneur infini parmi ce petit nombre de ceux
qui l'ont su, et qui doit lui être un exemple agréable pour le fortifier
dans cette conduite multipliée avec sa sagesse ordinaire à l'avenir.

Ce qui vient d'être dit sur les deux grands devoirs de Mgr le duc de
Bourgogne doit s'étendre avec encore plus de force sur d'autres devoirs
indirects que ceux-là lui imposent par lesquels il achève de remplir si
agréablement les principaux, que cela serait complet pour lui et pour
les personnes\footnote{M\textsuperscript{me} de Maintenon et
  M\textsuperscript{lle} Choin. (\emph{Idem}.)} qu'ils regardent, s'il
voulait prendre un soin plus libre de s'en approcher de plus en plus et
de le faire avec un naturel qui achèverait de charmer, et qu'il se peut
dire qu'il doit aux choses passées et au souvenir de ce qui s'est passé
ici pendant le cours de la dernière campagne et de l'hiver qui l'a
suivie\footnote{La disgrâce de M. de Vendôme. (\emph{Idem}.)}.

Entre tant de grâces si radieuses dont le ciel a comblé ce prince, il se
peut avancer qu'il n'y en a aucune dont il doive ressentir plus de joie
et de secours que de la princesse avec laquelle il se trouve uni par les
liens les plus saints et les plus tendres. Comme il n'est question ici
que de Mgr le duc de Bourgogne, je retiendrai l'effusion de mon coeur et
la pente naturelle de mon esprit sur M\textsuperscript{me} la duchesse
de Bourgogne. Je ne parlerai d'elle que par rapport à son époux, et je
ne craindrai point, après tout ce que j'ai dit de grand et d'élevé de
lui, de la lui proposer en plus d'une chose pour exemple. Et pour
ajouter encore ce mot à ce qui vient d'être dit des devoirs, de quelle
grâce n'accompagne-t-elle pas tous les siens, et de quelle réciproque
n'en est-elle pas en cela même récompensée\,? Le désir qu'elle a d'être
aimée lui inspire un noble soin et une attention qui lui a gagné tous
les coeurs. Vive, douce, accessible, ouverte, avec une sage mesure,
compatissante, peinée de causer le moindre malaise, dignement remplie
d'égards pour tout ce qui l'approche, elle en fait les constantes
délices, et les désirs même désintéressés de tout ce qui en est le plus
éloigné. C'est ce qui ne se peut qu'avec beaucoup d'esprit, mais à quoi
beaucoup d'esprit ne suffit pas\,; et c'est pour cela que Mgr le duc de
Bourgogne, qui en a tant lui-même, pourrait considérer ces dons dans son
épouse, et n'en pas dédaigner l'imitation et les grâces en tout
continuelles.

C'est un si grand bonheur que de savoir goûter celui qu'on possède,
qu'on doit voir avec ravissement combien le prince se plaît avec la
princesse\,; mais il serait à désirer aussi que, lui donnant tout le
temps dont tous deux doivent être contents et si jaloux, et qu'ajoutant
à leur entier particulier ce que la bienséance en exige encore pour sa
cour particulière, un milieu plus compassé entre la gravité et la bonté,
la liberté des privances et les familiarités trop usurpées, se
continssent par son propre exemple, et lui fissent rendre par les jeunes
dames\footnote{Les trois soeurs Noailles, toutes trois dames du palais.
  (\emph{Note de Saint-Simon}.)} le respect qu'elles lui doivent en tout
lieu et tout temps, et dont nulle gaieté n'excuse qui en sort ni qui
l'endure, bien moins qui y accoutume. Un peu d'attention à les remettre
peu à peu dans ce devoir par un air froid et surpris lorsqu'elles s'en
écartent, par quelques airs graves, mais toujours polis quand il est à
propos, par une petite affectation de silence et de sérieux un peu
continuée à l'égard de celles qui en auraient besoin, qui en même temps
instruirait les autres qui en seraient témoins, les corrigerait bientôt
toutes, et ferait un bien plus excellent effet qu'on ne se l'imagine
peut-être.

S'il est vrai que ces bagatelles intérieures sont vraiment importantes,
combien l'est-il plus de prendre garde qu'il n'échappe au dehors des
mouvements peu dignes de l'âge et du rang\,? Je ne me lasse point de
m'indigner du pernicieux usage que le monde en fait, et je gémis sans
cesse de voir encore des mouches étouffées dans l'huile, des grains de
raisin écrasés en rêvant, des crapauds crevés avec de la poudre, des
bagatelles de mécaniques, une paume et des volants déplacés\footnote{Pendant
  {[}le siège de Lille{]}. (\emph{Note de Saint-Simon}.)}, sans y
prendre garde des propos trop badins, soutenir avec un audacieux poids
les attentats de Flandre, et le trop continuel amusement de cire fondue,
et surtout de dessins griffonnés\footnote{Ces figures de l'abbé Genest.
  (\emph{Idem}.)}, augmenter les insolences par des problèmes
scandaleux. Plus ces bagatelles sont petites et paroissent innocentes,
plus elles blessent profondément et plus elles enfantent de
blasphèmes\,; c'est une vérité qui ne peut être suffisamment inculquée,
et qui doit marcher de front avec les vérités les plus solides et les
plus essentielles, puisque tel est le joug de la suprême grandeur que
tout se grossit en elle, et que les plus simples inadvertances sont
aussitôt tournées en symptômes qui retentissent aisément de tous côtés,
encore plus quand les fréquences de ces bagatelles peuvent passer pour
des habitudes, que le prince qui s'y laisse échapper, se rend d'ailleurs
difficile à se faire voir par l'arrangement de ses journées, et qu'il
demeure par là effectivement inconnu.

Cet arrangement des journées est tel dans Mgr le duc de Bourgogne, qu'on
ne peut pas contester que sa vie ne s'écoule dans son cabinet, ou parmi
une troupe de femmes.

Le monde, indulgent aux vices qu'il éprouve, passerait même
difficilement cette unique compagnie de femmes à un prince qui y serait
porté par ses plaisirs. Combien la trouvet-il donc surprenante dans Mgr
le duc de Bourgogne, dont il ne connaît que trop l'exactitude des
mesures qu'il n'est pas capable d'admirer\,?

C'est donc cet arrangement qu'il serait le plus important de rompre
comme mauvais et nuisible en soi-même, et comme obstacle encore à ce
qu'il y a de meilleur, je veux dire à cette connaissance si essentielle
des hommes à laquelle cette assiduité parmi des femmes qui au moins
n'apprend rien et perd cependant un temps précieux, sert de barrière
continuelle, et pour venir à quelques détails que cette grande matière
demande, il serait infiniment à souhaiter que Mgr le duc de Bourgogne ne
se contentât pas de tenir une cour mêlée par un jeu qu'il a néanmoins
été excellent d'établir, et qu'il est très à propos d'entretenir pour
avoir occasion de parler et de gracieuser le monde, mais qu'il
s'accoutumât aussi à un commerce d'hommes plus familier et plus
instructif, ce qui ne se peut que par des conversations particulières
qui lui concilieraient les esprits et les coeurs, qui les lui feraient
pénétrer, et qui le feraient connaître effectivement aux autres. Les
occasions en seront continuelles, pourvu qu'une volonté de bonne foi
soit le fruit de la persuasion de l'extrême importance et nécessité de
le faire. Il aime à se promener\,: pourquoi se fera-t-il une prison du
gros qui l'y accompagne, et pourquoi n'en prendra-t-il point quelqu'un,
tantôt un lieutenant général distingué, et puis un autre qui le sera
moins, mais qui sera instruit à fond des faits obscurs d'une campagne\,?
une autre fois un seigneur qui aura en soi autre chose que son nom,
ensuite un personnage de plume qui aura négocié\,? en un mot une fois
des uns, une autre fois des autres, mais presque toujours quelqu'un avec
lequel il s'avançât seul hors de sa cour\,; et se faisant suivre par son
officier des gardes hors de portée de l'entendre, il discoure avec celui
qu'il aura pris, et le fasse encore plus discourir lui-même, prenant
soin de le mettre à son aise, et surtout en sûreté, et de payer
d'attention les moindres choses qu'il lui dira. C'est ainsi que les
princes tirent du sein des hommes, avec application, art et
discernement, des vérités grandes et petites, mais toujours plus ou
moins importantes, qu'ils apprennent à distinguer à quoi ils sont
propres, à profiter de leurs lumières, de leurs humeurs, de leurs
intérêts\,; à démêler les choses d'avec les apparences, à tempérer une
discrète croyance par une discrète défiance, à se tenir en garde contre
les surprises, les artifices, les circonventions, pièges continuels des
princes, qui n'ont que ce moyen d'échapper, de savoir ce qu'eux seuls
bien souvent ignorent, d'éviter le poison en multipliant les canaux qui
conduisent jusqu'à eux\,; de découvrir la portée, les goûts, les amis,
les ennemis, les cabales des hommes\,; de saisir les instants où la
force de toutes ces diverses choses les fait malgré eux s'échapper à
eux-mêmes dans le tissu d'une conversation, de les pousser alors d'une
manière insensible au nuage de la passion qui s'échauffe en eux, et en
ne les rebutant sur rien d'attirer et de profiter de leur confiance qui
se refuse si difficilement à un prince qui ne dédaigne pas de la
rechercher.

Quand on ne parle qu'à un seul homme, l'idée de favori épouvante
aussitôt, mais lorsqu'on multiplie les conversations, dont on couvre le
choix d'un air d'indifférence, qu'on est surtout soigneux d'entretenir
les gens de parti ou de sentiment opposé, la crainte cesse, l'humanité,
l'accès attire, la bonté charme, les vertus, les connaissances, tout
l'esprit, tout le grand sens, tout l'usage qu'on en fait se découvre, et
en se découvrant se fait admirer, confond l'ignorance et la friponnerie,
s'insinue des uns aux autres à qui ces conversations de l'un à l'autre
reviennent, et par cette voie si facile un prince connaît et est connu,
et profitant du désir public de l'approcher se gagne le coeur de
l'esprit de ceux à qui il parle, et par eux de ceux encore à qui il ne
parle pas, devient difficile à se tromper et à se méprendre, compte
juste surtout, et, par une attentive combinaison de tout ce qu'il
entend, il porte sa vue sur le bon et sur le vrai autant qu'il est donné
de le découvrir ici-bas, et se guérit surtout de l'opinion mortelle que
la vérité est impénétrable aux princes, dont la condition serait dès là
trop déplorable s'ils ne pouvaient jamais agir qu'à tâtons. Par là
encore moins d'espérance et de hardiesse, et plus de danger à les
tromper, moins d'attentats et de possibilité à les gouverner, plus
d'émulation à se rendre capable et à bien faire, en un mot source
féconde de tout bien sans aucun péril à craindre\,; un temps toujours
bien employé, quelque stérilité qui se rencontrât quelquefois en
quelques-unes de ces conversations, dont il n'est pas possible qu'il n'y
eût toujours quelque chose à recueillir, et qui toutes s'allongent et
s'abrègent aisément, se remettent même au gré du prince, mais qui toutes
aussi doivent avoir un objet ou proposé à découvert, ou amené dans la
conversation avec adresse, et surtout ne pas parler toujours à un homme
de son métier\,; et tant pour apprendre que pour le sonder, le mettre
diverses fois sur les affaires présentes, sur la politique, le
gouvernement intérieur et extérieur, le commerce, la guerre de terre et
de mer, les divers personnages, en un mot sur une matière toujours
considérable, pousser les raisonnements et quelquefois les aiguiser en
entretenant doucement quelque dispute.

C'est un grand abus que de se persuader que des hommes ne soient pas
souvent fort instruits de bien des choses qui ne sont pas de la
profession à laquelle ils se sont particulièrement voués. L'esprit et le
bon sens portent à tout et sur tout\,; et encore que cela soit un
déréglement, il n'est pas rare de trouver des hommes médiocres dans le
métier qu'ils font, meilleurs et plus instructifs à entendre sur
d'autres choses, quelquefois même excellents. C'est donc à la patience
du prince à ne se rebuter pas, pour tâcher, en développant les hommes,
de tirer d'eux tout ce qui se peut sur toutes matières\,; à son bon
esprit à en faire le discernement, et à son bon sens à ne se laisser pas
trop facilement frapper des choses, et surtout à se bien persuader que
son temps ne peut être plus excellemment employé qu'en ces recherches
qui produisent en lui la science des sciences et le fondement des bons
conseils à prendre après avec lui-même en résumant ce qu'il a appris et
en démêlant bien toutes choses\,; quelquefois encore Mgr le duc de
Bourgogne ferait très-convenablement d'appeler dans son cabinet tantôt
un homme, tantôt un autre\,; mais cela semble devoir être beaucoup plus
rare, et réservé à des personnages principaux, ou à ceux qui reviennent
de quelque emploi considérable de guerre ou de négociation. J'ajouterai
encore que la liberté du tête-à-tête y fera trouver un plus grand profit
que dans les conversations de deux ou trois ensemble qui paroissent
bonnes seulement pour le salon de Marly et pour des lieux publics de la
sorte, dont l'oisiveté se peut mettre de cette manière à profit. Les
nouvelles et les occasions qui peu à peu se font naître les unes les
autres peuvent aisément servir d'ouverture et comme d'introduction à ces
conversations diverses, mais surtout un secret profond jusque des choses
les plus indifférentes qui s'y diraient en doit être l'âme et le fidèle
sceau, sans quoi elles deviendraient pires qu'inutiles.

Mgr le duc de Bourgogne est depuis si longtemps en habitude d'en garder,
et sa sûreté est même si connue, que ce n'est pas là une difficulté pour
lui. À l'égard des autres, leur intérêt y serait tout entière, et les
différentes conversations avec différentes personnes un bon moyen de
voir si elles y seraient fidèles, et de se conduire conformément avec
elles, ou avec plus de réserve, ou par l'exclusion de qui y aurait
manqué. Mais comme le dessein de ces conversations ne doit être rien
moins pour un prince que de s'y répandre, et qu'il y doit veiller
incessamment à demeurer aussi fermé qu'il se peut sans trop rebuter, il
ne peut jamais courir aucun risque\,; il n'est pas nécessaire de dire
que la flatterie toujours poison mortel le deviendrait doublement en ces
couversa-tions, et qu'un prince qui les lie ne peut jamais être assez en
garde contre elle, ni la bannir trop sévèrement par des réponses même
dures, sitôt qu'il s'en apercevrait, et j'en dis presque autant d'une
complaisance trop poussée. Le temps étant donc partagé de cette manière,
Mgr le duc de Bourgogne qui a tant et de si bon esprit, de sens, de
justesse, de lumières et de connaissances, occuperait une partie de ses
journées agréablement avec une infinie et double utilité. Quand la
promenade manque, à laquelle cette conduite attirerait bientôt tout ce
qu'il y a de meilleur, il peut prendre un homme à part au coin d'une
chambre du salon de Marly, de la galerie à Versailles, y en appeler
deux, quelquefois trois ensemble, les mettre aux mains, les faire
discourir, les échauffer un peu avec art, et recueillir comme les
abeilles le meilleur suc de ces différentes fleurs. Mais sur toutes
choses, il faudrait bannir de ces entretiens toute science, toutes
mécaniques, toutes chasses et toutes bagatelles, qu'il faut réserver
pour les entretiens et les propos publics, et ne se proposer dans
ceux-ci que la double mais centuple utilité que j'ai tâché de
représenter.

Que si cette pratique, qu'on ne peut assez relever et qui se lit encore
partout avoir été celle de tous les grands hommes chargés de quelque
gouvernement ou qui y étaient destinés par leur naissance, a paru depuis
assez longtemps s'anéantir en France, on sait qu'il y a des voies de
grands princes moins proposées à suivre qu'à admirer, et que la conduite
secrète des grands rois doit en quelques rencontres être respectée, par
un silence et une vénération qui tient quelque chose du religieux, et
qui pour cela même est au-dessus de l'imitation. Je reviens donc à dire
que, par cette communication fréquente et familière, on découvre où va
le général et le gros du raisonnement, et des sentiments du monde et sur
quels fondements\,; et le profit qui s'en tire est infini. Le prince
montre une estime et une facilité qui, peu à peu, malgré les hommes à
qui il parle, lui rend en quelque façon leur poitrine transparente,
tandis que le respect qui retient les questions et la trop grande
liberté des autres lui conserve à leur égard tous ses voiles sur la
sienne. Des hommes qui se croient consultés s'abandonnent aisément à
prendre un vif intérêt aux princes et aux choses de l'État, et cette
disposition se répand des uns dans les autres. Ceux même qui sont le
moins à portée de ces conversations ne peuvent que difficilement s'en
défendre, flattés en autrui, dès là que plusieurs y arrivent, de ce
qu'ils aimeraient pour eux-mêmes\,; et il ne faut pas penser que cet
intérêt d'affection ne soit pas un appui pour l'État infiniment utile
jusque dans les temps de la plus grande prospérité.

Mais il se présente une grande difficulté dans l'exécution si importante
de ces conversations, qui est la crainte qu'une trop scrupuleuse piété
inspire à Mgr le duc de Bourgogne de tout entretien qui ne roule pas
absolument sur les sciences et les bagatelles, et qui met sa langue et
ses oreilles dans de continuelles entraves, et son esprit dans une
pénible contrainte qui le raccourcit, et qui lui en empêche les
principaux usages qu'il ne tiendrait qu'à lui d'en faire. Son attention
à la charité du prochain le conduit à une ignorance entière de ses
défauts, et souvent aussi de ses vertus, et la frayeur de la blesser en
quoi que ce soit ou d'y donner occasion, va jusqu'à une terreur que les
supérieurs des plus saintes maisons regarderaient comme dangereuse en
eux pour le petit et simple gouvernement dont ils se trouvent chargés
pour un temps. Dieu, qui permet les défauts et les vices dans les hommes
et qui défend la calomnie et même la médisance, leur a cependant donné
des yeux pour voir et des oreilles pour entendre, et sa providence, dont
la sagesse est ineffable et qui a si diversement ordonné des diverses
sortes de fonctions de l'esprit humain, commande souvent aux uns ce
qu'elle défend aux autres, et forme une harmonie merveilleuse par cette
diversité qui tend également à sa gloire et au bien de la société des
hommes qui sont les États. Si donc le commun des hommes ne doit voir et
entendre qu'à travers la charité qui croit tout et qui souffre tout, et
si l'exacte exécution de ce devoir forme la paix et la concorde,
pourrait-on attendre le même fruit de cette même conduite fidèlement
gardée par ceux qui maintenant sont commis à quelque sorte de
gouvernement, et dans ceux encore entre les mains desquels est ou sera
remise un jour la souveraine administration du royaume\,? La confusion,
le chaos, les maux extrêmes, les pièges, les méprises grossières, les
artifices, les énormes ignorances, en un mot les désordres sans nombre
qui en résulteraient, sautent si vivement aux yeux, qu'il est superflu
de s'amuser aux preuves, et qu'il faut conclure que cette vigilance, si
fort recommandée à ceux qui sont en place, consiste très-principalement
à être bien instruits de ce que valent les hommes, à quoi il est
impossible qu'ils puissent parvenir sans s'en informer, sans en parler,
sans qu'on leur en dise le bien et le mal dans toute leur étendue, et
c'est après à eux à rechercher la vérité par des informations
multipliées et par un examen où ils apportent tout le discernement dont,
quelque esprit et quelque justesse qu'ils aient, ils ne peuvent être
capables que par ouvrir toutes leurs oreilles au bien et au mal, et leur
bouche à toutes les questions et à tous les propos indirects qui les
peuvent conduire par divers chemins à la connaissance de la vérité. Que
si des places subalternes donnent, je ne dis pas simplement cette espèce
de dispense dans l'usage du précepte de la charité pour la charité même,
puisqu'elle est due au public aux dépens du particulier, mais si ces
places imposent cette loi nécessaire et indispensable, on doit conclure
qu'elle oblige bien plus étroitement ceux dont les emplois sont plus
élevés, et à proportion de leur emploi et de leur importance, et plus
que tous ceux-là les princes qui sont par leur naissance destinés à
régner, surtout quand leur âge est devenu capable de porter leurs
devoirs, et qu'ils se trouvent appelés aux affaires.

C'est l'évidence et la force de cette juste considération qui doit non
pas affranchir Mgr le duc de Bourgogne de ses scrupules sur la charité
du prochain, mais les lui faire changer en d'autres, et l'obliger à
porter cette lampe, dont il se sert si soigneusement pour éclairer tous
les replis de son coeur et de sa conscience, non plus à l'examen
rigoureux de ce trop scrupuleux plus ou moins qui lui sera échappé sur
quelqu'un ou aux autres en sa présence, mais bien sur tout ce qu'il
aurait dû savoir, et qui lui est échappé par ce dangereux change de
scrupules, et dont l'ignorance ne va à rien moins qu'à ce qui vient
d'être dit plus haut, et à la perte de ce temps si précieux pour
acquérir la connaissance des hommes et leur communiquer la sienne, avant
que Dieu lui en diminue les moyens en l'appelant à la couronne, comme
j'ai tâché de l'expliquer au commencement de ce Discours.

Cette maxime si sûre, que la charité est due au public aux dépens du
particulier, ne peut donc être assez méditée par Mgr le duc de
Bourgogne. Il y découvrira que ce qui est défendu à la plupart des
hommes entre eux en qualité de discours inutiles, vains, dissipés,
légers, de médisance, de calomnie, de prévarication de charité, que tout
cela, dis-je, sont les viandes immondes de l'ancienne loi, permises dans
la nouvelle, commandées en certains cas\,; je veux dire que l'usage de
tout cela, réglé par la droiture de son intention et par la nécessité et
la charge de son état, lui est permis et commandé, et permis et commandé
aux autres envers lui, à qui ils doivent toute vérité et toute
information par respect pour ce qu'il est, et par la charité qu'ils
doivent au public et à l'État, au timon duquel Dieu même l'a mis, et
qu'il ne peut tenir avec un bandeau sur les yeux sous aucun prétexte,
pour saint qu'il paroisse, sans en devenir à l'instant responsable à
l'État et comptable au roi des rois, qui l'a revêtu d'honneur et de
gloire à condition expresse d'en acquitter toutes les charges et les
devoirs, dont le plus important et le plus continuel est d'être bien
instruit des hommes pour se servir d'eux bien à propos. Je sens qu'un
prince très-délicat sur la charité du prochain pourrait s'effaroucher
aisément de ce qui est dit un peu crûment par rapport à sa délicatesse,
par la comparaison des viandes immondes devenues permises et quelquefois
commandées\,; mais il ne doit pas séparer de cette expression la
réflexion du sens auquel elle est proposée, qui réservant aux délations
et aux mauvais offices toute l'horreur qui les doit toujours poursuivre
et proscrire, conserve également une sage et nécessaire liberté de
vérité et de lumières qui doit être le motif des instructions qu'il faut
rechercher, et l'âme de l'usage qui s'en doit faire.

Cette matière des conversations, m'ayant comme insensiblement conduit à
ce qui leur pouvait être opposé par la considération de la charité du
prochain, me fournit une occasion si naturelle de dire ce que je pense
de la dévotion de Mgr le duc de Bourgogne que je ne croirais pas remplir
ce que je me suis proposé, si, tout profane que je suis, je ne hasardais
d'en découvrir aussi mes pensées. Ce don de Dieu si grand, si saint, si
utile, même pour bien gouverner les choses de ce monde, pour le bonheur
temporel de ce monde même, ce don si rare, si désirable en tout homme,
l'est encore davantage à proportion de leur puissance et de leur
élévation\,; c'est un don qui apprend, avec une singulière excellence,
aux grands rois qu'ils ne sont faits que pour le bien et le bonheur de
leurs peuples, et que rien n'est plus particulièrement fait pour eux que
pour le dernier de leurs sujets. C'est encore ce don qui leur enseigne à
pratiquer éminemment cette justice qui s'étend à tout et dont ils sont
si étroitement redevables à Dieu et aux hommes, qui leur apprend à
découvrir leur petitesse parmi tant de grandeur, et à exercer l'humilité
avec une majestueuse douceur, qui augmente leur suprême dignité jusque
devant les hommes, et qui leur attire l'hommage de leurs coeurs avec une
bénédiction du ciel plus abondante. On ne peut donc regarder sans folie,
avec des yeux indifférents, ce grand don dans Mgr le duc de Bourgogne,
sur lequel il y a, outre les raisons générales, des grâces infinies à
rendre à Dieu pour le merveilleux effet qu'il a produit en lui, comme il
a été remarqué au commencement de ce Discours, et sans lequel les plus
libertins auraient pu admirer ses grandes qualités également, mais les
aimer moins et les redouter davantage. Je suis donc bien éloigné
non-seulement de ceux qui n'ont pas honte de s'en plaindre, mais de ceux
encore qui lui en désireraient moins, et je tiens fermement qu'il n'est
aucun sujet de ce royaume qui, à ne regarder même que son bien temporel,
ne doive autant ou presque autant rendre grâces à Dieu de la piété de
Mgr le duc de Bourgogne que ce prince lui-même\,; mais cette puissante
conviction de mon esprit ne le ferme pas aux réflexions qui se peuvent
faire sur l'austérité qui y est jointe, et qui pourrait être comparée à
quelque petite âpreté d'un fruit très-délicieux. Pour expliquer cette
importante matière, il est nécessaire de se permettre quelque détail
après avoir posé quelques principes qui puissent être reçus. On n'en
doit point chercher ailleurs ici que dans le Saint-Esprit même, parmi
les divines Écritures, où on trouve écrit qu'il faut que \emph{les forts
supportent les faibles}, ordonnance si conforme à la charité du
prochain, dont il était mention tout à l'heure\,; ailleurs, \emph{qu'il
faut être sage avec sobriété}. Sage ici doit, ce me semble, comprendre
piété, bonnes oeuvres, et tout ce qui appartient enfin à cette sagesse
qui renferme tout ce qui l'est devant Dieu. Pour peu que l'on médite ces
deux passages, on verra bientôt combien ils se soutiennent tous deux, et
combien de rapport ils ont l'un à l'autre. Que les forts supportent les
faibles, n'est-ce pas ne les point effrayer par des maximes trop sèches
et par une conduite trop à la lettre et trop attachée au scrupule, et à
une certaine exactitude que tous ne peuvent pas porter\,? Et garder la
sobriété jusque dans la sagesse, n'est-ce pas ne la pas porter au delà
de ce que l'ordinaire des hommes et les faibles peuvent aisément
faire\,? Ainsi un sage supérieur est en garde contre le zèle de ses
religieux, et en même temps qu'il a les yeux ouverts sur tout ce qui est
du précepte véritable de la règle, il les ferme sur un grand nombre de
bagatelles qui la rendent plus dure, qui se sont introduites par degrés
et en diverses rencontres\,; et sans y renoncer formellement, parce
qu'elles sont pieuses, quoique venues d'ailleurs que de l'instituteur,
il est charitablement soigneux de n'y être pas trop exact pour lui-même,
de peur de mortifier par son exemple, jusqu'au trouble, les faibles de
sa communauté, qui atteignant à peine les observances prescrites et
nécessaires, encore qu'ils y soient fidèles, viendraient à s'en dégoûter
par ce surcroît qui n'en fait point partie, et dont l'accablement leur
ferait peur. Voilà donc cette sagesse sobre qui supporte les faibles, et
cette force qui ménage ceux qui en ont besoin jusque dans les
monastères, qui fait les plus excellents supérieurs\,; et pour peut
qu'on ait occasion par hasard de fréquenter quelques maisons
religieuses, on y aura trouvé bien des exemples très-loués, et
très-recommandables par leurs succès, de ce que j'ose en avancer. S'il
en est donc ainsi parmi des victimes de pénitences cachées dans le
secret de la face de Dieu, et que rien ne détourne de tendre à lui de
toutes leurs forces, de quelle indulgence ne sont donc pas redevables
aux mondains les exemples qui, en caractérisant celui qui doit être leur
maître pour toujours, le leur rendent âprement ou doucement vénérable,
les attirent ou les intimident, et les repoussent par des considérations
diverses ou les invitent par une puissante facilité\,! À ces vérités, il
s'en doit ajouter une autre\,: c'est que la dévotion, qui est de tous
les états, doit être différemment pratiquée par tous les états, et
qu'elle devient d'autant plus parfaite qu'elle se trouve plus
proportionnément mesurée, non en elle-même, mais en sa pratique et en
ses effets, à l'état auquel on est appelé. Qu'un religieux ne doive
faire un autre usage de sa piété qu'un autre religieux d'un autre ordre,
ou qu'un autre du sien même, cela est constant, puisque les divers
instituts sont diversement appliqués à l'action et à la contemplation, à
la solitude et à l'instruction des autres, et les divers particuliers
qui en sont à gouverner les autres en divers degrés d'emplois et à être
gouvernés. D'où il résulte que si tous exerçaient leur dévotion en la
même manière, que les jésuites voulussent être solitaires, les chartreux
enseigner, ainsi du reste, et les supérieurs s'anéantir dans l'humilité,
et les inférieurs veiller sur leurs frères et les reprendre, une source
si sainte ne laisserait couler que le poison d'une confusion étrange,
qui ne contribuerait à rien moins qu'à la gloire de Dieu et au salut des
hommes. Que s'il est donc vrai que les divers instituts et les divers
offices des maisons religieuses doivent y diriger diversement la
dévotion, cette même nécessité se trouve encore plus formelle dans les
divers états du siècle, dont les devoirs et les fonctions, étant si
différents, doivent tourner aussi la dévotion de chacun si différemment.
Or celle d'un prince, et si proche du sceptre, le doit porter à tout ce
qui l'en peut rendre digne, et le faire paraître tel a tout le monde,
dont la voie la plus importante et la plus assurée est cette double
connaissance des hommes par tous les moyens qui la peuvent acquérir, et
une impression d'estime et de vénération qui se tire également de toutes
les actions du prince, et qui s'y reporte en même temps, en sorte qu'il
est très-vrai de dire que ce réciproque est tel qu'un prince devient
recommandable à proportion du mérite de ses actions, et les actions du
prince recommandables aussi à proportion de son propre mérite. Il ne
peut donc prendre garde de trop près à ce qui forme le tissu de sa
vie\,; et tout grand, tout sublime, tout au-dessus qu'il puisse être du
commun des hommes par des vertus extraordinaires, il doit ménager leur
faiblesse en s'abaissant à garder quelque proportion avec eux, et
puisqu'il est appelé à être un jour l'image de Dieu, il ne doit pas
dédaigner de voiler sa face devant eux, de peur que l'éclat de la
lumière dont elle brille ne les épouvante, et ne les fasse mourir, comme
il est écrit que pour cette raison Dieu même s'est voilé ainsi en se
découvrant à quelques-uns de son peuple\,; et comme Dieu n'y perdit rien
de son immutabilité, le prince aussi, par cette sage et nécessaire
condescendance, ne doit pas craindre aucun affaiblissement de ses
vertus.

Ainsi donc une assiduité moins exacte à l'office divin, tous les
dimanches et toutes les fêtes de l'année, n'ôterait rien devant Dieu à
Mgr le duc de Bourgogne des chastes délices qu'il trouve à ouïr chanter
ses louanges, et en se rapprochant plus de l'ordinaire des hommes, il
les rendrait plus capables d'admirer en lui les choses principales qui
forment l'essence de la religion. Ainsi une fuite moins rigoureuse de
certaines fêtes qui, dans tous les siècles, ont été nécessaires pour
l'amusement et la majesté des grandes cours, rendrait en lui la piété
plus aimable, je n'ose dire moins terrible. Ainsi un front plus serein,
un air plus aisé, quelque chose de plus leste en de certaines occasions,
dilateraient les coeurs que la vue du contraire resserre avec crainte.
Ainsi un art plus onctueux et plus doux d'allier la haute piété avec les
bienséances de l'âge et du rang, avec les convenances de grand prince,
dirais-je de fils, en quelques rencontres, ajouteraient au mérite de
l'intention de la victoire sur les répugnances, celui de la conformité à
son état, de la douce et charitable condescendance pour les autres, de
ce voile enfin sur la splendeur de sa face que les hommes supportent si
difficilement sans cela, pour ne pas dire qu'ils ne le peuvent, et
donnerait à la vertu une grâce et une douceur qui ne la rabaisserait pas
devant Dieu, et qui la rehausserait infiniment devant les hommes en les
rendant capables de l'admirer et de l'aimer avec transport, affranchie
alors de ces rides austères, de ces presque involontaires froncements,
de cette gêne de précisions qui ne sont pas la vertu, et qui, entées sur
elle, font tout fuir en sa présence, et creusent chez les hommes qui en
dépendent les plus profondes dissimulations. Elles y sèment une horrible
et abondante hypocrisie, et toutes les autres si dangereuses
transformations qu'opèrent l'intérêt et l'ambition dans les courtisans
et dans ceux qui veulent ou arriver, ou au moins plaire. De là s'élève
un mur entre le prince et les hommes, qui devient d'autant plus
impénétrable, que sa nature, la plus épaisse de toutes, se trouve aidée
de cette crainte de blesser la charité qui supprime avec sécurité tous
moyens de percer les masques, par quoi périt avant de pouvoir naître
cette double connaissance des hommes, devoir toutefois si grand et si
principal d'un prince.

Mais il ne me suffit pas d'avoir tâché d'expliquer l'excellence et la
nécessité du devoir d'un prince de connaître les hommes, si je ne
m'efforce de représenter aussi l'excellence et la nécessité du devoir
d'un prince de se faire connaître aux hommes, ce qui n'a pu jusqu'ici
être assez fortement touché. Il n'est personne qui ne convienne que, de
l'idée qui se conçoit d'un prince par l'effet des regards curieux qui le
percent et des réflexions que l'application y ajoute, ne se forment
toutes les démarches d'une cour, et par elle d'un État, qui ont rapport
à lui en quelque genre que ce puisse être, que chacun ne s'anime ou ne
se ralentisse à bien faire sur la mesure (je parle du gros qui est
l'important), sur la mesure d'utilité ou d'inutilité qu'il y voit pour
son intérêt, et qu'il ne s'accoutume au travail, ou ne se contente de
l'apparence suivant ce qu'il juge qu'il faut ou qu'il suffit\,; que
mesurant son respect et son zèle sur ce qu'il pense du prince, l'un et
l'autre ne soit vif ou éteint, et leurs effets de même, suivant ce que
son opinion ou l'expérience lui enseigne être le plus profitable, et que
de ce prince de tout bien ou de tout mal, qui est le respect de
l'opinion d'un prince, ne coulent pour lui tous les déportements de tous
ceux qui, en tout genre, composent l'État qui le regarde. Ces vérités si
grandes et si solides, que la raison et l'expérience de tous les siècles
rendent telles, n'ont besoin que d'un peu de méditation pour en faire
sentir tout le poids et toute l'étendue, sans avoir recours à une plus
grande explication, et ne demandent qu'un peu d'application à Mgr le duc
de Bourgogne, laquelle me force à un court examen qui m'a souvent coûté
bien des réflexions amères. Pour y entrer utilement tout d'un coup, il
serait infiniment à désirer que ce prince qui n'a point changé, et qui
si constamment est digne que l'on ne change point pour lui, fît quelque
comparaison de lui-même pendant ses deux premières campagnes et tout le
temps qui les a suivies, jusqu'à son départ pour la dernière, avec
lui-même pendant cette dernière campagne et depuis. Jamais le fameux
prince de Galles, dont toute l'Europe plaint encore aujourd'hui avec les
mêmes élans que l'Angleterre le sort trop promptement tranché, ne fit
plus véritablement les délices des siens, le plus doux espoir de son
pays, l'admiration la plus attentive de tous les grands hommes de son
temps, et de toutes les terres étrangères, que Mgr le duc de Bourgogne
dans ces premiers temps. Tout ce qui respire encore en est témoin, et
ses modestes yeux n'ont pu refuser de s'en apercevoir eux-mêmes.
Qu'est-il donc arrivé depuis qui ait pu affaiblir tant de lustre, et qui
ait rendu cet éclat moins vif dans tous les lieux, même les plus
reculés, où il avait pénétré\,? une pratique de piété la plus simple qui
soit conseillée par la vérité même, mais si contraire à l'état de Mgr le
duc de Bourgogne, que je crois pouvoir avancer, sans témérité, que de
cette pratique de vertu, le comble de toutes les autres pour le commun
des hommes, il ne doit pas être sans crainte d'en compter un jour devant
Dieu.

Je me garderai bien de tomber dans un détail cruel, qui rouvrirait en
moi des plaies encore sanglantes, de ce que j'ose nommer également des
deux côtés des attentats, en l'un d'impudence, en l'autre de patience,
et dont le châtiment est trop pesamment tombé de ce dernier côté, sans
que celui qui est tardivement arrivé à l'autre ait rien produit de
solide, que les acclamations les plus fortes et les plus tendres des
coeurs, l'espérance et l'admiration la plus vive, la gloire la plus
brillante mais la plus solide, qui suivront toujours la jeune mais
vénérable princesse qui, si continuellement et si constamment sensible à
la gloire de son époux, a triomphé seule, également grande devant Dieu
et devant les hommes, par un changement inattendu que seule elle a
produit\footnote{La chute sans retour du duc de Vendôme. \emph{(Note de
  Saint-Simon.)}}\,: action si conforme à l'état où la Providence l'a
placée, et qui en a si dignement rempli tous les divers et plus
importants devoirs. C'est de ces mêmes devoirs que ceux de Mgr le duc de
Bourgogne le devaient presser de se souvenir, et de ce qu'ils exigeaient
de lui pendant tout le cours de cette campagne dans le rang et la place
où il était, dont la conservation du respect et des droits sacrés de sa
naissance lui étaient si étroitement recommandés par tout ce que la
piété bien entendue a de plus indispensable, et auxquels il a donné lieu
et audace de penser qu'il ne songeait pas, alors ni depuis. Les
conséquences de cette omission sont telles, que la plus grande
application de Mgr le duc de Bourgogne doit se porter incessamment sur
elle, comme sur ce qu'il a et qu'il aura jamais de plus important,
puisqu'il n'y a rien qui expose un prince à de plus grands ni à de plus
continuels dangers que le malheur de s'être rendu soi-même évidemment
complice de l'opinion publique, du dedans et du dehors, ou qu'il n'est
pas sensible ou qu'il s'est fait une religion de ne l'être pas. J'irais
trop loin si j'en disais davantage\,; mais l'importance extrême de cette
matière, qui ne peut être assez comprise, m'a forcé d'aller aussi avant
que je fais, et que je n'ai au moins pu me dispenser de faire.

C'est cet amour de l'ordre qui conserve à chaque état ce qui lui
appartient, non par attachement, par goût, par amour-propre, mais par
respect pour la volonté de Dieu énoncée par la parole muette mais
toujours existante des devoirs respectifs des divers états, et par amour
pour cette justice distributive qui doit veiller sans cesse, qui est
tant recommandée à ceux qui se trouvent revêtus de puissance et sans
laquelle toute l'harmonie des états se défigure et se renverse peu à peu
d'une étrange manière et jusqu'à un point pernicieux. La négligence de
le maintenir remarquée dans un prince, par quelque considération que ce
soit, devient bientôt un mobile puissant de trouble qui dégénère en
destruction\,; et il n'est point de motif, pour saint qu'il soit en soi,
qui y puisse servir d'excuse devant Dieu ni devant les hommes. Mais il
faut mettre des bornes à l'abondance et à l'importance de cette matière,
qui est intarissable, et qui se présente presque à tous moments à un
grand prince par les occasions continuelles de méditation et de
pratique.

Une des choses du monde que doit le plus soigneusement éviter un prince
destiné à régner est l'opinion parmi les autres, que, frappé trop
fortement de quelque chose, il ne mesure toutes ses connaissances et
tous ses choix que là-dessus, et que l'impression que le monde a reçue
de la grande dévotion de Mgr le duc de Bourgogne, ne continue à le
persuader que ce prince ne juge de l'aptitude et de la capacité même des
hommes que par ce qu'il leur croit de piété, et qu'il ne préfère un
homme de bien pour tout emploi, sans nulle autre raison que celle de sa
vertu. Il suffit de présenter cette pensée toute nue pour en faire
apercevoir les suites funestes en réalité, si cette opinion était
fondée, et que l'exécution en fût réelle, ou même, étant fausse, qu'elle
ne cessât point de prévaloir parmi les hommes. C'est aussi ce qui mérite
tous les soins et toute l'attention possibles, pour ôter au monde une
impression si dangereuse, et si aisément féconde en toutes sortes de
grands inconvénients.

On ne peut exagérer assez la funeste croyance qu'a trouvée partout cette
prétendue consultation faite en Sorbonne, au moins à plusieurs docteurs
particuliers, par ordre de Mgr le duc de Bourgogne\,: savoir, si dans
les conjonctures présentes il est ou il n'est pas permis de faire la
guerre au roi d'Espagne. Nier ce fait à Paris et dans les provinces, on
s'élève avec impétuosité et on ne souffrira pas, dit-on, qu'on en
impose\,; le nier à la cour, aux personnages de l'un et l'autre sexe, on
sourit et on change dédaigneusement de propos. Si on est plus libre avec
eux, ils déclarent leur compassion pour les dupes qui ne le veulent pas
croire, et ils finissent souvent par l'indignation. Leur opiniâtreté se
soutient par la fréquence et la longueur des entretiens de Mgr le duc de
Bourgogne avec son confesseur, auquel on souhaite longue vie, parce
qu'on l'estime et qu'on en craindrait un autre. On regarde cette place
comme la première dans le conseil du prince, et à l'avenir dans le
conseil du roi qu'il sera un jour. On pense avec angoisse que le
ministère ne sera plus séparable de la théologie\,; que les affaires,
que les grâces, que tout enfin deviendra point de conscience et de
religion\,; et on jette tristement les yeux sur les derniers princes de
la maison d'Autriche qui ont porté la couronne d'Espagne. À ces frayeurs
des bons se joignent les réflexions malignes des fripons. Toute réplique
est exclue, proscrite, inutile\,; et voilà de ces inconvénients profonds
qu'un prince ne soit pas connu des hommes.

C'est ce qui doit puissamment convier le nôtre de ne perdre plus un seul
instant à travailler de toutes ses forces à parvenir à cette double
connaissance des hommes si souvent répétée, à y arriver par tous les
moyens possibles, à s'en faire une loi par principe de religion, et à
renfermer tellement la sienne dans la justesse de ce qu'elle lui impose,
par rapport à son état, qu'il s'affranchisse de tout ce qui n'en est pas
l'essence par cette douce liberté des enfants de Dieu, qui de
l'intérieur se répand aux choses extérieures. Qu'il cesse de mettre sous
le boisseau cette pure et brillante lumière que Dieu même, en l'en
revêtant, a placée sur le plus haut chandelier\,; qu'il paroisse donc
tout ce que véritablement il est, et, pour ne point tomber dans la
répétition des justes éloges qui ont commencé et qui doivent finir ce
Discours, qu'il s'assure qu'il paraîtra, comme autrefois Tite, les
délices du genre humain, et que, sans rien perdre de la sainteté de
saint Louis, il se montrera aussi grand que les derniers rois ses
illustres et magnanimes pères\,; que pour cela il n'a qu'à le bien
vouloir, puisqu'il ne s'agit que d'en développer la vérité et la réalité
au monde, lesquelles sont avec tant d'abondance en Mgr le duc de
Bourgogne.

Vous avez si absolument voulu que je vous écrivisse mes pensées sur Mgr
le duc de Bourgogne, et qu'en même temps je vous rendisse compte de
celles qui ont prévalu dans le monde sur ce prince, que je n'ai pas cru
qu'il me fût permis de rien omettre des miennes ni de celles du public.
J'ai remarqué, en commençant, que l'oisiveté devenue l'apanage de mon
état me répand plus que vous dans le monde, et m'y expose à entendre ses
sottises. Vous m'êtes témoin combien souvent et vivement elles m'ont
irrité, par rapport à Mgr le duc de Bourgogne\,; et, outre le public que
je n'ai pas redouté sur cela, j'en ai autant de témoins que d'amis
particuliers et ce qu'il y a de personnes principales des deux sexes
avec lesquelles je vis en privance. C'est maintenant à votre profonde
sagesse et votre judicieux discernement à juger de ce que vous m'avez
contraint d'exposer sous vos yeux, et à moi à m'y abandonner sans
réserve. La matière en est telle, qu'il ne faut pas un moindre ni un
moins ancien respect que celui que je vous ai voué pour vous donner
cette marque si singulière de mon entière obéissance. L'usage en sera
pour vous seul, s'il vous plaît\,; et la confiance qu'une longue et
douce habitude me commande d'avoir en vous, jointe à celle que vous avez
de garder impénétrablement les plus grands secrets de l'État, me fait
compter sans crainte que vous ne me garderez pas celui-ci moins
religieusement que vous faites ceux-là, puisque vous jugez bien
vous-même qu'il m'est d'une importance infinie.

\hypertarget{chapitre-x.}{%
\chapter{CHAPITRE X.}\label{chapitre-x.}}

1710

~

{\textsc{Crayon de Mgr le duc de Bourgogne pour lors.}} {\textsc{-
Succès de ce Discours.}} {\textsc{- Intrigue du mariage de M. le duc de
Berry.}} {\textsc{- Obstacles contre Mademoiselle.}} {\textsc{- Causes
de ma partialité sur ce mariage.}} {\textsc{- Fondement de ma
détermination de former une cabale pour Mademoiselle.}} {\textsc{- Duc
et duchesse d'Orléans.}} {\textsc{- Duc et duchesse de Bourgogne.}}
{\textsc{- Duchesse de Villeroy.}} {\textsc{- M\textsuperscript{me} de
Lévi.}} {\textsc{- M. et M\textsuperscript{me} d'O, par ricochet.}}
{\textsc{- Duc du Maine, par ricochet.}} {\textsc{- Ducs et duchesses de
Chevreuse et de Beauvilliers.}} {\textsc{- Jésuites.}} {\textsc{- Noeud
intime de la liaison du P. Tellier avec les ducs de Chevreuse et de
Beauvilliers.}} {\textsc{- Maréchal de Boufflers.}}

~

Une courte anatomie de ce Discours ne sera pas inutile pour la suite. Il
faut dire d'abord que Mgr le duc de Bourgogne était né avec un naturel à
faire trembler. Il était fougueux jusqu'à vouloir briser ses pendules
lorsqu'elles sonnaient l'heure qui l'appelait à ce qu'il ne voulait pas,
et jusqu'à s'emporter de la plus étrange manière contre la pluie quand
elle s'opposait à ce qu'il voulait faire. La résistance le mettait en
fureur\,: c'est ce dont j'ai été souvent témoin dans sa première
jeunesse. D'ailleurs un goût ardent le portait à tout ce qui est défendu
au corps et à l'esprit. Sa raillerie était d'autant plus cruelle qu'elle
était plus spirituelle et plus salée, et qu'il attrapait tous les
ridicules avec justesse. Tout cela était aiguisé par une vivacité de
corps et d'esprit qui allait à l'impétuosité, et qui ne lui permit
jamais dans ces premiers temps d'apprendre rien qu'en faisant deux
choses à la fois. Tout ce qui est plaisir il l'aimait avec une passion
violente, et tout cela avec plus d'orgueil et de hauteur qu'on n'en peut
exprimer\,; dangereux de plus à discerner et gens et choses, et à
apercevoir le faible d'un raisonnement et à raisonner plus fortement et
plus profondément que ses maîtres. Mais aussi, dès que l'emportement
était passé, la raison le saisissait et surnageait à tout\,; il sentait
ses fautes, il les avouait, et quelquefois avec tant de dépit, qu'il
rappelait la fureur. Un esprit vif, actif, perçant, se roidissant contre
les difficultés, à la lettre transcendant en tout genre. Le prodige est
qu'en très-peu de temps la dévotion et la grâce en firent un autre
homme, et changèrent tant et de si redoutables défauts en vertus
parfaitement contraires. Il faut donc prendre à la lettre toutes les
louanges de ce Discours.

Ce prince, qui avait toujours eu du goût et de la facilité pour toutes
les sciences abstraites, les mit à la place des plaisirs dont l'attrait
toujours subsistant en lui les lui faisait fuir avec frayeur, même des
plus innocents, ce qui, joint à cet esclavage de charité du prochain, si
on ose hasarder ce terme, dans un novice qui tend d'abord en tout à la
perfection, et qui ignore les bornes des choses, et à une timidité qui
l'embarrassait partout faute de savoir que dire et que faire à tous
instants, entre Dieu qu'il craignait d'offenser en tout et le monde avec
lequel cette gêne perpétuelle le mettait de travers, le jeta dans ce
particulier sans bornes, parce qu'il ne se trouvait en liberté que seul,
et que son esprit et les sciences lui fournissaient de reste de quoi ne
s'y pas ennuyer, outre que la prière y occupait beaucoup de son temps.
La violence qu'il s'était faite sur tant de défauts et tous véhéments,
ce désir de perfection, l'ignorance, la crainte, le peu de discernement
qui accompagne toujours une dévotion presque naissante, le faisait
excéder dans le contre-pied de ses défauts, et lui inspirait une
austérité qui outrait en tout, et qui lui donnait un air contraint, et
souvent, sans s'en apercevoir, de censeur, qui éloigna Monseigneur de
lui de plus en plus et dépitait le roi même. J'en dirai un trait entre
mille qui, parti d'un excellent principe, mit le roi hors des gonds, et
révolta toute la cour deux ou trois ans auparavant. Nous étions à Marly,
où il y eut un bal le jour des Rois\,; Mgr le duc de Bourgogne n'y
voulut seulement pas paraître, et s'en laissa entendre assez tôt pour
que le roi, qui le trouva mauvais, eût le temps de lui en parler d'abord
en plaisanterie, puis plus amèrement, enfin en sérieux et piqué de se
voir condamné par son petit-fils. M\textsuperscript{me} la duchesse de
Bourgogne, ses dames, M. de Beauvilliers même, jamais on n'en put venir
à bout. Il se renferma à dire que le roi était le maître, qu'il ne
prenait pas la liberté de blâmer rien de ce qu'il faisait, mais que
l'Épiphanie étant une triple fête et celle des chrétiens en particulier
par la vocation des gentils et par le baptême de Jésus-Christ, il ne
croyait pas la devoir profaner en se détournant de l'application qu'il
devait à un si saint jour, pour un spectacle tout au plus supportable un
jour ordinaire. On eut beau lui représenter qu'ayant donné la matinée et
l'après-dînée aux offices de l'Église et d'autres heures encore à la
prière dans son cabinet, il en pouvait et devait donner la soirée au
respect et à la complaisance de sujet et de fils\,: tout fut inutile,
et, hors le temps de souper avec le roi, il fut enfermé tout le soir
seul dans son cabinet.

Avec cette austérité il avait conservé de son éducation une précision et
un littéral qui se répandait sur tout, et qui gênait lui et tout le
monde avec lui, parmi lequel il était toujours comme un homme en peine
et pressé de le quitter, comme ayant tout autre chose à faire, qui sent
qu'il perd son temps et qui le veut mieux employer. D'un autre côté, il
ressemblait fort à ces jeunes séminaristes qui, gênés tout le jour par
l'enchaînement de leurs exercices, s'en dédommagent à la récréation par
tout le bruit et toutes les puérilités qu'ils peuvent, parce que toute
autre chose de plaisir est interdite dans leurs maisons. Le jeune prince
était passionnément amoureux de M\textsuperscript{me} la duchesse de
Bourgogne\,; il s'y livrait en homme sévèrement retenu sur toute autre,
et toutefois s'amusait avec les jeunes dames de leurs particuliers,
souvent en séminariste en récréation, elles en jeunesse étourdie et
audacieuse. On trouvera donc dans cette courte exposition les raisons de
bien des traits du Discours qu'on vient de lire, qu'on ne comprendrait
pas aisément sans cet éclaircissement, et surtout celle qui m'a fait
étendre en raisonnement de piété, pour tourner un peu plus au monde la
piété de ce prince qui n'était pas susceptible d'écouter, bien moins de
se rendre, par d'autres raisons que par celles de la piété même.

Ses deux premières campagnes lui avaient été extrêmement favorables, en
ce que, étant éloigné des objets de son extrême timidité et de celui de
son amour, il était plus à lui-même et se montrait plus à découvert,
délivré des entraves de la charité du prochain par les matières de
guerre et de tout ce qui y a rapport, qui, dans le cours de ces
campagnes, faisait le sujet continuel des discours et de la
conversation\,; tellement qu'avec l'esprit, l'ouverture, la pénétration
qu'il y fit paraître, il donna de soi les plus hautes espérances. La
troisième campagne lui fut funeste, comme je l'ai raconté en son lieu,
parce qu'il sentit de bonne heure, et toujours de plus en plus, qu'il
avait affaire, chose également monstrueuse et vraie, à plus fort que lui
à la cour et dans le monde, et que l'avantageux Vendôme, secondé des
cabales qui ont été expliquées, saisit le faible du prince, et poussa
l'audace au dernier période. Ce faible du prince fut cette timidité si
déplacée, cette dévotion si mal entendue qui fit si étrangement du
marteau l'enclume et de l'enclume, le marteau, dont il ne put revenir
ensuite.

C'est en peu de mots ce qui forme toute la matière de mon Discours, par
lequel, après les louanges méritées et ailleurs encore entrelacées pour
faire passer ce qui les suit, je tâche de faire voir quel est l'usage
que Mgr le duc de Bourgogne doit tirer de son cabinet, l'abus qu'il en
fait et dont il ne sort rien de ce qu'il y fait peut-être de plus
convenable à son état pour son instruction particulière. Après avoir
essayé à faire voir ce qu'il y doit faire en beaucoup moins de temps
qu'il n'y en donne, je viens à combattre sa timidité, et si cette
expression se peut hasarder, ce pied gauche où il est avec le roi et
Monseigneur, avec le monde, par tout ce qu'il m'est possible, et encore
avec M\textsuperscript{me} de Maintenon et M\textsuperscript{lle} Choin,
choses toutes si principales\,; enfin à combattre son éternel
particulier avec M\textsuperscript{me} la duchesse de Bourgogne seule,
que je loue d'ailleurs avec sincérité, et avec ce fatras de femmes qui
abusent avec indécence de sa bonté, de ses distractions, de sa dévotion
et de ses gaietés peu décentes qui sentent si fort le séminaire. Après
avoir parlé des indécences des autres à son égard, je viens aux siennes,
et c'est où la plume me tourne dans les doigts, frappé des énormes abus
qui se sont faits en Flandre, et de là partout de ces sortes de fautes
dont la continuité y ajoute un fâcheux poids. Je m'y arrête néanmoins
tout aussi peu qu'il est possible, et je viens à l'objet principal de
mon Discours qui est la connaissance des hommes\,; je m'y étends avec
une liberté égale à la nécessité, et j'entre clans un détail de moyens
par le besoin d'y conduire comme par la main le prince, et de lui ôter
occasion et prétexte de ne savoir comment s'y prendre. En même temps je
sens très-bien que ce que je propose avec tant de force et d'étendue est
entièrement contraire à l'usage du roi, auprès duquel les anciens
ministres, et les nouveaux après eux, n'ont rien craint davantage ni
détruit avec plus de soin, d'application et d'industrie\,; ainsi je
pallie cela comme je puis, en me jetant dans l'apothéose à travers
laquelle on peut sentir que je ne suis pas convaincu par cet exemple.
Jusque-là ce Discours est à la portée de tous les gens du monde.

La manière de penser de Mgr le duc de Bourgogne si austère, si
littérale, et la dévotion du duc de Beauvilliers et quoique tout
autrement formée et raisonnable, m'ont forcé de me jeter ici dans une
discussion du goût de peu de gens, mais sans laquelle ce qui précède
n'aurait pu entrer dans la tête du prince ni si aisément dans l'esprit
de son ancien gouverneur. J'avais besoin de quelque discussion sur la
médisance pour apprivoiser le prince au raisonnement avec les hommes, et
sur la dévotion, pour le préparer par des comparaisons monacales à
m'écouter sur sa conduite en Flandre pendant sa dernière campagne et à
son retour encore, et pour en sentir tous les profonds inconvénients.
Cette préparation m'était absolument nécessaire pour oser toucher ceux
de l'opinion qu'il a donné lieu de prendre\,; qu'il n'estime et ne
mesure rien que par la dévotion, et que tout devient pour lui cas de
conscience. On se persuada tellement en effet qu'il avait fait consulter
la guerre d'Espagne, pour, sur l'avis des docteurs, former le sien au
conseil, que le roi lui demanda ce qu'il en était, et qu'il ne fut pas
peu surpris de la réponse nette et précise du prince\,: qu'il n'y avait
pas seulement pensé. C'est ce qui m'a obligé à traiter en deux mots la
messéance de ses longs et fréquents entretiens avec son confesseur, et
comme j'avais loué le précepteur pour mieux faire recevoir dès l'entrée
tout ce que j'avais à dire, louer aussi ce confesseur pour ne pas
choquer le pénitent, et lui mieux faire entrer dans la tête la
considération des réflexions et de la comparaison des règnes des
derniers rois d'Espagne, et je reviens par tout cela aux grands
inconvénients de n'être pas connu des hommes. Les louanges terminent le
Discours comme elles l'ont commencé. C'est un adoucissement
indispensable devant et après tout ce qu'il y avait à dire. Mais la
grâce, qui avait commencé par des miracles rapides, acheva bientôt son
ouvrage, et en fit un prince accompli. Les petitesses, les scrupules,
les défauts disparurent et ne laissèrent plus que la perfection en tout
genre. Mais, hélas\,! la perfection n'est pas pour ce monde, qui n'en
est pas digne. Dieu la montra pour montrer sa bonté et sa puissance, et
se hâta de la retirer pour récompenser ses dons et pour châtier nos
crimes.

Ce Discours, des vérités duquel j'étais plein, fut bientôt jeté sur le
papier. Je n'y corrigeai rien du premier trait de plume, et je le lus au
duc de Beauvilliers tel qu'il se voit ici. J'ose dire qu'il lui plut
extrêmement. De tout son tissu il ne me contesta que deux choses\,:
l'assiduité rigoureuse aux offices de l'Église les fêtes et les
dimanches, qu'à la fin il me céda, et les spectacles, que je ne pus
jamais lui faire passer. Il loua toute la discussion sur la médisance et
sur la dévotion\,; fut entièrement de mon avis sur la communication avec
les hommes, telle que je la proposais\,; il approuva tout ce que je dis
sur M. de Vendôme, que j'avais évité de nommer, et sur la conduite de
Mgr le duc de Bourgogne, en sa dernière campagne de Flandre et à son
retour. En un mot, tout le Discours se trouva de son goût. Il en voulut
une seconde lecture\,; à mon tour, je le priai de peser l'endroit des
mouches, des crapauds, et de ces sortes de badinages que je trouvais
moi-même trop frappé\,; il en convint, mais ces choses lui parurent si
importantes à vivement représenter, qu'il ne put consentir à le
supprimer ni même à l'adoucir. Je lui fis faire attention sur l'article
du confesseur, mais il s'écria d'approbation.

Après cet examen il fut question de l'usage, et ce fut là où s'émut la
plus longue et la plus vive dispute que j'aie guère eue avec lui. Il
voulait montrer ce Discours au prince et le lui montrer sous mon nom, en
lui racontant naturellement comment il me l'avait demandé. Je me récriai
sur le danger\,; et après un long combat il ne put obtenir de moi que
j'y consentisse, ni moi de lui qu'il en quittât le dessein, tellement
qu'il me proposa de nous en rapporter au duc de Chevreuse. Il était à
Paris, où un grand procès de la duchesse de Luynes contre Matignon le
retenait et nous à Marly le même voyage dont j'ai déjà parlé. J'acceptai
ce tiers parti, plutôt dans le dessein de gagner temps et de me
consulter, que dans celui d'acquiescer au désir de M. de Beauvilliers,
quand même l'avis de M. de Chevreuse y eût été conforme.
M\textsuperscript{me} de Saint-Simon avait été fort fâchée de
l'engagement où je m'étais laissé aller à Vaucresson, dans la crainte
que je ne fusse plus maître de mon Discours après que je l'aurais fait.
Elle la fut bien davantage quand elle sut la passion du duc de
Beauvilliers à le montrer, et elle y résista de toutes ses forces\,;
j'étais combattu entre sa peine et son grand sens si souvent éprouvé, et
mon extrême déférence pour M. de Beauvilliers, en tout véritablement
aiguisée en cette occasion d'un peu de sot amour-propre. Nous convînmes,
elle et moi, d'en passer par l'avis d'un homme fort de nos amis et tout
propre à consulter là-dessus, par sa probité, son esprit, sa
connaissance du monde, et surtout de Mgr le duc de Bourgogne\,; ce fut
Cheverny, que le roi avait attaché à lui, et dont j'ai quelquefois
parlé. Le Discours fut donc lu entre nous trois. Je fus payé de louanges
et M\textsuperscript{me} de Saint-Simon d'approbation. Il trouva comme
elle qu'il était très-dangereux à montrer à celui pour qui seul il était
fait, et même de le lui faire voir par parties, et sans me nommer, parce
que j'y étais trop reconnaissable par le style, parce qu'il était
impossible que le duc de Beauvilliers l'eût demandé à un autre que moi,
par le zèle pour le prince, par sa connaissance intime, par cette
impatience des choses de Flandre et des calomnies, par la connaissance
si particulière de la cour qui y était répandue. Ainsi nous convînmes
que, quoi que pussent dire et vouloir les deux ducs, je ne permettrais
point que ce Discours fût livré à Mgr le duc de Bourgogne, qui, tout
saint qu'il était, souffrirait peut-être impatiemment, sinon à présent
au moins dans la suite, d'être si transparent à mes yeux, et plus encore
désapprouvé dans des choses qu'il ne changerait pas, et dont le
changement était difficilement espérable.

Cette sage résolution prise, je subis l'examen du duc de Chevreuse à qui
j'avais envoyé une copie, afin qu'il eût tout le temps d'y penser. Il
approuva extrêmement l'ouvrage, mais il fut heureusement d'avis de ne le
point donner, par quoi je sortis d'embarras\,; mais il me condamna à
leur laisser ma copie avec sûreté entière qu'elle ne sortirait point de
leurs mains, et à consentir que, sans faire mention de moi ni du
discours même, ils pussent de fois à autre et de loin en loin en lâcher
des morceaux détachés au prince, ce qui pouvait se faire sans danger. M.
de Beauvilliers s'y soumit et moi pareillement, après que Cheverny et
M\textsuperscript{me} de Saint-Simon eurent jugé aussi que, de cette
façon, il n'y avait point d'inconvénient. Les deux ducs ignorèrent
toujours que M\textsuperscript{me} de Saint-Simon et moi eussions mis
Cheverny dans cette confidence\,: tel est le malheur des meilleurs
princes et les plus attentifs à leur salut, à leur mortification, à leur
anéantissement, d'être plus capables de porter les opprobres jusqu'à la
dernière indécence et au danger, que les avertissements les plus
salutaires et les plus mesurés de leurs plus affidés serviteurs.

Maintenant il est temps d'expliquer une puissante intrigue qui partagea
toute la cour. Il faut retourner beaucoup en arrière, parce qu'elle fut
commencée longtemps avant tout ceci, et la suivre jusqu'à sa fin pour ne
la pas interrompre par des mélanges de ce qui se passa cependant aux
armées, dont les divers succès ne veulent pas être suspendus.

J'ai touché légèrement, à l'occasion de la rupture de M. le duc
d'Orléans avec M\textsuperscript{me} d'Argenton, et du règlement du rang
des princesses du sang entre elles, quelque chose du désir de M. le duc
et de M\textsuperscript{me} la duchesse d'Orléans de marier Mademoiselle
à M. le duc de Berry, du peu qu'il s'était passé là-dessus, de la même
passion de M\textsuperscript{me} la Duchesse pour M\textsuperscript{lle}
de Bourbon, et plus en détail de la haine de M\textsuperscript{me} la
Duchesse pour M. et M\textsuperscript{me} la duchesse d'Orléans, de la
liaison de celle-ci avec M\textsuperscript{me} la duchesse de Bourgogne
et de l'extrême et réciproque éloignement de cette princesse et de
M\textsuperscript{me} la Duchesse. Ces deux derniers points sont traités
avec étendue à l'occasion des cabales de la campagne de la perte de
Lille, et c'est de toutes ces choses qu'il est nécessaire de se souvenir
pour bien entendre ce qui va être raconté.

Les obstacles qui s'opposaient à ce mariage de Mademoiselle étaient
également nombreux et considérables. En général, un temps de guerre la
plus vive et la plus infortunée, la misère extrême du royaume qui ôtait
les moyens de fournir aux choses les plus pressantes, la dépense du
mariage, l'apanage à fournir, une double maison à entretenir, l'âge et
le naturel de M. le duc de Berry doux et craignant le roi à l'excès, qui
n'avait que vingt-quatre ans, et qui parmi plusieurs commencements de
galanteries n'avait encore su ni les embarquer, ni les conduire, ni en
mettre aucune à fin, ce qui devait guérir les scrupules\,; l'âge et
l'union de Mgr et de M\textsuperscript{me} la duchesse de Bourgogne qui
leur avait donné des enfants, et qui leur en promettait pour longtemps
encore\,; enfin la perspective si naturelle d'un mariage étranger, sans
comparaison plus décent, et qui pouvait servir de prétexte à rapprocher
l'empereur, ou à détacher le Portugal qui était dans la guerre présente
une si dangereuse épine à l'Espagne. En particulier, l'état personnel de
M. le duc d'Orléans pour qui le roi n'était point revenu à fond, à qui
M\textsuperscript{me} de Maintenon ne pardonnerait jamais ce cruel bon
mot d'Espagne, la considération du roi d'Espagne, toujours persuadé que,
de concert avec les alliés, il avait voulu usurper sa couronne\,; l'idée
du public et de la cour en France qui n'était point déprise de cette
même opinion, et qui déjà froncée de voir tous ses princes légitimes si
mêlés avec les bâtards, le seraient bien autrement d'un mélange qui
remonterait si près du trône\,; enfin il s'agissait du fils de
Monseigneur et de son fils favori\,: de Monseigneur qui marquait sans
cesse jusqu'à l'indécence sa haine pour M. le duc d'Orléans depuis
l'affaire d'Espagne, qui était gouverné par les ennemis personnels de ce
prince, et par des ennemis, qui ayant la même prétention, au mariage de
M. le duc de Berry, se porteraient à tout pour rompre celui de
Mademoiselle par Monseigneur, malgré lequel il faudrait l'emporter.
L'union récente, et qui s'entretenait, que les menées qui avaient perdu
Chamillart avaient mises entre M\textsuperscript{me} de Maintenon,
M\textsuperscript{lle} Choin et Monseigneur, et le crédit nouveau qui
avait paru en ce prince sur le roi son père dans l'éclat de cette
disgrâce, tout cela se réunissait contre Mademoiselle, et ne paraissait
pas possible à être surmonté\,; de raison d'État aucune, et de famille
moins encore s'il se pouvait avec cette opposition de Monseigneur et
cette offense du roi d'Espagne, nulle considération qui pressât un
mariage, et si la paix n'en fournissait point d'étranger, ce qui était
impossible à croire, le domestique toujours aisé à retrouver dans une
des trois branches du sang légitime. Enfin, après ce dont M. le duc
d'Orléans avait été accusé en Espagne, avec ses talents et son esprit,
{[}il semblait{]} dangereux à le faire beau-père de M. le duc de Berry
pour un temps ou pour un autre.

Tant et de tels obstacles généraux et particuliers, à pas un desquels M.
et M\textsuperscript{me} la duchesse d'Orléans n'avaient quoi que ce fût
à répondre, les tenaient dans une inaction glacée et dans un état de
désir sans espérance, qui était le premier de tous les obstacles à
vaincre et qui m'étaient tous bien présents et bien distincts dans
l'esprit. Je continuerai ici à parler de moi dans la même vérité que je
fais des autres. Un intérêt sensible me faisait souhaiter le mariage de
Mademoiselle avec passion\,; je voyais que tout tendait au mariage de
M\textsuperscript{lle} de Bourbon. Outre qu'elle était fille de feu M.
le Duc, je ne pouvais pardonner à M\textsuperscript{me} la Duchesse ses
procédés à mon égard sur l'affaire de M\textsuperscript{me} de Lussan\,;
et quelques ménagements que j'eusse saisis pour elle à l'occasion de la
mort de M. le Duc, il était difficile qu'elle me pardonnât les procédés
dont j'avais osé payer les siens, et ma liaison intime avec ce qu'elle
et sa cabale haïssait le plus, cabale qui avait pris pour moi la plus
grande aversion depuis les choses de Flandre, et d'Antin seul, que la
politique en avait écarté sur ce périlleux article, aussi attentif à me
nuire et pour les choses passées et pour mes liaisons toutes opposées à
lui. Je redoutais déjà assez la situation présente de
M\textsuperscript{me} la Duchesse avec Monseigneur, combien plus après
le mariage de leurs enfants, qui la porterait à une grandeur et à une
autorité auprès de lui sans bornes pour le présent, et pour le futur,
arriverait par un autre biais à ce que la cabale avait tâché par les
attentats de Flandre, et du même coup écraserait M. {[}le duc{]} et
M\textsuperscript{me} la duchesse d'Orléans et moi, tant d'avec eux que
d'avec Mgr le duc de Bourgogne, que de mon chef personnellement.

En même temps je considérais que si Mademoiselle était préférée, le
crédit et la faveur de M\textsuperscript{me} la Duchesse se pouvaient
balancer auprès de Monseigneur\,; et qu'en prenant dès ce règne de
bonnes et sages mesures pour l'avenir, il n'était pas impossible de
faire avorter ses grandes espérances de gouverner, et par l'union des
enfants de Monseigneur embarrasser cette redoutable cabale qui s'était
déjà montrée avec une audace si criminelle, et la réduire même sous les
fils de la maison. Je me trouvais ainsi dans la fourche fatale de voir
dès maintenant, et plus encore dans le règne futur, ce qui n'était le
plus contraire, ou ceux à qui j'étais le plus attaché, sur le pinacle ou
dans l'abîme, avec les suites personnelles de deux états si différents,
sans compter le désespoir ou le triomphe, et la part que je pouvais
avoir à parer l'un et à procurer l'autre. Il n'en fallait pas tant pour
exciter puissamment un homme fort sensible et qui savait si bien aimer
et haïr, que je ne l'ai que trop su toute ma vie. Une seule chose me
retenait, le désir extrême d'un mariage étranger qui, convenable à M. le
duc de Berry et à l'État, sauvait ce rejeton si prochain de la couronne
de cette souillure de bâtardise qui me faisait horreur, et qui ne
pouvait qu'appuyer les bâtards dont le rang m'était si odieux.

Dans cette balance de mon esprit, je mis toute mon application à bien
examiner les choses, et je vis nettement les menées de
M\textsuperscript{me} la Duchesse, qui saisissait toutes les avenues, et
qui n'oubliait rien pour assurer, hâter, brusquer même le mariage de
M\textsuperscript{lle} de Bourbon. Elle-même avait fait écarter l'idée
d'une étrangère dans l'esprit du roi, qui s'était laissé aller à en
marquer du dégoût, {[}parce{]} que la paix était trop éloignée pour
différer jusque-là à marier un prince sain et vigoureux, dont le goût
pour les femmes lui donnait du scrupule de ce qui en pourrait arriver,
et qui enfin, ennemi de toute pensée de la plus légère et la plus courte
contrainte, trouvait plus commode de choisir dans sa famille qu'au
dehors. Je compris donc que, tandis que déçu par le désir et l'espérance
d'un mariage étranger, je laisserais couler le temps, celui de
M\textsuperscript{lle} de Bourbon s'avancerait sourdement et nous
tomberait, et à moi en particulier, un matin sur la tête, qui comme une
meule m'écraserait et froisserait les princes à qui j'étais attaché, de
manière à ne s'en relever jamais. Je vis clairement que je ne pouvais
éviter la bâtardise, dès là qu'on était réduit à la volontaire nécessité
d'un mariage domestique, et ce fut ce qui me détermina à agir.

Cette résolution bien mûrement prise, je repassai dans mon esprit tous
les obstacles généraux et particuliers pour m'accoutumer à n'en être
point effrayé et pour chercher les moyens de les vaincre. J'en examinai
les divers genres\,; je les balançai, je les pesai à part et ensemble\,;
je les pénétrai tous pour me former un plan de conduite pour attaquer à
découvert ou en biaisant par à côté, selon leurs diverses natures, les
uns indispensables à renverser, les autres trop forts passer à côté et
n'en effleurer que le purement nécessaire, persuadé qu'il fallait que je
commençasse par l'être moi-même de la possibilité du succès avant d'en
pouvoir persuader les autres, et ceux-là mêmes qui y avaient tout
intérêt. Je conçus aussi que toutes mes combinaisons devaient être dans
ma tête et bien débrouillées, et que nous fussions tous persuadés et
d'accord avant de remuer aucune machine. Une triste expérience, mais
continuelle, sur la plupart des événements principaux, m'avait depuis
longtemps convaincu que le solide, l'essentiel, le grand avait changé de
place avec la bagatelle, le futile, la commodité momentanée\,; que les
plus importants effets étaient depuis longtemps toujours sortis de cette
dernière source, et je compris que je pouvais en tirer un grand parti
dans cette occasion.

La plus grande raison contre Mademoiselle était celle d'un mariage
étranger pour lequel tout parlait. Ce n'était point cela qu'il y avait à
combattre par les raisons qui viennent d'en être rapportées. Le roi n'en
voulait point, et il n'y avait rien à craindre des réflexions qui lui
pouvaient être présentées là-dessus par ceux que leur naissance ou leurs
places dans le conseil mettaient en droit de le faire. Le silence
profond que le roi gardait toujours avec eux tous sur ces choses
intérieures de sa famille, dont lui seul disposait sans s'ouvrir à
personne, rassurait pleinement là-dessus\,; à l'égard des autres
obstacles, je conçus qu'il n'y avait de moyen que d'opposer cabale à
cabale et puis de lutter d'adresse et de force. Le fondement de tout
était M. {[}le duc{]} et M\textsuperscript{me} la duchesse d'Orléans,
qui s'épuisaient inutilement en désirs et qui les noyaient dans une
oisiveté profonde. Je leur mis vivement devant les yeux l'état des
choses du côté de M\textsuperscript{me} la Duchesse, je leur fis sentir
sans ménagement quelle serait leur situation, même de ce règne, si elle
réussissait, et combien pire après, je les piquai d'orgueil, de
jalousie, de dépit\,; croirait-on que j'eusse besoin de tout cela avec
eux\,? et, à force de les exciter par les plus puissants motifs, je les
rendis enfin capables d'entendre à leur plus pressant intérêt. La
paresse naturelle mais extrême de M\textsuperscript{me} la duchesse
d'Orléans céda pour cette fois, moins peut-être à ce grand intérêt qu'à
la puissante émulation d'une soeur si ennemie, et, ce premier pas fait,
elle et moi nous concertâmes pour nous aider de M. le duc d'Orléans.

Ce prince, avec tout son esprit et sa passion pour Mademoiselle, qui
n'avait point faibli du premier moment qu'elle était née, était comme
une poutre immobile qui ne se remuait que par nos efforts redoublés, et
qui fut tel d'un bout à l'autre de toute cette grande affaire. J'ai
souvent réfléchi en moi-même sur cette incroyable conduite de M. le duc
d'Orléans, dont je ne pouvais allier l'incurie avec le désir, le besoin
et tant et de si puissantes raisons qui le poussaient à mettre vivement
la main à l'oeuvre, sans qu'après lui avoir souvent, longuement et
fortement représenté, M\textsuperscript{me} la duchesse d'Orléans en
tiers, toutes les puissantes considérations qui le devaient exciter, il
se prêtât ensuite à la moindre démarche, et déconcertait ainsi tous nos
projets. Certainement, quelque peu de suite qu'il eût dans l'esprit,
quelque mollesse qui lui fût naturelle, quelque peu capable qu'il fût
d'agir effectivement sur un plan, quelque légère et faible que fût sa
volonté sur toutes choses, il n'est pas possible de croire que ces
défauts causassent en lui une conduite si surprenante, si étrange en
elle-même et pour nous si radicalement embarrassante\,; et j'ai toujours
soupçonné qu'en sachant plus que personne sur son affaire d'Espagne,
cette bride non-seulement l'arrêtait, mais le persuadait si pleinement
qu'elle était obstacle insurmontable au mariage dont il s'agissait,
qu'il ne faisait que se prêter avec nonchalance et par reprises légères
à ce dont nous le pressions souvent, certain qu'il se croyait de
l'entière inutilité de toute démarche et de tout soin, sans toutefois
nous en vouloir avouer la cause véritable, et que pour nous mieux cacher
il agissait faiblement, pressé à un certain point, plutôt que de nous
déclarer une fois pour toutes sa vraie raison de désespérer et de nous
arrêter tout à fait pour s'en épargner les regrets plus à découvert.
C'est ce qui me fut d'un travail dur et extrême, parce qu'il ne fallut
jamais cesser de forcer de bras auprès de lui, ni de se rebuter des
contre-temps continuels de sa part, qui pensèrent plusieurs fois faire
tout échouer.

Moins je vis de ressource à espérer de celui qui y avait le plus grand
intérêt, plus je m'appliquai à en trouver d'ailleurs et à former et
diriger une puissante cabale, et de plusieurs différentes à en faire une
seule qui se proposât puissamment le but où je tendais, puissamment,
dis-je, pour son intérêt propre, premier mobile ou plutôt unique de tous
les mouvements des cours. M\textsuperscript{me} la duchesse de
Bourgogne, unie avec M\textsuperscript{me} la duchesse d'Orléans,
infiniment mal avec M\textsuperscript{me} la Duchesse, avait plus d'un
intérêt à la préférence de Mademoiselle sur M\textsuperscript{lle} de
Bourbon\,; le premier sautait aux yeux de qui savait la situation de
M\textsuperscript{me} la duchesse de Bourgogne avec
M\textsuperscript{me} la Duchesse, et celle de M\textsuperscript{me} la
Duchesse auprès de Monseigneur, des volontés duquel elle disposait
absolument, et qui, reliée à lui par le mariage de leurs enfants,
usurperait une puissance sous laquelle tout plierait sous son règne et
dès celui-ci même\,; M\textsuperscript{me} la duchesse de Bourgogne
tomberait peu à peu dans un éloignement de Monseigneur qui, approfondi
par la dévotion mal entendue de Mgr le duc de Bourgogne et par le dégoût
que Monseigneur avait pris de lui depuis les choses de Flandre,
soigneusement entretenu depuis, les plonge-roit tous deux dans l'abîme
que la cabale dont il a été parlé avait si hardiment commencé à leur
creuser. À ce grand intérêt il s'en joignait un autre aussi fort
sensible et qui avait sa solidité.

M\textsuperscript{me} la duchesse de Bourgogne connaissoit le roi
parfaitement, elle ne pouvait ignorer la puissance de la nouveauté sur
son esprit, dont elle-même avait fait une expérience si heureuse. Elle
avait donc à redouter une autre elle-même, je veux dire une princesse au
même degré du roi qu'elle, qui, plus jeune qu'elle, le pourrait amuser
par des badinages nouveaux et enfantins qui lui avaient si bien réussi,
mais qui n'étaient plus guère de son âge, quoiqu'elle s'en aidât encore,
et qui lui siéraient d'autant moins alors qu'ils seraient plus de saison
pour une autre\,; que cette autre, égale à elle en rang, en
particuliers, en privances, aurait lieu d'en user autant qu'elle,
peut-être plus que si le roi y prenait\,; que conduite par sa mère,
M\textsuperscript{me} la Duchesse, elle serait au fait de tout, ne
donnerait prise sur rien par aucuns contretemps, n'aurait point comme
elle un époux à soutenir, et que soutenue elle-même par Monseigneur et
par cette terrible cabale qui voulait perdre Mgr le duc de Bourgogne, et
qui ne le pouvait sans la perdre elle-même, irritée sur l'un par le
désir de gouverner, sur l'autre par la même cause et par la passion qui
s'y était jointe contre elle, depuis qu'elle avait pour le présent fait
avorter ses desseins et perdu leur instrument principal, sa belle-soeur
deviendrait un espion dangereux dans le plus intérieur de son sein, par
qui les choses les plus innocentes seraient tournées en poison\,: une
rivale cuisante et dominante, à qui tout droit par la considération de
l'avenir, une égale avec laquelle il faudrait se mesurer et compter en
toutes choses\,; épouse enfin du fils favori dont la vie libre plaisait
par conformité à père et à grand-père, tous deux en gêne avec Mgr le duc
de Bourgogne, ses scrupules, ses précisions, sa vie à part et cachée
dans le littéral de sa dévotion.

Ces deux grands intérêts qui portaient également sur l'agréable et sur
le considérable, sur le présent et sur l'avenir, et tout ensemble sur
tout ce qu'il peut y avoir de plus important dans la vie, et dont
M\textsuperscript{me} la duchesse de Bourgogne était plus capable d'être
touchée qu'aucune autre personne de son âge et de son rang, avaient
néanmoins besoin de lui être fortement inculqués pour n'être pas
suffoqués par le futile et l'amusement du courant des journées. Elle
sentait bien d'elle-même ces choses en général, et qu'il lui était
essentiel de n'avoir pour belle-soeur qu'une princesse qui ne pût et ne
voulût lui faire ombrage, et de qui elle fût maîtresse assurée. Mais
quelque esprit, quelque sens qu'elle eût, elle n'était pas capable de
sentir assez vivement d'elle-même toute l'importance de ces choses, à
travers les bouillons de sa jeunesse, l'enchaînement et le cercle des
devoirs successifs, l'offusquement de sa faveur intime et paisible, la
grandeur d'un rang qu'attendait une couronne, la continuité des
amusements qui dissipaient l'esprit et les journées\,; douce, légère,
facile d'ailleurs, peut-être à l'excès. Je sentis que c'était de l'effet
de ces considérations sur elle que je tirerais le plus de force et de
secours, par l'usage qu'elle en saurait bien faire avec le roi, et plus
encore avec M\textsuperscript{me} de Maintenon, qui tous deux l'aimaient
uniquement\,; et je sentis aussi que M\textsuperscript{me} la duchesse
d'Orléans n'aurait ni la grâce ni la force nécessaire pour le lui bien
enfoncer, à cause de son trop grand intérêt.

Je me tournai donc vers d'autres instruments plus propres, et qui
eussent aussi leurs intérêts personnels en la préférence de
Mademoiselle. La duchesse de Villeroy m'y parut infiniment propre par
tout ce que j'en ai raconté, et par une fermeté souvent peu éloignée de
la rudesse qui, jointe au bon sens, tient quelquefois lieu d'esprit, et
frappe plus fortement et plus utilement des coups que plus d'esprit avec
plus de mesure. Elle était depuis longtemps instruite des désirs de
M\textsuperscript{me} la duchesse d'Orléans\,; je lui fis sentir que ses
désirs étaient trop languissants, combien il était pressé d'agir avec
force, et je suppléai à tout avec grand fruit de ce côté-là.
M\textsuperscript{me} de Lévi me parut un autre instrument triplement
considérable. Elle joignait infiniment d'esprit à une fermeté, qui un
peu gouvernée par l'humeur était égale, et quelquefois supérieure, à
celle de la duchesse de Villeroy. Presque aussi mal qu'elle avec
M\textsuperscript{me} la Duchesse, et dès longtemps bien et ménagée par
M\textsuperscript{me} la duchesse d'Orléans, son intérêt la portait à
Mademoiselle. D'ailleurs sensible au dernier point à l'amitié, et
très-bien alors avec M\textsuperscript{me} la duchesse de Bourgogne,
l'intérêt de cette princesse, qui la frappa en entier, la porta
rapidement à tout ce que je désirais d'elle. Deux autres raisons me la
rendirent encore utile. Nonobstant son âge, elle était dès lors à portée
de tout avec M\textsuperscript{me} de Maintenon\,; et le hasard ou, pour
mieux dire, la Providence voulut qu'ayant été personnellement très-mal
avec M\textsuperscript{me} la duchesse de Bourgogne, et à cause de sa
famille fort éloignée de M\textsuperscript{me} de Maintenon, toutes les
deux l'avaient rapprochée, puis goûtée, au point qu'elle était arrivée
jusqu'à l'intimité de la princesse, et à toute celle qui se pouvait
espérer de M\textsuperscript{me} de Maintenon. L'autre raison, c'est
qu'elle était tendrement aimée, considérée, estimée et comptée dans sa
famille, qui pouvait beaucoup influer sur le mariage, et admise dans ses
conseils. Elle me fut un excellent second auprès des ducs et des
duchesses de Chevreuse et de Beauvilliers, en sorte qu'elle et moi
concertâmes souvent les choses qu'il ne fallait pas leur présenter trop
crues, ni toujours par la même main.

De ces deux femmes résulta un troisième instrument, faible à la vérité
par un désir constant de tout ménager à la fois, et une politique vaste,
mais qui, mis en oeuvre selon son talent, nous servit. Ce fut
M\textsuperscript{me} d'O, que de puissantes raisons parmi les dames
tenaient dans l'intime confidence de M\textsuperscript{me} la duchesse
de Bourgogne. D'O y servit aussi en sa froide et profonde matière. Il
était attaché aux ducs de Chevreuse et de Beauvilliers. Il leur était
redevable en beaucoup de choses, sur toutes d'avoir évité d'être perdu
au retour de la campagne de Lille. Le comte de Toulouse était
intérieurement plus porté pour M\textsuperscript{me} la duchesse
d'Orléans que pour M\textsuperscript{me} la Duchesse\,; et M. du Maine
bien plus encore, qui, depuis la mort de M. le Prince, ne regardait plus
cette soeur que comme une ennemie.

Cette raison fut un grand instrument dans la main de
M\textsuperscript{me} la duchesse d'Orléans et de M. d'O pour exciter la
peur de M. du Maine, qui de toutes les passions était celle qui de tous
les genres avait le plus d'empire sur lui. Ils lui montrèrent les enfers
ouverts sous ses pieds par le mariage de M\textsuperscript{lle} de
Bourbon, toutes ses prétentions à la succession de M. le Prince sans
ressources, son rang à l'avenir fort en l'air, ses survivances
très-hasardées, et le rang de ses enfants perdu, toutes choses à quoi la
haine de M\textsuperscript{me} la Duchesse n'aurait pas grand'peine à
réussir dès à présent pour le procès, avec la part que M. le duc de
Berry et Monseigneur même ne se cacheraient plus d'y prendre, et dans
l'avenir pour le reste, avec la répugnance que Monseigneur y avait
montrée, et qui n'avait pu être fléchie par les prières du roi les plus
touchantes, et pour lui les plus nouvelles, de sorte que ne s'agissant
que d'agir auprès du roi dans les ténèbres des tête-à-tête dont il avait
plusieurs occasions tous les jours, et de même avec
M\textsuperscript{me} de Maintenon sur qui il pouvait tout, et qu'il
voyait seule tant qu'il voulait, son propre salut le mit d'autant plus
puissamment en oeuvre qu'il conçut dès lors le dessein de s'en faire
payer comptant par le mariage qu'il ne tarda pas à proposer, et à
presser de régler, de signer et de déclarer, d'une soeur de Mademoiselle
avec son fils qui deviendrait ainsi beau-frère de M. le duc de Berry,
qui fut une chose qui me coûta bien du manége à éviter. Telle fut la
cabale des femmes, si principales dans les cours, si continuellement
dans la nôtre. Je crus que c'en était assez pour bien remplir mes vues
de ce côté-là, et que le secret, si fort l'âme et le salut de cette
affaire, ne souffrait pas qu'on y en mît davantage. Je n'eus sur cela
aucun commerce avec les d'O ni avec M. du Maine\,; mais je lui faisais
dire tout ce que je voulais par M\textsuperscript{me} la duchesse
d'Orléans, et savais par elle toutes ses démarches, mais sans jamais
proférer un mot d'un rang auquel je ne voulais pas monter aucune
inclination pour me réserver entier et libre pour des temps plus
heureux, et je me contentai du procès de la succession de M. le Prince,
et de la haine qu'il avait fait éclater, dont toutes les justes
conséquences sautaient aux yeux sans que j'eusse à en particulariser
aucune. Pour les d'O, jamais je ne leur fus nommé, mais je les dirigeais
par la duchesse de Villeroy en gros, qui me rendait exactement tout le
détail qui se passait d'elle à eux et d'eux à elle\,; et elle et moi
avec la même délicatesse et le même silence sur des rangs qui ne lui
étaient pas moins odieux qu'à moi. Le rare est qu'il me fallut presque
tout imaginer, mâcher et conduire avec M\textsuperscript{me} la duchesse
d'Orléans même, et souvent encore l'arracher à sa paresse avec effort.

Quelque content que je fusse de ces ressorts, j'estimai qu'il en fallait
encore ajouter d'autres, et saisir tous les côtés possibles. Bien que
toute la tendresse de M\textsuperscript{me} de Maintenon fût pour M. du
Maine et M\textsuperscript{me} la duchesse de Bourgogne, et qu'elle
n'aimât point M\textsuperscript{me} la Duchesse qui avait secoué son
joug dès qu'elle l'avait pu, l'avait toujours depuis négligée de peur de
s'y rempêtrer, et à qui même il était échappé des moqueries d'elle, je
redoutais sur ce mariage les mesures qui, depuis la grande affaire de la
disgrâce de Chamillart, subsistaient entre elle et Monseigneur, ses
liaisons prises en même temps avec M\textsuperscript{lle} Choin, ses
réserves quelquefois timides avec le roi. Je craignais encore
M\textsuperscript{me} de Caylus, sa nièce, son goût et son coeur, qui la
connaissoit parfaitement, qui avait tout l'esprit et tout le manège
possible, que les plaisirs, la galanterie, et des vues ensuite plus
solides avaient attachée de tout temps à M\textsuperscript{me} la
Duchesse, bien par elle avec Monseigneur et avec tout ce qui le
gouvernait, mais bien solidement et en dessous, et qui de tout cela
comptait se faire une ressource après sa tante, et plus encore après le
roi.

Ainsi je compris qu'il ne fallait rien omettre, parce que M. le duc de
Berry était une place que nous n'emporterions que par mine et par
assaut\,; et je parlai puissamment aux ducs de Chevreuse et de
Beauvilliers, et aux duchesses leurs femmes, qui avaient grand crédit
sur eux, surtout M\textsuperscript{me} de Beauvilliers. À ceux-là je
représentai le schisme radical de la cour, l'abîme certain de Mgr le duc
de Bourgogne si M\textsuperscript{lle} de Bourbon prévalait,
conséquemment le danger futur de l'État, la haine inévitable entre les
deux frères, jusqu'à présent si unis par leurs soins, et qui serait
l'ouvrage de leurs épouses et de leur situation forcée, le danger
extrême d'attendre un mariage étranger dont le roi était tout à fait
aliéné, avec les menées si avancées de M\textsuperscript{me} la
Duchesse, le scrupule enfin, pour les hâter, de laisser davantage sans
épouse un prince de l'âge et de la santé de M. le duc de Berry. Le fort
de mon raisonnement porta sur ces considérations. J'y mêlai celle de
l'extinction totale de tout ce qu'il pouvait rester de l'affaire
d'Espagne, et dans l'esprit même de M. le duc d'Orléans, de toute idée
nouvelle que pourrait exciter dans d'autres temps la grandeur de
M\textsuperscript{me} la Duchesse et sa propre oppression. Je montrai en
éloignement, sur le compte de ce prince, ce que pourrait opérer le
retour de M\textsuperscript{me} des Ursins, si le malheur du roi
d'Espagne la rappelait en France, dont il était déjà sourdement
question, et je m'adressais à des gens qui ne désiraient pas champ libre
à cette femme dans notre cour. Dans ce même esprit, je leur parlai de
M\textsuperscript{me} la duchesse et de d'Antin, ouvertement leurs
ennemis, et je sentis que je ne parlais pas à des sourds. Bref, je
m'assurai d'eux, j'en obtins l'aveu de leurs craintes et de leurs
désirs, enfin je les mis en mouvement, moi en possession d'eux
là-dessus, eux en toutes mesures avec moi, et en compte presque
journalier de leurs démarches. Ce n'était pas peu faire avec des gens de
système si fort mesuré, à marches si profondes, si compassées, si
difficiles, moines, profès d'indifférence et d'impuissance, mais qui se
souvenaient parfois qu'ils n'en avaient pas fait les voeux.

Ce côté-là saisi, je mis la main sur un autre qui n'était pas moins
important\,: ce fut les jésuites. L'affaire de mon ambassade de Rome, où
d'Antin avait vainement été mon concurrent, m'avait appris combien ils
le haïssaient\,; et tout ce qu'ils avaient employé pour l'exclure,
jusqu'à son su, me répondait qu'ils l'en craignaient bien davantage.
J'étais bien informé qu'ils n'avaient ni moins d'éloignement ni moins
d'appréhension de M\textsuperscript{me} la Duchesse. Je ne pouvais
ignorer qu'ils affectionnaient assez M. le duc d'Orléans, ce que j'avais
pris soin de cultiver. Je crus donc facile de profiter de si heureuses
dispositions. J'obtins de M. {[}le duc{]} et de M\textsuperscript{me} la
duchesse d'Orléans qu'ils fissent confidence de leurs désirs au P. du
Trévoux. Ce jésuite avait été confesseur de Monsieur jusqu'à sa mort. M.
le duc d'Orléans, dont la vie ne cadrait pas avec la fonction d'un
pareil officier, n'avait pas laissé de lui en conserver le titre et
l'utile, pour faire avec lui la nomination des abbayes et des autres
bénéfices de son apanage, dont le roi avait donné le droit à la mort de
Monsieur.

Ce P. du Trévoux, gentilhomme de Bretagne de bon lieu, était un petit
homme assez ridicule, bon homme, qui se prenait par l'amitié et la
confiance, de fort peu d'esprit, et de sens assez court, et qui avec
tout cela ne laissait pas d'être ami intime et à toute portée du P.
Tellier, qui en avait si peu jusque dans sa compagnie. Mais il n'y avait
que de certaines choses que M. {[}le duc{]} et M\textsuperscript{me} la
duchesse d'Orléans pussent dire à ce cerveau étroit, et d'autres qui
eussent perdu leur grâce et leur force dans leur bouche. Ce fut à quoi
je suppléai amplement et utilement par le P. Sanadon, autre ami intime
du P. Tellier, mais à leur insu, parce que je ne voulais pas leur
montrer tous mes ressorts, quoique ce fût pour eux que je les misse en
oeuvre, pour ne les pas ralentir et apparesser par compter trop sur mon
industrie. Je fis donc entendre à ce père les mêmes choses qu'ils
disaient au P. du Trévoux, mais avec plus de force. Je les paraphrasai
de tout ce que j'y pus ajouter, surtout de ce qui pouvait entrer dans
l'intérêt des jésuites, leur donner envie pour l'amour d'eux-mêmes du
mariage de Mademoiselle, et toute la frayeur que je pus de celui de
M\textsuperscript{lle} de Bourbon. Comme je parlais à un homme qui était
pour moi de toute confiance, je le fis nettement et sans mesure\,; et
comme je disais effectivement la vérité, je ne craignis pas de la
présenter toute nue et dans toute son âpreté. Cela passa de même façon
au P. Tellier\,; et quoique je fusse fort à portée de lui, ces choses
lui firent une tout autre impression de la bouche d'un jésuite bien
endoctriné et bien affectionné à moi que de la mienne. Toutefois nous ne
laissâmes pas de nous en parler souvent lui et moi avec ce tour.

Les jésuites, à qui rien n'est indifférent, et moins les choses majeures
que les autres, c'est-à-dire le P. Tellier, et ce conseil si étroit, si
inconnu même des autres jésuites, par qui tout le grand et l'important
se régit parmi eux, s'affectionnèrent à celle-ci comme à la leur propre,
et se rendirent d'eux-mêmes capables de tout concerter avec nous, et
d'entrer en part des conseils et des exécutions. Ils devinrent donc un
très-puissant instrument, avec cela d'heureux qu'il était de soi
très-concordant avec les ducs de Chevreuse et de Beauvilliers, par le
plus secret et le plus sensible recoin de la cabale. On se souviendra
ici que, lors de l'orage du quiétisme, la politique société se divisa.
Le gros, avec le P. La Chaise, le P. Bourdaloue, le P. La Rue, en un mot
les jésuites de cour et du grand monde, furent contre M. de Cambrai,
mais sans agir. Un petit nombre, et ce qui se peut appeler leur
sanhédrin secret, fut pour ce prélat, et le servit sous main de toutes
ses forces. Ainsi les puissances de Rome et de France ne furent point
choquées, et les bons pères ne laissèrent pas d'aller à leur fait.
Ceux-là demeurèrent intimement unis à M. de Cambrai, et par ceux-là en
effet la société entière. Dans cette intimité de parti le P. Tellier
avait toujours tenu les premiers rangs, et la liaison était d'autant
plus étroite qu'elle était moins connue, et c'est ce qui avait le plus
contribué au choix que les deux ducs en firent pour confesseur du roi.
Je ne l'appris qu'après, mais j'en étais parfaitement instruit lors de
ces menées pour le mariage, et c'était là le noeud secret de l'union du
P. Tellier avec les deux ducs, d'où l'identité de leurs vues en faveur
de Mademoiselle tirait une force dont ne s'apercevaient pas ceux-là
mêmes qui étaient le plus avant dans l'intrigue du mariage.

Causant un jour avec M. le duc d'Orléans sur son départ alors pour
l'Italie, la conversation tomba sur M. de Cambrai. Il échappa au prince
que, si, par de ces hasards qu'il est impossible d'imaginer, il se
trouvait le maître des affaires, ce prélat vivant et encore éloigné, le
premier courrier qu'il dépêcherait serait à lui, pour le faire venir et
lui donner part dans toutes. Ce mot ne tomba pas. J'eus grand soin d'en
faire part aux deux ducs, dans le coeur et l'esprit desquels il fonda
une bienveillance qui germa toujours, et que je parvins à porter jusqu'à
un attachement, dans le secret profond mais intime duquel je fus seul
entre eux, mais qui n'aurait pas ployé des gens si vertueux au mariage
de Mademoiselle s'ils avaient eu la moindre lueur d'espérance d'un
mariage étranger, et s'ils n'eussent pas très-distinctement vu les
dangereuses suites de celui de M\textsuperscript{lle} de Bourbon pour
Mgr le duc de Bourgogne, pour toute la famille royale immédiate et pour
l'État, quoiqu'en particulier M. de Chevreuse eût déjà assez de liaison
avec M. le duc d'Orléans par celles que son pauvre fils, le duc de
Mont-fort, y avaient eues, par le goût des mêmes sciences, et par des
dissertations que le duc de Chevreuse ne fuyait pas, parce qu'il les
ramenait toutes à la religion, à laquelle il voulait ramener M. le duc
d'Orléans. L'accord si peu connu, si sûr, si profond de tous ces
ressorts, par des motifs divers et si cachés, fut un bonheur très-rare.
Je me gardai bien d'en découvrir toutes les trames et la force à la
paresse de M\textsuperscript{me} la duchesse d'Orléans, ni à la
nonchalance et à l'indiscrétion de M. le duc d'Orléans.

Avec ces secours, je voulus encore m'aider d'un personnage qui, tout
abattu qu'il fut auprès du roi, conservait toute sa juste considération
dans le monde et les mêmes accès auprès de M\textsuperscript{me} de
Maintenon, et qui, une fois bien persuadé en faveur de Mademoiselle,
était capable de porter de grands coups. Ce fut le maréchal de
Boufflers. Outre ces fortes raisons, je fus bien aise de l'attirer dans
une union de desseins avec le duc de Beauvilliers, et peu à peu les
disposer à s'unir solidement pour les suites\,; de l'écarter ainsi
doucement de la cabale des seigneurs, et d'ôter à ceux-ci tout usage du
maréchal, si, éventant la mine par quelque intérêt ou par celui seul de
contrecarrer le duc de Beauvilliers, il leur prenait envie de nuire à
Mademoiselle. Je n'eus pas peine à persuader Boufflers, mon ami si
particulier, déjà enclin à M. le duc d'Orléans par la confidence qu'il
lui avait faite de sa rupture avec M\textsuperscript{me} d'Argenton et
de ce qui l'avait accompagnée. Une autre raison le jeta encore vers
Mademoiselle\,: d'Antin était ami du maréchal de Villars. On a vu en son
lieu combien il tomba dans les cabinets, et parlant au roi, sur la
seconde lettre de Boufflers sur la bataille de Malplaquet, que je le sus
aussitôt, et que j'en avertis Boufflers à son arrivée de Flandre. Il
n'ignorait pas l'union intime de d'Antin avec M\textsuperscript{me} la
Duchesse, si bien que, ravi de trouver des raisons solides pour le
mariage de Mademoiselle, il me donna parole de la servir de tout son
pouvoir. Il y avait cela de commode avec le maréchal de Boufflers que
promettre et tenir, et bien exécuter, était pour lui même chose, et
qu'avec ses amis intimes, comme je l'étais, il disait franchement ce
qu'il pouvait, jusqu'à quel point et comment, tellement qu'on ne prenait
point avec lui de fausses mesures, quand on était à cette portée avec
lui et qu'il faisait tant que d'en vouloir bien prendre.

Telles furent les machines et les combinaisons de ces machines, que mon
amitié pour ceux à qui j'étais attaché, ma haine pour
M\textsuperscript{me} la Duchesse, mon attention sur ma situation
présente et future, surent découvrir, agencer, faire marcher d'un
mouvement juste et compassé, avec un accord exact et une force de
levier, que l'espace du carême commença et perfectionna, dont je savais
toutes les démarches, les embarras et les progrès par tous ces divers
côtés qui me répondaient, et que tous les jours aussi je remontais en
cadence réciproque.

\hypertarget{chapitre-xi.}{%
\chapter{CHAPITRE XI.}\label{chapitre-xi.}}

1710

~

{\textsc{Adresse de M\textsuperscript{me} la duchesse de Bourgogne.}}
{\textsc{- Mot vif de Monseigneur contre le mariage de Mademoiselle, qui
y sert beaucoup.}} {\textsc{- Tables réformées à Marly, où le roi ne
nourrit plus les dames.}} {\textsc{- M\textsuperscript{me} la Duchesse à
Marly dans le premier temps de son veuvage, et obtient d'y avoir ses
filles.}} {\textsc{- Marly offert et refusé pour Mademoiselle.}}
{\textsc{- Raisons et mesures pour presser le mariage.}} {\textsc{-
Timidité de M. le duc d'Orléans, qui ne peut se résoudre de parler au
roi, et s'engage à peine à lui écrire.}} {\textsc{- Nul homme logé à
Marly au château.}} {\textsc{- Lettre de M. le duc d'Orléans au roi sur
le mariage.}} {\textsc{- Courte analyse de la lettre.}} {\textsc{-
Petits changements faits à la lettre, et pourquoi.}} {\textsc{-
Difficultés à rendre la lettre au roi.}} {\textsc{- Étrange timidité de
M. le duc d'Orléans, qui enfin la rend.}} {\textsc{- Succès de la
lettre.}}

~

Vers la fin du carême, M\textsuperscript{me} la duchesse de Bourgogne,
ayant sondé le roi et M\textsuperscript{me} de Maintenon, l'avait
trouvée bien disposée et le roi sans éloignement. Un jour qu'on avait
mené Mademoiselle voir le roi chez M\textsuperscript{me} de Maintenon,
où par hasard Monseigneur se trouva, M\textsuperscript{me} la duchesse
de Bourgogne la loua, et quand elle fut sortie, hasarda avec cette
liberté et cette étourderie de dessein prémédité qu'elle employait
quelquefois, de dire que c'était là une vraie femme pour M. le duc de
Berry. À ce mot Monseigneur rougit de colère et répondit vivement que
cela serait fort à propos pour récompenser le duc d'Orléans de ses
affaires d'Espagne. En achevant ces paroles, il sortit brusquement et
laissa la compagnie bien étonnée, qui ne s'attendait à rien moins d'un
prince d'ordinaire si indifférent et toujours si mesuré.
M\textsuperscript{me} la duchesse de Bourgogne, qui n'avait parlé de la
sorte que pour tâter Monseigneur en présence, fut habile et hardie
jusqu'au bout. Se tournant d'un air effarouché vers
M\textsuperscript{me} de Maintenon\,: «\,Ma tante, lui dit-elle, ai-je
dit une sottise\,?» Le roi, piqué, répondit pour M\textsuperscript{me}
de Maintenon, et dit avec feu que si M\textsuperscript{me} la Duchesse
le prenait sur ce ton-là et entreprenait d'empaumer Monseigneur, elle
compterait avec lui. M\textsuperscript{me} de Maintenon aigrit la chose
adroitement, en raisonnant sur cette vivacité si peu ordinaire à
Monseigneur, et dit que M\textsuperscript{me} la Duchesse lui en ferait
faire bien d'autres, puisqu'elle en était déjà venue jusque-là. La
conversation diversement coupée et reprise s'avança avec émotion, et
avec des réflexions qui nuisirent plus à M\textsuperscript{lle} de
Bourbon que l'amitié de Monseigneur pour M\textsuperscript{me} la
Duchesse ne la servit.

Cette aventure, que M\textsuperscript{me} la duchesse d'Orléans sut
aussitôt par M\textsuperscript{me} le duchesse de Bourgogne, et qu'elle
me rendit dès qu'elle l'eut apprise, me confirma dans ma pensée qu'il
fallait presser et emporter d'assaut sur Monseigneur, en piquant
d'honneur le roi contre M\textsuperscript{me} la Duchesse, lui faire
sentir que l'effet de l'empire de cette princesse sur Monseigneur serait
de le lui rendre difficile à conduire, combien plus si elle emportait
avec lui le mariage de leurs enfants\,; qu'il ne fallait perdre aucune
occasion de bien imprimer au roi la crainte d'avoir à commencer à
compter avec Monseigneur, à ménager M\textsuperscript{me} la Duchesse, à
n'oser leur refuser rien, non de ce que Monseigneur voudrait, mais de ce
que M\textsuperscript{me} la Duchesse lui ferait vouloir, que, de maître
absolu et paisible qu'il avait toujours été dans sa famille, il s'y
verrait à son âge réduit en tutelle par des entraves qui, une fois
usurpées, iraient toujours en augmentant. Je crus également nécessaire
d'effrayer M\textsuperscript{me} de Maintenon, haïe comme elle l'était
de M\textsuperscript{me} la Duchesse, et originellement de Monseigneur,
laquelle à la longue serait rapprochée du roi par lui, par leur fille,
par les menées et les artifices de d'Antin\,; que son crédit
s'affaibliroit par là auprès du roi, et sans cela encore par les
brassières où le roi se trouverait lui-même. J'en fis faire toute la
peur à M\textsuperscript{me} la duchesse de Bourgogne, et pour elle-même
encore, par la duchesse de Villeroy et par M\textsuperscript{me} de
Lévi\,; à Mgr le duc de Bourgogne, par M. de Beauvilliers\,; à
M\textsuperscript{me} de Maintenon par le maréchal de Boufflers\,; au
roi même, par le P. Tellier\,; et toutes ces batteries réussirent.

Les choses en cet état, j'estimai qu'il les fallait laisser reposer et
mâcher, ne les point gâter par un empressement à contre-temps, surtout
ne pas exciter M\textsuperscript{me} la Duchesse par des mouvements,
auxquels ce mot échappé et si fort relevé par Monseigneur la rendrait
attentive, et la laisser assoupir dans la confiance en ses forces et le
mépris de celles qui lui étaient opposées. Toutes ces mesures gagnèrent
la semaine sainte. Je pris ce temps ordinaire d'aller à la Ferté, d'où
je revins droit à Marly, le premier où le roi alla après l'audience
qu'il m'avait accordée, comme je l'ai dit en son temps. Je le répète ici
pour rendre les époques de toute cette grande intrigue plus certaines.
J'appris en y arrivant une petite alarme qui ne m'effraya pas, mais dont
je me servis pour faire renouveler, et de plus en plus inculquer à
M\textsuperscript{me} la duchesse de Bourgogne tout ce qui était vrai à
son égard et {[}à celui{]} de Mgr le duc de Bourgogne, dont je m'étais
servi d'abord pour l'intéresser puissamment en Mademoiselle, et qui a
été expliqué déjà. J'appris donc qu'un soir, pressée peut-être plus que
de raison sur Mademoiselle par M\textsuperscript{me} d'O, et
impatientée, elle lui montra du penchant pour un mariage étranger\,; et
plût à Dieu qu'il eût pu se faire\,! c'eût bien été aussi le mien, comme
je l'ai dit plus haut, mais j'y ai rapporté en même temps les raisons
qui le rendaient impossible, et il l'était devenu de plus en plus alors,
tant par les menées de M\textsuperscript{me} la Duchesse que par les
mesures en faveur de Mademoiselle\,; ainsi je ne répéterai rien
là-dessus. Arrivant à Marly, j'y trouvai tout en trouble, le roi chagrin
à ne le pouvoir cacher, lui toujours si maître de soi et de son visage,
la cour dans l'opinion de quelque nouveau malheur qu'on ne se pouvait
résoudre à déclarer. Quatre ou cinq jours s'y passèrent de la sorte\,; à
la fin on sut et on vit de quoi il s'agissait. Le roi, informé que Paris
et tout le public murmurait fort des dépenses de Marly, dans des temps
où on ne pouvait fournir aux plus indispensables d'une guerre forcée et
malheureuse, s'en piqua cette fois-là plus que tant d'autres qu'il avait
reçu les mêmes avis, sans raison plus particulière, ou qu'au moins elle
soit venue jusqu'à moi. Mais le dépit fut si grand que
M\textsuperscript{me} de Maintenon eut toutes les peines du monde de
l'empêcher, et par deux fois, de retourner tout court à Versailles,
quoique ce voyage eût été annoncé pour dix-huit jours au moins. La fin
fut que, au bout de ces quatre ou cinq jours, le roi déclara, avec un
air de joie amère, qu'il ne nourrirait plus les dames à Marly\,; qu'il y
dînerait désormais seul à son petit couvert comme à Versailles\,; qu'il
souperait tous les jours à une table de seize couverts avec sa famille,
et que le surplus des places serait rempli par des dames qui seraient
averties dès le matin\,; que les princesses de sa famille auraient
chacune une table pour les dames qu'elles amenaient, et que
M\textsuperscript{me}s Voysin et Desmarest en tiendraient chacune une,
pour que toutes les dames qui ne voudraient pas manger dans leur chambre
eussent à choisir où aller. Il ajouta avec aigreur qu'il ne
travaillerait plus à Marly qu'en amusements de bagatelles, et que de
cette façon, n'y dépensant pas plus qu'à Versailles, il aurait au moins
le plaisir d'y pouvoir être tant qu'il voudrait, sans qu'on pût le
trouver mauvais. Il se trompa d'un bout à l'autre, et personne autre que
lui n'y fut trompé, si tant est qu'il le fut en effet, sinon en croyant
en imposer au monde.

Il fallut établir des tables, comme à Versailles, pour le bas étage de
ce qui y avait bouche à cour, et qui vivait de la desserte des trois
tables qui jusque-là étaient soir et matin servies dans un des petits
salons pour le roi et les dames. Il fallut des cuisines aux princesses
et d'autres appartenances, et tout aussitôt réparer ce qu'on avait pris
pour cela par des bâtiments nouveaux, qui furent fort étendus pour
pouvoir mener plus de monde. Les ateliers et les noms furent changés,
mais d'Antin laissa subsister les ouvrages sous une autre face.
L'épargne en effet demeura nulle, les ennemis se moquèrent de ce
retranchement avec insulte, les plaintes des sujets ne cessèrent point,
et l'interruption du courant des affaires, souvent importantes et
pressées, ne fit qu'augmenter par l'allongement et la fréquence de ces
voyages dont le roi avait compté de s'acquérir ainsi toute la liberté.

M\textsuperscript{me} la Duchesse, qui voulait tenir Monseigneur de
près, et qui connaissoit le danger de l'interruption d'un continuel
commerce, avait, contre toute bienséance, dans ses deux premiers mois de
deuil, obtenu d'être de tous les voyages de Marly. Ce n'avait pas été
sans peine, sans autre raison toutefois que le roi, qui voulait que ses
tables fussent toujours remplies sans que personne y manquât (et ce ne
fut que dans la première huitaine de ce Marly qu'il les retrancha), et
qui était jaloux aussi que le salon fût toujours vif et plein, craignait
que l'appartement de M\textsuperscript{me} la Duchesse, qui n'en pouvait
sortir que par le cabinet du roi après son souper, fît une diversion qui
éclaircirait fort l'un et l'autre. Elle promit là-dessus l'attention la
plus discrète, à laquelle le roi se rendit voyant qu'il fallait céder ou
défendre, à quoi il ne voulut pas se porter. Le retranchement des
tables, qui suivit de si près le commencement de ce voyage, élargit
M\textsuperscript{me} la Duchesse pour les suites. Elle voulut avoir à
Marly M\textsuperscript{lle}s de Bourbon et de Charolais, les deux de
ses six filles qui étaient élevées auprès d'elle, et qui avaient eu pour
elles deux un méchant petit logement, tout en haut, à Versailles,
lorsque Cavoye le quitta pour celui de M. de Duras.
M\textsuperscript{me} la Duchesse allégua l'épargne, l'état de ses
affaires et la dépense d'avoir une table et un détachement de sa maison
pour ses filles à Versailles, pendant les Marlys. Le roi y consentit.
Elle avait d'autres raisons\,: elle voulait en amuser Monseigneur\,;
suppléer par elles à ce dont son état de veuve l'empêchait\,; accoutumer
le roi à leur visage, avec qui il était difficile qu'elles ne soupassent
pas souvent\,; détourner Monseigneur, qui ne pouvait jouer chez elle
dans ces premiers temps et qui s'ennuyait chez M\textsuperscript{me} la
princesse de Conti, de s'adonner chez M\textsuperscript{me} la duchesse
de Bourgogne, et par ses filles, bourdonnant dans le salon autour de
lui, des particuliers momentanés qu'il pouvait avoir avec
M\textsuperscript{me} la duchesse de Bourgogne, souvent si utiles à
faute d'autres que ces gens-là ne savent pas se donner dans leur
famille\,; enfin les tenir avec lui à jouer chez M\textsuperscript{me}
la princesse de Conti, sa dupe éternelle, qui espérait se rapprocher de
Monseigneur en la servant à son gré, et qui, pour les yeux, était une
autre elle-même dans le salon, où avec sa cabale, M\textsuperscript{me}
la Duchesse n'ignorait rien de ce qui s'y passait de plus futile.

Dès que cela fut accordé, le roi, qui voulait toujours tenir égale la
balance entre ses filles, proposa à M\textsuperscript{me} la duchesse
d'Orléans que Mademoiselle fût de tous les Marlys. Elle était à
Versailles, son rang était réglé avec les princesses du sang\,; ainsi
nulle difficulté. Cette proposition fut la matière d'une délibération
entre M. et M\textsuperscript{me} la duchesse d'Orléans et moi. Après
avoir bien discuté le pour et le contre, nous nous trouvâmes tous trois
du même avis de laisser Mademoiselle à Versailles, et de ne
s'embarrasser point de voir M\textsuperscript{lle} de Bourbon passer les
journées dans le même salon, et souvent à la même table de jeu que M. le
duc de Berry, se faire admirer de la cour, voltiger autour de
Monseigneur, et accoutumer le roi à elle. Ce n'était aucune de ces
bagatelles qui ferait son mariage\,; mais d'avoir Mademoiselle à Marly
pouvait rompre le sien, exposée comme elle serait à toutes les pièces,
qu'une malice si intéressée et si connue, et à toutes les affaires les
plus fausses ou les plus imprévues, que la même malignité lui
susciterait, soutenues de cette audacieuse cabale, et de Monseigneur
même, sous les yeux de M. le duc de Berry qu'on dégoûterait, du roi
qu'on embarrasserait, et qui se trouverait infiniment importuné des
éclaircissements et des plaintes que M\textsuperscript{me} la duchesse
de Bourgogne ne pourrait pas toujours soutenir, et qui lasseraient la
faiblesse de M\textsuperscript{me} de Maintenon, toutes choses
très-dangereuses au mariage et très-inutiles à hasarder. Nous conclûmes
donc à remercier, et à ne rien changer à la vie séparée de Mademoiselle,
et ce refus fut fort approuvé.

Dans cet état de choses, je fus frappé de l'importance d'aller
rapidement en avant. Je sentis toute la force de ces nouvelles mesures
de M\textsuperscript{me} la Duchesse, et je prévis que, plus on perdrait
de temps, moins il deviendrait favorable à Mademoiselle.
M\textsuperscript{me} la duchesse de Bourgogne, que je fis presser, fut
du même avis\,; le P. Tellier, avec qui j'avais souvent conféré, et qui
passait deux jours chaque semaine à Marly, pensa de même\,; M. de
Beauvilliers aussi. Un jour que M\textsuperscript{me} la duchesse
d'Orléans se trouva légèrement indisposée, il monta avec moi dans sa
chambre, où, dans un coin écarté de la compagnie, il traita cette
matière à découvert entre M. le duc d'Orléans et moi, et j'en fus ravi
dans l'espérance que cela encouragerait ce prince. Le maréchal de
Boufflers fut du même sentiment, et pressa M\textsuperscript{me} de
Maintenon utilement. Je fis en sorte que M\textsuperscript{me} la
duchesse d'Orléans, qui n'était pas en état de descendre, fît prier
M\textsuperscript{me} la duchesse de Bourgogne, par la duchesse de
Villeroy, de monter chez elle. Je dis que je fis en sorte, parce que,
paresse ou timidité, avec un désir extrême, cette princesse ne se
remuait qu'à force de bras. M\textsuperscript{me} la duchesse de
Bourgogne y monta. Le tête-à-tête dura plus d'une heure.

Les fréquents particuliers entre la duchesse de Villeroy,
M\textsuperscript{me} de Lévi, M. et M\textsuperscript{me} d'O,
M\textsuperscript{me} la duchesse de Bourgogne les uns avec les autres,
les miens surtout chez M\textsuperscript{me} la duchesse d'Orléans, et à
toutes heures, quoiqu'ils parussent moins, donnèrent à parler au
courtisan curieux et oisif\,; ce qui, suivi de cette longue conférence
de M\textsuperscript{me} la duchesse de Bourgogne chez
M\textsuperscript{me} la duchesse d'Orléans, alarma
M\textsuperscript{me} la Duchesse\,; dont il résulta que Monseignenr se
fronça encore plus qu'à l'ordinaire avec M. le duc d'Orléans, se
rengorgea avec M\textsuperscript{me} la duchesse de Bourgogne, et se
montra plus rêveur et plus froid au roi pour en être moins accessible.
Toutes ces choses me hâtèrent de plus en plus. Après avoir fort concerté
toutes choses, et m'être assuré du succès de diverses tentatives et de
M\textsuperscript{me} de Main-tenon, nous proposâmes
M\textsuperscript{me} la duchesse d'Orléans et moi, à M. le duc
d'Orléans, de parler au roi. D'abord il se hérissa, mais battu presque
sans cesse un jour et demi de suite, et ne pouvant nous résister, armés
comme nous l'étions de l'avis et du concert de M\textsuperscript{me} la
duchesse de Bourgogne, de M\textsuperscript{me} de Maintenon, de M. de
Beauvilliers, du maréchal de Boufflers et du P. Tellier, il nous dit
franchement qu'il ne savait comment s'y prendre, que le mariage en soi
était ridicule à proposer dans un temps de guerre et de misère, et le
mariage de sa fille plus fou et plus insensé que nul autre.
M\textsuperscript{me} la duchesse d'Orléans se trouva étrangement
étourdie de cet aveu si nettement négatif\,; pour moi, il ne me mit
qu'en colère.

Je répondis qu'il se faisait tous les jours tant de sottises gratuites
qu'il en pouvait bien espérer une en sa faveur, et n'être retenu de la
demander, puisqu'elle lui était si importante. Je n'y gagnai rien. Après
avoir longtemps disputé, il nous dit franchement qu'il n'avait ni le
front ni le courage de parler, et que, s'il le faisait dans cette
disposition, ce serait si mal qu'il ne ferait que gâter son affaire.
Toutefois sans cela elle ne se pouvait amener au delà des termes où elle
se trouvait conduite, et il s'agissait de la bâcler sous peine de la
manquer sans retour. Réduite en ces termes, et M\textsuperscript{me} la
duchesse d'Orléans, pour ainsi dire pétrifiée de surprise et de douleur,
je pris mon parti\,; ce fut de proposer à M. le duc d'Orléans, puisqu'il
était fermé à ne point parler au roi, au moins de lui écrire et de lui
rendre sa lettre lui-même. Cette proposition rendit la vie et la parole
à M\textsuperscript{me} la duchesse d'Orléans, qui applaudit à cet avis
qu'elle-même avait mis en avant d'abord à la proposition de parler, et
j'y avais résisté comme beaucoup plus faible que la parole, et j'y étais
revenu lorsque je ne vis point d'autre ressource. Pour
M\textsuperscript{me} la duchesse d'Orléans, elle crut toujours qu'une
lettre qui demeurait, et qui se pouvait relire plus d'une fois dans un
intérieur de gens favorables, valait mieux que le discours. Le succès
montra qu'elle avait raison. M. le duc d'Orléans y consentit. Je
craignis ses réflexions, et je le pressai d'écrire sur-le-champ. Il
logeait toujours en bas du premier pavillon du côté de la chapelle, avec
M. le Prince ou M. le prince de Conti en haut, après leur mort avec M.
de Beauvilliers. Le voyage suivant cela fut changé. Il eut pour toujours
un logement au château, en haut, de suite de celui de
M\textsuperscript{me} sa femme, où, pour le dire en passant, il n'y
avait eu au château d'hommes logés que les fils de France, et le
capitaine des gardes en quartier, et aucune femme mariée que les filles
de France et enfin M\textsuperscript{me} la duchesse d'Orléans. Tant que
le roi vécut cela ne fut point autrement, sinon en faveur de M. et
M\textsuperscript{me} la duchesse d'Orléans.

Comme M. le duc d'Orléans sortait, M\textsuperscript{me} la duchesse
d'Orléans me dit d'un air peiné\,: «\,Allez-vous le quitter\,?» puis\,:
«\,N'écrirez-vous point\,?» Elle voulait que je fisse la lettre. Je
suivis donc M. le duc d'Orléans qui, en arrivant chez lui, où il n'eut
jamais ni plume, ni encre, ni papier, demanda à ses gens de quoi écrire,
qui en apportèrent de fort mauvais. Il me proposa que nous fissions la
lettre ensemble\,; mais importuné dès la première ligne, je lui en
remontrai l'inconvénient et le priai de faire sa lettre et moi une
autre\,; qu'il chaisirait après, ou corrigerait et ajoute-roit ce qu'il
voudrait\,; et là-dessus je me mis à écrire. Vers le milieu de ma
lettre, que je fis rapidement tout de suite, le hasard me fit lever les
yeux sur lui, en prenant de l'encre dans ma plume\,; et je vis qu'il
n'avait pas écrit un mot depuis que nous avions cessé de faire ensemble,
et que, couché dans sa chaise, il me voyait écrire tranquillement. Je
lui en dis mon avis en un mot, et continuai. Il me dit pour raison qu'il
n'était non plus en état d'écrire que de parler. Je ne voulus pas
contester. Cette lettre qui emporta le mariage, et qui peint mieux que
les portraits l'intérieur du roi, par le tour dont elle s'exprime, pour
l'emporter comme elle fit, mérite par ces raisons d'être insérée ici, et
n'est pas d'ailleurs assez longue pour être renvoyée aux Pièces. La
voici telle que je la fis d'un seul trait de plume en présence de M. le
duc d'Orléans, comme je viens de le dire\,:

«\,Sire,

«\,Plusieurs pensées m'occupent et me pénètrent depuis longtemps, que je
ne puis plus me refuser de représenter à Votre Majesté, puisqu'elles ne
peuvent lui déplaire, et que depuis peu diverses occasions ont tellement
grossi dans mon coeur et dans mon esprit les sentiments qu'elles y ont
fait naître que je ne puis que je ne les porte aux pieds de Votre
Majesté, avec cette confiance que vos anciennes bontés, et, si j'ose
l'ajouter, que le sang inspirent\,; et je le fais par écrit dans la
crainte de ma plénitude, qui est telle que j'aurais appréhendé de vous
parler trop diffusément. Il y a deux ans, Sire, que Votre Majesté fit
naître en moi des espérances flatteuses du mariage de M. le duc de Berry
avec ma fille. Elle me fit l'honneur de me dire qu'il n'y avait point en
Europe de princesse étrangère qui lui convînt, et j'ose ajouter que la
France ne lui en peut offrir aucune au préjudice de ma fille. J'ai vécu
depuis dans ce raisonnable désir que vous-même m'avez accru. Je vois
cependant que le temps s'écoule, et qu'en s'écoulant vous prenez plaisir
à combler votre famille de nouveaux biens. Quelles grâces à la fois pour
M\textsuperscript{me} la Duchesse que sa pension, celle de son fils, la
charge de grand maître et le gouvernement de Bourgogne\,! Quelle faveur
à M. du Maine que la survivance de colonel général des Suisses et
Grisons, et de grand maître de l'artillerie pour ses enfants, et un rang
qui les égale au mien\,! Vous m'avez fait son beau-frère, et je suis
bien aise de ses avantages\,; mais qu'il me soit permis de vous
représenter, avec toute sorte de respect, que l'état de ma famille est
tel, que, si je mourais, il ne serait pas en la puissance de votre
amitié de lui en donner des marques semblables\,; puisque les honneurs
que je tiens de vous ne lui passeraient pas, et que, n'ayant ni
gouvernement ni charge, elle ne peut être revêtue de rien, par quoi mes
enfants seraient bien moindres en effet, quoique si fort aînés des
autres, et vos petits-enfants comme eux. Qu'est-il donc au pouvoir de
Votre Majesté de faire, pour eux et pour moi, qu'un mariage que je ne
puis douter qui ne soit de son goût, par ce qu'elle m'a fait la grâce de
m'en dire le premier, qui réunit tous ses enfants, et qui assure une
protection aux miens, quelque dénués qu'il soient d'ailleurs, jusqu'à
l'accomplissement duquel je suis sans cesse entre la crainte et
l'espérance\,? Voilà, Sire, mes raisons de père qui me touchent
sensiblement\,; mais j'en ai d'autres qui me tiennent encore bien plus
vivement au coeur, et qui me le serrent, de sorte qu'il n'est pas que
vous ne vous intéressassiez à me rendre le repos, si vous étiez informé
de tout ce que je souffre.

«\,Vous avez nouvellement comblé toute votre famille de biens, et moi
seul je me trouve excepté. Vous avez cherché à consoler
M\textsuperscript{me} du Maine du chagrin qu'elle s'est voulu faire sur
son rang, moi seul je me trouve encore égalé aux princes du sang à votre
communion. Je me trouve condamné, en la personne de mes filles, sur le
rang que j'avais cru devoir prétendre pour elles. J'étouffe mon chagrin
par soumission, et pour vous rendre un plus profond respect. Rien
cependant ne me console, et rien ne s'avance pour l'unique chose qui
pourrait le faire. Que puis-je penser là-dessus, Sire, sinon de craindre
de n'être pas avec Votre Majesté comme j'ose dire que le mérite mon
coeur pour elle, ou qu'il se présente un autre obstacle, que je vois
depuis longtemps se former avec art et se grossir de même\,? Car pour la
conjoncture des temps, tout apprend, et ces derniers exemples, que vous
êtes trop grand, trop absolu, trop maître pour qu'une semblable raison
arrête ce que vous voulez faire\,; et puisque l'état des princesses de
l'Europe est tel que le mariage de M. le duc de Berry ne peut rien
influer à la paix, votre amitié et votre autorité peuvent trouver les
expédients nécessaires de passer en ma faveur, comme vous avez fait pour
les autres, par-dessus la conjoncture des temps. Mon malheur est donc
tel que je ne puis plus attribuer le silence sur ce mariage qu'à votre
volonté, et j'en mourrais de douleur, et qu'à l'éloignement qu'on ne
cesse de donner contre moi, avec toute la malignité et l'artifice
possible, à celui dont la bonté et l'équité naturelle, l'ancienne amitié
pour moi en rendrait tout à fait incapable sans un crédit aussi grand,
et dont l'augmentation continuelle ne promet qu'une division que rien ne
pourra éteindre dans votre maison si j'en deviens la victime, dans un
temps surtout où, contents ou jamais, on ne devrait avoir aucune aigreur
de reste. C'est donc, Sire, mon extrême et respectueuse tendresse pour
votre personne, mon attachement pour celle de Monseigneur, qui plus que
tout me fait du désir de me voir rapprocher de Votre Majesté et de lui
par les liens les plus étroits et les plus intimes, et qui, d'ailleurs,
terminant toute aversion, et me donnant lieu de m'unir par ma seconde
fille avec M\textsuperscript{me} la Duchesse, liera son fils à M. le duc
de Berry par un honneur semblable à celui que mon fils en recevra
lui-même. Ces considérations sont telles que j'espère enfin qu'elles
toucheront le bon coeur de Votre Majesté, et je lui demande avec toute
l'instance dont peut être capable, avec le plus profond respect, Sire,
de Votre Majesté,

«\,Le très-humble,\,» etc.

J'avais tâché de faire entrer dans cette lettre tout ce qui pouvait
porter à une détermination prompte\,: une préface touchante par le
respect, la confiance et le souvenir que la pensée de ce mariage était
d'abord venue du roi\,; une énumération ensuite des prodigieux bienfaits
si récemment répandus sur M\textsuperscript{me} la Duchesse et sur M. du
Maine\,; une comparaison forte, mais légère, de sa nudité, en faisant
délicatement souvenir le roi qu'il l'avait marié, et ne faisant que
montrer, comme à la dérobée, la grandeur de sa naissance en leur
comparaison\,; ne tirer droit que parce que ses enfants étaient aussi
ses petits-enfants, flatterie la plus puissante sur le roi. J'essaye de
découvrir avec douceur et sacrifice les divers griefs de rang, et de
montrer qu'en tout il ne peut y avoir de dédommagement que le mariage.
Passant de là à des tendresses bienséantes à un neveu et à un gendre si
élevé, je présente l'empire de M\textsuperscript{me} la Duchesse sur
Monseigneur, avec la force précisément nécessaire pour se faire sentir,
et la mesure propre a écarter de soi l'amertume\,; d'où, après les
louanges, l'excuse de Monseigneur et une échappée de tendresse pour lui,
sort tout à coup une menace qui sans rien exprimer dit tout, et le dit
avec force, sans toutefois pouvoir blesser\,; de là, se rabattant sur
l'union, propose de la rendre effective par un autre mariage, et adoucit
ainsi tout ce qui a échappé de fort, mais laisse ces idées vives en leur
entier, en finissant tout court par des tendresses les plus pressantes
de terminer enfin ce mariage.

Dès que la lettre fut achevée, je la lus à M. le duc d'Orléans, qui de
bonne foi, ou de paresse, la trouva admirable, sans y vouloir changer
rien. Comme je l'avais écrite rapidement et d'une petite écriture, dont
je me sers pour écrire vite et me suivre moi-même, je me défiai des
mauvais yeux de M. le duc d'Orléans\,; ainsi je la lui donnai pour voir
s'il la lirait bien. La précaution fut sage. Il ne put en venir à bout,
de sorte que je m'en allai chez moi en faire une copie qu'il pût lire,
avec promesse de la lui porter le soir même chez M\textsuperscript{me}
la duchesse d'Orléans. Il était tard quand je l'eus achevée. Je trouvai
Pontchartrain à table, chez qui je devais souper, et que je quittai au
sortir de table, pour aller chez M\textsuperscript{me} la duchesse
d'Orléans. Cela fit deux contre-temps qu'il n'y eut pas moyen d'éviter
et qui me fâchèrent. Pontchartrain était d'une curiosité insupportable,
grand fureteur et inquisiteur, sur ses meilleurs amis comme sur les
autres\,; cette arrivée à table, et cette retraite immédiatement après,
le mit en éveil et sa compagnie, quoiqu'ils n'eussent pu rien remarquer
en moi pendant le souper, et dans la suite il ne m'épargna pas les
questions, qui ne lui acquirent pas la moindre lumière. L'autre fut que
je trouvai le roi retiré. Cela fut cause que je ne voulus pas m'arrêter
chez M\textsuperscript{me} la duchesse d'Orléans, où elle et M. le duc
d'Orléans m'attendaient avec impatience. Ils voulurent me retenir à lire
la lettre, mais je me contentai de leur laisser la copie que j'avais
faite pour leur donner, et ne voulus pas être remarqué pour sortir si
tard de chez elle. Je n'y gagnai rien. On le sut, on en fut en
curiosité, mais elle fut peu satisfaite. Le lendemain ils me
remercièrent l'un et l'autre plus en détail. M. le duc d'Orléans avait
copié la lettre et brûlé la copie qu'il en avait de moi, et sa lettre
était toute cachetée. Ils me dirent qu'ils avaient un peu abrégé la
préface, omis la communion du roi, et adouci cette phrase, \emph{trop
grand, trop absolu, trop maître}, et que du reste elle avait été copiée
mot pour mot. S'ils y avaient fait d'autres changements, ils me les
auraient dits tout de même\,; ainsi j'ai inséré ma lettre ici en
marquant ces changements. Le préambule abrégé, je l'avais fait tel qu'il
était pour disposer le roi à n'être pas effarouché\,; la communion,
grief qui me touchait à la vérité, mais qui ne blessait pas moins le
rang de M. le duc d'Orléans, je l'avais mise pour faire sentir au roi
que ce prince était maltraité pour l'amour des autres, et l'exciter
d'autant au seul dédommagement qu'il pouvait lui donner\,; je sentis à
l'instant la double raison qui l'avait fait supprimer à
M\textsuperscript{me} la duchesse d'Orléans\,: l'intérêt de M. son fils
que, depuis le règlement fait contre sa prétention pour ses filles, elle
ne pouvait espérer de faire plus que prince du sang\,; et celui des
bâtards égalés en tout aux princes du sang, qui lui était encore bien
plus cher que celui de M. son fils, chose monstrueuse, mais qui se
trouvera bien au net dans la suite. L'adoucissement de la phrase, je
n'en compris pas la raison, d'autant que rien ne flattait plus le roi
que l'opinion et l'étalage de son autorité, et qu'il s'agissait là de
l'en piquer pour l'engager à forcer Monseigneur\,; mais la lettre étant
copiée et cachetée, et ces changements au fond n'altérant rien
d'important à représenter, je ne fis nul semblant de ne les approuver
pas. M\textsuperscript{me} la duchesse d'Orléans fut fort touchée de
l'énumération des grâces nouvellement faites à M\textsuperscript{me} la
Duchesse et à M. du Maine, de la mention du poids de ce pouvoir de
M\textsuperscript{me} la Duchesse sur Monseigneur, surtout de la menace
mêlée de tendresse\,; et elle espéra beaucoup de l'effet de cette
lettre.

Il fut après question de la donner au roi, et ce ne fut pas une petite
affaire. La confidence en fut faite à M\textsuperscript{me} la duchesse
de Bourgogne, et par elle à M\textsuperscript{me} de Maintenon, de la
lettre s'entend, non du vrai auteur, et toutes deux l'approuvèrent, mais
pressèrent de la remettre. La même confidence fut aussi faite au P.
Tellier par le P. du Trévoux, afin qu'elle fût plus obligeante par cette
voie que par la mienne, comme venant plus purement de M. {[}le duc{]} et
de M\textsuperscript{me} la duchesse d'Orléans. Le confesseur promit
d'agir en conséquence. Lui et moi en conférâmes, et il tint bien sa
parole. Je la fis aussi à M. de Beauvilliers pour Mgr le duc de
Bourgogne. Elle n'alla pas au delà, pour en mieux conserver le secret
dans le pur nécessaire au succès. Pour rendre cette lettre, il fallait
trouver une jointure où le roi et M\textsuperscript{me} de Maintenon,
toute bien intentionnée qu'elle était, fussent de bonne humeur\,; où
elle passât la journée à Marly, car elle allait presque tous les jours à
Saint-Cyr, et ces jours-là le roi ne la voyait que le soir\,; où le P.
Tellier fût à Marly, qu'il n'y venait que le mercredi ou souvent le
jeudi jusqu'au samedi\,; enfin éviter que d'Antin vît donner la lettre,
qui était toujours dans les cabinets, et qui, sur une démarche aussi peu
ordinaire, ne manquerait pas d'alarme et de soupçons, et de les donner à
l'instant à Monseigneur et à M\textsuperscript{me} la Duchesse, qu'il
s'agissait sur toutes choses de maintenir dans la tranquille sécurité
qu'ils avaient prise. Tant de choses à ajuster à la fois étaient affaire
bien difficile. Toutefois le hasard les présenta toutes le vendredi et
le samedi suivant, sans que l'extrême timidité de M. le duc d'Orléans à
l'égard du roi eût osé en profiter, quoique sa lettre toujours en poche.

Cependant M\textsuperscript{me} la duchesse de Bourgogne pressait sans
cesse M\textsuperscript{me} la duchesse d'Orléans, tant de sa part que
de celle de M\textsuperscript{me} de Maintenon. Huit jours après, le
vendredi matin, je sus par Maréchal que le roi se portait bien, et avait
été gaillard avec eux à son premier petit lever\,; que
M\textsuperscript{me} de Maintenon ne sortait point de chez elle de tout
le jour\,; car elle avait un autre petit appartement avec une tribune
sur la chapelle qu'on appelait le Repos, sanctuaire tout particulier où
elle allait souvent se cacher quand elle n'allait pas à Saint-Cyr. Le P.
Tellier était à Marly comme tous les vendredis\,; et de grande fortune,
d'Antin était allé faire une course à Paris. Je trouvai M. le duc
d'Orléans dans le salon, comme le roi revenant de la messe entrait chez
M\textsuperscript{me} de Maintenon, comme il faisait toujours à Marly
quand elle y était les matins. Je dis à M. le duc d'Orléans ce que
j'avais appris, je lui demandai combien de temps encore il avait résolu
de garder sa lettre en poche. Je lui dis que j'étais bien informé que
M\textsuperscript{me} la duchesse de Bourgogne et M\textsuperscript{me}
de Maintenon même blâmaient fort sa lenteur. Je lui appris que le monde
s'apercevait à son air rêveur et embarrassé qu'il avait quelque chose
dans la tête, et que la visite et le tête-à-tête de
M\textsuperscript{me} la duchesse de Bourgogne chez
M\textsuperscript{me} la duchesse d'Orléans, et nos divers particuliers,
avaient été fort remarqués. Il voulait, il n'osait. Nous fûmes ainsi
trois bons quarts d'heure en dispute dans ce salon, rempli à ces
heures-là des plus considérables courtisans qui nous voyaient et que je
mourais de peur qui ne nous remarquassent. Enfin le roi passa de chez
M\textsuperscript{me} de Maintenon chez lui, et le salon se vida dans le
petit salon entre-deux, et dans sa chambre. Alors je pressai M. le duc
d'Orléans de toute ma force d'aller donner sa lettre. Il s'avançait vers
le petit salon, puis tournait le dos à la mangeoire. Moi toujours
l'exhortant, je le serrais de l'épaule vers le petit salon, je faisais
le tour de lui pour le remettre entre ce petit salon et moi quand il
s'en était écarté, et ce manége se fît à tant de reprises que j'étais
sur les épines de ce peu de gens du commun restés dans le grand salon,
et des courtisans qui, du petit, nous pouvaient voir pirouettant de la
sorte, à travers la grande porte vitrée. Toutefois je fis tant qu'à
force de propos, de tours, et d'épaule, je le poussai dans le petit
salon, et de là encore avec peine jusqu'à la porte de la chambre du roi,
tout ouverte. Alors il n'y eut plus à rebrousser, il fallut pousser
jusque dans le cabinet. Restait s'il oserait enfin y donner sa lettre.

J'entrai lentement pour ne pas traverser la chambre avec lui, et je
gagnai la croisée la plus proche du cabinet, dans la profondeur de
laquelle on me fit place sur un ployant, où je m'assis auprès du
maréchal de Boufflers, avec M. de Bouillon, M. de La Rochefoucauld, le
duc de Tresmes et le premier écuyer, en attendant que le roi sortît pour
prendre d'autres habits et aller dans ses jardins. Je n'avais pas été
trois ou quatre \emph{Pater} assis que je vis avec surprise sortir M. le
duc d'Orléans, qui brossa la chambre et disparut. Je ne fis que me lever
et me rasseoir avec les autres, bien en peine de ce qui s'était pu
passer dans des instants si-courts. Le roi fut assez longtemps sans
sortir. Enfin il vint, changea d'habit, et alla à la promenade, où je le
suivis. Tant à son habiller qu'à sa promenade j'observai soigneusement
son maintien. Jamais homme n'en fut plus le maître\,; mais comme il
était impossible qu'il se pût douter que qui que ce fût de ce qui
l'environnait sût que M. le duc d'Orléans devait lui avoir donné une
lettre, je voulus voir s'il serait gai, ou sérieux et concentré. Je ne
le trouvai rien de tout cela, mais entièrement à son ordinaire, de sorte
que je demeurai fort en peine de ce que la lettre était devenue. Après
quelques tours, le roi s'arrêta au bassin des carpes, du côté de
M\textsuperscript{me} la Duchesse. M. le duc d'Orléans l'y vint joindre
sans se trop approcher de lui. Un peu après, le roi tourna pour se
promener ailleurs. Je me tins en arrière, M. le duc d'Orléans aussi,
dans l'impatience réciproque de nous parler. Il me dit qu'il avait donné
sa lettre, que d'abord le roi surpris lui avait demandé ce qu'il lui
voulait, qu'il lui avait dit\,: rien qui lui pût déplaire, qu'il le
verrait par sa lettre, et que ce n'était pas chose dont il pût aisément
lui parler\,; que sur cette réponse, le roi, plus ouvert, lui avait dit
qu'il la lirait avec attention, sur quoi il était sorti pour ne pas
laisser refroidir la première curiosité\,; et qu'en effet, étant près de
la porte, il avait tourné un peu la tête, et vu que le roi ouvrait sa
lettre. Ce mot dit, nous rejoignîmes la queue de la suite du roi, pour
nous mêler et reparaître séparément. Je me sentis bien soulagé d'une si
grande affaire faite, et j'avoue que ce ne fut pas sans émotion que
j'attendis le succès de mon travail.

L'attente ne fut pas longue. J'appris le lendemain, par M. le duc
d'Orléans, que le roi lui avait dit qu'il avait lu deux fois sa
lettre\,; qu'elle méritait grande attention\,; qu'il lui avait fait
plaisir de lui écrire plutôt que de lui parler\,; qu'il désirait lui
donner contentement, mais que Monseigneur serait difficile, et qu'il
perdrait son temps pour lui en parler. En même temps je sus de
M\textsuperscript{me} la duchesse d'Orléans que le roi avait lu la
lettre, dès le vendredi au soir, chez M\textsuperscript{me} de
Maintenon, entre elle et M\textsuperscript{me} la duchesse de
Bourgogne\,; qu'il l'avait goûtée, louée, et approuvé le désir et les
raisons qu'elle contenait\,; que M\textsuperscript{me} de Maintenon et
M\textsuperscript{me} la duchesse de Bourgogne l'avaient fortement
appuyée\,; que leur embarras était Monseigneur, sur lequel ils avaient
fort raisonné ensemble, et conclu qu'il fallait l'y faire consentir avec
douceur et amitié, bien prendre son temps, n'en point perdre, et que ce
fût le roi qui parlât pour forcer d'autant plus Monseigneur qui ne lui
avait encore jamais dit non à rien. C'était de M\textsuperscript{me} la
duchesse de Bourgogne que M\textsuperscript{me} la duchesse d'Orléans
tenait tout ce récit. Peu de jours après, nous sûmes par le P. du
Trévoux que le roi avait parlé de la lettre au P. Tellier en même sens
que je viens de le dire\,; que le confesseur l'avait confirmé dans ces
sentiments, l'avait affermi sur Monseigneur et persuadé de finir tout le
plus tôt qu'il serait possible. Dans cette heureuse situation je fus
d'avis que M. {[}le duc{]} et M\textsuperscript{me} la duchesse
d'Orléans ne gâtassent rien par un empressement que l'engagement si
formel du roi rendait pire qu'inutile, et gardassent une conduite unie
et serrée pour ne réveiller pas M\textsuperscript{me} la Duchesse et les
siens, et ne troubler pas leur sécurité parfaite, tandis que la mine se
chargeait sous leurs pieds sans qu'ils s'en aperçussent, et que le feu
était déjà au saucisson, et que l'effet n'en pouvait être que fort peu
éloigné. Ils m'en crurent. Leur joie, qu'ils contraignaient au dehors,
était sans pareille, la mienne était égale à la leur, mais elle ne fut
pas sans amertume.

\hypertarget{chapitre-xii.}{%
\chapter{CHAPITRE XII.}\label{chapitre-xii.}}

1710

~

{\textsc{Attaques de M\textsuperscript{me} la duchesse d'Orléans à moi
pour faire M\textsuperscript{me} de Saint-Simon dame d'honneur de sa
fille, devenant duchesse de Berry.}} {\textsc{- Mesures pour éviter la
place de dame d'honneur.}} {\textsc{- Audience de M\textsuperscript{me}
la duchesse de Bourgogne à M\textsuperscript{me} de Saint-Simon sur la
place de dame d'honneur.}} {\textsc{- Situation personnelle de
M\textsuperscript{me} la duchesse d'Orléans avec Monseigneur, guère
meilleure que celle de M. le duc d'Orléans.}} {\textsc{- Projet
d'approcher M. et M\textsuperscript{me} la duchesse d'Orléans de
M\textsuperscript{lle} Choin.}} {\textsc{- Curieux tête-à-tête
là-dessus, et sur la cour intérieure de Monseigneur, entre Bignon, ami
intime de la Choin, et moi.}}

~

Dès les premiers temps du mouvement effectif de ce mariage,
M\textsuperscript{me} la duchesse d'Orléans me demanda, d'un ton trop
significatif pour n'être pas entendu, qui on pourrait mettre dame
d'honneur de sa fille si elle devenait duchesse de Berry. Je saisis donc
incontinent sa pensée, et lui répondis exprès, d'un ton ferme et élevé,
de faire seulement le mariage, et qu'elle aviserait après de reste à une
dame d'honneur, dont elle ne manquerait pas. Elle se tut tout court, M.
le duc d'Orléans ne dit pas un mot, et je changeai sur-le-champ de
discours. De ce moment, jusqu'à la grande force de l'affaire, elle ne me
parla plus de dame d'honneur\,; mais, deux jours avant que je fisse la
lettre dont il vient d'être parlé, elle dans son lit et moi tête à tête
avec elle, au milieu d'une conversation très-importante sur le
mariage\,: «\,Pour cela, interrompit-elle tout à coup en me regardant
attentivement, si cette affaire réussit, nous serions trop heureux si
nous avions M\textsuperscript{me} de Saint-Simon pour dame d'honneur.
Madame, lui répondis-je, votre bonté pour elle vous fait parler ainsi.
Elle est trop jeune et point du tout capable de cet emploi. Mais
pourquoi\,?» interrompit-elle, et se mit sur ses louanges en tout genre.

Après l'avoir écoutée quelques moments je l'interrompis à mon tour, en
l'assurant qu'elle ne convenait point à cette place\,; et je me mis à
lui en nommer d'autres, les plus dans son intimité ou dans sa liaison. À
chacune elle trouvait un cas à redire, que je combattais à mesure
vainement. Sur une entre autres tout à fait son intime et aussi
extrêmement de mes amies, elle me fit entendre qu'il y avait eu un court
espace de sa vie qui ne cadrait pas avec la garde d'une jeune princesse.
Je souris et je dis que par cela même elle y était plus propre\,; que
rien n'était plus rare qu'une femme aimable sans galanterie, mais qu'il
était si extraordinaire de n'en avoir eu qu'une seule en sa vie,
conduite modestement et finie sans retour ni rechute, que cela devait
tenir lieu d'un mérite fort singulier. M\textsuperscript{me} la duchesse
d'Orléans sourit à son tour\,; elle me répondit que rien n'était plus
avantageusement tourné pour cette dame que ce que je disais, mais qu'il
fallait que je lui avouasse aussi qu'une femme aimable qui n'avait
jamais eu ni galanterie ni soupçon était fort au-dessus de celle qui
n'en avait eu qu'une\,; que c'était une chose encore bien plus rare, et
que M\textsuperscript{me} de Saint-Simon était celle-là. Je convins de
cette vérité, mais je me rabattis tout court sur l'âge, le peu de
capacité à cet égard, et je continuai tout de suite à lui en nommer un
grand nombre, et elle de ne s'accommoder d'aucune. J'en pris occasion de
la trouver aussi trop difficile, et de lui dire que, pour soulager sa
mémoire et la mienne, je lui apporterais une liste de toutes les dames
titrées, parce que je comprenais bien qu'avec l'étrange exemple et si
nouveau des deux dames d'honneur de Madame on n'en voudrait point
d'autres pour une duchesse de Berry\,; que dans ce nombre il était
impossible qu'il ne s'en trouvât plusieurs très-convenables, et
qu'elle-même en demeurerait convaincue. Cela dit, je changeai tout de
suite de propos.

Le lendemain j'allai chez elle, ma liste en poche, résolu de lui bien
faire entendre que mes réponses n'étaient pas modestie, mais refus
absolu civilement tourné. Je trouvai chez elle un très-petit nombre de
compagnie très-familière\,; mais il fallait le tête-à-tête pour
reprendre les propos de la veille. Je tournai doucement la conversation
sur le grand-nombre de tabourets, je parvins naturellement à les faire
nommer, et, sous prétexte de soulager la mémoire, de tirer la liste de
ma poche disant, en regardant bien M\textsuperscript{me} la duchesse
d'Orléans, que je l'y avais oubliée depuis quelques jours que j'en avais
eu besoin. Je la lus et la remis dans ma poche après lui avoir ainsi
témoigné que je lui tenais promptement parole, autant que cela se
pouvait sans être seuls, résolu, après ce que je venais de faire, de ne
remettre pas ce propos le premier avec elle, qui devait bien entendre ce
que je pensais là-dessus, et qui ne l'entendit que de reste, mais qui
avait résolu de ne le pas entendre. Ce redoublement d'attaque si vive,
et si à découvert, me donna beaucoup d'inquiétude, et à
M\textsuperscript{me} de Saint-Simon encore plus. Elle et moi abhorrions
également une place si au-dessous de nous en naissance et en dignité\,;
et, bien que nous comprissions que l'orgueil royal n'y mettrait qu'une
femme assise, nous ne voulions pas au moins que ce ravalement portât sur
nous. Nous crûmes donc qu'il était à propos de prendre nos mesures de
bonne heure, moi de parler net au duc de Beauvilliers, et
M\textsuperscript{me} de Saint-Simon à M\textsuperscript{me} la duchesse
de Bourgogne, puisque d'en dire davantage à M\textsuperscript{me} la
duchesse d'Orléans, après ce qui s'était passé avec elle, n'arrêterait
ni ses désirs ni ses pas, et ne servirait au contraire qu'à la faire
agir à son insu et plus fortement.

Cette résolution prise, je fis souvenir le duc de Beauvilliers de ce que
je lui avais dit, il y avait deux ans, lorsqu'on crut, et non sans
quelque fondement, que M. le duc de Berry allait épouser la princesse
d'Angleterre. Je lui exposai ce qui s'était passé entre
M\textsuperscript{me} la duchesse d'Orléans et moi\,; je lui réitérai
mon éloignement et celui de M\textsuperscript{me} de Saint-Simon pour
une telle place. Je l'assurai que, si on venait jusqu'à nous la donner,
nous la refuserions\,; et je le conjurai d'en détourner la pensée si
ceux qui ont ou qui prennent droit de choisir venaient à l'avoir et
qu'elle lui fût communiquée, et il nous approuva et me le promit. De
retour à Versailles, nous contâmes notre fait au chancelier sous le
sceau de la confession. Il fut bien étonné que le mariage en fût là. Il
était aliéné de M. le duc d'Orléans par le tissu de sa vie, et plus
encore par son affaire d'Espagne. Il pensait d'ailleurs sainement sur un
mariage étranger, tellement qu'il me reprocha beaucoup d'avoir si
utilement travaillé, et il ne s'apaisa qu'à peine, lorsque je lui eus
fais sentir combien, sans ce mariage, celui de M\textsuperscript{lle} de
Bourbon était certain et imminent, fille comme Mademoiselle d'une
bâtarde, ce que, avec raison, il ne pouvait supporter. Il trouva que
nous pensions dignement de ne vouloir point de la place de dame
d'honneur et sagement de prendre là-dessus des mesures de bonne heure.

M\textsuperscript{me} la duchesse de Bourgogne continuait sans
interruption depuis bien des années de témoigner une amitié solide à
M\textsuperscript{me} de Saint-Simon, dont elle lui avait toujours donné
des marques. Le chancelier approuva fort que, les choses en cet état,
elle s'adressât à elle. M\textsuperscript{me} de Saint-Simon lui fît
donc demander une audience, de façon que cela fût ignoré, s'il était
possible, et, pour en mieux tenir le secret, elle se servit de
M\textsuperscript{me} Cantin, première femme de chambre, plutôt que des
dames du palais si fort de nos amies, dont nous voulûmes éviter la
curiosité. L'audience fut aussitôt accordée que demandée.
M\textsuperscript{me} de Saint-Simon se rendit chez
M\textsuperscript{me} la duchesse de Bourgogne à onze heures du matin,
comme elle sortait de son lit, qui à l'instant la fit entrer dans son
cabinet, et asseoir sur un petit lit de repos auprès d'elle. Après le
premier compliment, M\textsuperscript{me} de Saint-Simon lui dit
qu'étant toute sa ressource, et toujours sa ressource éprouvée, elle
venait à elle lui demander une grâce avec confiance, mais avec instance
de ne pas être refusée\,; qu'elle avait balancé longtemps, mais que la
chose pressant et ne pouvant craindre de manquer à la fidélité du
secret, puisqu'il s'agissait d'un mariage qu'elle-même désirait et
fai-soit.. À ces mots M\textsuperscript{me} la duchesse de Bourgogne
l'interrompit en l'embrassant avec empressement\,: «\,Le mariage de M.
le duc de Berry, dit-elle, et vous voulez être dame d'honneur\,? J'y ai
déjà pensé. Il faut que vous la soyez. C'est justement de ne la pas être
que je viens vous demander.\,»

À cette repartie, on ne peut rendre quel fut l'étonnement de
M\textsuperscript{me} la duchesse de Bourgogne. Après un moment de
silence, elle demanda la raison d'un éloignement qui la surprenait tant,
et lui dit à quel point elle en était étonnée. M\textsuperscript{me} de
Saint-Simon répondit que peut-être lui paraîtroit-il étrange qu'elle
prît ainsi des devants auprès d'elle sur une chose dont l'occasion
n'existait pas encore, et sur une place qu'elle serait plus éloignée que
personne de croire qu'on la lui voulût donner\,; que, pour l'occasion,
elle était instruite par moi, si avant dans l'affaire, de l'état si
prochain auquel elle se trouvait\,; que, sur la place, elle ne pouvait
pas douter que M\textsuperscript{me} la duchesse d'Orléans ne l'y
désirât, par tout ce qu'elle m'avait dit, dont elle lui conta le
détail\,; que, ne craignant donc point de parler à faux sur l'un ni sur
l'autre, ni de s'adresser mal sur tous les deux, elle venait à elle lui
demander à temps, et avec toute l'instance dont elle était capable, de
lui éviter une place dont je ne voulais point, et elle beaucoup moins
encore\,; que tout son désir était borné à une place de dame du palais
auprès d'elle\,; qu'elle avait tout son coeur et tout son respect\,;
qu'elle ne pouvait regarder une autre qu'elle, ni souffrir d'être mise
ailleurs\,; et que, si elle ne devenait point dame du palais, elle
serait contente et heureuse de demeurer à lui faire sa cour, pourvu
qu'elle n'eût point d'attachement ailleurs. M\textsuperscript{me} la
duchesse de Bourgogne lui fit là-dessus toutes les amitiés imaginables.
Elle lui dit ensuite que c'était par amitié pour elle et par intérêt
pour soi, comptant sur son attachement avec goût et confiance, qu'elle
avait aussitôt pensé à elle pour dame d'honneur dès qu'elle avait vu le
mariage en apparence de se faire\,; que cette belle-soeur, fille de M.
et de M\textsuperscript{me} d'Orléans, étant de sa main et de son choix,
elle comptait vivre beaucoup avec elle, par conséquent vivre beaucoup
avec celle qui sera sa dame d'honneur\,; avoir avec elle un particulier
de confiance nécessaire sur mille choses\,; qu'une dame d'honneur avec
qui elle ne serait pas fort libre la contraindrait donc beaucoup\,; et
qu'une sur l'attachement véritable de qui elle ne pourrait pas compter
l'embarrasserait continuellement\,; que dans cette vue elle avait jeté
les yeux sur elle, comme la seule convenable à cette place, qui eût ces
qualités à son égard\,; que, pour ce qui était de dame du palais, il
était vrai que cela ne lui conviendrait plus après avoir été dame
d'honneur de la duchesse de Berry\,; mais que la duchesse de Lude, déjà
si infirme, n'était point éternelle\,; qu'elle la pourrait très-bien et
très-dignement remplacer\,; qu'elle le souhaitait passionnément, et que
dans cette vue encore elle avait songé à la faire dame d'honneur de la
duchesse de Berry, pour ôter par là l'obstacle de sa jeunesse, et
l'approcher cependant du roi, s'il lui fallait bientôt à elle une autre
dame d'honneur.

Après les remercîments M\textsuperscript{me} de Saint-Simon répondit,
sur le fait de sa dame d'honneur, que l'autre place l'en élaignerait
plutôt que de l'en approcher\,; et sur ce que M\textsuperscript{me} la
duchesse de Bourgogne lui répliqua avec vivacité qu'elle voulait bien
qu'on sût, et la duchesse de Berry la première, quand il y en aurait
une, qu'une duchesse de Berry était faite pour lui céder ses dames quand
il lui plairait de les vouloir prendre, M\textsuperscript{me} de
Saint-Simon lui représenta que, si elle ne s'acquittait pas de l'emploi
suivant ce qu'on attendrait d'elle, ce serait une exclusion pour la
première place\,; que, si au contraire elle y réussissait, ce serait une
raison de l'y laisser\,; qu'ainsi à tous égards elle ne pouvait entrer
dans la pensée que cette place lui pût servir à en avoir une à laquelle
elle n'avait jamais osé songer, s'étant toujours bornée à lui être plus
particulièrement attachée par une place de dame du palais. Elle se
rabattit ensuite sur son incapacité, que M\textsuperscript{me} la
duchesse de Bourgogne releva par toutes sortes de marques d'estime, sur
quoi M\textsuperscript{me} de Saint-Simon lui représenta fortement la
différence extrême de se bien acquitter des fonctions de dame d'honneur
auprès d'elle, à qui à présent il n'était plus question de rien dire,
mais seulement de la bien faire servir ou de la suivre, ou auprès d'une
princesse de moins de quinze ans, dont il faudrait devenir la
gouvernante, et répondre de sa conduite à tant de différentes personnes
et au public\,; qu'elle ne se sentait ni force ni capacité pour bien
remplir tous ces différents devoirs, et moins encore d'humilité pour en
essuyer le blâme\,; que, si elle avait le bonheur de se conduire
elle-même au gré du monde, et d'une manière qu'on la jugeât capable d'en
bien conduire une autre, elle redoutait si fort le poids de cette
attente, que, trop contente de cette opinion qu'on voulait bien avoir
d'elle, elle s'en voulait tenir là sans hasard\,; que d'ailleurs, outre
son invincible répugnance à gouverner et à contredire, comme il ne se
pouvait éviter qu'il n'y eût bien des choses à dire et à faire faire à
un enfant contre son goût, elle ne pouvait se résoudre à passer pour
sotte si elle ne faisait faire ce qu'il faudrait, ni en le faisant à
devenir la bête de la princesse auprès de qui elle serait mise\,; et
qu'elle ne s'y résoudrait jamais. M\textsuperscript{me} la duchesse de
Bourgogne n'oublia rien, avec tout l'art et l'esprit possible, pour
combattre ces raisons\,; et finalement lui dit que Mademoiselle ayant
père et mère et grand'mère à la cour, et elle par-dessus eux tous, ils
se chargeraient de sa conduite et de celle de sa dame d'honneur.

Après quantité de raisonnements, M\textsuperscript{me} de Saint-Simon
tint ferme. Elle lui avoua qu'elle craignait encore que, si en exécutant
ses ordres sur la conduite de la princesse et sur la sienne elle se
brouillait avec la future duchesse de Berry, elle n'en fût que peu
approuvée\,; et que la princesse ensuite en faisant mieux, ou tournant
les choses d'une autre façon auprès d'elle, elle-même enfin par douceur,
par complaisance, par amitié pour M\textsuperscript{me} sa belle-soeur,
ne vint à la blâmer elle-même, et à se refroidir d'estime et de bonté
pour elle. À cette nouvelle difficulté M\textsuperscript{me} la duchesse
de Bourgogne opposa les protestations d'un côté, les reproches de
l'autre, de la croire capable de cette faiblesse et de cette légèreté.
Après avoir fort insisté là-dessus, elle mit le doigt plus
particulièrement sur la lettre, et lit entendre qu'elle comprenait bien
qu'elle et moi trouvions cette place au-dessous de nous, sur quoi
M\textsuperscript{me} de Saint-Simon s'étant modestement et brèvement
étendue\footnote{Nous avons reproduit fidèlement le texte du manuscrit,
  quoique le mot \emph{étendue} paraisse inexact\,; les précédents
  éditeurs l'ont remplacé par \emph{défendue}.}, M\textsuperscript{me}
la duchesse de Bourgogne lui dit qu'il fallait vivre selon les temps\,;
que d'ailleurs, quant à présent, elle et sa belle-soeur seraient égales
en rang et en toutes choses. Après avoir quelque temps battu là-dessus,
et avoué pourtant ce que M\textsuperscript{me} de Saint-Simon en avait
dit, elle lui représenta notre situation à la cour, les ennemis que j'y
avais, les espèces de disgrâces que j'avais essuyées, avec combien de
temps et de peine j'en étais sorti\,; que cette place y serait un
puissant bouclier, un chemin facile de me mieux faire connaître, d'être
plus approché du roi, assuré des agréments de toutes les sortes. Elle
s'étendit fort sur ces avantages qu'un homme de mon esprit pouvait
solidement pousser. Elle ajouta qu'il était flatteur que j'eusse été
choisi pour l'ambassade de Rome, à l'âge où j'étais lors de ce choix\,;
et elle, au sien, regardée comme si convenable à la place dont il était
à présent question, la première de la cour à remplir, puisque celle
d'auprès d'elle n'était point vacante\,; qu'on savait si fort que nous
ne voulions point aller à Rome\,; que cela, joint avec notre dégoût pour
la place de dame d'honneur dont il s'agissait, irriterait le roi, et lui
ferait demander avec justice ce que nous voulions donc, puisque les deux
premiers et plus considérables emplois pour homme et pour femme, enviés
et désirés de toute la cour en tout âge, ne nous semblaient pas
convenables à nous, et dans la situation d'âge et de fortune où nous
nous trouvions\,; que l'envie et la jalousie du monde en crierait encore
bien plus haut contre nous\,; qu'enfin, nous nous avisassions bien, et
que nous comprissions qu'un refus ne se pardonnerait point et nous
romprait le cou pour jamais. Après les remercîments proportionnés à la
bonté avec laquelle elle entrait dans toute la discussion de cette
affaire, M\textsuperscript{me} de Saint-Simon convint qu'un refus nous
noierait en effet sans retour\,; que c'était aussi pour l'éviter qu'elle
s'adressait à elle, puisque enfin nous étions fermement résolus au refus
si les choses en venaient là. Après quelques autres propos plus
généraux, M\textsuperscript{me} la duchesse de Bourgogne lui dit qu'elle
ne voyait point d'autre dame d'honneur à faire qui convînt,
non-seulement à elle, mais à la place\,; et le tout, après beaucoup
d'amitiés, se termina à promettre enfin à M\textsuperscript{me} de
Saint-Simon qu'elle tâcherait d'empêcher que la place lui fût offerte,
puisque elle et moi étions si obstinés au refus, qu'elle ne comprenait
pas\,; que, néanmoins, il pourrait bien arriver que la chose se ferait
sans elle, ou que, se faisant avec elle, elle ne serait pas maîtresse de
rien empêcher, mais qu'elle promettait de bonne foi d'y faire tout son
possible, encore que ce fût contre son goût et contre son sentiment.

Après une audience si favorable et si longue, car elle dura jusques
après midi et demi, M\textsuperscript{me} de Saint-Simon sortit du
cabinet, et trouva la toilette dans la chambre, et des dames qui
attendaient à y faire leur cour, dont elle fut bien fâchée, surtout des
dames du palais de nos amies intimes qui s'y trouvèrent\,; mais nous
tînmes bon au secret, qui par sa nature n'était pas le nôtre.
M\textsuperscript{me} de Saint-Simon me fit le récit de son audience, de
laquelle nous fûmes bien contents, persuadés tous deux que
M\textsuperscript{me} la duchesse de Bourgogne arrêterait
M\textsuperscript{me} la duchesse d'Orléans\,; que le choix ne se ferait
pas sans la première\,; que sûre comme elle était, et ayant donné sa
parole, elle la voudrait et la pourrait tenir\,; tellement que dans une
pleine et juste espérance d'avoir de toutes parts écarté le danger, et
les princes n'ayant pas accoutumé de prendre les gens par force pour des
places après lesquelles tant d'autres ne sont pas honteux de soupirer,
même en public, nous comptâmes en être en repos. Nous n'avions en effet
oublié aucune des voies possibles de détourner cette place de nous,
aucun des meilleurs, des plus forts, des plus directs moyens à pouvoir
employer d'avance, mais à temps\,; ainsi rendu au calme et à la liberté
d'esprit, je me rendis aussi aux soins de ne pas laisser refroidir ce
qui avait été si bien reçu sur le mariage, ni les mouvements, tous si
justes et si bien ensemble de notre part, pour le brusquer, tandis que
M\textsuperscript{me} la Duchesse et les siens, si sûrs de Monseigneur
et si peu avertis de nos menées, vivaient dans une parfaite sécurité.

Dès la fin du voyage de Marly l'embarras du roi sur Monseigneur, grand
et de bonne foi, nous avait fort embarrassés nous-mêmes. Il s'agissait
d'un mariage pour le fils de Monseigneur, d'un mariage domestique et
particulier, où le bien de la paix, ni l'honneur ou l'avantage de l'État
n'avaient aucune part, conséquemment où un père de cinquante ans devait
en avoir davantage. On a vu combien personnellement il était éloigné de
vouloir du bien à M. le duc d'Orléans, et à quel point il était livré à
ceux dont le double intérêt était d'entretenir et d'augmenter cette
aversion, et quel était ce double intérêt. Maintenant il faut dire que
M\textsuperscript{me} la duchesse d'Orléans n'était guère mieux avec lui
de son propre chef, avec cette différence de M. son mari que c'était par
sa pure faute, et par ces sortes de fautes qui se font le plus sentir\,:
c'est ce qu'il faut expliquer. Monseigneur, très-refroidi avec
M\textsuperscript{me} la princesse de Conti, dont l'ennui et l'aigreur
le mettait en continuel malaise, ne savait que devenir, parce que ces
princes-là, et lui plus que pas un, n'ont pire lieu à se tenir que chez
eux. D'Antin, je parle de loin, qui avait peut-être meilleure opinion de
M\textsuperscript{me} la duchesse d'Orléans que de M\textsuperscript{me}
la Duchesse, voulut le tourner vers la première\,; et, dans l'espérance
de recueillir de sa connaissance les fruits d'une liaison si avantageuse
pour elle, il n'oublia rien pour la former.

Monseigneur ne pouvait guère se délivrer, du réduit continuel qu'il
s'était fait chez M\textsuperscript{me} la princesse de Conti depuis
tant d'années, dont l'affaire de M\textsuperscript{lle} Choin venait de
bannir toute la confiance et la douceur, qu'en se faisant un autre
réduit chez une des deux autres bâtardes, et il ne lui importait pour
lors chez laquelle des deux, moins conduit en tout par son choix que par
le hasard ou par l'impulsion d'autrui. M\textsuperscript{me} la duchesse
d'Orléans, qui aurait dû être charmée d'une si heureuse conjoncture,
ivre de sa grandeur et de sa paresse de corps et d'esprit, ne vit que de
l'ennui, des complaisances, des amusements à donner, des mouvements de
corps à essuyer pour des parties de chasse, d'Opéra et de petits
voyages. Elle devenait non plus la divinité qu'on allait adorer, mais la
prêtresse d'une divinité supérieure dont sa maison deviendrait le
temple. Son orgueil ne put s'y ployer, peut-être moins que sa paresse.
Son dédain ferma son esprit à toute politique, et à toute vue d'un futur
que l'âge et la santé du roi montraient fort éloigné\,; point d'enfants
à établir\,; au-dessous d'elle dépenser aux besoins de l'avenir. Elle
fut sourde aux cris de d'Antin et si froide aux avances réitérées de
Monseigneur, {[}elle mit{]} tant de langueur et de négligence à le
recevoir chez elle, qu'il s'en aperçut bientôt avec un dépit qu'il
n'oublia jamais, et se livra à M\textsuperscript{me} la Duchesse, qui le
reçut avec les grâces, les jeux et les ris, et qui ne songea qu'à
profiter d'une si bonne fortune par se lier Monseigneur de façon qu'elle
se rendît en tout maîtresse de son temps et de son esprit, et y réussit
de la manière la plus complète. Ainsi M\textsuperscript{me} la duchesse
d'Orléans fit à sa soeur, avec qui alors elle n'était point encore mal,
un présent volontaire de l'intimité parfaite qui se lia entre
Monseigneur et elle, ouvrit la porte à son triomphe et à tout ce qui en
sortit après contre elle, en se la fermant à elle-même, et croupit
longues années sur son canapé, non-seulement sans regret d'une faute si
démesurée, mais avec un orgueilleux et dédaigneux gré de l'avoir
commise. Il ne tint encore après qu'à elle de se rapprocher de
Monseigneur, chez M\textsuperscript{me} la Duchesse, où, à son refus,
d'Antin l'avait tout à fait jeté\,; mais les mêmes misérables raisons
qui l'avaient empêchée de le vouloir chez elle, quelque dépit aussi de
voir sa soeur en profiter, l'en détournèrent encore. L'éloignement, puis
l'inimitié des deux soeurs vint ensuite, et se combla enfin par les
occasions qui naquirent et dont j'ai touché quelques-unes, et où
Monseigneur, tout à sa façon pesante et indolente, ne fut pas tout à
fait neutre. Ainsi, M\textsuperscript{me} la duchesse d'Orléans se vit
réduite à continuer, par raison et par nécessité, ce qu'{[}un sentiment
qu'{]}on ne peut s'empêcher de nommer folie lui avait fait commencer.
Dans cette situation de M. {[}le duc{]} et de M\textsuperscript{me} la
duchesse d'Orléans, chacune à part et ensemble, si fâcheuse avec
Monseigneur, je ne cessai de pourpenser, à part moi, quels pourraient
être les moyens d'émousser dans ce prince tant de pointes hérissées, et
de le rendre capable de se ployer volontairement au mariage de la fille
de deux personnes dont il était si fortement aliéné. Je sentis l'extrême
danger de démarches, qui d'elles-mêmes avertiraient la cabale contraire
de penser à soi et à M\textsuperscript{lle} de Bourbon. D'autre part,
l'affaire commençait imperceptiblement à pointer par tous les mouvements
qui ne s'étaient pu cacher à Marly\,; et je fus bien étonné que,
rencontrant le premier écuyer, avec qui j'étais fort libre, dans la
porte de l'antichambre du roi, dans la galerie, une après-dînée qu'il
n'y avait personne, il m'arrêta, me dit qu'il y avait des compliments à
me faire, et qu'on savait bien que je faisais le mariage de Mademoiselle
avec M. le duc de Berry. J'en sortis par hausser les épaules, couper
court et admirer les beaux bruits. De bruits, il n'y en avait pas le
moindre\,; c'était transpiration à un homme toujours fort informé, que
j'eus grand'peur qui ne perçât plus loin, qui nous fut un nouveau motif
de serrer la mesure. Je ne pus me persuader que le roi bâclât l'affaire,
de lui à Monseigneur, avec tant d'autorité et si court que
M\textsuperscript{me} la Duchesse et les siens n'eussent le temps de se
tourner\,; et je ne trouvais pas, sans de pernicieux écueils, la manière
de marier le fils de Monseigneur malgré un tel père, si ce père, aigri
de lui-même et violemment poussé par tous ceux qui pouvaient tout sur
lui, augmentait tacitement son ressentiment par un consentement forcé,
et je n'étais pas à dire mon avis avec colère à M\textsuperscript{me} la
duchesse d'Orléans, sur sa conduite à l'égard de Monseigneur, et sa
manière conséquente d'être avec lui, qu'elle-même m'avait racontée.

Venant, à part moi, à l'examen des personnages de la cabale opposée,
pour voir à en détacher quelqu'un qui pût nous servir puissamment auprès
de Monseigneur, je considérai que d'Antin, si intimement uni à
M\textsuperscript{me} la Duchesse par tant de liens anciens et nouveaux,
et par une si grande conformité de vie, de moeurs et d'esprit, n'était
pas l'instrument qu'il nous fallait\,; M\textsuperscript{me} de
Lislebonne et sa soeur encore moins, avec tout ce qu'on a vu ici plus
d'une fois de leurs vues, de leurs hautes menées et de leurs vastes
projets. Enfin je ne vis de ressource, s'il y en pouvait avoir, qu'en
M\textsuperscript{lle} Choin, qui eût assez de pouvoir sur Monseigneur,
et assez d'indépendance de la cabale et de M\textsuperscript{me} la
Duchesse même, pour oser entreprendre, si elle le voulait, de le rendre
plus accessible au mariage de Mademoiselle\,; et je crus qu'il ne serait
peut-être pas impossible de le lui faire vouloir, en lui faisant sentir
qu'il y allait de son intérêt\,; je conçus donc le dessein de traiter
cette matière en la tâtant d'abord, puis en l'approfondissant plus ou
moins, selon que j'y verrais jour, mais sans m'ouvrir du tout sur le
mariage, avec Bignon, intendant des finances, le plus intime ami et
confident qu'eût la Choin, et fort le mien, duquel je m'étais déjà servi
utilement en contre-poison auprès d'elle, et par elle auprès de
Monseigneur, lorsque l'affaire de M\textsuperscript{me} de Lussan me
brouilla avec M\textsuperscript{me} la Duchesse.

Je proposai ce dessein à M\textsuperscript{me} la duchesse d'Orléans,
qui le goûta fort, à M. le duc d'Orléans ensuite, qui l'approuva aussi.
Tous deux le discutèrent, puis moi avec eux. Ils jugèrent qu'à tout le
moins, la tentative n'était qu'honnête et respectueuse de leur part\,;
qu'il n'y avait rien à risquer en s'y conduisant sagement\,; que le
temps pressait. Ils me donnèrent donc toute commission de parler en leur
nom. Ainsi je vis Bignon dans cette chambre que le chancelier son oncle
m'avait forcé de prendre chez lui au château, et là, tête à tête, je
l'entretins des brigues et des cabales qui partageaient la cour. Je le
mis sur celles de la cour intime de Monseigneur. Comme de moi à lui, je
lui parlai sur le peu de retour que M. le duc d'Orléans sentait avec
tant de peine de Monseigneur à lui\,; je lui vantai en même temps celui
du roi et celui de M\textsuperscript{me} de Maintenon vers lui qui
devenait tous les jours plus intime. Je lui dis que M. et
M\textsuperscript{me} la duchesse d'Orléans avaient une estime infinie
pour M\textsuperscript{lle} Choin\,; qu'il était vrai que leur respect
pour Monseigneur y entrait bien pour quelque chose, mais qu'il était
vrai aussi que tout ce qui paraissait et revenait de la conduite si
sage, si mesurée si unie de cette-personne, la manière si soumise et si
intime avec laquelle elle entretenait Monseigneur avec le roi, donnait
d'elle une haute opinion, et allumait M. {[}le duc{]} et
M\textsuperscript{me} la duchesse d'Orléans un désir sincère de la voir
et de devenir de ses amis\,; que je savais combien soigneusement elle
évitait l'éclat et le monde, mais que, ayant bien voulu lier dans les
derniers temps avec feu M. le prince de Conti, quoiqu'il en fût à
l'extérieur si mal avec le roi, et de plus si bien avec
M\textsuperscript{me} sa belle-soeur, il serait encore plus convenable
que M\textsuperscript{lle} Choin voulût bien lier avec M. {[}le duc{]}
et M\textsuperscript{me} la duchesse d'Orléans, maintenant si unis et si
bien avec le roi et avec M\textsuperscript{me} de Maintenon, avec
laquelle elle était si bien elle-même. Bignon me répondit, en tâtant,
par les mesures infinies d'obscurité et de dégagement, que
M\textsuperscript{lle} Choin gardait. Je n'étais pas à savoir en combien
de choses elle entrait, avec quelle liberté elle tenait chez elle, en sa
petite maison de la rue Saint-Antoine, cour plénière de ce qu'il y avait
de plus important, où n'était pas admis qui voulait, mais par goût et
par choix des personnes, et non par crainte d'en trop voir\,; mais ce
n'était pas le cas de disputer et de contredire. J'entrai donc avec
docilité dans ce qu'il voulut, pour ne pas choquer un esprit plein et
médiocre au plus, duquel seul je pouvais me faire un instrument. Par
cette méthode, je le conduisis peu à peu à l'aveu de diverses choses, et
singulièrement à la part entière que cette fille avait eue en tout ce
que Monseigneur avait fait auprès du roi contre Chamillart, sans quoi,
me dit-il, ce ministre n'eût jamais été chassé de sa place. De ce que
Bignon me dit qu'elle s'était conduite de la sorte de concert avec
M\textsuperscript{me} de Maintenon, j'en pris thèse pour lui représenter
que M\textsuperscript{me} de Maintenon aimait tendrement
M\textsuperscript{me} la duchesse d'Orléans, et protégeant sincèrement
M. le duc d'Orléans à cette heure, rien ne serait plus convenable à
M\textsuperscript{lle} Choin que de se prêter aux désirs d'amitié dont
M\textsuperscript{me} de Maintenon lui donnait l'exemple après le roi
même si parfaitement revenu sur son neveu\,; que j'irais plus loin avec
lui, qui était mon ami, en faveur de son amie\,; que je pouvais
l'assurer qu'en cela elle ferait chose agréable au roi, et qui le serait
infiniment à M\textsuperscript{me} de Maintenon, et que, pour n'avoir
nulle réserve avec lui, je ne balancerais pas à épuiser la matière. Je
lui dis donc que l'union présente de M\textsuperscript{lle} Choin avec
M\textsuperscript{me} la Duchesse, et celle de toutes deux avec
M\textsuperscript{lle} de Lislebonne et M\textsuperscript{me} d'Espinoy
et tout ce côté-là, n'était qu'apparente et ne pouvait subsister au delà
du règne sous lequel nous étions. Que tous ensemble aspiraient à
gouverner un prince qui, n'étant que Dauphin, les faisait tous compatir,
dans la vue de se soutenir et de ne se commettre point à une lutte
prématurée, mais qui éclaterait à l'instant que ce prince devenu roi,
chacun alors voudrait saisir le timon. Je m'étendis ensuite à mon gré
sur les deux Lorraines, tant pour le pomper que pour lui en donner, et
par lui à son amie, les plus sinistres pensées, qui néanmoins étaient
vraies et solides, et radicalement telles. Je n'eus pas été bien loin
là-dessus qu'il sourit et me dit que, pour celles-là,
M\textsuperscript{lle} Choin les con-noissoit bien à peu près telles que
je les lui dépeignais\,; qu'elle vivait avec elles avec tous les dehors
d'amitié et de cette liaison ancienne qu'il n'était pas à propos de
rompre, mais que, bien convaincue des retours qu'elle en devait attendre
en leur temps, il y en avait déjà beaucoup qu'il n'y avait plus de
confiance réelle, et que son amie se précautionnait\,; qu'ainsi il était
inutile de lui en dire là-dessus davantage, puisqu'il les connaissoit
bien, et son amie encore mieux, et dans le sens dont je lui en parlais.

Dilaté à l'extrême en moi-même sur un si précieux chapitre, et sûr d'un
sentiment si important, quoique j'en eusse déjà soupçonné quelque chose
par l'évêque de Laon, frère de Clermont, perdu pour la Choin lorsqu'elle
fut chassée par M\textsuperscript{me} la princesse de Conti. Je tournai
court où j'en voulais\,; je me mis sur M\textsuperscript{me} la
Duchesse, mais avec mesure, pour ne pas décréditer par une apparence de
haine ce que je voulais persuader. Je pris le même tour que j'avais pris
sur les deux Lorraines et avec la même vérité\,; je dis que Monseigneur,
devenu roi, allumerait dans le courage de M\textsuperscript{me} la
Duchesse une telle volonté de gouverner seule, et si violente que depuis
longtemps elle se mettait en tout devoir de pouvoir satisfaire et qui
commençait bien déjà à transpirer que vainement M\textsuperscript{lle}
Choin prétendrait-elle pouvoir modérer autrement qu'en lui en ôtant les
moyens\,; que par une familiarité et un empire que de jour en jour elle
acquérait plus grands sur Monseigneur, joints aux avantages de son rang,
elle se rendrait très-dangereuse à M\textsuperscript{lle} Choin, quelle
que pût être cette fille à l'égard de Monseigneur\,; qu'il pouvait se
souvenir de ce que je lui avais dit, il y avait déjà longtemps, de
l'attaque contre elle faite à Monseigneur avec tant d'audace, quoique
avec peu de succès, qui manifestait bien ses plus secrètes pensées\,; et
que M\textsuperscript{lle} Choin avait et aurait en elle la plus
redoutable ennemie qu'elle trouverait jamais\,; que le grand intérêt de
gouverner seule lui rendrait telle, quelques mesures qu'elle prît et
qu'elle crût prendre avec elle. Bignon se souvint très-bien du fait que
je lui avais raconté autrefois, et qu'il me dit alors avoir rendu à son
amie, sur laquelle il avait fait impression dans ce temps-là. Mais il me
dit cependant que M\textsuperscript{lle} Choin comptait absolument sur
l'amitié de M\textsuperscript{me} la Duchesse, dont elle croyait pouvoir
ne pas douter\,; qu'il ne croyait pas lui-même qu'elle s'y trompât, ni
qu'elle en pût être séparée, convenant cependant avec moi de la solidité
de ce que je lui représentais de tant de volonté et tant d'avantages
dans cette volonté de gouverner absolument, et par conséquent seule,
dans M\textsuperscript{me} la Duchesse, aussitôt que la couronne
tomberait sur la tête de Monseigneur\,; et dans cet aveu je crus
entrevoir que le coeur de M\textsuperscript{lle} Choin avait moins de
part à cette liaison intime avec M\textsuperscript{me} la Duchesse que
l'esprit, qui sentant l'attachement incroyable de Monseigneur pour cette
soeur, que Bignon me releva beaucoup, ne croyait pas qu'il fût sûr pour
elle de lui laisser naître aucun soupçon sur leur intimité,
M\textsuperscript{me} la Duchesse toujours présente, elle presque
toujours absente, et M\textsuperscript{me} la princesse de Conti encore
palpitante par des restes de bienséance et de considération. Que ce fût
coeur ou esprit qui produisît dans la Choin cette union intrinsèque avec
M\textsuperscript{me} la Duchesse, ce n'était pas chose aisée à
nettement pénétrer\,; et bien que cette alternative ne me pût être
indifférente pour des temps éloignés, c'était pour l'objet présent la
même chose\,; et dans le fond, désirs à part, quelques raisons qu'eût
M\textsuperscript{lle} Choin de craindre M\textsuperscript{me} la
Duchesse, tout sens, toute sagesse, toute raison était pour qu'elle la
ménageât si parfaitement, dans la position où elles se trouvaient l'une
et l'autre, qu'elle lui ôtât tout son soupçon de défiance et de
jalousie, ce qu'elle ne pouvait avec une personne d'autant d'esprit et
d'application que l'était celle-là, avec l'air futile de ne songer qu'à
s'amuser et à se divertir elle et les autres, que par un entier abandon
à elle pour le temps présent, qui était justement ce que je voulais
tâcher d'ébranler. Dans ce dessein je continuai à m'étendre sur tout le
danger de la puissance de M\textsuperscript{me} la Duchesse, sur son peu
de coeur, de foi, de principes en tout genre, et en louanges sur la
conduite de M\textsuperscript{lle} Choin avec elle\,; mais je remontrai
à Bignon qu'au milieu de tout cela un abandon effectif à elle serait le
comble de l'imprudence\,; qu'il me paraissait que, sans offenser
M\textsuperscript{me} la Duchesse, elle pouvait entendre à quelque
liaison avec M. {[}le duc{]} et M\textsuperscript{me} la duchesse
d'Orléans, d'autant plus sûrement que ni à présent, pour l'amitié de
Monseigneur, ni dans d'autres temps pour le timon de toutes choses, elle
n'aurait point à lutter avec eux\,; qu'il pouvait arriver des
conjonctures où cette liaison lui deviendrait utile à elle-même\,;
qu'elle était bien avec Mgr et M\textsuperscript{me} la duchesse de
Bourgogne, ce qui lui serait toujours important à ménager, quelle
qu'elle fût, et à bien ménager de plus en plus\,; qu'elle savait à quel
point M\textsuperscript{me} la duchesse de Bourgogne et
M\textsuperscript{me} la Duchesse étaient mal ensemble, et y devaient
être, et au contraire l'union étroite qui liait M\textsuperscript{me} la
duchesse de Bourgogne et M\textsuperscript{me} la duchesse d'Orléans\,;
que, quoi qu'il arrivât, c'était là ce qui environnait le trône le plus
près\,; que M. le duc d'Orléans était le seul homme du sang royal en âge
et en expérience de figurer, qui, écarté de Monseigneur par les
artifices de M\textsuperscript{me} la Duchesse, trouverait tôt ou tard
dans sa naissance, dans son état d'homme connu pour en être un, dans sa
liaison avec Mgr et M\textsuperscript{me} la duchesse de Bourgogne, des
ressources pour se rapprocher de Monseigneur\,; que les choses étant
donc en effet telles que je les lui représentais, il ne pouvait nier
qu'il ne fût au moins sûr et honnête pour M\textsuperscript{lle} Choin,
et même bon, de se laisser approcher par M. {[}le duc{]} et
M\textsuperscript{me} la duchesse d'Orléans, qui, pour le dire encore
une fois, était lui, si bien avec le roi, si intimement avec Mgr et
M\textsuperscript{me} la duchesse de Bourgogne, si recueilli de
M\textsuperscript{me} de Maintenon, avec qui M\textsuperscript{lle}
Choin était si bien elle-même, d'entrer au moins en connaissance avec
des personnes de cet état, qui ne pouvaient en aucun temps lui faire
d'ombrage, quitte après pour lier plus ou moins avec eux, selon qu'elle
s'en accommoderait et le jugerait à propos.

Bignon trouva si fort que je lui parlais raison, qu'il entra en
discussion avec moi du personnel de M. {[}le duc{]} et de
M\textsuperscript{me} la duchesse d'Orléans. Imbu par les sarbacanes
ennemies, il ne me cacha pas que M\textsuperscript{lle} Choin craignait
M. le duc d'Orléans, et en pensait d'ailleurs peu favorablement. Je lui
répondis là-dessus avec une sorte d'ouverture qui lui plut, et qui, sans
blesser ce prince, donna plus de confiance au reste de mes propos.
Ensuite je lui dis que je comprenais que M\textsuperscript{lle} Choin
pouvait être peinée de la liaison qui avait paru si longtemps entre M.
le duc d'Orléans et M\textsuperscript{me} la princesse de Conti, mais
que je lui disais avec vérité que depuis longtemps aussi un reste
d'honnêteté et de bienséance avait succédé à une amitié plus étroite\,;
qu'il devait comprendre que, outre l'aliénation produite par la querelle
du rang de Mademoiselle, le pauvreteux personnage que
M\textsuperscript{me} la princesse de Conti faisait auprès de
M\textsuperscript{me} la Duchesse avait extrêmement refroidi M. le duc
d'Orléans\,; que, même au dernier voyage de Marly d'où nous arrivions,
M. le duc d'Orléans, étant entré chez M\textsuperscript{me} la princesse
de Conti, l'avait extrêmement déconcertée pour l'avoir trouvée tête à
tête avec M\textsuperscript{me} de Bouzols, si intime de
M\textsuperscript{me} la Duchesse, et fort proche l'une de l'autre,
écrivant sur une table et comme en conférence importante\,; qu'après le
premier trouble, M\textsuperscript{me} la princesse de Conti l'avait
excusé en disant qu'elle faisait une réponse au prince de Monaco qui lui
avait écrit sur la mort de M. le Duc, et à qui elle n'avait pas encore
fait réponse, sans que M. le duc d'Orléans lui eût fait aucune question,
et sans aucune apparence, depuis deux mois de la mort de M. le Duc, ni
que M\textsuperscript{me} de Bouzols eût aucune liaison avec M. de
Monaco. Bignon fit assez d'attention à cette bagatelle, que le hasard
m'avait à propos fournie, pour me faire espérer que cette amitié
apparente blessait son amie plus que toute autre chose\,; mais après
m'avoir toujours rebattu sa crainte du caractère de M. le duc d'Orléans,
nous parlâmes fort à fond de celui de M\textsuperscript{me} la duchesse
d'Orléans, pour laquelle il me dit que M\textsuperscript{lle} Choin
n'avait que de l'estime, et puis nous traitâmes de la manière de se voir
qui pour cette princesse ne laissait pas d'être une difficulté. Je la
levai bientôt en l'assurant qu'elle irait à Paris dès que
M\textsuperscript{lle} Choin voudrait, et que toutes deux en même ville
conviendraient bientôt d'un lieu pour se voir. Enfin Bignon me dit que,
quelque éloignement qu'il eût de se mêler d'aucune affaire avec son
amie, qui même n'en avait pas moins aussi et le lui avait souvent
témoigné, tout ce que je lui disais lui paraissait si bon, si peu
engageant pour elle, si utile à la concorde et {[}à{]} l'union de toutes
les personnes principales, et si raisonnable en soi, qu'il se chargerait
volontiers d'en rendre compte à son amie, à deux conditions\,: la
première, qu'il me nommerait à elle, pour donner, me dit-il, plus de
poids à son discours, et ne lui point faire de mauvaise finesse\,; la
deuxième que M. {[}le duc{]} et M\textsuperscript{me} la duchesse
d'Orléans, qu'il sentit bien qui me faisait agir, lorsqu'il les verrait
en leur faisant sa cour ou ailleurs, ignoreraient jusque avec lui-même
qu'il entrât en rien de tout cela. Je lui permis l'un et lui promis
l'autre, après quoi il m'assura qu'il ferait incessamment tout son
possible pour persuader son amie de voir au moins M\textsuperscript{me}
la duchesse d'Orléans, mais que, Monseigneur allant ce jour-là à Meudon,
où il devait demeurer huit ou dix jours, il ne pourrait sitôt voir son
amie. Il convint avec moi qu'aussitôt qu'il l'aurait vue, il m'enverrait
prier à dîner pour éviter jusqu'aux apparences de rendez-vous, et que je
n'y manquerais pas, pour savoir la réponse.

Après un entretien si long, si confident, si fort approfondi, je conçus
quelques espérances, et M. {[}le duc{]} et M\textsuperscript{me} la
duchesse d'Orléans encore plus. Ce ne fut pas sans admirer ensemble en
quelle réduction on vivait, et la singularité non jamais assez admirée
de ce besoin général de tout le monde, et des plus proches du trône, de
passer par M\textsuperscript{me} de Maintenon pour aller au roi, et par
M\textsuperscript{lle} Choin pour aller jusqu'à Monseigneur, et cela en
même temps avec l'humilité des avances d'une part, l'orgueil des
réserves de l'autre, et la nécessité avec l'incertitude d'une secrète et
difficile négociation pour, à toute condition, obtenir audience, et que
deux créatures de si vil aloi voulussent bien prêter chez elles quelques
précieux moments aux désirs empressés et réitérés de ce qu'il y avait de
plus important, de plus grand et de plus proche de la couronne. Je ne
voulus pas effaroucher M. le duc d'Orléans de l'éloignement que cette
Choin avait pris de lui, mais je le confiai à M\textsuperscript{me} la
duchesse d'Orléans.

\hypertarget{chapitre-xiii.}{%
\chapter{CHAPITRE XIII.}\label{chapitre-xiii.}}

1710

~

{\textsc{Le roi résolu au mariage.}} {\textsc{- Contre-temps de
M\textsuperscript{me} la duchesse d'Orléans adroitement réparé.}}
{\textsc{- M. {[}le duc{]} et M\textsuperscript{me} la duchesse
d'Orléans éconduits entièrement de tout commerce avec
M\textsuperscript{lle} Choin.}} {\textsc{- Conférence à Saint-Cloud.}}
{\textsc{- Horreurs semées sur M. le duc d'Orléans et Mademoiselle.}}
{\textsc{- Le roi fait consentir Monseigneur au mariage.}} {\textsc{-
M\textsuperscript{me} la Duchesse, etc., en émoi.}} {\textsc{-
Déclaration du mariage.}} {\textsc{- Souplesse de d'Antin. M. {[}le
duc{]} et M\textsuperscript{me} la duchesse d'Orléans très-bien reçus de
Monseigneur, et fort mal de M\textsuperscript{me} la Duchesse.}}

~

Cependant l'affaire traînait trop à mon gré. Je n'avais pas compté de
détacher M\textsuperscript{lle} Choin de M\textsuperscript{me} la
duchesse, aussi peu, dans l'inespérable cas que ce détachement se fît,
que ce fût assez promptement pour en faire un instrument en faveur de
Mademoiselle. Mon but n'avait été que d'émousser l'intimité, de jeter
des craintes et des nuages qui s'augmentassent avec du soin et du temps,
et cependant de la rendre moins empressée pour le mariage de
M\textsuperscript{lle} de Bourbon. Au bout de sept ou huit jours que
nous fûmes revenus de Marly, je pressai M. le duc d'Orléans de parler au
roi, au moins en monosyllabes, de la lettre qu'il lui avait donnée.
Après bien des instances il le fit un matin. Ce même matin, comme
j'étais dans la petite chambre de M\textsuperscript{me} la duchesse
d'Orléans seul avec elle, M. le duc d'Orléans y entra, venant de chez le
roi\,; il nous conta tout joyeux qu'aussitôt qu'il lui avait ouvert la
bouche, le roi, en l'interrompant, lui avait répondu que sa lettre
l'avait entièrement persuadé de ses bonnes raisons, et de lui donner
toute satisfaction\,; qu'il comptât qu'il voulait faire le mariage de sa
fille avec son petit-fils\,; qu'il en était encore à trouver l'occasion
d'en parler comme il fallait à Monseigneur, parce qu'il prévoyait sa
résistance et qu'il la voulait vaincre en toutes façons\,; qu'il sentait
bien aussi que les retardements ne feraient qu'augmenter l'obstacle,
mais qu'il le laissât faire et qu'il ne se mît point en peine, et qu'il
ferait bien et bientôt.

Une si favorable réponse et si décisive nous combla de joie. Nous
conclûmes que, pour engager le roi de plus en plus sans l'importuner en
l'excitant davantage, M\textsuperscript{me} la duchesse d'Orléans se
trouverait le soir de ce même jour chez M\textsuperscript{me} de
Maintenon lorsque le roi y entrerait, où il n'y avait presque personne
de contrebande pour elle\,; que là elle le remercierait comme d'une
chose faite de ce qu'il avait dit le matin à M. le duc d'Orléans. Comme
elle n'avait pas accoutumé d'y aller sans affaire, M\textsuperscript{me}
de Maintenon et M\textsuperscript{me} la duchesse de Bourgogne, qui y
était à son ordinaire, lui demandèrent avec surprise ce qui l'amenait.
La sienne fut extrême lorsque toutes les deux lui dirent de se bien
garder d'exécuter son dessein, qui étonnerait le roi, gâterait tout à
fait son affaire. Le roi survint si promptement, qu'elles n'eurent pas
le temps de lui en dire davantage, et le roi, la trouvant là,
l'embarrassa encore plus en lui demandant ce qu'elle venait y faire. À
l'instant M\textsuperscript{me} de Maintenon prit la parole pour elle,
et répondit qu'elle l'était venue voir un peu sur le tard\,; et
M\textsuperscript{me} là duchesse d'Orléans ajouta quelques propos sur
la difficulté de la trouver seule entre son retour de Saint-Cyr et
l'arrivée du roi chez elle. Le roi lui dit que puisqu'elle était venue
elle pouvait s'asseoir un peu. Elle, qui vit là plusieurs dames ou du
palais ou de la privance de M\textsuperscript{me} de Maintenon qui ne
vidaient point pour couler dans le grand cabinet à l'ordinaire, eut
parmi son trouble l'esprit assez présent pour trouver à leur donner le
change. Elle parla bas à M\textsuperscript{me} de Maintenon sur ses deux
filles cadettes qu'elle avait pris le dessein de mettre en religion, et
s'aida du prétexte de la petite surdité de M\textsuperscript{me} de
Maintenon pour, en parlant bas d'un air de mystère, laisser entendre aux
dames quelques mots de ses filles et du couvent, à quoi
M\textsuperscript{me} de Maintenon, qui entra aussitôt dans sa pensée,
aida elle-même. Peu après, M\textsuperscript{me} la duchesse de
Bourgogne fit signe à M\textsuperscript{me} la duchesse d'Orléans de
s'en aller, qui se retira infiniment déconcertée, ne sachant plus où
elle en était, entre ce que M. le duc d'Orléans lui avait dit le matin
et ce qui venait de lui arriver en un lieu si instruit, et si avant
entré dans ses intérêts.

Le soir même elle se trouva au souper auprès de M\textsuperscript{me} la
duchesse de Bourgogne à table, et après dans le cabinet. Elle
s'éclaircit avec elle, et apprit que tout ce que M. le duc d'Orléans lui
avait dit était vrai\,; que le roi en avait parlé en mêmes termes à
M\textsuperscript{me} de Maintenon et à elle, mais qu'il avait si fort
en tête qu'il n'en parût rien, qu'elles avaient jugé qu'il serait choqué
de la trouver chez M\textsuperscript{me} de Maintenon, parce que cela
ferait une nouvelle, et plus choqué encore si elle lui parlait\,; ce qui
les avait engagées à lui conseiller de n'en rien faire\,; et en effet le
roi avait paru mal content de la trouver là. M\textsuperscript{me} la
duchesse de Bourgogne ajouta que depuis quelques jours le roi tournait
Monseigneur pour lui parler\,; qu'il remarquait que Monseigneur le
sentait et l'évitait en particulier, et lui paraissait rêveur et
morgue\,; que cela peinait et embarrassait le roi, et lui faisait
désirer qu'il ne se fît aucune démarche qui réveillât davantage
M\textsuperscript{me} la Duchesse, afin de lui donner lieu de se
rassurer de ce qui l'avait alarmée des mouvements du dernier Marly, et à
Monseigneur d'être moins en garde et froncé avec lui. Il est vrai que la
visite de M\textsuperscript{me} la duchesse d'Orléans fit tout aussitôt
du bruit\,; mais sa présence d'esprit y mit le remède. Le dessein de
mettre ses filles en religion avait été entendu de quelques dames parmi
cet air de secret, et passa aussitôt pour l'objet de la visite. La chose
me revint de la sorte par des dames du palais de mes amies, et nous en
rîmes bien, M. {[}le duc{]} et M\textsuperscript{me} la duchesse
d'Orléans et moi.

Le roi retourna à Marly le lundi 26 mai, et c'est le seul voyage que
j'aie manqué depuis l'audience qu'il m'avait accordée. M. {[}le duc{]}
et M\textsuperscript{me} la duchesse d'Orléans, qui trouvèrent que je
leur y manquais fort m'en écrivirent souvent, et me firent aller
plusieurs fois à l'Étoile et à Saint-Cloud faire des repas rompus pour
avoir lieu de m'entretenir sans afficher les rendez-vous. J'en avais un
de ceux-là à Saint-Cloud le jour de l'Ascension (29 mai), mais Bignon
m'ayant envoyé prier à dîner, qui était le signal de la réponse dont lui
et moi étions convenus\,; je le mandai à M. le duc d'Orléans, et le
priai de faire son repas sans moi, mais de m'attendre au sortir du mien,
que j'irais lui dire ce que j'aurais appris. J'allai de bonne heure chez
Bignon. Il acheva quelque chose qu'il faisait dans son cabinet, et me
mena après dans sa galerie. Là il me dit qu'il avait raconté à
M\textsuperscript{lle} Choin les choses principales de notre
conversation, et celles qui étaient les plus propres à la porter à
entrer en commerce avec M. {[}le duc{]} et M\textsuperscript{me} la
duchesse d'Orléans, qu'elle se sentait très-obligée à leur désir, mais
que n'étant que déjà trop vue, elle ne voulait augmenter ni le nombre ni
l'éclat de ceux qu'elle voyait\,; que les uns étaient de ses amis
particuliers, les autres des gens que Monseigneur avait désiré qu'elle
vît\,; qu'elle ne voyait personne de nouveau d'elle-même, mais seulement
par Monseigneur, et de lui-même sans qu'elle le proposât, et quantité de
fausses excuses et de verbiages semblables, qu'elle l'avait même grondé
de s'être chargé de la commission. Puis s'ouvrant avec moi davantage, il
me dit franchement qu'elle craignait à tel point le caractère de M. le
duc d'Orléans que pour rien au monde, elle ne lierait avec lui, qui
d'ailleurs était trop mal avec Monseigneur pour qu'elle l'osât faire\,;
qu'à l'égard de M\textsuperscript{me} la duchesse d'Orléans, elle
l'estimait et serait volontiers portée à la voir, mais qu'au point où
elle en était avec M\textsuperscript{me} la Duchesse, et dont celle-ci
était mal avec cette soeur, elle croirait lui manquer essentiellement si
elle entrait en commerce avec son ennemie\,; que, quoi que Bignon eût pu
lui dire sur M\textsuperscript{me} la Duchesse, il n'avait pu
l'ébranler, et qu'il n'était pas possible pour peu que ce fût de l'en
détacher\,; qu'encore que M\textsuperscript{lle} Choin pût connaître de
M\textsuperscript{me} la Duchesse, elle se louait tellement de son
amitié et de ses soins qu'elle se persuadait que le tout était
sincère\,; qu'en un mot, elle lui avait fermé la bouche sur M. et
M\textsuperscript{me} la duchesse d'Orléans, et défendu de lui en jamais
plus parler, non en air de chagrin et de colère, mais au contraire
d'amitié, comme ayant si fortement pris son parti là-dessus que rien
n'était capable de la faire changer, par quoi elle n'en voulait pas être
tourmentée. C'en fut assez pour me fermer la bouche à moi-même.

Je remerciai fort Bignon, qui ne désira point que je rendisse ce détail
à M. {[}le duc{]} et a M\textsuperscript{me} la duchesse d'Orléans, mais
bien que je leur disse clairement que M\textsuperscript{lle} Choin
s'excusait respectueusement de les voir sur l'obscurité qu'elle
recherchait, avec force beaux compliments, et que je leur fisse entendre
que toute tentative était désormais superflue. Je dis encore deux mots à
Bignon en conformité de notre conversation de Versailles, afin qu'il ne
demeurât pas convaincu que son amie eût raison, comme en effet il ne le
demeura pas, avec quoi, nous mîmes fin à ce propos, et moi à mon dessein
de ce côté-là. Je parus gai à l'ordinaire pendant le dîner, je demeurai
du temps après avec la compagnie pour ne point laisser sentir
d'empressement, et je vins ensuite chez moi prendre six chevaux et m'en
aller à Saint-Cloud. J'y trouvai M. {[}le duc{]} et
M\textsuperscript{me} la duchesse d'Orléans à table avec Mademoiselle et
quelques dames, dans une ménagerie la plus jolie du monde, joignant la
grille de l'avenue près le village, qui avait son jardin particulier,
charmant, le long de l'avenue. Tout cela était, sous le nom de
Mademoiselle, à M\textsuperscript{me} de Maré sa gouvernante. Je m'assis
et causai avec eux\,; mais l'impatience de M. le duc d'Orléans ne lui
permit pas d'attendre sans me demander si j'étais bien content et bien
gaillard. «\,Entre-deux,\,» lui dis-je, pour éviter de troubler le
repas, mais il se leva de table aussitôt, et m'emmena dans le jardin.

Là je lui rendis compte du peu de succès de la négociation, et par ce
récit, quoique ménagé, je l'affligeai beaucoup. Il revint à table parler
bas à M\textsuperscript{me} la duchesse d'Orléans\,; le reste du repas
fut triste et abrégé. En sortant de table elle m'emmena dans un cabinet,
où je fus assez longtemps seul avec elle, et où sur la fin M. le duc
d'Orléans nous vint trouver. Je leur dis que cette impatience de savoir,
et cette tristesse après avoir su, convenait mal avec la compagnie et
avec le domestique, et deviendrait nouvelle, et matière de curiosité\,;
qu'il fallait se promener et après cela raisonner. M. le duc d'Orléans,
toujours extrême, dit qu'il ne s'en souciait point\,; et sur la chose
même nous tint des propos d'aller planter ses choux dans ses maisons,
qui ne revenaient à rien et qui lui étaient ordinaires quand il était
mécontent. M\textsuperscript{me} la duchesse d'Orléans fut de mon avis.
Enfin à grand'peine nous visitâmes la ménagerie qu'ils me montrèrent,
d'où nous allâmes nous promener en calèche dans les jardins de
Saint-Cloud. Sur le soir, ayant mis pied à terre dans ceux de
l'orangerie, ils s'y promenèrent tous deux quelque temps seuls avec moi
à l'écart. Je leur dis que pour tout ceci il ne fallait pas perdre
courage\,; que dès l'entrée de l'affaire nous avions compris qu'elle ne
s'emporterait que d'assaut\,; que dans la suite cette pensée de
M\textsuperscript{lle} Choin m'était venue comme une chose bonne à
tenter, mais fort peu sûre à s'y appuyer\,; qu'au fond c'était une
honnêteté qui ne pouvait être prise qu'en bonne part par cette créature,
et par Monseigneur même, quoique rejetée\,; que le meilleur était que je
m'étais tenu parfaitement clos et couvert sur le mariage, dont je
n'avais pas laissé sentir le moindre vent\,; qu'au fond nous avions
toute la force et l'autorité pour nous, puisqu'ils avaient
M\textsuperscript{me} la duchesse de Bourgogne, M\textsuperscript{me} de
Maintenon et le roi même déclarés pour le mariage, lequel s'en était
nettement expliqué avec M. le duc d'Orléans\,; que c'était ces voies
qu'il fallait suivre et suivre vivement\,; que ceci marquait deux
choses\,: la première qu'il était perdu si le mariage ne se faisait
point\,; l'autre que, s'il retardait, il ne se ferait jamais, partant
que c'était à lui à prendre ses mesures là-dessus. Je ne leur rendis
point les détails que Bignon m'avait engagé à taire, mais je leur en dis
assez pour leur faire bien sentir le tout.

J'étais convenu avec M. {[}le duc{]} et M\textsuperscript{me} la
duchesse d'Orléans qu'ils feraient confidence à M\textsuperscript{me} de
Maintenon et à M\textsuperscript{me} la duchesse de Bourgogne de leur
démarche auprès de M\textsuperscript{lle} Choin, mais sans me nommer, ni
le canal de cette démarche. Elles l'avaient goûtée, et le roi, à qui
elles l'avaient dit, l'avait approuvée. Ma raison d'en avoir été d'avis
était de leur marquer dépendance et confiance entière pour les engager
de plus en plus, et, si la démarche ne réussissait pas, leur faire plus
de peur de l'éloignement de Monseigneur, et du concert et du pouvoir sur
lui de la cabale qui le dominait. Je conseillai donc fortement à M.
{[}le duc{]} et à M\textsuperscript{me} la duchesse d'Orléans de faire
un grand usage de ce refus. Je leur inculquai le plus fortement qu'il me
fut possible, que, si dans ce reste de Marly ils ne ve-naient à bout du
mariage, jamais il ne se ferait, parce que l'ardeur du roi diminuerait,
son embarras sur Monseigneur augmenterait, les impressions de la lettre
qui avait déterminé le roi s'élaigneraient et s'effaceraient,
Monseigneur, par M\textsuperscript{me} la Duchesse et par les Meudons,
où la Choin était toujours, se fortifierait, l'affaire ainsi éloignée
s'évanouirait par insensible transpiration\,; que par cela même qu'ils
seraient, eux, justement fâchés, touchés, mécontents, {[}ils
deviendraient à charge au roi, qui, embarrassé avec eux de ses
ouvertures, et outré qu'ils vissent à découvert qu'il n'osait parler ni
exiger de Monseigneur, s'élaignerait absolument d'eux, tellement que,
mal pour le présent, ils devaient penser ce qu'ils pourraient devenir
pour l'avenir, surtout si la même faiblesse d'une part, et la même force
de cabale de l'autre, emportait le mariage de M\textsuperscript{lle} de
Bourbon, comme il y avait peu à en douter. Après un raisonnement si
nerveux, et que tous deux approuvèrent sans le moindre débat,
M\textsuperscript{me} la duchesse d'Orléans rentra au château pour
écrire au P. du Trévoux. Je suivis M. le duc d'Orléans à rejoindre la
compagnie, qui un moment après s'éparpilla.

M. le duc d'Orléans se mit à l'écart avec Mademoiselle, et moi par
hasard avec M\textsuperscript{me} de Fontaine-Martel. Elle était fort de
mes amies et très-attachée à eux, et, comme je l'ai rapporté en son
lieu, c'était elle qui m'avait relié avec M. le duc d'Orléans. Elle
sentit bien à tout ce qu'elle vit là qu'il y avait quelque chose sur le
tapis, et ne douta point qu'il ne s'agît du mariage de Mademoiselle.
Elle me le dit sans que j'y répondisse, ni que je lui donnasse lieu de
le croire par un air trop réservé. Prenant occasion de la promenade de
M. le duc d'Orléans avec Mademoiselle, elle me dit confidemment qu'il
ferait bien de hâter ce mariage s'il voyait jour à le faire, parce qu'il
n'y avait rien d'horrible qu'on n'inventât pour l'empêcher\,; et sans se
faire trop presser elle m'apprit qu'il se débitait les choses les plus
horribles de l'amitié du père pour la fille. Les cheveux m'en dressèrent
à la tête. Je sentis en ce moment bien plus vivement que jamais à quels
démons nous avions affaire, et combien il était pressé d'achever. Cela
fut cause qu'après nous être promenés assez longtemps après la fin du
jour, je repris M. le duc d'Orléans comme il rentrait au château, et lui
dis qu'encore un coup il avisât bien à ses affaires, qu'il n'y avait
aucune ressource pour lui si le mariage ne se faisait, et qu'en comptant
bien là-dessus il ne comptât pas moins que, si dans le reste de ce Marly
il ne l'emportait jusqu'à la déclaration, jamais il ne se ferait.

Soit par ce qui avait précédé, soit par cette vive reprise, je le
persuadai, et le laissai plus animé et plus encouragé d'agir que je ne
l'avais encore vu. Il s'amusa je ne sais où dans la maison. Je fis
encore quelques tours de parterre avec M\textsuperscript{me} de Mare, ma
parente et mon amie de tout temps, où on me vint dire que
M\textsuperscript{me} de Fontaine-Martel me de-mandait au château. En y
entrant on me fit passer dans le cabinet où M\textsuperscript{me} la
duchesse d'Orléans écrivait. C'était elle qui, sous cet autre nom,
m'avait envoyé chercher. M\textsuperscript{me} de Fontaine-Martel lui
avait dit dans cet entre-deux de temps l'horreur dont elle m'avait
glacé, et M\textsuperscript{me} la duchesse d'Orléans en voulait
raisonner avec moi. Nous déplorâmes ensemble le malheur d'avoir affaire
à de telles furies. Elle me protesta que l'apparence n'y était pas même
avec une étrangère, combien moins avec une fille, que M. le duc
d'Orléans avait tendrement aimée dès l'âge de deux ans, où il pensa se
désespérer dans une grande maladie qu'elle eut, pendant laquelle il la
veillait jour et nuit, et que toujours depuis cette tendresse avait été
la même, et fort au-dessus de celle qu'il avait pour son fils. Nous
convînmes qu'il était non-seulement cruel et inutile d'en parler à M. le
duc d'Orléans, mais dangereux pour n'augmenter pas son embarras et ses
peines, mais aussi qu'il n'y avait pas une minute de temps à perdre pour
finir le mariage. Enfin ils partirent dans la ferme résolution de
redoubler de force et de courage pour précipiter le mariage, et de faire
leurs derniers efforts pour une très-prompte conclusion.

Dès le lendemain vendredi, ils firent bon usage auprès de
M\textsuperscript{me} la duchesse de Bourgogne et de
M\textsuperscript{me} de Maintenon du refus opiniâtre de
M\textsuperscript{lle} Choin, que je leur avais porté à Saint-Cloud,
qui, par M\textsuperscript{me} la duchesse de Bourgogne et
M\textsuperscript{me} de Maintenon, passa au roi avec tout
l'assaisonnement nécessaire le même soir du vendredi. Le lendemain matin
samedi, M. le duc d'Orléans parla au roi, et lui demanda avec cette
sorte de hardiesse qui quelquefois ne lui déplaisait pas, quand ce
n'était pas pour le contredire, ce qu'il faisait dans ses cabinets de
d'Antin qui y était toujours, et qui était si bien avec Monseigneur,
s'il ne lui était pas bon à lui faire entendre raison. Le roi rejeta
cette ouverture avec cette sorte de mépris pour d'Antin, qui
persuaderait aux gens des dehors qu'un homme est perdu, mais qui aux
intérieurs et aux connaisseurs ne faisait qu'augmenter l'opinion du
crédit de ce même homme, parvenu à toute familiarité, et dont l'apparent
mépris ne servait qu'à cacher tout son pouvoir à celui-là même qui,
croyant de bonne foi le mépriser, et le voulant parfois montrer aux
autres dans des occasions importantes, n'en était que moins en garde
contre lui, et de plus en plus en proie à l'autorité qu'il lui laissait
usurper sur lui-même. Mais le roi, pressé de la sorte sans le trouver
mauvais, et par cette proposition de se servir de d'Antin, piqué de son
propre embarras sur Monseigneur qu'il voyait clairement aperçu, et
{[}dont{]} il craignit les suites, promit de nouveau, et si
positivement\,; à son neveu qu'il agirait incessamment, qu'il n'y eut
pas matière à réplique.

En effet, le lendemain matin dimanche, le roi saisit enfin Monseigneur
dans son cabinet, où, après un court préambule, il lui proposa le
mariage\,; Il le fit d'un ton de père mêlé de ton de roi et de maître,
qu'adoucit la tendresse avec une mesure si juste et si compassée,
qu'elle ne fit que faciliter, sans donner courage à la résistance,
manière rare, mais très-ordinaire et facile au roi, quand il voulait
s'en servir. Monseigneur hésita, balbutia\,; le roi pressa, profitant de
son trouble. Je n'entre pas dans un plus grand détail, parce qu'il n'en
est pas venu jusqu'à moi davantage. Finalement Monseigneur consentit et
donna parole au roi, mais il lui demanda la grâce de suspendre la
déclaration de quelques jours pour lui donner le temps de s'accoutumer
et d'achever de se résoudre avant que l'affaire éclatât. Le roi donna à
l'obéissance et à la répugnance de son fils le temps illimité qu'il lui
demanda, et encore une fois prit sa parole pour éviter toute remontrance
et tout effort de cabale, le pria de se vaincre le plus tôt qu'il
pourrait, et de l'avertir dès qu'il pourrait souffrir la déclaration.

Le coup décisif ainsi frappé, le roi, infiniment à son aise, le dit à
son neveu une demi-heure après, lui permit de porter cette bonne
nouvelle à M\textsuperscript{me} la duchesse d'Orléans, trouva bon qu'il
en parlât à M\textsuperscript{me} la duchesse de Bourgogne et à
M\textsuperscript{me} de Maintenon uniquement et à la dérobée, et imposa
sur tout le reste un silence exact à sa bouche et jusqu'à sa contenance.
M. le duc d'Orléans lui embrassa les genoux, car il était seul avec lui,
lui exprima sa juste reconnaissance, et le supplia instamment de ne lui
pas refuser d'avancer une si grande joie à Mademoiselle, en lui
répondant de son secret. Après l'avoir obtenu, il lui représenta avec
respect, mais sans empressement, pour ne le pas gêner, combien Madame
aurait lieu de se plaindre de lui, s'il ne la mettait pas dans la
confidence. Le roi trouva bon que, sous le même secret, il le lui dît
aussi, en la priant de sa part de ne lui en parler pas à lui-même. M. le
duc d'Orléans alla tout de suite chez Madame, qui, ne s'étant jamais
flattée que ce mariage pût réussir, et ayant parfaitement ignoré toutes
les démarches qui s'étaient faites, se trouva tout à coup comblée de la
plus extrême joie\,; de là il monta chez M\textsuperscript{me} la
duchesse d'Orléans, où, à portes fermées, ils se livrèrent ensemble à
toute la leur. Bientôt après ils s'en allèrent tous deux à Saint-Cloud,
et revinrent de bonne heure, en grand désir de voir la déclaration
éclater.

D'Antin avait écumé depuis le jeudi jusqu'au dimanche, car les dates
sont ici importantes, que M. le duc d'Orléans avait donné une lettre au
roi qu'il lui avait écrite, et s'était écrié en l'apprenant qu'il ne
comprenait pas comment il avait pu faire pour la donner en son absence,
tant il fut frappé du fait. Ce fut un trait qui nous revint bientôt, et
qui nous montra à plein combien il était attentif à espionner et à
contraindre M. le duc d'Orléans dans les cabinets du roi, dans la
crainte du mariage. Or le jeudi fut le jour que Bignon me fît la réponse
négative de M\textsuperscript{lle} Choin que je fus tout de suite porter
le même jour à Saint-Cloud, et le dimanche suivant est le jour auquel le
roi parla à Monseigneur, et tira parole de lui pour le mariage. Entre
ces deux jours-là je n'ai pu démêler celui où d'Antin apprit que M. le
duc d'Orléans avait donné une lettre au roi\,; mais ce ne fut
certainement que ce jeudi même ou un des deux {[}jours{]} suivants. Par
ce qui suivit, et que j'expliquerai en son lieu, je ne puis douter que
la Choin, à qui Bignon voulut me nommer, et à qui je permis, comme je
l'ai dit, se hâta d'avertir Monseigneur et M\textsuperscript{me} la
Duchesse de la démarche que M. {[}le duc{]} et M\textsuperscript{me} la
duchesse d'Orléans avaient faite, vers elle par moi, par l'entremise de
Bignon.

Ces notions, qui se suivirent coup sur {[}coup{]} si fort en cadence,
après des mouvements peu éloignés qui avaient été remarqués à l'autre
Marly, réveillèrent la cabale\,; et comme elle n'était pas intéressée au
secret sinon de ses notions, il en échappa à quelqu'un d'eux assez pour
que, dès le samedi au soir, veille du dimanche que le roi parla enfin à
Monseigneur, il se murmurât bien bas dans le salon quelque bruit sourd
et incertain du mariage comme d'une chose qui s'allait faire, mais qui
demeura entre les plus éveillés et les plus instruits. Monseigneur, qui
n'avait osé résister au roi pour la première fois de sa vie, lui demanda
peut-être ce délai illimité de la déclaration, dans l'embarras où il se
trouva avec M\textsuperscript{me} la Duchesse et sa cabale, qui, sur ce
que je viens d'expliquer, était bien en émoi, mais fort éloignée de
croire rien d'avancé, et que Monseigneur voulut avoir le temps de les y
préparer. Quoi qu'il en soit, le lundi 2 juin, le lendemain du jour que
le roi avait parlé la première fois à Monseigneur, le roi prit en
particulier M. le duc de Berry le matin, et lui demanda s'il serait bien
aise de se marier. Il en mourait d'envie, comme un enfant qui croit en
devenir plus grand homme et plus libre, et en qui on avait pris soin des
deux côtés d'en nourrir le désir. Mais il était tenu de longue main dans
la crainte secrète de M\textsuperscript{lle} de Bourbon et dans le désir
de Mademoiselle, par Mgr le duc de Bourgogne et surtout par l'adresse de
M\textsuperscript{me} la duchesse de Bourgogne, avec qui il vivait dans
la plus intime amitié et confiance. Il sourit donc à la question du roi
et lui répondit modestement qu'il attendrait sur cela tout ce qui lui
plairait de faire sans empressement et sans éloignement. Le roi lui
demanda ensuite s'il n'aurait point de répugnance à épouser
Mademoiselle, la seule en France, ajouta-t-il, qui pût lui convenir,
puisque, dans les conjonctures présentes, on ne pouvait songer à aucune
princesse étrangère. M. le duc de Berry répondit qu'il obéirait au roi
avec plaisir. Aussitôt le roi lui déclara qu'il avait le dessein de
faire incessamment le mariage, que Monseigneur y consentait, mais il lui
défendit d'en parler. Sortant de chez le roi, M. le duc de Berry fut
courre le loup avec Monseigneur et Mgr le duc de Bourgogne, et la chasse
même fut assez longue.

Cette même journée, M. {[}le duc{]} et M\textsuperscript{me} la duchesse
d'Orléans l'allèrent encore passer à Saint-Cloud. Il faisait déjà chaud
alors, et le roi sortait plus tard pour la promenade. Monseigneur ne lui
avait point reparlé du mariage, mais d'Antin le devina ou le sut par
Monseigneur, et se tourna lestement à en hâter la déclaration pour s'en
faire un mérite. En cette saison le roi donnait chez lui les premiers
temps de l'après-dînée au ministre qui aux jours d'hiver travaillait le
soir avec lui chez M\textsuperscript{me} de Maintenon, se promenait
après, rentrait tard chez elle et y travaillait seul et souvent point.
D'Antin, occupé de son projet, entra par les derrières dans les cabinets
aussitôt que le travail fut achevé. Il y hasarda des demi-mots qui
firent que le roi lui dit le mariage. Il applaudit avec cet engouement
de flatterie qu'il avait si fort en main et qui lui coûtait si peu pour
les choses qui le fâchaient le plus et avec cette liberté qu'il savait
usurper si à propos\,; il dit au roi qu'il ne savait pas pourquoi {[}on
faisait{]} un mystère d'une affaire aussi convenable et déjà même si
découverte, qu'à l'heure même qu'il en faisait un secret dans son
cabinet, plusieurs gens s'en parlaient à l'oreille dans le salon. En ce
moment Monseigneur entra dans le cabinet, ou naturellement et revenant
de la chasse, ou de concert avec d'Antin pour lui en procurer le gré, et
s'épargner la peine de reparler au roi de chose qui lui était si peu
agréable. Le roi et d'Antin continuèrent cette conversation devant lui.
Cela donna occasion et courage au roi de lui demander que lui en
semblait, et d'ajouter tout de suite que, puisque la chose commençait à
se savoir, autant valait-il aller de ce pas, avant la promenade, faire
la demande à Madame. Monseigneur s'y laissa aller comme il avait fait au
mariage, mais pour cette fois sans résistance.

À l'instant le roi envoya chercher Mgr le duc de Bourgogne, à qui, pour
la forme, ils dirent ce qu'il savait bien, et aussitôt après sortirent
tous trois par le second cabinet, vis-à-vis la porte duquel était celle
de la chambre de Madame, le petit salon entre-deux, et entrèrent chez
elle. Pendant ce moment de Mgr le duc de Bourgogne, d'Antin sortit,
s'alla montrer gaiement dans le salon, où il dit ce qui se passait pour
l'avoir dit le premier, et avisant à travers la porte vitrée du salon un
laquais à lui dans le petit salon de la Perspective, où tous les valets
attendaient leurs maîtres, il l'envoya à pied à Saint-Cloud porter
verbalement cette nouvelle de sa part, pour ne perdre pas de temps à
seller un cheval et à écrire. Du moment qu'il eut dit ce qu'il savait,
il se fit une telle presse à la porte du petit salon de la chapelle de
tout ce qui se trouva dans le salon, qu'on s'y étouffait à qui verrait
passer et repasser le roi. Madame, qui écrivait à son ordinaire, et qui
savait ce qui se devait passer, ne douta plus que le moment n'en fût
arrivé, dès qu'elle vit entrer chez elle le roi, Monseigneur et Mgr le
duc de Bourgogne. Le roi lui fit en forme la demande de Mademoiselle. On
peut juger si elle l'accorda et quelle fut son extrême joie. Le roi
envoya chercher M. le duc de Berry et le présenta à Madame sur le pied
de gendre. Tout cela fut fort court, le roi repassa chez lui par ses
cabinets, et de là dans ses jardins. Dès qu'on l'y eut vu entrer, toute
la cour fondit chez Madame et de là chez Monseigneur et chez M. le duc
de Berry, chacun avide de se faire voir et plus encore de pénétrer les
visages. Si peu de gens et depuis si peu en avaient eu de simples
soupçons, que cette déclaration subite jeta tout le monde dans le plus
grand étonnement. La rage pénétra les uns et jusqu'aux plus indifférents
de la cour et de la ville\,; ce mariage ne fut approuvé de personne, par
les raisons que j'ai expliquées dès l'entrée du récit de cette puissante
intrigue. Mais il est des choses dont on ne peut et on ne doit pas
rendre raison, et alors il faut laisser dire. Tel fut le coup de foudre
qui tomba sur M\textsuperscript{me} la Duchesse, si à coup\footnote{Si à
  l'improviste.} au premier voyage de ses filles à Marly. Je n'ai point
su ce qui se passa chez elle dans ces étranges moments, où j'aurais
acheté cher une cache derrière la tapisserie. M. {[}le duc{]} et
M\textsuperscript{me} la duchesse d'Orléans revenaient de Saint-Cloud,
lorsqu'ils rencontrèrent ce laquais de d'Antin, qui les arrêta et qui
poursuivit après son chemin vers Mademoiselle.

On peut juger du soulagement de M. {[}le duc{]} et de
M\textsuperscript{me} la duchesse d'Orléans. En arrivant ils allèrent
droit chez Monseigneur, qui était à table chez lui, faisant un retour de
chasse avec des dames et Mgrs ses fils. Débarrassé de l'éclat et bon
homme au fond, il ne voulut pas déplaire au roi par une mauvaise grâce
inutile\,; il prit donc en les voyant entrer un air non-seulement gai,
mais épanoui\,; il les embrassa et les fit embrasser par Mgrs ses Fils,
leur présentant le second comme leur gendre, et voulut que les plus
considérables de la table les embrassassent aussi. Il fit asseoir
M\textsuperscript{me} la duchesse d'Orléans près de lui, lui prit les
mains à sept ou huit reprises, l'embrassa cinq ou six autres, but au
beau-père, à la belle-mère, à la belle-fille sous ses noms, porta leurs
santés à la compagnie, et quoique M. et M\textsuperscript{me} d'Orléans
ne fussent pas à table, les fit boire à lui et faire raison aux autres,
en un mot, on ne vit jamais Monseigneur si gai, si occupé, si rempli de
quelque chose. Le repas fut allongé, les santés réitérées, en un mot,
allégresse complète. De leur vie, M. {[}le duc{]} et
M\textsuperscript{me} la duchesse d'Orléans ne furent si surpris que
d'une réception si fort inespérée. On peut croire qu'ils n'eurent pas
peine à faire merveilles de joie, de reconnaissance, de respect.
M\textsuperscript{me} la duchesse de Bourgogne, qui se tint toujours là,
anima tout, et Mgr le duc de Bourgogne fut si aise et du mariage, et de
le voir si bien pris, qu'il en haussa le coude jusqu'à tenir des propos
si joyeux, qu'il ne pouvait les croire le lendemain. Monseigneur poussa
la chose jusqu'à vouloir mener le lendemain M. le duc de Berry à
Saint-Cloud voir Mademoiselle\,; mais le roi, plus mesuré, dit qu'il
fallait qu'elle le vînt voir auparavant, qu'il lui présenterait le duc
de Berry, et que ce ne serait que le surlendemain, pour donner un jour à
la préparation de l'entrevue.

Le retour de chasse et la visite achevée, M. {[}le duc{]} et
M\textsuperscript{me} la duchesse d'Orléans, allèrent chez
M\textsuperscript{me} la Duchesse lui donner part du mariage auquel en
effet elle en prenait tant. Soit que dans ces premiers moments elle
craignît les compliments et les curieux, soit qu'elle ne sût que
devenir, comme il arrive dans ces crises d'angoisses, elle était sortie
de chez elle et se promenait dans les jardins, fort peu accompagnée.
M\textsuperscript{me} la duchesse d'Orléans parla la première, et lui
fit excuse de n'avoir pu le lui dire plus tôt sur ce qu'elle arrivait de
Saint-Cloud et ne faisait que sortir de chez Monseigneur. Le remercîment
fut d'un froid à glacer. M. le duc d'Orléans prit un peu la parole pour
les soulager toutes deux\,; ensuite M\textsuperscript{me} la duchesse
d'Orléans, pour adoucir ces premiers moments, ou plutôt pour agir en
conformité de la lettre de M. le duc d'Orléans au roi qui détermina le
mariage, dit à M\textsuperscript{me} la duchesse que ce qui lui faisait
un nouveau plaisir dans une affaire si agréable était qu'il y avait dans
leur famille de quoi se communiquer une alliance si honorable. À
l'instant M\textsuperscript{me} la Duchesse échappant à elle-même\,:
«\,Quoi\,! votre fille\,? répondit-elle d'un ton aigre\,; mon fils quant
à présent est un trop mauvais parti, ses affaires sont dans un désordre
étrange, on lui dispute tout, et on ne sait encore ce qui lui restera de
bien, et votre fille est trop jeune pour la pouvoir marier.\,»
M\textsuperscript{me} la duchesse d'Orléans, à mon avis trop bonne
d'avoir dès lors fait cette ouverture, et trop douce de l'avoir après
continuée, repartit que M. le Duc aurait toujours de quoi la satisfaire,
ce que M. le duc d'Orléans reprit aussi, et M\textsuperscript{me} la
duchesse d'Orléans ajouta l'âge de M\textsuperscript{lle} sa fille.
M\textsuperscript{me} la Duchesse le disputa pour la soutenir trop
jeune, et toutes deux poussèrent jusqu'aux dates et aux époques\,;
M\textsuperscript{me} la Duchesse vaincue, conclut plus aigrement encore
qu'elle ne voulait marier son fils de longtemps. La pluie et le beau
temps relevèrent quelques moments de silence. M\textsuperscript{me} la
duchesse d'Orléans dit qu'elle avait beaucoup d'affaires, et pria
M\textsuperscript{me} la Duchesse de tenir sa visite pour reçue,
puisqu'elle allait chez elle lorsqu'elle l'avait rencontrée dans le
jardin. M\textsuperscript{me} la Duchesse se jeta aux compliments et dit
qu'elle monterait incessamment chez elle. M\textsuperscript{me} la
duchesse d'Orléans la pria de n'en rien faire, M. le duc d'Orléans
aussi, enfin ils se quittèrent réciproquement les visités, et se
séparèrent, M\textsuperscript{me} la Duchesse soulagée d'avoir au moins
insolenté sa soeur, et celle-ci riant de bon coeur de cette rage montée
au point de ne la pouvoir cacher. Je supprime le reste de cette belle
journée pour M. {[}le Duc{]} et M\textsuperscript{me} la duchesse
d'Orléans\,; mais cette visite à M\textsuperscript{me} la Duchesse m'a
paru trop plaisante et trop curieuse pour ne la pas rapporter.

\hypertarget{chapitre-xiv.}{%
\chapter{CHAPITRE XIV.}\label{chapitre-xiv.}}

1710

~

{\textsc{M\textsuperscript{me} de Blansac, et sa rare retraite, et son
rare héritage.}} {\textsc{- Fortune de ses enfants.}} {\textsc{-
J'apprends la déclaration du mariage de M. le duc de Berry avec
Mademoiselle.}} {\textsc{- Spectacle de Saint-Cloud.}} {\textsc{- Vive,
dernière et inutile attaque de M\textsuperscript{me} la duchesse
d'Orléans à moi, sur la place de dame d'honneur.}} {\textsc{- Oubli sur
l'audience de M\textsuperscript{me} la duchesse de Bourgogne à
M\textsuperscript{me} de Saint-Simon.}} {\textsc{- Présentation de
Mademoiselle à Marly.}} {\textsc{- Consultation entre le roi,
M\textsuperscript{me} de Maintenon et M\textsuperscript{me} la duchesse
de Bourgogne, sur une dame d'honneur.}} {\textsc{- Bruit à Marly sur
M\textsuperscript{me} de Saint-Simon, et mouvements.}} {\textsc{- Le
chancelier, par l'état des choses, change d'avis sur la place de dame
d'honneur.}} {\textsc{- Avis menaçant de nos amis.}} {\textsc{-
M\textsuperscript{me} la duchesse de Bourgogne nous fait avertir du
péril du refus, et de venir à Versailles.}} {\textsc{- Nous nous
résolvons par vive force à accepter.}} {\textsc{- Conspiration de toutes
les personnes royales à vouloir M\textsuperscript{me} de Saint-Simon.}}
{\textsc{- Singulier dialogue bas entre M. le duc d'Orléans et moi.}}
{\textsc{- M\textsuperscript{me} la duchesse de Bourgogne me fait parler
sur le péril du refus.}} {\textsc{- Droiture et bonté de cette
princesse.}} {\textsc{- Propos très-francs de moi à M. {[}le duc{]} et à
M\textsuperscript{me} la duchesse d'Orléans sur la place de dame
d'honneur.}}

~

Ce même lundi, 2 juin, nous allames, M\textsuperscript{me} de
Saint-Simon et moi, dîner à Saint-Maur avec M\textsuperscript{me} de
Blansac, à qui M\textsuperscript{me} la Duchesse avait prêté le petit
château, c'est-à-dire la maison que feu M. le Duc avait eue de la
déconfiture de la Touanne, et qu'il avait enfermée dans ses jardins.
J'ai assez expliqué ailleurs quelle était M\textsuperscript{me} de
Blansac. J'ajouterai seulement qu'ayant mangé plus de deux millions à
elle ou à Nangis, son fils du premier lit, et mieux encore sans avoir
jamais, elle ni Blansac, montré aucune dépense, elle emprunta cette
maison pour y prendre du lait, et y est demeurée vingt ans sans en
sortir\,: sur la fin de sa vie elle revint à Paris, où elle devint riche
par la succession de M. de Metz, qui jusqu'à la mort, lui dit et lui fit
dire qu'il ne lui donnerait rien, et qui en même temps qu'il l'en
persuadait lui avait tout donné, comme il parut par son testament. Les
deux fils du premier et du second lit de M\textsuperscript{me} de
Blansac ont été plus heureux que père et mère. Nangis est mort maréchal
de France, chevalier de l'ordre et chevalier d'honneur de la reine avec
toute sa confiance\,; l'autre, outre ce grand bien de M. de Metz,
enrichi par d'autres voies dont il n'a négligé aucune, a eu un brevet de
duc, en épousant une fille du duc de La Rochefoucauld. Il me faut passer
cette courte digression assez mal placée, mais dont je n'aurais su où
placer mieux la singularité.

Revenant de Saint-Maur où nous avions passé presque la journée avec
l'abbé de Verteuil, frère du duc de La Rochefoucauld que nous y avions
mené, rentrant chez moi sur les sept heures du soir, je trouvai un
billet de M. le duc d'Orléans qu'un de ses gens avait apporté fort peu
après midi, comme cela m'arrivait souvent pendant ce Marly. Je n'ouvris
le billet que lorsque, monté chez ma mère, j'y fus seul avec elle et
M\textsuperscript{me} de Saint-Simon\,; le dessus était de l'écriture de
M. le duc d'Orléans, le dedans, fort court, de celle de
M\textsuperscript{me} la duchesse d'Orléans, dont les trois premiers
mots étaient ceux-ci\,: \emph{Veni, vidi, vici}. Elle ajoutait que je
verrais bien que c'était M. le duc d'Orléans qui les avait dictés, et
sans en dire davantage, m'imposait le secret jusqu'à la déclaration qui
ne tarderait pas. Après ma première effusion de joie, à laquelle, par un
secret pressentiment, M\textsuperscript{me} de Saint-Simon ne prit
qu'une part de complaisance, j'entrai en inquiétude du délai de la
déclaration. Tandis que j'agitais ce qui pouvait la retarder, on
m'annonça un valet de pied de M. le duc d'Orléans qui, sans lettre me
vint apprendre de la part de Mademoiselle la déclaration de son mariage,
et qu'elle m'envoya dans l'instant qu'elle l'eut apprise, par le laquais
que d'Antin lui avait dépêché de Marly. Alors ma joie fut complète\,: le
triomphe et la sûreté de ceux à qui j'étais attaché, la surprise et
l'extrême dépit de ceux a qui je ne l'étais pas, l'amour-propre d'un tel
succès où j'avais eu une part si principale en tant de sortes, la
différence entière qui en résultait pour ma situation présente et
future, toutes ces choses me flattèrent à la fois. J'écrivis aussitôt à
M. {[}le duc{]} et M\textsuperscript{me} la duchesse d'Orléans qui le
lendemain matin mardi me mandèrent de les aller trouver ce même jour à
Saint-Cloud, de bonne heure.

Ce voyage fut bien différent du dernier, où je leur avais porté la
négative de la Choin. M\textsuperscript{me} de Saint-Simon et moi
trouvâmes Saint-Cloud retentissant de joie. La foule brillante y était
déjà\,; tout s'empressa de me témoigner sa joie\,: je fus complimenté de
chacun, environné sans cesse. À un accueil si surprenant, je me crus
presque le visité. La plupart me parlèrent de cette grande affaire comme
de mon ouvrage, ce que je ne fis jamais semblant d'entendre. Environné,
accolé, entraîné de part et d'autre, dont M\textsuperscript{me} de
Saint-Simon eut aussi toute sa part, je fus poussé à travers ce vaste
appartement, au fond duquel était Mademoiselle avec
M\textsuperscript{me} la princesse de Conti, M\textsuperscript{lle}s ses
filles et un groupe de personnes considérables qui de Marly et de Paris
étaient accourues. Sitôt que Mademoiselle m'aperçut, elle s'écria,
courut à moi, m'embrassa des deux côtés, et tout de suite me prit par la
main, laissa là tout le monde, et du salon me mena dans l'orangerie qui
y est contiguë, et l'enfila. Là, en liberté de ce grand monde qui ne
nous voyait que de loin, elle se répandit en remercîments dont ma
surprise fut telle que je demeurai sans répondre. Elle le sentit et
croyant m'en tirer, elle m'y plongea de plus en plus en me racontant les
choses principales que j'avais faites ou conseillées sur son mariage, et
y mit le comble en m'apprenant que M. le duc d'Orléans lui contait tout
à mesure\,; qu'elle n'avait jamais rien ignoré de tout ce qui s'était
passé dans cette affaire\,; que c'était pour cela qu'elle sortait
presque toujours du cabinet de M\textsuperscript{me} la duchesse
d'Orléans dès que j'y entrais et avant qu'on le lui dît, et m'avoua
qu'elle avait souvent observé mon visage entrant et sortant de ces
conversations.

À un si étonnant récit je ne pus désavouer la vérité des faits, ni
m'empêcher de m'écrier sur la facilité de M. son père à lui faire de
telles confidences. Tout cela fut coupé par des témoignages de la plus
vive reconnaissance dont l'esprit, les grâces, l'éloquence, la dignité
et la justesse des termes ne me surprirent pas moins, mêlés d'élans et
de trouble de joie qu'elle ne contraignit pas avec moi. Elle me dit que
j'avais tout perdu, et qu'elle m'avait bien regretté une demi-heure
auparavant\,; que M\textsuperscript{me} la Duchesse était venue avec
M\textsuperscript{lle}s ses filles lui faire leurs compliments\,; que
cette bonne tante avait essayé de voiler son désordre par une joie si
feinte, que la sienne s'en était augmentée\,; qu'elle lui avait présenté
M\textsuperscript{lle}s ses filles déjà avec un air de respect, en la
suppliant de conserver de la bonté pour elles, à quoi elle avait
malignement répondu qu'elle les aimerait toujours autant qu'elle avait
fait, m'ajoutant en riant de bon coeur qu'elle n'y aurait pas
grand'peine. M\textsuperscript{me} la Duchesse abrégea sa visite en
témoignant son regret de n'avoir pas trouvé M. {[}le duc{]} et
M\textsuperscript{me} la duchesse d'Orléans à Saint-Cloud, et se retira
comme avec avidité de se délivrer d'un état si violent. Mademoiselle me
dit qu'elle l'avait conduite, et malicieusement affecté de lui céder
partout la droite et les portes, quoique toutes ouvertes, et que
M\textsuperscript{me} la Duchesse l'avait si bien senti, qu'elle lui
avait fait des reproches comme d'amitié de ce qu'elle la traitait ainsi
avec cérémonie, dont elle s'était donné le plaisir de ne s'en point
départir jusqu'au bout.

Elle me conta ensuite comment M. le duc d'Orléans lui avait appris son
bonheur, combien elle avait été fidèle au secret, enfin le beau message
de d'Antin, dont elle se moqua fort, sur lequel elle m'avait dépêché
aussitôt, sachant tout ce que j'y avais fait. On ne peut comprendre le
nombre de choses qui se dirent en tête à tête en nous promenant dans
cette orangerie, pendant une demi-heure. La duchesse de La Ferté le vint
interrompre, d'où incontinent nous nous retrouvâmes dans le gros du
monde, que je laissai aussitôt pour aller faire mes compliments à Madame
qui écrivait, et qui me reçut avec des larmes de joie. En même temps
M\textsuperscript{me} de Saint-Simon était environnée de foule et de
compliments, et de gens qui lui en faisaient d'autres à découvert sur ce
qu'elle allait être dame d'honneur de la future duchesse de Berry. Elle
répondit avec modestie sur son incapacité, son âge, ses empêchements,
sur le grand nombre d'autres personnes convenables, et parmi tout cela
fit si bien sentir ce qu'elle sentait elle-même, qu'il lui fut dit par
M\textsuperscript{me} de Châtillon qu'elle se portait donc elle-même
pour trop jeune\,: à quoi elle répondit très-franchement que oui.
Mademoiselle qui à peine la connaissoit, lui fit toutes les prévenances
et les caresses imaginables\,; enfin cette opinion de la place qu'elle
allait remplir se trouva si répandue parmi ce peuple femelle de la cour,
que les bassesses lui furent prodiguées à en avoir honte et pitié, et
que ses craintes se renouvelèrent\,; elle fut en calèche avec quelque
peu de dames au-devant de M. {[}le duc{]} et de M\textsuperscript{me} le
duchesse d'Orléans qui venaient de Sceaux, donner part du mariage.
L'allégresse fut grande, ils se pressèrent pour les mettre dans leur
carrosse, et arrivèrent ainsi dans la cour. Tout y courut\,; dès qu'ils
m'aperçurent ce furent des cris de joie, et en mettant pied à terre, des
embrassades réitérées et des compliments réciproques.

La foule illustre les environna, Madame et Mademoiselle les
rencontrèrent et descendirent pour se promener avec eux et se faire voir
au peuple, dont fourmillaient la cour et les jardins. En montant en
calèche ils me prièrent instamment de les attendre, afin qu'un peu
débarrassés d'une cour si nombreuse, ils me pussent entretenir et se
répandre avec moi, et je me promenai en les attendant en bonne et grande
compagnie. Sitôt qu'ils se furent séparés de Madame, qui retournait de
bonne heure à Marly, ils m'envoyèrent dire de les aller trouver au haut
des jardins de l'orangerie. Dès qu'ils me virent, ils quittèrent le gros
qui les environnait, vinrent à moi, s'écartèrent loin de tout le monde,
et là me racontèrent tout ce qui s'était passé à Marly, et que j'ai
expliqué ci-dessus pour conserver l'ordre des temps de chaque chose\,;
nous nous épanouîmes au port après les dangers courus, nous repassâmes
mille choses avec plaisir sur la joie des uns, sur la surprise et le
dépit des autres, nous nous divertîmes de l'incroyable souplesse de
d'Antin, surtout nous ne pouvions nous lasser de nous parler du procédé
si surprenant de Monseigneur, ni moi de les exhorter d'en profiter pour
se rapprocher de lui, et d'en saisir ces premiers moments si favorables.
Ils me dirent après, que le roi ne donnerait ni maison ni apanage aux
futurs époux jusqu'à la paix, et qu'en attendant, ils mangeraient chez
M\textsuperscript{me} la duchesse de Bourgogne, et se serviraient des
officiers et des équipages du roi.

Tout en devisant ils me menèrent insensiblement tout de l'autre côté du
parterre, où il n'y avait personne, et fort loin d'où ils m'avaient
joint, encore plus de la compagnie qu'ils avaient quittée, que tout à
coup M. le duc d'Orléans alla rejoindre, et me laissa seul avec
M\textsuperscript{me} la duchesse d'Orléans.

Elle s'assit sur un banc qui se trouva là, et m'invita de m'y asseoir
avec elle. Quelque liberté que j'eusse avec eux, jamais, hors en
discours seul avec eux et pour eux-mêmes, je n'en ai séparé le respect,
persuadé que, quelque familiarité que ces gens-là donnent, on en est au
fond mieux et plus à l'aise avec eux en gardant cette conduite, dont la
décence tient aussi à ce qu'on se doit à soi-même. Quoique je dusse être
assis et que je le fusse toujours devant M. {[}le duc{]} et
M\textsuperscript{me} la duchesse d'Orléans, je ne crus pas devoir
m'asseoir sur le même banc tête à tête avec elle, vus surtout à travers
ce grand parterre de tout le monde qui était demeuré de l'autre côté, et
je me tins debout vis-à-vis d'elle\,; elle acheva assise quelque reste
court de discours commencés en gagnant ce banc, puis tout à coup, et
sans aucune liaison qui conduisît où elle en voulait venir, elle me dit
que, maintenant que le mariage s'allait faire, il était question d'une
dame d'honneur\,; que j'avais assez mal reçu ce qu'elle m'en avait jeté
d'abord, puis proposé pour M\textsuperscript{me} de Saint-Simon d'une
manière plus expresse\,; qu'elle ne m'en avait plus parlé depuis, mais
qu'à présent qu'il fallait se déterminer, elle me disait franchement
qu'elle n'en voyait point d'autre qu'elle pût désirer. Je lui répondis
par un remercîment auquel j'ajoutai que je lui avais parlé de bonne foi
là-dessus\,; que M\textsuperscript{me} de Saint-Simon ne convenait point
à cette place\,; qu'elle n'en avait point l'âge\,; qu'elle n'en avait
point la santé pour les fatigues, ni la capacité pour conduire une si
jeune princesse, ni la liberté par nos affaires domestiques et notre
situation avec sa mère\,; que j'étais extrêmement sensible à la bonté
qu'elle nous témaignait, mais que ce serait y mal répondre que de ne le
pas faire avec la môme franchise\,; qu'il y en avait beaucoup d'autres
qui y seraient très-propres, sur qui elle pouvait jeter les yeux. Elle
me répliqua qu'après y avoir bien pensé, sur le peu de goût qu'elle
m'avait vu pour cette place, elle n'en trouvait aucune sans inconvénient
et avec toutes les qualités à souhait, que M\textsuperscript{me} de
Saint-Simon seule, qu'elle m'avouait qu'elle souhaitait uniquement et
passionnément. Je repartis les mêmes choses, sur chacune desquelles elle
me dit en m'interrompant\,: «\,Mais c'est notre affaire à nous de voir
si nous la voulons bien comme cela, et c'est la vôtre de voir si vous
nous la voulez bien donner.\,»

Après avoir ainsi contesté un bon quart d'heure, elle me dit que son nom
pour l'honneur, son mérite et sa réputation pour la confiance, était
tout ce qu'ils désiraient, qu'après cela elle ne ferait de fonctions
qu'autant et en la manière qu'elle pourrait et qui lui plairait. Rien
n'était plus flatteur, et les façons de dire ajoutaient encore aux
paroles, mais je demeurai ferme sur mes mêmes excuses, si bien qu'après
m'avoir un moment regardé avec plus de tristesse\,: «\,Je vois bien ce
que c'est, me dit-elle, c'est qu'une seconde place ne vous accommode
pas,\,» et à l'instant ses yeux rougissant et s'emplissant d'eau, elle
les baissa et demeura fort embarrassée\,; je le fus moins que je
n'aurais dû, parce que mon parti était bien pris. Je ne répondis rien à
ce qu'elle me dit sur la deuxième place, parce qu'en effet c'était cela
même qui nous tenait, et je demeurai deux bons \emph{Miserere} sans
parler ni elle aussi, vis-à-vis l'un de l'autre. Enfin je ne sus mieux,
pour assurer mon refus, en le ménageant avec le respect dû au rang et à
l'amitié, que de sortir de ce silence par une disparate expresse et tout
à fait déplacée. «\,Madame, lui dis-je tout d'un coup et d'un ton ferme,
Mademoiselle a bien de l'esprit, et je n'ai pas ouï dire que M. le duc
de Berry en ait autant qu'elle. Il faut qu'elle s'insinue tout de son
mieux auprès de lui\,; elle le gouvernera.\,» Puis me mettant à battre
la campagne et à parler précisément pour parler, je continuai assez
longtemps, jusqu'à ce que M\textsuperscript{me} la duchesse d'Orléans
ayant repris ses esprits et surmonté son embarras et son dépit, elle fit
effort pour rencogner ses larmes, entra dans ce que je disais par cinq
ou six paroles, se leva aussitôt brusquement, dit qu'il était temps de
s'en retourner, et marcha vers son carrosse en silence jusqu'à ce
qu'elle eût rencontré quelqu'un.

La foule se rapprocha promptement\,; et, sans me dire un mot, {[}elle{]}
me fit une révérence civile et monta en carrosse, {[}avec{]} M. le duc
d'Orléans et M\textsuperscript{me} de Castries, et tous trois s'en
retournèrent à Marly, non, je pense, sans parler de ce qui venait de se
passer avec moi. Je me remis ensuite parmi le grand monde\,; et, après
fort peu de tours, M\textsuperscript{me} de Saint-Simon et moi prîmes
congé de Mademoiselle, et nous retournâmes à Paris, moins occupés tous
deux du brillant spectacle que nous venions de voir que de ce qu'il
venait de m'arriver avec M\textsuperscript{me} la duchesse d'Orléans.
Tout ce qui était alors de l'autre côté du parterre avec M. le duc
d'Orléans et Mademoiselle, avaient les yeux fichés sur nous, et lui plus
qu'aucun, à ce que je remarquai bien. Nous fûmes fort surpris,
M\textsuperscript{me} de Saint-Simon et moi, de cette persévérance,
après les refus, l'un général, l'autre si particulier, que j'avais faits
à M\textsuperscript{me} la duchesse d'Orléans, le premier à Versailles,
l'autre si exprès à Marly\,; et de ce que, après cela, avec toute sa
hauteur et sa fierté, elle s'était exposée au troisième à Saint-Cloud,
au jour de son triomphe. Nous sentîmes bien que cette dernière tentative
était un concert entre elle et M. le duc d'Orléans, qui, me connaissant
bien et comptant que je n'avais pas avec elle la même liberté qu'avec
lui, et bien plus de mesure\,; je serais moins ferme et plus hors de
garde, livré à un tête-à-tête avec elle\,; pour quoi de guet-apens ils
m'avaient conduit de l'autre côté du jardin, où il n'y avait personne,
et lui s'était aussitôt après retiré pour me laisser seul avec elle et
me livrer à l'embarras, sans qu'il eût encore osé m'ouvrir la bouche de
cette place de dame d'honneur. De tout cela nous conclûmes qu'il n'était
pas possible de refuser, ni plus nettement ni plus respectueusement que
je l'avais fait, et fort difficile qu'après cela ils poussassent leur
pointe davantage. Nous nous sûmes bon gré de plus en plus des devants si
à propos pris avec M. de Beauvilliers et M\textsuperscript{me} la
duchesse de Bourgogne, sur l'audience de laquelle je m'aperçois que le
désir d'abréger ce qui ne regarde que moi m'en a fait omettre une partie
essentielle que je restituerai ici.

Après avoir inutilement épuisé toutes les raisons d'incapacité et d'âge,
et toutes celles d'attachement personnel pour M\textsuperscript{me} la
duchesse de Bourgogne, M\textsuperscript{me} de Saint-Simon se jeta sur
la délicatesse de sa santé, sur les soins domestiques que je laisserais
toujours rouler entièrement sur elle, sur l'âge de ma mère, qui avec
toute sorte de justice et de raison demandait une assiduité auprès
d'elle, incompatible avec celle de dame d'honneur d'une si jeune
princesse. Elle exagéra même ces trois bonnes raisons fort au delà de
leur juste mesure, et pour tout cela ne trouva pas M\textsuperscript{me}
la duchesse de Bourgogne plus flexible. Sur sa santé elle lui répondit
qu'on ne prétendait pas lui demander plus qu'elle pourrait et voudrait
faire\,; que la dame d'atours était faite pour porter sans murmure, du
moins sans appui, toutes les corvées fatigantes qu'une dame d'honneur de
sa sorte ne voudrait pas essuyer\,; sur les affaires, qu'elle était
très-louable de s'y attacher, qu'elle l'assurait de tous les congés
qu'elle voudrait, même pour des absences et des voyages à la Ferté, que
le roi ne trouverait point mauvais pendant les voyages de Marly\,; à
l'égard de ma mère, que ce devoir devait aller toujours avant tout
autre\,; qu'elle y vaquerait avec liberté, et qu'elle lui répondait de
prendre tout cela sur elle. M\textsuperscript{me} de Saint-Simon
répliqua que tout cela était bon en spéculation, mais que pour la
pratique il fallait convenir qu'elle serait impossible\,; et apporta
l'exemple de toutes les autres dames d'honneur, à quoi
M\textsuperscript{me} la duchesse de Bourgogne répondit toujours par les
exceptions les plus obligeantes, et finalement ne se rendit, comme je
l'ai rapporté, que pour nous éviter de nous perdre totalement par un
refus auquel elle vit M\textsuperscript{me} de Saint-Simon résolue, quoi
qu'elle eût pu lui dire.

Toutes ces choses devaient nous rassurer, puisque aucune voie ni aucunes
raisons n'avaient été omises et à temps. Néanmoins M\textsuperscript{me}
de Saint-Simon, sujette à espérer peu ce qu'elle désire, ne pouvait se
délivrer d'inquiétude par le désir extrême que nous voyions dans eux
tous, jusqu'à ce qu'il y eût une dame d'honneur nommée. Cela ne pouvait
guère être différé, puisque le mariage était déclaré, et qu'on
n'attendait pour le célébrer que l'arrivée de la dispense du pape.

Le jour même de la déclaration du mariage, il partit deux courriers pour
Rome\,: l'un par Turin, adressé par M. le duc d'Orléans à M. de Savoie,
à qui, nonobstant la guerre, on donnait part du mariage, et qui était
prié en même temps de faire passer et repasser le courrier sûrement et
diligemment\,; l'autre à tout hasard par Marseille, et par la voie de la
mer. Mais M. de Savoie en usa en cette occasion avec toute la politesse
et toute la diligence possible.

Le mardi, qui était le lendemain de ce que je viens de raconter de
Saint-Cloud, Mademoiselle alla dîner à Marly avec M. {[}le duc{]} et
M\textsuperscript{me} la duchesse d'Orléans, sans voir personne. Au
sortir de table, ils la menèrent chez Madame, et de là chez le roi par
les derrières, qu'ils trouvèrent dans son grand cabinet environné de
Monseigneur, de Mgr {[}le duc{]} et M\textsuperscript{me} la duchesse de
Bourgogne, M. le duc de Berry et des principaux officiers seulement des
deux sexes. Les dames d'honneur et d'atours de Madame et de
M\textsuperscript{me} la duchesse d'Orléans, et M\textsuperscript{me} de
Mare, gouvernante de Mademoiselle, les y suivirent. Madame présenta
Mademoiselle au roi, qui se prosterna, et que le roi releva et embrassa
aussitôt, et tout de suite la présenta à Monseigneur, à Mgr {[}le duc{]}
et à M\textsuperscript{me} la duchesse de Bourgogne et à M. le duc de
Berry, qui tous la baisèrent, puis à toute la compagnie. Le roi, pour
ôter tout embarras, avec cette grâce qu'il avait en tout, défendit à
Mademoiselle de dire un mot à personne, à M. le duc de Berry de lui
parler, et abrégea promptement l'entrevue. M\textsuperscript{me} la
duchesse de Bourgogne alla montrer un moment Mademoiselle au salon, où
tout ce qui était à Marly s'était rassemblé, et la mena ensuite chez
M\textsuperscript{me} de Maintenon. Au sortir de là, Mademoiselle passa
chez Madame, et s'en alla coucher à Versailles, où, le surlendemain
jeudi, le roi retourna, contre l'ordinaire, qui était toujours le
samedi. La raison fut que la Pentecôte était le dimanche suivant, 8
juin, et que le roi faisait toujours ses dévotions la veille.

Nous avions fort balancé, M\textsuperscript{me} de Saint-Simon et moi,
d'aller ou n'aller pas à Versailles, jusqu'à ce qu'il y eût une dame
d'honneur. Néanmoins nous crûmes trop marqué de ne nous pas présenter
devant le roi, dans une occasion où la bienséance ferait aller chez un
particulier en pareil cas, et où le respect menait à la cour ceux même
qui n'y allaient plus que pour de véritables occasions. Comme nous
dînions ce jour-là, mercredi, le chancelier, et son fils, qui, faute de
conseils, dont il n'y avait jamais le jeudi, le vendredi et la veille de
la Pentecôte, étaient venus à Paris, nous envoya prier de passer chez
lui après dîner, parce qu'il avait à nous parler, et voici ce que nous
apprîmes d'eux. Le soir du jour de la déclaration du mariage, il fut
question de la dame d'honneur dans la petite chambre de
M\textsuperscript{me} de Maintenon, entre elle, le roi et
M\textsuperscript{me} la duchesse de Bourgogne. Le roi proposa la
duchesse de Roquelaure. On a vu ailleurs que le roi avait eu autrefois
plus que du goût pour elle, et qu'il lui avait toujours conservé de
l'amitié et de la considération. Par cette même raison,
M\textsuperscript{me} de Maintenon ne l'aimait pas, et aurait été outrée
de la voir nécessairement admise dans tout, singulièrement dans les
particuliers, comme il serait arrivé par cette place. C'était une
personne extrêmement haute, impérieuse, intrigante, dont le grand air
altier rebroussait tout le monde, et avec cela de la dernière bassesse
et de la plus abjecte flatterie, qui la faisait fort mépriser.
M\textsuperscript{me} de Maintenon profita de tout cela, sourit, et
répondit qu'on ne pouvait mieux choisir si on avait résolu de faire
enrager toute la compagnie, aucun ne la pouvant souffrir. Le roi, avec
cet air de surprise, demanda à M\textsuperscript{me} la duchesse de
Bourgogne si cela était vrai, qui le confirma, sur quoi le roi dit qu'il
n'y fallait donc pas songer.

Là-dessus il tira de sa poche une liste des duchesses, et s'arrêta à
M\textsuperscript{me} de Lesdiguières, veuve du vieux Canaples, dont
j'ai parlé en son lieu, et fille du duc de Vivonne, frère de
M\textsuperscript{me} de Montespan. C'était une personne de beaucoup de
douceur, de mérite, de vertu et d'infiniment d'esprit, de ce langage à
part si particulier aux Mortemart, mais qui de sa vie n'avait vu la cour
ni le monde, et qui vivait avec très-peu de bien dans une grande piété,
sans presque voir personne. D'Antin, son cousin germain et son ami
intime, en avait fort parlé au roi, qui en dit du bien, mais qu'elle ne
convenait pas à cause du jansénisme dont elle était un peu suspecte. Ce
fut un soliloque auquel il ne fut pas répondu un mot.

Mon érection suivant de fort près celle de Lesdiguières, le roi tomba
incontinent sur le nom de M\textsuperscript{me} de Saint-Simon, et dit
qu'il n'y voyait que celle-là à prendre dans toute la liste, qu'il
venait de parcourir des yeux. «\, Qu'en dites-vous, madame\,? en
s'adressant à M\textsuperscript{me} de Maintenon. Il m'en est toujours
revenu beaucoup de bien\,; je crois qu'elle conviendra fort.\,»
M\textsuperscript{me} de Maintenon répondit qu'elle le croyait aussi,
qu'elle ne la connaissoit point du tout, mais qu'on lui en avait
toujours dit toute sorte de bien et en tous genres, et jamais de mal sur
aucun. «\,Mais, ajouta-t-elle, voilà M\textsuperscript{me} la duchesse
de Bourgogne qui la connaît et qui vous en dira davantage.\,»
M\textsuperscript{me} la duchesse de Bourgogne répondit froidement, la
loua, mais conclut qu'elle ne savait pas si elle conviendrait bien.
«\,Mais pourquoi\,?» dit le roi, et pressa sur chaque qualité et sur
chaque louange qui avait été donnée, auxquelles toutes
M\textsuperscript{me} la duchesse de Bourgogne consentit, mais ajoutant
toujours qu'enfin elle ne croyait pas qu'elle convînt. Le roi surpris
insista sur l'esprit\,; et M\textsuperscript{me} la duchesse de
Bourgogne, qui ne voulait pas nuire à M\textsuperscript{me} de
Saint-Simon, mais seulement la servir à sa mode en écartant la place,
mollit sur l'esprit, comme moins important que les autres qualités\,;
sur quoi le roi, importuné des difficultés, répliqua qu'il n'en fallait
pas tant aussi, tant d'autres qualités se trouvant ensemble, et poussa
M\textsuperscript{me} la duchesse de Bourgogne au point qu'il lui
échappa qu'elle doutait qu'elle acceptât. Le roi, presque piqué, reprit
vivement\,: «\,Oh\,! pour refuser, non pas cela, quand on lui dira comme
il faut et que je le veux.\,» M\textsuperscript{me} la duchesse de
Bourgogne le pria de regarder encore dans sa liste, et dit qu'assurément
il y en trouverait qui conviendraient mieux. Le roi, avec action, la
repassa encore, et conclut qu'il n'y en avait du tout que
M\textsuperscript{me} de Saint-Simon, et qu'en un mot il fallait bien
qu'elle le fût. Peiné cependant de n'en point trouver d'autre, parce
qu'il crut que M\textsuperscript{me} la duchesse de Bourgogne ne voulait
point M\textsuperscript{me} de Saint-Simon, il lui demanda si elle avait
quelque chose contre elle. Elle lui répondit que non, mais de manière à
ne pas faire tout à fait cesser ce scrupule. Cette matière de dame
d'honneur en demeura là pour cette fois.

À ce récit Pontchartrain ajouta que, dès le moment de la déclaration du
mariage, tout le monde avait dit hautement que M\textsuperscript{me} de
Saint-Simon serait dame d'honneur, mais personne que nous le
désirassions, beaucoup que nous ne le voudrions pas, et quelques-uns
même que nous refuserions, et que depuis on n'avait parlé d'autre chose.
Il nous dit encore que M. {[}le duc{]} et M\textsuperscript{me} la
duchesse d'Orléans avaient affecté de répandre qu'ils m'avaient écrit et
dépêché à l'instant qu'ils avaient été assurés du mariage, et qu'ils ne
se cachaient point de toutes sortes d'efforts pour que
M\textsuperscript{me} de Saint-Simon fût dame d'honneur, jusque-là que
M. le duc d'Orléans lui avait dit franchement qu'il y faisait tous ses
cinq sens de nature, et que, lui ayant demandé s'il était sûr de mes
sentiments là-dessus, parce que m'exposer au refus était me perdre, M.
le duc d'Orléans lui avait répondu qu'il disait très-vrai, qu'il savait
bien que je ne voulais pas demander, mais que j'accepterais si on
voulait. Là-dessus, Pontchartrain, qui aimait à se mêler de tout,
quoiqu'en peine de n'avoir point de nos nouvelles et de la froideur de
M\textsuperscript{me} de Lauzun là-dessus, avait pressé les dames du
palais de nos amies d'exciter M\textsuperscript{me} la duchesse de
Bourgogne, qui avait répondu à M\textsuperscript{me} de Nogaret qu'elle
ne savait que faire, sachant ce qu'elle savait.

Pontchartrain se voulut mettre sur les remontrances. Je l'arrêtai fort
court par une sortie que je lui fis sur ce qu'il se mêlait toujours de
ce qu'il n'avait que faire\,; que le froid de M\textsuperscript{me} de
Lauzun et notre silence lui auraient dû faire comprendre nos sentiments,
puisque nous étions bien assez grands, M\textsuperscript{me} de
Saint-Simon et moi, pour nous aviser tout seuls qu'il fallait une dame
d'honneur, et pour écrire à lui et à nos amis si nous avions désiré
cette place. Il se voulut défendre sur ce que M. le duc d'Orléans lui
avait dit, sur quoi je répliquai qu'à ce que j'avais dit à
M\textsuperscript{me} la duchesse d'Orléans, qui ne pouvait ignorer, je
ne pouvais pas imaginer cette conduite ni ce bruit universel du monde si
sottement occupé. Des larmes de M\textsuperscript{me} de Saint-Simon lui
en dirent encore plus, en sorte que je ne vis jamais homme plus étonné.
Nous passâmes là-dessus dans le cabinet du chancelier, qui ne le fut
guère moins que son fils, quoiqu'il sût bien que nous ne voulions point
de la place, mais {[}qui fut surpris{]} des larmes et de ma colère. Il
nous répéta en peu de mots le fait passé chez M\textsuperscript{me} de
Maintenon, et il ajouta qu'il savait sûrement, qu'il y avait pensé avoir
depuis un ordre d'accepter. M\textsuperscript{me} de Saint-Simon,
outrée, lui répéta tout ce que nous avions fait pour éviter cette place,
ce que son fils qui était présent ignorait, et mes trois refus si
positifs et si nets à M\textsuperscript{me} la duchesse d'Orléans toutes
les trois seules fois qu'elle m'en avait parlé, sans que M. le duc
d'Orléans l'eût jamais osé une seule. Je m'exhalai fort là contre lui de
ce qu'il faisait là-dessus contre mon gré, qu'il ne pouvait ignorer\,;
et de ce qu'il avait dit que j'accepterais, ne pouvant douter du
contraire.

Le chancelier laissa exhaler la colère d'une part, les larmes de
l'autre, puis nous dit que les choses se trouvaient maintenant en tel
état qu'elles le faisaient changer d'avis\,; qu'il trouvait un péril si
certain au refus, et si peu réparable, qu'il n'y pouvait plus consentir.
Il nous fit sentir combien le roi y était peu accoutumé, combien il y
serait sensible\,; que ce crime à son égard serait par sa nature
irréparable, et toujours subsistant\,; que nous nous retrouverions dans
un état pire que jamais, et dans une disgrâce dont le roi se plairait et
s'appliquerait à nous faire porter tout le poids, à nous et aux nôtres,
en toutes choses\,; que plus il avait pensé, de lui-même, à
M\textsuperscript{me} de Saint-Simon\,; plus j'étais nouvellement bien
remis auprès de lui, dont ce choix était une grande marque, plus il
voyait M\textsuperscript{me} de Saint-Simon souhaitée de toutes les
parties intéressées, et unanimement nommée avec une approbation
générale, plus il se trouverait embarrassé d'en faire un autre, plus cet
autre lui serait étranger, incommode, forcé, plus il serait outré, et
plus il se plairait à appesantir sa vengeance\,; au lieu que, cédant de
bonne grâce à son goût et à sa volonté, toute notre répugnance, qu'il
connaissoit bien, nous tournerait à sacrifice, à gré, à distinction, et
à tout genre de bien\,; et qu'il n'y avait pas à balancer dans une
situation si extrême. Deux heures se passèrent dans cette consultation
et cette dispute, qui finit enfin par nous faire résoudre d'aller
coucher à Versailles, et, si nous ne pouvions doucement conjurer
l'orage, ne nous en pas laisser accabler par un refus qui nous perdrait
sans ressource. Nous partîmes donc de chez le chancelier. En chemin le
duc de Charost, qui revenait de Marly, nous arrêta, qui nous apprit à
peu près les mêmes choses, et que nos amis avaient chargé de nous dire
en arrivant qu'ils ne voyaient point de milieu entre refuser et nous
perdre.

Nous n'avions point de logement au château que cette chambre pour nous
tenir le jour, que le chancelier m'avait forcé de prendre chez lui,
depuis qu'à la chute de Chamillart nous avions rendu celui du duc de
Lorges. Nous allâmes donc descendre chez M\textsuperscript{me} de
Lauzun. M\textsuperscript{me} la duchesse de Bourgogne, qui avait
reconnu à la livrée un laquais, dans la salle des gardes où elle passait
en arrivant de Marly, l'avait appelé, et lui avait demandé à deux
reprises si M\textsuperscript{me} de Saint-Simon venait ce soir-là\,;
puis, jouant avec Monseigneur chez M\textsuperscript{me} la princesse de
Conti, où elle vit qu'on vint parler à M\textsuperscript{me} de Lauzun,
elle lui dit avec joie que nous étions apparemment arrivés, sur ce que
ce laquais lui avait dit. Le fait était qu'elle avait ordonné à
M\textsuperscript{me} de Lauzun, par quatre reprises, de demander à
M\textsuperscript{me} de Saint-Simon de sa part que, sur toutes choses,
elle ne manquât pas de se trouver à Versailles le soir même du retour de
Marly\,; que nous avisassions bien à ce que nous voudrions faire\,; que
la place de dame d'honneur lui serait offerte\,; et qu'elle et moi nous
étions perdus sans fond et sans ressource si nous la refusions. La
lettre n'était point arrivée par la négligence et la paresse des
valets\,; nous ne la sûmes que par le récit de M\textsuperscript{me} de
Lauzun, et sa surprise qu'elle se fût égarée.

Je ne répéterai point la colère, les larmes, les raisonnements. Nous
apprîmes là une chose nouvelle avec la confirmation des autres\,: c'est
que M\textsuperscript{me} la duchesse de Bourgogne, étant seule à Marly
dans sa chambre, avec les duchesses de Villeroy et de Lauzun et M. le
duc de Berry, à parler de l'affaire du jour, elle lui avait demandé
franchement qui il nommerait dame d'honneur si le choix lui en était
laissé. Il se défendit avec embarras. Pour le lever, ces deux dames
l'assurèrent qu'elles ne seraient point fâchées de lui en entendre
nommer une autre qu'elles, et le pressèrent de se déclarer. Enfin poussé
à bout, il dit sans balancer que M\textsuperscript{me} de Saint-Simon
était celle qu'il préférerait, et qu'il souhaitait uniquement.
M\textsuperscript{me} la duchesse de Bourgogne en dit autant après lui.
Tout cela pouvait être flatteur, mais nous tirait par le licou où nous
ne voulions pas. Il fallut aller voir M\textsuperscript{me} la duchesse
de Bourgogne dans ce cabinet des soirs de M\textsuperscript{me} de
Maintenon. À peine les deux soeurs y parurent qu'elles se trouvèrent
environnées. M\textsuperscript{me} la duchesse de Bourgogne, qui ne se
contraignit plus en public de son désir, joignit ses compliments aux
autres. M\textsuperscript{me} de Saint-Simon, dans l'embarras, répondait
qu'on se moquait d'elle\,; M\textsuperscript{me} la duchesse de
Bourgogne lui maintint que cela serait. Le souper du roi produisit
d'autres bordées. Pour les éviter je ne sortis point de chez
M\textsuperscript{me} de Lauzun de tout le soir. J'étais si piqué de ce
que Pontchartrain m'avait dit de M. le duc d'Orléans que j'eus besoin,
pour ne pas rompre avec lui, de toutes les considérations d'ancienne
amitié, de son intérêt pressant qui l'emportait, de la situation où je
me voyais sur le point d'être forcé d'entrer, qui m'approcherait d e
plus en plus de lui d'une manière indispensable.

Je le trouvai le lendemain marchant devant le roi qui allait à la messe.
Aussitôt il me joignit et me dit à l'oreille, pour la première fois de
sa vie qu'il m'en parla jamais\,: «\,Savez-vous bien qu'on parle fort de
vous pour nous ? --- Oui, monsieur, lui répondis-je d'un air
très-sérieux, et je l'apprends avec une extrême surprise, car rien ne
nous convient moins. --- Mais pourquoi\,? reprit-il avec embarras. ---
Parce que, lui repartis-je, puisque vous le voulez savoir, une seconde
place ne nous va et ne nous ira jamais. --- Mais refuserez-vous\,?
dit-il. --- Non, lui dis-je avec feu, parce que je ne suis pas comme le
cardinal de Bouillon (dont la félonie dont je parlerai venait d'être
consommée). Je suis sujet du roi et lui dois obéir\,; mais il faut qu'il
commande, et alors j'obéirai, mais ce sera avec la plus vive douleur
dont je sois capable, et que n'émoussera guère qu'à grand'peine votre
qualité de père de la princesse, et qui n'empêchera pas en nous une
amertume effroyable.\,» Avec ce dialogue nous avancions vers la
chapelle. Mgr le duc de Bourgogne, qui nous suivait sur les talons,
s'avança encore davantage pour écouter ce que mon émotion lui donnait
curiosité d'entendre, et souriait, car je tournai la tête et le vis. M.
le duc d'Orléans ne répliqua point. Mais mes réflexions augmentant à
mesure, je lui demandai, en approchant de la chapelle, s'il pensait au
moins à une dame d'atours raisonnable. Je craignais
M\textsuperscript{me} de Caylus à cause de sa tante et pour beaucoup
d'autres raisons\,; sur quoi, en la lui nommant, il me dit qu'il
espérait que ce ne serait pas elle.

L'entrée de la tribune mit fin à ce bizarre colloque. Après la messe je
montai chez M\textsuperscript{me} de Nogaret. Dès qu'elle me vit elle me
dit qu'elle en était dans l'impatience\,; que M\textsuperscript{me} la
duchesse de Bourgogne l'avait chargée de me parler sur la place de dame
d'honneur, et de me représenter telles et telles choses, les mêmes
qu'elle avait dites à M\textsuperscript{me} de Saint-Simon dans son
cabinet\,; surtout de me bien faire entendre que j'étais perdu à fond et
sans ressources, moi et les miens, si je refusais\,; que le roi savait
que je n'en voulais point\,; qu'après avoir cherché qui la pourrait
remplir, il n'en avait trouvé nulle autre que M\textsuperscript{me} de
Saint-Simon\,; qu'il était buté (ce fut le terme) à ce qu'elle
acceptât\,; et que non-seulement le dépit du refus me perdrait, mais la
nécessité encore de lui en faire choisir une autre qu'il ne trouvait
point, et de le forcer à la prendre désagréable et malgré lui\,; ce
qu'il ne me pardonnerait jamais, et se plairait à me faire sentir en
tout le poids de sa disgrâce. Alors M\textsuperscript{me} de Nogaret
m'avoua que M\textsuperscript{me} la duchesse de Bourgogne lui avait
raconté, à la fin de Marly, toute son audience à M\textsuperscript{me}
de Saint-Simon, et lui avait dit que, pressée par le roi à l'excès sur
M\textsuperscript{me} de Saint-Simon, elle n'avait pu en sortir, sans
mensonge ou sans lui nuire que par l'aveu de notre résolution au refus,
dont le roi s'était, conditionnellement, extrêmement irrité,
c'est-à-dire si nous y persistions, comme au contraire l'acceptation
ferait sur lui un effet tout différent.

Je contai à M\textsuperscript{me} de Nogaret tout ce qui s'était passé
là-dessus entre M\textsuperscript{me} la duchesse d'Orléans et moi, et
tout à l'heure encore entre M. le duc d'Orléans et moi, dont le mot
lâché que j'obéirais fit un grand plaisir à M\textsuperscript{me} de
Nogaret, dans l'aspect de l'extrême péril où elle nous voyait. En effet,
il était sans ressources de tous côtés, présents et futurs, parce que
tous s'étaient mis dans la tête cette place avec tant de volonté ou
d'intérêt, que le dépit du refus les aurait offensés tous à n'en jamais
revenir, et que Monseigneur, le seul d'eux qui n'y prenait point de
part, était conduit par tout ce qui m'était le plus contraire, et qui,
ravis du refus pour eux-mêmes, n'auraient point laissé de nous en faire
un crime auprès de lui. Les menaces ne pouvaient pas être plus
multipliées, mieux inculquées, ni venir plus nettement de la première
main\,; et il faut avouer que, dans la dépendance si totale où le roi
avait mis de lui tout le monde, c'eût été folie que s'opiniâtrer contre
une volonté si ferme, si entière, et encore si générale. Bientôt après
j'appris de la même M\textsuperscript{me} de Nogaret, que dans le
premier moment que M\textsuperscript{me} la duchesse de Bourgogne
l'aperçut depuis, elle lui avait demandé avec empressement si elle
m'avait vu et avec quel succès\,; qu'elle avait été ravie d'apprendre
que nous ne nous perdrions point\,; qu'elle se hâta de le dire au roi,
pour le tirer de peine, parce que rien ne le met en si aigre malaise que
la crainte d'être désobéi, et qu'il s'en sentit en effet très-soulagé et
à nous un gré infini.

L'après-dînée j'allai chez M\textsuperscript{me} la duchesse d'Orléans,
que je trouvai dans le cabinet de M. le duc d'Orléans avec lui. Dès
qu'elle me vit, elle me dit d'un air plein de joie qu'elle espérait
toujours qu'elle nous aurait. Je répondis, fort sérieux, qu'elle me
permettrait d'espérer jusqu'au bout le contraire\,; que le respect
m'empêchait de lui répéter ce que j'avais dit le matin à M. le duc
d'Orléans, que je croyais bien qui le lui avait rendu. Elle l'avoua et
s'en tint là. Je saisis cette occasion de lui en parler une bonne fois
pour toutes. Je lui dis donc qu'il était vrai que la seconde place nous
répugnait à l'excès, quelque adoucissement qu'y pût mettre la
considération que la princesse était leur fille\,; qu'indépendamment de
tant d'autres raisons qui nous rendaient cette place pesante, elle
n'était faite ni pour notre naissance ni pour notre dignité\,; que
M\textsuperscript{me}s de Ventadour et de Brancas, qui en avaient fait
l'étrange planche, avaient toutes les deux étonné le roi, la cour et le
monde, qui, à commencer par le roi, ne s'en était pas tu\,; mais {[}que
le roi{]} s'y était enfin accoutumé, et voulait sur ces exemples une
duchesse pour sa petite-fille\,; mais que M\textsuperscript{me}s de
Ventadour et de Brancas s'y étaient jetées toutes deux pour trouver du
pain qui leur manquait absolument, et plus encore pour trouver un asile
contre la persécution de leurs maris, l'un plus que jaloux, l'autre plus
qu'extravagant, deux motifs les plus pressants qui n'avaient, Dieu
merci, aucune application à nous, et qui, dans les autres de même
dignité, ne nous rendraient pas la chose meilleure. Elle essaya de
relever les différences d'être séparée de tout avec la belle-soeur du
roi, ou de se trouver de tout avec sa belle-petite-fille\,; de suivre
une princesse de l'âge de Madame, ou d'avoir la confiance, à l'âge de
M\textsuperscript{me} de Saint-Simon, d'être mise auprès d'une princesse
de celui de la future duchesse de Berry, et par tout ce qui se pouvait
dire avec le plus d'agrément et de flatterie. Je lui répétai qu'en un
mot c'était la seconde place, que rien ne pouvait rendre la première\,;
que j'espérerais jusqu'au bout que M\textsuperscript{me} de Saint-Simon
n'y serait point, mais qu'au cas que l'absolue nécessité de l'obéissance
l'y fît être, j'étais bien aise de lui dire une bonne fois ce qu'il nous
en semblait également à M\textsuperscript{me} de Saint-Simon et à moi,
pour qu'elle en fût bien instruite, et qu'il n'y fallût pas revenir,
parce que rien ne me paraissait si déplacé, ni si de mauvaise grâce, que
de chercher à faire sentir qu'on honore sa place, qu'on l'a à dégoût et
à mépris\,; qu'aussi, après tout ce que je prenais la liberté de lui en
dire, je ne lui en parlerais jamais plus\,; que M\textsuperscript{me} de
Saint-Simon, forcée de l'accepter, tâcherait d'en remplir les devoirs
comme si elle lui était agréable, et n'éviterait rien plus que d'imiter
la maréchale de Rochefort\,: c'est que la maréchale, qui croyait avec
raison honorer fort sa place de dame d'honneur de M\textsuperscript{me}
la duchesse d'Orléans, la désolait de plaintes et de reproches\,; et
puisque je voyais la chose devenir un \emph{faire-le-faut}, je voulus
éloigner la crainte de la même chose, après avoir montré tant de
répugnance et dit si franchement ce que nous en pensions. J'avais aussi
mêlé force reproches sur l'amitié de tout ce qu'ils avaient fait
là-dessus malgré notre résistance\,; et puisqu'il fallait vivre
désormais avec eux en liaison nécessaire et plus continuelle que jamais,
je crus de la sagesse de n'y arriver que sur le pied gauche, et de
hasarder brouillerie, qui ne ferait qu'ôter à une place désagréable en
soi tout ce qui d'ailleurs pouvait, autant qu'il était possible, réparer
notre dégoût, à quoi je voyais tout si entièrement disposé.
M\textsuperscript{me} la duchesse d'Orléans rit de l'exemple de sa dame
d'honneur, et ne se montra pas le moins du monde peinée de tant de dures
vérités, et sans que M. le duc d'Orléans eût mis un seul mot dans cette
conversation.

\hypertarget{chapitre-xv.}{%
\chapter{CHAPITRE XV.}\label{chapitre-xv.}}

1710

~

{\textsc{Motifs de la volonté si fort déterminée de faire
M\textsuperscript{me} de Saint-Simon dame d'honneur de
M\textsuperscript{me} la duchesse de Berry.}} {\textsc{- Menées pour
empêcher que cette place ne fût donnée à M\textsuperscript{me} de
Saint-Simon.}} {\textsc{- Leur inutilité singulière.}} {\textsc{-
M\textsuperscript{me} de Caylus arrogamment refusée pour dame d'atours
par M\textsuperscript{me} de Maintenon à Monseigneur.}} {\textsc{- Je
propose et conduis fort près du but M\textsuperscript{me} de Cheverny
pour dame d'atours.}} {\textsc{- Quelle elle était.}} {\textsc{-
Exhortations et menaces par le maréchal de Boufflers, avec tout l'air de
mission du roi.}} {\textsc{- Motifs qui excluent M\textsuperscript{me}
de Cheverny.}} {\textsc{- M\textsuperscript{me} de La Vieuville
secrètement choisie.}} {\textsc{- Inquiétude du roi d'être refusé par
moi.}} {\textsc{- Le roi me parle dans son cabinet, et y déclare
M\textsuperscript{me} de Saint-Simon dame d'honneur de la future
duchesse de Berry.}} {\textsc{- Sa réception du roi et des personnes
royales.}} {\textsc{- Je vais chez M\textsuperscript{me} de Maintenon\,;
son gentil compliment.}} {\textsc{- Assaisonnements de la place de dame
d'honneur.}} {\textsc{- La marquise de La Vieuville déclarée dame
d'atours de la future duchesse de Berry.}} {\textsc{- Sa naissance et
son caractère\,; et de son mari.}} {\textsc{- M. le duc d'Orléans
mortifié par l'Espagne.}} {\textsc{- Mouvements sur porter la queue de
la mante.}} {\textsc{- Facilité de M. le duc d'Orléans.}} {\textsc{-
Baptême de ses filles.}} {\textsc{- Fiançailles.}} {\textsc{- Mariage de
M. le duc de Berry et de Mademoiselle.}} {\textsc{- Festin où les
enfants de M. du Maine sont admis, ainsi qu'à la signature du contrat,
pour la première fois.}} {\textsc{- Le duc de Beauvilliers, comme
gouverneur, est préféré au duc de Bouillon, grand chambellan, à
présenter au roi la chemise de M. le duc de Berry.}} {\textsc{- Visite
et douleur de la reine et de la princesse d'Angleterre.}} {\textsc{-
M\textsuperscript{me} de Mare refuse obstinément d'être dame d'atours.}}
{\textsc{- Son traitement.}} {\textsc{- Causes de ce refus trop
sensées.}} {\textsc{- Tristes réflexions.}}

~

Il serait difficile de comprendre comment le roi et ces autres personnes
royales ne furent pas rebutés de nos refus, ni assez piqués pour passer
à un autre choix. On ne peut se dissimuler qu'elles ne se crussent une
espèce tout à fait à part du reste des hommes, continuellement induits
en cette douce erreur par les empressements, les hommages, la crainte,
l'espèce d'adoration qui leur étaient prodigués par tout le reste des
hommes, une ivresse de cour uniquement {[}appliquée{]} à tout sacrifier
pour plaire, surtout occupée à étudier, à deviner, à prévenir leurs
goûts, et au mépris de la raison et souvent de plus encore, à s'immoler
à eux par toutes sortes de flatteries, de bassesses et d'abandon. Il
était donc fort surprenant de voir des personnes si absolues et si
accoutumées à voir tout ramper sous leurs pieds, prévenir leurs moindres
désirs, s'opiniâtrer jusqu'à cet excès à nous faire accepter une place
qui faisait l'envie générale, jusqu'à remuer tant de machines en menaces
et en flatteries pour ne nous pas livrer à un ressentiment, qui en toute
autre occasion aurait eu le plus prompt effet. Mais un motif puissant
avait emporté toute autre considération.

Le roi avait envie d'approcher M\textsuperscript{me} de Saint-Simon de
sa cour particulière, dès lors que M\textsuperscript{me} de La Vallière
eut la place de dame du palais à la mort de M\textsuperscript{me} de
Montgon. Nous sûmes depuis que ce qui l'avait empêché d'en disposer
pendant six semaines fut qu'il la destinait à M\textsuperscript{me} de
Saint-Simon, et qu'il espéra par ce délai lasser M\textsuperscript{me}
la duchesse de Bourgogne, qui, entraînée par les Noailles et par des
raisons de femmes de leur âge, fit tant d'instance pour obtenir
M\textsuperscript{me} de La Vallière, qu'à la fin le roi s'y rendit.
Heureusement que j'avais demandé cette place, parce qu'il se publia sur
notre résistance à celle-ci, que je trouvais même celle des dames du
palais au-dessous des duchesses. L'imputation était pitoyable. La reine
en avait eu plusieurs, elle avait eu encore M\textsuperscript{lle}
d'Elboeuf, M\textsuperscript{me} d'Armagnac, la princesse de Bade, fille
d'une princesse du sang, femme d'un souverain d'Allemagne, qui dans leur
service de dames du palais ne différaient en rien des autres, sans
préférence, sans distinction, mêlées avec les dames du palais duchesses,
et sans dispute ni prétentions de rang, en toute égalité ensemble. Outre
cette bonne volonté, le roi, à qui la seule complaisance mêlée de la
crainte de la cabale de M\textsuperscript{me} la Duchesse avait fait
vouloir le mariage qui approchait les bâtards de M. le duc de Berry (et
c'en était là le grand et secret ressort), au même degré qu'eût fait
celui de M\textsuperscript{lle} de Bourbon, ne le voulait accompagner
que de choses agréables à ceux qui l'y avaient induit et utiles à leurs
intérêts. Rien ne leur était plus important que d'avoir dans cette place
une personne dont la vertu de tout temps sans atteinte, le bon esprit,
le sens et les inclinations fussent de concert pour une éducation
désirable.

Il faut que cette vérité m'échappe\,: il n'y avait point de femme qui
eût jamais mérité ni joui d'une réputation plus pleine, plus unanimement
reconnue, ni plus solide que M\textsuperscript{me} de Saint-Simon, sur
tout ce qui forme le mérite des plus honnêtes et des plus vertueuses. Il
n'y en avait point aussi qui en usât avec plus de douceur et de
modestie, ni qui fût plus généralement respectée dans cet âge où elle
était\,; ni avec cela plus aimée\,; jusque-là que les jeunes femmes les
moins retenues n'en pensaient pas autrement et n'en avaient pas même de
crainte, malgré la distance des moeurs et de la conduite. Sa piété
solide, et qui ne s'était affaiblie en aucun temps, n'étrangeait
personne, tant on s'en apercevait peu et tant elle était uniquement pour
elle. Tant de choses ensemble, et si rares, remplissaient avec abondance
toutes les vues de l'éducation, et suppléaient avantageusement au nombre
des années. La naissance, les alliances, les entours, les noms, la
dignité flattaient extrêmement l'orgueil et l'amour-propre, en sorte
qu'il ne se trouvait en ce choix quoi que ce pût être qui ne satisfît
pleinement en tout genre.

L'intimité qui me liait à M. le duc et à M\textsuperscript{me} la
duchesse d'Orléans, les services que je leur avais rendus, la part que
j'avais eue au mariage, rendaient ce choix singulièrement propre. La
bonté très-marquée de M\textsuperscript{me} la duchesse de Bourgogne, et
son désir pour M\textsuperscript{me} de Saint-Simon, mon attachement
pour Mgr le duc de Bourgogne qu'on sentait dès lors n'être pas ingrat,
ma liaison plus qu'intime avec tout ce qui environnait le plus
principalement et le plus intérieurement ce prince, ajoutaient
infiniment à toute convenance. Ce qui y mettait le sceau était ma
situation de longue main si éloignée de M\textsuperscript{me} la
Duchesse et de toute cette cour intérieure de Mlonseigneur, que venait
de combler la part qu'ils ne savaient que trop, comme j'aurai bientôt
occasion de le dire, que j'avais eue à l'exclusion de
M\textsuperscript{lle} de Bourbon et à la fortune de Mademoiselle. Il ne
leur pouvait rester d'espérance que d'avoir occasion de tomber sur la
nouvelle fille de France, et alors il importait au dernier point à tout
ce qui la faisait telle d'avoir auprès d'elle une dame d'honneur qui,
non-seulement eût les qualités requises à l'emploi, mais qui fût encore
incapable, quoi qu'il pût arriver de radieux dans les suites à
M\textsuperscript{me} la Duchesse et à cette cabale, de s'en laisser
entamer à quelques intérêts particuliers que ce pût être, et c'est ce
qui ne pouvait se rencontrer en nulle autre avec la même sûreté, tant
par la vertu et la probité de M\textsuperscript{me} de Saint-Simon, que
par un éloignement personnel si peu capable d'aucun changement entre
nous et cette cabale. Ce furent, à ce que j'ai toujours cru, ces
puissantes raisons qui portèrent M. {[}le duc{]} et
M\textsuperscript{me} la duchesse d'Orléans à ne se rebuter de rien et à
pousser, s'il faut user de ce terme, l'acharnement jusqu'où il pouvait
aller pour emporter M\textsuperscript{me} de Saint-Simon.

M\textsuperscript{me} la duchesse de Bourgogne, dans sa situation avec
M\textsuperscript{me} la Duchesse et cette cabale telle qu'elle a été
montrée, comblée par ce mariage, qui était de plus son ouvrage, avait
les mêmes raisons, et de plus celles de son aisance, comme elle ne
l'avait pas caché à M\textsuperscript{me} de Saint-Simon. Ce qui
environnait Mgr le duc de Bourgogne avec le plus de poids pensait peu
différemment, parce que les éloignements et les intérêts étaient les
mêmes. Le roi, avec son ancienne prévention que rien n'avait détruite
depuis l'affaire de la dame du palais, pressé par les menées de
M\textsuperscript{me} la duchesse d'Orléans, sûr que
M\textsuperscript{me} de Saint-Simon était au moins très-agréable à
M\textsuperscript{me} la duchesse de Bourgogne, instruit peut-être par
ce que j'ai rapporté du maréchal de Boufflers de toute la part que
j'avais eue à la séparation de M. le duc d'Orléans avec
M\textsuperscript{me} d'Argenton, qui sûrement avec sa mémoire n'avait
pas oublié ce que je lui avais dit sur feu M. le Duc en l'audience du
mois de janvier que j'ai racontée, accoutumé au visage de
M\textsuperscript{me} de Saint-Simon par les Marlys et par la voir
souvent à la suite de M\textsuperscript{me} la duchesse de Bourgogne,
choses d'habitude qui lui faisaient infiniment, tout cela forma un amas
de raisons qui non-seulement le déterminèrent, mais le décidèrent, et
une fois déclaré et averti du refus en poussant à bout
M\textsuperscript{me} la duchesse de Bourgogne, il se piqua de n'avoir
pas cette espèce de démenti, et il voulut si fermement être obéi qu'il
en vint jusqu'à prodiguer les menaces et à nous en faire avertir de tous
côtés. Je dis faire avertir, par le lieu qu'il y donna exprès à
plusieurs reprises et peut-être, comme on le verra bientôt, par quelque
chose de plus fort.

Il ne fallait pas moins qu'un aussi puissant groupe de choses et
d'intérêts pour l'emporter sur le dépit de nos refus et sur tout l'art
qui fut mis en oeuvre pour les seconder, et que je découvris peu de
jours après que M\textsuperscript{me} de Saint-Simon fut déclarée.
M\textsuperscript{me} la Duchesse, d'Antin et toute cette cabale intime
outrée du mariage, s'échappèrent à dire que tout était perdu si
M\textsuperscript{me} de Saint-Simon était dame d'honneur, soit qu'ils
regardassent à l'importance d'y avoir quelqu'un dont ils pussent faire
usage, au moins qui pût être accessible, enfin neutre, s'ils ne
pouvaient mieux. Ils considérèrent comme un coup de partie de l'empêcher
de l'être. Les prétendants et les curieux de cour, qui regardaient cette
place d'un autre oeil que nous ne faisions, et qui pour eux ou pour les
leurs l'ambitionnaient, les ennemis dont on ne manque jamais, tous
enfin, occupés de la crainte que cette place ne me frayât chemin à
mieux, se distillèrent l'esprit à travailler à la détourner. Faute de
mieux, ils cherchèrent une ressource dans l'exactitude de la vie de
M\textsuperscript{me} de Saint-Simon\,: ils furetèrent de quel côté elle
penchait, qui était son confesseur, et ils se crurent assurés de
l'exclure, lorsqu'ils eurent découvert que c'était depuis longues années
M. de La Brue, curé de Saint-Germain de l'Auxerrois, mis en place et
protégé par le cardinal de Noailles, et qui passait pour suspect de
jansénisme.

Ce crime, auprès du roi, était le plus irrémissible et le plus
certainement exclusif de tout. Être de la paroisse de Saint-Sulpice,
passer sa vie à la cour, n'avoir jamais cessé d'être dans la piété,
quoique sans enseigne, et ne se confesser, ni à sa paroisse de
Saint-Sulpice, ni à Versailles, ni aux jésuites, et aller de tout temps
à ce curé étranger et si suspect, leur parut une preuve complète qu'ils
surent bien faire valoir. Leur malheur voulut que cette accusation
portée au roi le trouva si décidé pour M\textsuperscript{me} de
Saint-Simon, qu'elle ne fit que l'alarmer, lui à qui il n'en aurait pas
fallu davantage pour ne vouloir jamais ouïr parler de ce choix, bien
qu'arrêté, s'il s'en était moins entêté, ce qui lui était entièrement
inusité\,; et, sans perquisition, l'affaire aurait été finie\,: ce qui
avait rompu le cou à bien des gens qui ne se doutaient pas du comment ni
du pourquoi, et ce qui était avec lui d'une expérience certaine. On
n'oublia rien pour réaliser les soupçons sur le curé, mais on ne trouva
que de la mousse qui ne put prendre. On fit toutefois tout l'usage qu'on
put de ces choses. Le roi s'en alarma, mais ce fut tout, et voulut
s'éclaircir contre sa coutume en ce genre. Il s'adressa au P. Tellier,
et il ne pouvait consulter un plus soupçonneux ennemi du plus léger
fantôme.

Le P. Tellier était assuré sur mon compte par mon ancienne confiance au
P. Sanadon, son ami et de même compagnie\,; il savait par lui dans
quelle union nous vivions, M\textsuperscript{me} de Saint-Simon et moi,
depuis le jour de notre mariage. Il était dans la bouteille avec moi de
celui que nous avions fait réussir\,; il me courtisait comme j'ai
commencé ailleurs à en dire quelque chose, par rapport à Mgr le duc de
Bourgogne et à ses plus intimes entours avec lesquels il me savait
indissolublement lié depuis que j'étais à la cour\,; il glissa donc avec
le roi sur le sieur de La Brue, dont il ne dit pas grand bien, mais sans
rien de marqué, parce qu'il n'y ayait pas matière\,; il répondit
nettement de moi et, par moi, de M\textsuperscript{me} de Saint-Simon,
parce qu'il savait que nous étions unis en toutes choses. Il affermit le
roi dans le choix qu'il avait résolu, et l'assura qu'en tout genre il
n'y en avait point de si bon à faire, tellement que le poison se tourna
en remède, et que ce qui avait été si malignement présenté pour exclure
M\textsuperscript{me} de Saint-Simon de cette place, et par le genre
d'accusation de toute espérance et de tout agrément, opéra précisément
le contraire.

Je ne sus que longtemps après par M. le duc d'Orléans cette ferme parade
du P. Tellier. Il eut peine et à me l'avouer et à me la dissimuler pour
ne pas trop découvrir cette espèce d'inquisition, pourtant fort connue
déjà, et pour ne pas perdre aussi le mérite qu'il s'était acquis auprès
de moi, d'autant plus grand que je ne pouvais le deviner, et que, sans
ce bon office, nous nous trouvions perdus de nouveau sans savoir
pourquoi, et sûrement sans retour. On peut juger de la rage de la cabale
de manquer un coup si à plomb pour toujours et si continuellement
certain. Nous eûmes bien quelque vent, avant la déclaration de la place,
mais fort superficiellement, de ces manéges. Le curé de Saint-Germain,
peu curieux de pénitentes considérables, mais attaché d'estime à
M\textsuperscript{me} de Saint-Simon, tâcha de lui persuader de le
quitter, par la considération des effets pour toute la vie, et sans
ressource, de ce genre de soupçon\,; mais aucune n'entra là-dessus dans
son esprit ni dans le mien, persuadés l'un et l'autre de la liberté et
de la simplicité avec lesquelles on doit se conduire en choses
spirituelles, qui ne doivent jamais tenir aux temporelles, beaucoup
moins en dépendre. Depuis sa nomination on lui fit des attaques
indirectes pour changer de confesseur, qui ne durèrent guère, parce
qu'elle en fît doucement mais fermement sentir l'inutilité. Elle n'en a
jamais eu d'autre tant que ce sage et saint prêtre a vécu, près de
quarante ans depuis. Tel est l'usage des partis de religion quand les
princes s'en mêlent.

Notre parti enfin amèrement pris, après tout ce que j'ai raconté, de
céder à la violence, nous commençâmes à penser à éviter une dame
d'atours avec qui il aurait fallu compter. M\textsuperscript{me} de
Caylus était, à cause de sa santé, la seule de cette sorte. Elle avait
précisément toutes les raisons contraires à celles qui déterminaient au
choix de M\textsuperscript{me} de Saint-Simon\,; de tout temps liée avec
M\textsuperscript{me} la Duchesse, et, dans les derniers, autant que les
défenses de sa tante lui en pouvaient laisser de liberté\,; insinuée par
cette princesse et par Harcourt, son cousin, assez avant auprès de
Monseigneur pour s'en faire une ressource pour l'avenir, et un appui
même pour le présent s'il arrivait, faute de sa tante. Cela était bien
éloigné de ce que, pour abréger, je dirai toute notre cabale.
M\textsuperscript{me} la duchesse de Bourgogne de plus la craignait et
ne la pouvait souffrir, excitée peut-être par la jalousie brusque et
franche de la duchesse de Villeroy du goût toujours subsistant de son
mari pour elle, bien que commencé longtemps avant son bail, et dont
l'éclat avait fait chasser M\textsuperscript{me} de Caylus de la cour.
M\textsuperscript{me} la duchesse d'Orléans avait bien compris qu'elle
penserait à cette place, et à cause de M\textsuperscript{me} de
Maintenon, se trouvait embarrassée de lui en barrer le chemin,
quoiqu'elle ne se fût encore pu déterminer à personne.

Cet embarras ne fut pas long\,: elle m'apprit qu'aussitôt que le mariage
fut déclaré, Monseigneur avait parlé à M\textsuperscript{me} de
Maintenon en sa faveur pour cette place, que M\textsuperscript{me} de
Maintenon fut outrée de ce détour de sa nièce qui, au lieu de lui parler
elle-même, avait cru l'emporter par une recommandation de ce poids en ce
genre, et que dans sa colère il lui était échappé de dire qu'elle
voulait bien que Monseigneur sût que, si elle eût voulu que
M\textsuperscript{me} de Caylus eût une place, elle avait bien assez de
crédit pour y réussir sans lui\,; mais qu'il ne lui arriverait jamais de
la laisser mettre dans aucune après la vie qu'elle avait menée, pour se
donner le ridicule de faire dire qu'elle mettait sa nièce auprès d'une
jeune princesse pour la former à ce qu'elle avait pratiqué, et à ce qui
l'avait fait chasser avec éclat. Ce propos, pour une dévote soi-disant
repentie, s'oubliait un peu de la poutre dans l'oeil et du fétu de
l'Évangile. M\textsuperscript{me} de Caylus qui le sut, et cela n'avait
pas été dit à autre dessein, en tomba malade. N'osant plus rien tenter,
ni espérer là-dessus, ni même témoigner son chagrin à sa tante, elle
s'en dédommagea secrètement avec ses plus intimes par les plaintes les
plus amères.

La pensée me vint de faire dame d'atours la femme de Cheverny, duquel
j'ai parlé plus d'une fois, et qui était fort de mes amis. La naissance
et la place du mari auprès de Mgr le duc de Bourgogne, et les entours si
proches de la femme avaient de quoi satisfaire du côté de l'orgueil, et
le reste était à souhait. La femme était fille du vieux Saumery et d'une
soeur de M. Colbert, cousine germaine par conséquent, et en même temps
fort amie des duchesses de Chevreuse et de Beauvilliers. Avec cela
rompue au monde, quoique toujours dans Versailles elle allât fort peu\,;
beaucoup d'esprit et de sens, de l'agrément dans la conversation, et qui
avait très-bien réussi à Vienne et à Copenhague, où son mari avait été
envoyé, et ambassadeur. J'en parlai à M. et à M\textsuperscript{me} de
Beauvilliers qui, à la vue du danger, avaient été fort ardents à nous
faire résoudre d'accepter. Ils furent ravis de ma pensée, qui d'ailleurs
entrait dans leur projet d'unir étroitement la future duchesse de Berry
à Mgr {[}le duc{]} et à M\textsuperscript{me} la duchesse de Bourgogne,
à quoi il était important de former cette nouvelle cour des gens
principaux qui eussent les mêmes vues. Ils n'étaient pas même
indifférents qu'elle se composât de gens fort à eux autant que cela se
pourrait sans paraître par leur maxime d'embrasser tout, pourvu que cela
ne leur coûtât rien du tout, et qu'on ne s'en aperçut pas. Dès que ce
choix fut résolu entre nous et M. et M\textsuperscript{me} de Chevreuse,
j'en parlai à M. {[}le duc{]} et à M\textsuperscript{me} la duchesse
d'Orléans. Ils s'étaient servis de Cheverny pour sonder Monseigneur par
du Mont. Quoique cela n'eût pas réussi, le gré en était demeuré, de
sorte que M\textsuperscript{me} de Cheverny fut aussitôt acceptée que
proposée. M\textsuperscript{me} la duchesse de Bourgogne y entra fort
dès le lendemain, à qui M\textsuperscript{me} la duchesse d'Orléans et
M\textsuperscript{me} de Lévi en parlèrent, et la résolution en fut
prise tout de suite entre M\textsuperscript{me} la duchesse de Bourgogne
et M\textsuperscript{me} de Maintenon. Cheverny, quoique vieux et sans
enfants, y consentit avec joie par le goût et l'habitude de la cour.
Jamais partie ne fut si promptement et si bien liée.

Cela fait, nous comptâmes tout devoir plus que rempli d'avoir cédé et
demeuré trois jours à Versailles, où nous ne pouvions paraître nulle
part sans essuyer de fâcheux compliments. Je dis à M. {[}le duc{]} et à
M\textsuperscript{me} la duchesse d'Orléans, et nous fîmes dire aussi à
M\textsuperscript{me} la duchesse de Bourgogne que nous n'y pouvions
plus tenir, et nous nous en retournâmes à Paris la veille de la
Pentecôte, où nous barricadâmes bien notre porte et où
M\textsuperscript{me} de Saint-Simon se trouva fort incommodée de tous
ces chagrins et d'une si étrange violence. Au bout de huit jours,
persécuté par nos amis, je retournai seul à Versailles. Au bout du pont
de Sèvres, le maréchal de Boufflers qui revenait à Paris m'arrêta, et me
fit mettre pied à terre pour me parler à l'écart. Il m'avait écrit le
matin que mon absence de la cour ne pouvait plus se soutenir sans être
de très-mauvaise grâce. Il me confirma la même chose, puis me témoigna
que le roi était en peine si j'obéirais\,; que cette inquiétude le
blessait toujours, quoique M\textsuperscript{me} la duchesse de
Bourgogne lui eût dit, et de là se mit à m'exhorter comme sur une chose
nouvelle, et à me faire entendre nettement qu'un refus me perdrait sans
ressource, et avec des tons et des airs de réticence si marqués, et
toujours ajoutant qu'il savait bien ce qu'il disait, et qu'il savait
bien pourquoi il me le disait, que je ne doutai point que le roi ne l'en
eût expressément chargé. Le maréchal savait que j'étais enfin résolu\,;
il me rencontrait allant à Versailles, pourquoi il m'avait écrit\,; il
n'avait donc rien à me dire, pourquoi donc m'arrêter, m'exhorter, me
menacer\,? car il me dit encore qu'on m'enverrait si loin et si mal à
mon aise que j'aurais de quoi me repentir longtemps-Pourquoi tout ce
propos, désormais inutile, avec cette inquiétude du roi s'il n'avait pas
eu ordre de lui de le faire, et de s'assurer bien de l'obéissance qu'il
craignait tant de hasarder\,? Je sus à Versailles que ce qui retenait la
déclaration de la dame d'honneur était l'indétermination sur la dame
d'atours. M\textsuperscript{me} de Saint-Simon n'osa demeurer à Paris
que peu de jours après moi. Nous étions cependant fort mal à notre aise
parmi les divers regards, les propos différents, et sûrement les mauvais
offices qui pleuvent toujours sur les personnes du jour. Cela me
détermina à presser M. {[}le duc{]} et M\textsuperscript{me} la duchesse
d'Orléans de faire finir ces longueurs importunes. La dame d'atours
était toujours le rémora\,; M\textsuperscript{me} la duchesse de
Bourgogne et M\textsuperscript{me} de Maintenon s'étaient butées pour
M\textsuperscript{me} de Cheverny.

Avec tout son mérite elle avait un visage dégoûtant, dont le roi, qui se
prenait fort aux figures, ne se pouvait accommoder. Elle et son mari
avaient essuyé le scorbut en Danemark, dont peu de gens du pays et
beaucoup moins d'étrangers échappent. Ils y avaient laissé l'un et
l'autre presque toutes leurs dents, et eussent peut-être mieux fait de
n'en rapporter aucune. Ce défaut, avec un teint fort couperosé, faisait
quelque chose de fort désagréable dans une femme qui n'était plus jeune,
et qui avait pourtant une physionomie d'esprit. En un mot, ce fut un
visage auquel le roi qui en était fort susceptible ne put jamais
s'apprivoiser. C'était son unique contredit qui n'en eût pas été un pour
tout autre que le roi. M\textsuperscript{me} de Maintenon et
M\textsuperscript{me} la duchesse de Bourgogne, qui ne voulaient qu'elle
et qui à force de barrer toute autre avaient compté de surmonter cette
fantaisie, s'y trompèrent. À force d'attention à saisir toute occasion
de lui parler en faveur de M\textsuperscript{me} de Cheverny elles
achevèrent de l'éloigner. Il s'imagina une cabale en sa faveur\,;
c'était la chose qu'il haïssait le plus, qu'il craignait davantage et où
il était le plus continuellement trompé. Il le dit même nettement à
M\textsuperscript{me} de Maintenon et à M\textsuperscript{me} la
duchesse de Bourgogne, qui ne purent jamais lui en ôter l'idée.
Finalement, lassé de ce combat, il leur déclara qu'il ne pouvait
supporter d'avoir toujours le visage de M\textsuperscript{me} de
Cheverny à sa suite, et souvent à sa table et dans ses cabinets, et se
détermina au choix de M\textsuperscript{me} de La Vieuville, qui fut en
même temps résolu.

Dès que cela fut fait, il voulut déclarer le choix de
M\textsuperscript{me} de Saint-Simon, et il le déclara le dimanche matin
15 juin. M. le duc d'Orléans me dit à la fin de la messe du roi qu'il
l'allait faire, et deux heures après il me conta qu'avant la messe,
étant avec le roi et Monseigneur dans les cabinets à parler de cela, le
roi lui avait encore demandé avec un reste d'inquiétude\,: «\,Mais votre
ami, je le connais, il est quelquefois extraordinaire, ne me
refusera-t-il point\,?» que, rassuré sur ce qu'il lui avait dit de ma
comparaison du cardinal de Bouillon, le roi avait parlé de ma vivacité
sur diverses choses vaguement, mais avec estime, néanmoins comme
embarrassé à cet égard et désirant que j'y prisse garde, ce qu'il ne dit
à son neveu sûrement que pour que cela me revînt\,; que Monseigneur
avait parlé de même, mais honnêtement\,; que lui, saisissant l'occasion,
avait dit que depuis qu'il était question de cette place, il ne doutait
point qu'on ne m'eût rendu de mauvais offices comme lors de l'ambassade
de Rome, sur quoi le roi avait répondu avec ouverture que c'était la
bonne coutume des courtisans. Là-dessus ils allèrent à la messe.

En revenant de la messe, le roi m'appela dans la galerie, et me dit
qu'il me voulait parler, et de le suivre dans son cabinet. Il s'y avança
à une petite table contre la muraille, éloigné de tout ce qui était dans
ce cabinet, le plus près de la galerie par où il était entré. Là il me
dit qu'il avait choisi M\textsuperscript{me} de Saint-Simon pour être
dame d'honneur de la future duchesse de Berry\,; que c'était une marque
singulière de l'estime qu'il avait de sa vertu et de son mérite, de lui
confier, à trente-deux ans, une princesse si jeune et qui lui était si
proche, et une marque aussi qu'il était tout à fait persuadé de ce que
je lui avais dit, il {[}y{]} avait quelques mois, de m'approcher si fort
de lui. Je fis une révérence médiocre et répondis que j'étais touché de
l'honneur de la confiance en M\textsuperscript{me} de Saint-Simon à son
âge, mais que ce qui me faisait le plus de plaisir était l'assurance que
je recevais de Sa Majesté qu'elle était persuadée et contente. Après
cette laconique réponse, qui en tout respect lui laissait sentir ce que
je sentais moi-même de la place, il me dit assez longtemps toutes sortes
de choses obligeantes sur M\textsuperscript{me} de Saint-Simon et moi,
comme il savait mieux faire qu'homme du monde lorsqu'il savait gré, et
qu'il présentait surtout un fâcheux morceau qu'il voulait faire avaler.
Puis, me regardant plus attentivement avec un sourire qui voulait
plaire\,: «\,Mais, ajouta-t-il, il faut tenir votre langue,\,» d'un ton
de familiarité qui semblait en demander de ma part, avec lequel aussi je
lui répondis que je l'avais bien tenue, et surtout depuis quelque temps,
et que je la tiendrais bien toujours. Il sourit avec plus
d'épanouissement encore, comme un homme qui entend bien, qui est soulagé
de n'avoir pas rencontré la résistance qu'il avait tant appréhendée, et
qui est content de cette sorte de liberté qu'il a trouvée, et qui lui
fait mieux goûter le sacrifice qu'il sent sans en avoir les oreilles
blessées. En même temps il se tourna le dos à la muraille, qu'il
regardait auparavant, un peu vers moi et moi vers lui\,; et d'un ton
grave et magistral, mais élevé, il dit à la compagnie\,:
«\,M\textsuperscript{me} la duchesse de Saint-Simon est dame d'honneur
de la future duchesse de Berry.\,» Aussitôt chorus d'applaudissement du
choix et de louange de la choisie\,; et le roi, sans parler de dame
d'atours, passa dans ses cabinets de derrière.

À l'instant j'allai à l'autre bout du cabinet vers Monseigneur, qui de
Meudon y était venu pour le conseil, et lui dis, en m'inclinant
faiblement, que je lui faisais là ma révérence en attendant que je pusse
m'en acquitter à Meudon. Il me répondit, mais froidement en me saluant,
qu'il était fort aise de ce choix, et que M\textsuperscript{me} de
Saint-Simon ferait fort bien. Je voulus aller ensuite à Mgr le duc de
Bourgogne qui était éloigné, mais il fit la moitié du chemin, où sans me
laisser le loisir de parler, il me dit avec épanouissement, et me
serrant la main, que je savais combien il avait toujours pris et prenait
part en moi, que rien n'était plus de son goût que ce choix\,; et me
comblant de bontés et M\textsuperscript{me} de Saint-Simon d'éloges, me
mena au bout du cabinet, où je me tirai à peine d'avec ce qui y était
assemblé sur mon passage. J'eus plutôt fait de sortir par la porte de la
galerie qu'on m'ouvrit\,; puis, songeant que le chancelier était dans la
chambre du roi avec les ministres, attendant le conseil, j'allai lui
dire ce qu'il venait de se passer, car pour M. de Beauvilliers il y
avait été présent. Je fus suffoqué de toute la nombreuse compagnie,
comme il arrive en ces occasions. Je m'en dépêtrai avec peine et
politesse, mais avec sérieux, dédaignant jusqu'au bout de montrer une
joie que je n'avais point, comme j'avais soigneusement évité tout terme
de remercîment avec le roi et Monseigneur, et comme je l'évitai avec
tous, de la réception la plus empressée desquels je ne parlerai pas.

Je mandai aussitôt à M\textsuperscript{me} de Saint-Simon qu'elle était
nommée et déclarée. Cette nouvelle, quoique si prévue, la saisit presque
comme si elle ne l'eût pas été. Après avoir un peu cédé aux larmes, il
fallut faire effort et venir s'habiller chez la duchesse de Lauzun, où
malgré les précautions, les portes furent souvent forcées. Les deux
soeurs allèrent chez M\textsuperscript{me} la duchesse de Bourgogne qui
était à sa toilette, fort pressée d'aller dîner à Meudon, où, non sans
cause, Monseigneur lui reprochait souvent d'arriver tard. L'accueil
public fut tel qu'on le peut juger, celui de M\textsuperscript{me} la
duchesse de Bourgogne admirable. En se levant pour aller à la messe,
elle l'appela, la prit par la main, et la mena ainsi jusqu'à la tribune.
Elle lui dit que, quelque joie qu'elle eût de la voir où elle la
désirait, elle voulait qu'elle fût persuadée qu'elle l'avait servie
comme elle l'avait souhaité\,; que pour cela elle lui avait fait le plus
grand sacrifice qu'il fût possible de lui faire, parce que, la désirant
passionnément, elle avait mis tout en usage pour en détourner le roi,
jusque-là même qu'il avait cru un temps qu'elle avait quelque chose
contre elle\,; qu'à la vérité elle avait été fort embarrassée, parce que
l'aimant trop et la vérité aussi pour lui vouloir nuire, et ayant sur
elle le dessein dont elle lui avait parlé de la faire succéder à la
duchesse du Lude, elle n'avait trop su qu'alléguer pour empêcher le roi
de lui donner une place qu'il lui avait destinée\,; que néanmoins elle
n'avait rien oublié pour lui tenir parole jusqu'au bout, parce qu'il
faut servir ses amis à leur mode et pour eux, non pour soi-même, ce fut
son expression\,; qu'au surplus elle l'avait fait avertir de notre perte
qu'elle voyait certaine par un refus\,; qu'elle était très-aise que nous
nous fussions rendus capables de croire conseil là-dessus\,; qu'enfin,
puisque la chose était faite, elle ne pouvait lui en dissimuler sa joie,
d'autant plus librement que, encore une fois, elle lui répondait avec
vérité qu'elle avait fait contre son gré tout ce qu'elle avait pu
jusqu'à la fin pour détourner cette place d'elle, uniquement pour lui
tenir parole\,; que maintenant que la chose avait tourné autrement, elle
en était ravie pour soi, pour la princesse auprès de laquelle on la
mettait, et pour elle-même, parce qu'elle croyait que cela nous était
bon et nous porterait de plus en plus à des choses agréables et
meilleures.

Tout ce long chemin se passa en pareilles marques de bonté et d'amitié,
parmi lesquelles la princesse parlant toujours, M\textsuperscript{me} de
Saint-Simon eut peine à lui en témoigner sa reconnaissance.
M\textsuperscript{me} la duchesse de Bourgogne finit par lui dire
qu'elle l'aurait menée chez le roi sans l'heure qu'il était, où elle
était attendue à Meudon\,: Madame se mit à pleurer de joie en voyant
entrer M\textsuperscript{me} de Saint-Simon chez elle. Elle l'avait
toujours singulièrement estimée, quoique sans autre commerce que celui
d'une cour rare. Elle n'avait pu se tenir de lui dire à un souper du
roi, lorsque M\textsuperscript{me} de La Vallière fut dame du palais,
qu'elle en était outrée, mais qu'elle avait toujours bien cru qu'ils
n'auraient pas assez bon sens pour lui donner cette place.
M\textsuperscript{me} de Saint-Simon ne vit point M. {[}le duc{]} et
M\textsuperscript{me} la duchesse d'Orléans chez eux, ils étaient déjà
chez Mademoiselle, où elle les trouva. L'allégresse y fut poussée aux
transports. Mademoiselle dit même en particulier à M\textsuperscript{me}
de Lévi que ce choix rendait son bonheur complet.

M\textsuperscript{me} la duchesse d'Orléans ne s'offrit point de mener
M\textsuperscript{me} de Saint-Simon chez le roi\,; nous en fûmes
surpris. Elle y alla avec la duchesse de Lauzun comme le conseil venait
de lever. Le roi les reçut dans son cabinet. Il ne se put rien ajouter à
tout ce que le roi dit à M\textsuperscript{me} de Saint-Simon sur son
mérite, sa vertu, la singularité sans exemple d'un tel choix à son âge.
Il parla ensuite de sa naissance, de sa dignité, en un mot, de tout ce
qui peut flatter. Il lui témoigna une confiance entière, trouva la jeune
princesse bien heureuse de tomber en de telles mains si elle en savait
profiter, prolongea la conversation un bon quart d'heure, parlant
presque toujours\,; M\textsuperscript{me} de Saint-Simon peu,
modestement, et avec non moins d'attention que j'en avais eue à faire
sentir par ses expressions pleines de respect, qu'elle ne se tenait
honorée et ne faisait rouler ses remercîments que sur la confiance. Mgr
le duc de Bourgogne, qu'elle vit chez lui, la combla de toutes les
sortes\,; et M. le duc de Berry ne sut assez lui témoigner sa joie. Le
soir elle fut chez M\textsuperscript{me} de Maintenon, toujours avec
M\textsuperscript{me} sa soeur. Comme elle commençait à lui parler, elle
l'interrompit par tout ce qui se pouvait dire de plus poli et de plus
plein de louanges sur un choix de son âge, et finit par l'assurer que
c'était au roi et à la future duchesse de Berry qu'il fallait faire des
compliments sur une dame d'honneur dont la naissance et la dignité
honoraient si fort cette place. La visite fut courte, mais plus pleine
qu'il ne se peut dire. Je fus fort surpris de ce que
M\textsuperscript{me} de Maintenon sentait et s'expliquait si nettement
sur l'honneur que M\textsuperscript{me} de Saint-Simon faisait à son
emploi. Nous le fûmes bien plus encore de ce que dans la suite elle le
répéta souvent, et en termes les plus forts, en présence et en absence
de M\textsuperscript{me} de Saint-Simon, et à plus d'une reprise à
M\textsuperscript{me} la duchesse de Berry même, tant il est vrai qu'il
est des vérités qui, à travers leur accablement, se font jour jusque
dans les plus opposés sanctuaires.

Ce même jour Madame, Mademoiselle et M. le duc de Berry même, qui me
reçurent avec une extrême joie, s'expliquèrent tout aussi franchement
tous trois avec moi sur l'honneur en propres termes, et la satisfaction
qu'ils res-sentaient d'un choix qu'ils avaient uniquement désiré.
J'allai avec M. de Lauzun l'après-dînée à Meudon, où Monseigneur me
reçut avec plus de politesse et d'ouverture que le matin.

Le soir, au retour, on m'avertit fort sérieusement qu'il fallait aller
chez M\textsuperscript{me} de Maintenon. Je n'y avais pas mis le pied
depuis qu'au mariage de la duchesse de Noailles j'y avais été avec la
foule de la cour. M\textsuperscript{me} de Saint-Simon ni moi n'avions
jamais eu aucun commerce avec elle, pas même indirectement, et jamais
nous ne l'avions recherché. Je ne savais pas seulement comment sa
chambre était faite. Il fallut croire conseil. J'y allai le soir même.
Sitôt que je parus on me fit entrer. Je fus réduit à prier le valet de
chambre de me conduire à elle, qui m'y poussa comme un aveugle. Je la
trouvai couchée dans sa niche, et auprès d'elle la maréchale de
Noailles, la chancelière, M\textsuperscript{me} de Saint-Géran qui
toutes ne m'effrayaient pas, et M\textsuperscript{me} de Caylus. En
m'approchant, elle me tira de l'embarras du compliment en me parlant la
première. Elle me dit que c'était à elle à me faire le sien du rare
bonheur et de la singularité inouïe d'avoir une femme qui, à trente-deux
ans, avait un mérite tellement reconnu, qu'elle était choisie, avec un
applaudissement universel, pour être dame d'honneur d'une princesse de
quinze {[}ans{]}, toutes choses sans exemple et si douces pour un mari
qu'elle ne pouvait assez m'en féliciter. Je répondis que c'était de ce
témoignage même que je ne pouvais assez la remercier\,; puis, regardant
la compagnie, j'ajoutai tout de suite, avec un air de liberté, que je
croyais que les plus courtes visites étaient les plus respectueuses, et
fis la révérence de retraite. Oncques depuis je n'y ai retourné.
M\textsuperscript{me} de Maintenon me dit, en s'inclinant à moi, de bien
goûter le bonheur d'avoir une telle femme, et, en souriant agréablement,
ajouta tout de suite d'aller à M\textsuperscript{me} de Noailles, qui
avait bien affaire à moi. Elle l'avait dit en m'entendant annoncer, la
plaisantant de ce qu'elle saisissait toujours tout le monde. Elle me
prit en effet comme je me retirais, et me voulut parler, derrière la
niche, de je ne sais quel emploi dans mes terres. Je lui dis que
ailleurs tant qu'elle voudrait, mais qu'elle me laissât sortir de là, où
je ne voyais plus qu'un étang. Nous nous mîmes à rire, et je me tirai
ainsi de cette grande visite.

Le lendemain lundi, tout à la fin de la matinée, M\textsuperscript{me}
de Saint-Simon fut avec M\textsuperscript{me} sa soeur à Meudon.
Monseigneur était sous les marronniers, qui les vint recevoir au
carrosse. C'était sa façon de familiarité quand il était en cet endroit,
avec les gens avec qui il en avait. Quoique avec M\textsuperscript{me}
de Saint-Simon la sienne fût moins que médiocre, il lui fit toutes les
honnêtetés qu'il put, et la promena dans ce beau lieu. L'heure du dîner
s'approchait fort. Biron et Sainte-Maure, fort libres avec Monseigneur,
lui dirent qu'il ne serait pas honnête de ne pas prier ces dames.
Monseigneur répondit qu'il n'osait parmi tant d'hommes\,; que néanmoins
lui et une dame d'honneur serviraient bien de chaperons\,; et que de
plus le duc de Bourgogne allait venir, qui l'était plus que personne.
Elles demeurent donc. Le repas fut très-gai, Monseigneur leur en fit les
honneurs. Il s'engoua de la dame d'honneur comme il avait fait à Marly
du mariage\,; leurs santés furent bues, et Mgr le duc de Bourgogne fit
merveilles. Il prit après dîner M\textsuperscript{me} de Saint-Simon un
moment en particulier, et lui parla de son dessein arrêté, et de
M\textsuperscript{me} la duchesse de Bourgogne, de la faire succéder à
la duchesse du Lude. M\textsuperscript{me} de Saint-Simon en revint si
étonnée, mais si peu flattée, qu'elle ne pouvait s'accoutumer à croire
qu'il n'y eût plus d'espérance d'éviter d'être dame d'honneur.

Ceux qui nous aimaient le moins, les plus envieux et les plus jaloux,
ceux qui craignaient le plus que cette place ne nous portât à d'autres
et qui avaient le plus cabale pour y en mettre d'autres, tout se
déchaîna en applaudissements, en éloges, en marques d'attachement et
d'amitié, avec tant d'excès que nous ne pouvions cesser de chercher ce
qui nous était arrivé, ni d'admirer qu'une si médiocre place fît tant
remuer les gens de toutes les sortes pour nous accabler de tout ce
qu'ils ne pensaient point, et de ce dont aussi ils ne pouvaient
raisonnablement croire qu'ils nous pussent persuader. Mais telle est la
misère d'une cour débellée. Il faut pourtant dire que ce choix fut aussi
généralement approuvé que le mariage le fut peu, et que ce qui contribua
à cette désespérade universelle de protestations fut l'empressement fixe
avec lequel il se fît, malgré nous, par le roi et par toutes les
personnes royales, qui ne se cachèrent ni de leur désir ni de nos
refus\,; {[}ce{]} qui fut en tout une chose sans exemple.

Le roi y mit tous les autres assaisonements pour rendre la place moins
insupportable, sans que nous en eussions dit ni fait insinuer la moindre
chose. Il déclara que, tant que M. le duc de Berry demeurerait
petit-fils ou fils du roi, les places de la duchesse du Lude et de
M\textsuperscript{me} de Saint-Simon étaient égales. Il voulut que les
appointements fussent pareils en tout et de même sorte, c'est-à-dire de
vingt mille livres, ce qui égala la dame d'atours à la comtesse de
Mailly, et lui valut neuf mille livres d'appointements de même. Il prit
un soin marqué de nous former le plus agréable appartement de
Versailles. Il délogea pour cela d'Antin et la duchesse Sforce, pour des
deux nous en faire un complet à chacun. Il y ajouta des cuisines dans la
cour au-dessous, chose très-rare au château, parce que nous donnions
toujours à dîner, et souvent à souper, depuis que nous étions à la cour.
En même temps le roi déclara que tout le reste de la maison de la future
duchesse de Berry serait formé sur le pied de celle de Madame. Ainsi
toute la distinction fut pour M\textsuperscript{me} de Saint-Simon et
pour la dame d'atours, qui en profita à cause d'elle, et cela fit un
nouveau bruit. Le personnel a peu contribué à l'étendue que j'ai donnée
au récit de l'intrigue de ce mariage, et à ce qui se passa sur le choix
de M\textsuperscript{me} de Saint-Simon\,; le développement et les
divers intérêts des personnes et des cabales, la singularité de
plusieurs particularités, et l'exposition naturelle de la cour dans son
intérieur m'ont paru des curiosités assez instructives pour n'en rien
oublier.

Le jour que M\textsuperscript{me} de Saint-Simon fut déclarée,
M\textsuperscript{me} de Maintenon manda à la duchesse de Ventadour de
faire savoir à M\textsuperscript{me} de La Vieuville qu'elle était dame
d'atours de la future duchesse de Berry. Elle vint dès le soir à
Versailles. Le roi ne la vit que le lendemain, et en public, dans la
galerie en allant à la messe. Elle ne fut reçue en particulier nulle
part, et froidement partout, même de Monseigneur, quoique protégée et
menée par M\textsuperscript{me} d'Espinoy. M\textsuperscript{me} de
Maintenon fut encore plus franche avec elle. Elle interrompit ses
remercîments, l'assura qu'elle ne lui en devait aucun, ni à personne, et
que c'était le roi tout seul qui l'avait voulue. C'était une demoiselle
de Picardie qui s'appelait La Chaussée d'Eu, comme La Tour d'Auvergne,
parce qu'elle était de la partie du comté d'Eu qui s'étend en Picardie.
Elle était belle, pauvre, sans esprit, mais sage, élevée domestique de
M\textsuperscript{me} de Nemours où on l'appelait M\textsuperscript{lle}
d'Arrez\footnote{Jérôme, seigneur de La Chaussée d'Eu, prenait le titre
  de comte d'Arrest. Saint-Simon écrit Arrez et non \emph{Aurez}, comme
  on l'a imprimé dans les précédentes éditions.}, et où M. de La
Vieuville s'amouracha d'elle et l'épousa, ayant des enfants de sa
première femme qui avait plu au roi étant fille de la reine et qui était
soeur du comte de La Mothe\footnote{Voy. à la fin du volume une note sur
  M\textsuperscript{lle} de La Mothe-Houdancourt, qui avait été une des
  filles de la reine Anne d'Autriche.}, duquel il n'a été fait que trop
mention sur le siège de Lille et depuis. M\textsuperscript{me} de La
Vieuville était, comme on l'a dit ailleurs, amie intime de
M\textsuperscript{me} de Roquelaure et fort bien avec
M\textsuperscript{me} de Ventadour, M\textsuperscript{me} d'Elboeuf,
M\textsuperscript{me} d'Espinoy et M\textsuperscript{lle} de Lislebonne.
Son art était une application continuelle à plaire à tout le monde, une
flatterie sans mesure, et un talent de s'insinuer auprès de tous ceux
dont elle croyait pouvoir tirer parti, mais c'était tout\,; du reste,
appliquée à ses affaires, avec l'attachement que donnent le besoin et la
qualité de deuxième femme qui trouve des enfants de la première et des
affaires en désordre\,; souvent à la cour frappant à toutes les portes,
rarement à Marly. Elle vint aussitôt et plusieurs fois chez
M\textsuperscript{me} de Saint-Simon, en grands compliments et respects
infinis. Nous ne la connaissions point, et nous la croyions bonne femme
et douce\,; nous espérâmes qu'elle serait là aussi commode qu'une autre.
L'expérience nous montra bientôt qu'intérêt et bassesses, sans aucun
esprit pour contre-poids, sont de mauvaise compagnie. Cette pauvre femme
s'attira par sa conduite des coups de caveçon dont elle perdit toute
tramontane, sans avoir reçu secours ni consolation de personne, et
obtint enfin pardon de M\textsuperscript{me} de Saint-Simon après bien
des soumissions et des larmes.

Son mari était une manière de pécore lourde et ennuyeuse à l'excès, qui
ne voyait personne à la cour, et à qui personne ne parlait, quoique
cousin germain de la maréchale de Noailles, enfants du frère et de la
soeur. Il avait eu le gouvernement de Poitou et la charge de chevalier
d'honneur de la reine, en survivance de son père, en se mariant la
première fois. Son père était aussi un fort pauvre homme, qui, par la
faveur du sien, avait eu un brevet de duc, et mourut gouverneur de M. le
duc de Chartres, depuis d'Orléans, en 1689, un mois après avoir été fait
chevalier de l'ordre. C'étaient de fort petits gentilshommes de Bretagne
dont le nom était Coskaër, peu ou point connu avant 1500 qu'Anne de
Bretagne les amena en France. Le petit-fils de celui-là s'allia bien,
fut grand fauconnier après le comte de Brissac, et ne laissa qu'un fils,
qui fit une grande fortune avec la charge de son père. Il fut premier
capitaine des gardes du corps de Louis XIII, chevalier de l'ordre et
surintendant des finances en 1623. Il fit chasser Puysieux, secrétaire
d'État, à qui il devait sa fortune, et le chancelier de Sillery, père de
Puysieux, et fut payé en même monnaie. Le cardinal de Richelieu, qu'il
avait introduit dans les affaires, le supplanta bientôt après, et le fit
accuser de force malversations avec Bouhier, successeur de Beaumarchais,
trésorier de l'épargne, dont il était gendre. Il fut mis en prison,
sortit après du royaume, et son procès lui fut fait par contumace. Après
la mort de Louis XIII, il profita grandement de l'affection et de la
protection dont la haine de la reine mère contre le cardinal de
Richelieu, et plus haut encore, se piqua envers tous les maltraités du
règne de Louis XIII. Il fut juridiquement rétabli dans tous ses biens et
dans toutes ses charges, même dans celle des finances, et lui et son
fils furent faits ducs à brevet, dont il ne jouit qu'un an, étant mort
le 2 janvier 1653. Son fils, mort gouverneur de M. le duc de Chartres,
avait acheté, un an avant la mort de son père, le gouvernement du
Poitou, du duc de Roannais, quand on l'en fit défaire, et douze ans
après la charge de chevalier d'honneur de la reine, du marquis de
Gordes. Ils avaient eu autrefois une terre en Artois. Je ne sais d'où
ils s'avisèrent de prendre le nom et les armes de La Vieuville\,; je ne
vois ni alliance ni rien qui ait pu y donner lieu, si ce n'est que le
choix était bon et valait beaucoup mieux que les leurs. Mais ils n'y ont
rien gagné\,; cette bonne et ancienne maison d'Artois et de Flandre ne
les a jamais reconnus, et personne n'ignore qu'ils n'en sont point.

M. le duc d'Orléans, au milieu de sa joie, se trouva embarrassé sur
l'Espagne, où il ne pouvait douter que le mariage ne plairait pas à
cause de lui. Il était difficile qu'il se dispensât d'y en donner part.
N'osant s'y conduire par lui-même, il hasarda d'en consulter le roi, qui
ne fut non plus sans embarras. Après quelques jours de réflexions, il
lui conseilla de suivre tout uniment l'usage. M. le duc d'Orléans
écrivit donc au roi et à la reine d'Espagne, qui ne lui firent aucune
réponse ni l'un ni l'autre, mais qui tous deux récrivirent à
M\textsuperscript{me} la duchesse d'Orléans. Le duc d'Albe affecta de la
venir complimenter un jour que M. le duc d'Orléans était à Paris, auquel
il ne donna pas le moindre signe de vie. On garda même à Madrid peu de
mesures en propos sur le mariage. Madame, qui était en commerce de
lettres avec la reine d'Espagne, lui fît sentir inutilement qu'elle s'en
prenait à la princesse des Ursins\,; et la reine d'Espagne traita ce
chapitre avec M\textsuperscript{me} la duchesse de Bourgogne avec autant
de légèreté et de grâce, qu'en pouvait être mêlé un dépit amer qui
voulait être senti. M. le duc d'Orléans en fut vivement peiné et
mortifié\,; mais il n'osa en laisser échapper la moindre plainte.

Les dispenses étaient attendues à tous moments, et il n'était question
que de la prompte célébration du mariage. En ces cérémonies, il s'en
pratique une qui s'étend jusqu'aux noces des duchesses, mais qu'elles
ont laissée tomber depuis quelque temps\,: c'est que la fiancée porte
une mante, dont j'ai fait la description, il n'y a pas longtemps, à
l'occasion des accoutrements de veuve de M\textsuperscript{me} la
Duchesse. La queue de cette mante est portée par une personne de rang
égal, lors des fiançailles\,; et, quand il n'y en a point, par celle qui
en approche le plus. Il ne se trouvait alors ni fille ni petite-fille de
France\,; la fonction en tombait à la première des princesses du sang.
Les filles de M. le duc d'Orléans avaient été mises à Chelles, cela
tombait donc naturellement à M\textsuperscript{lle} de Bourbon. On peut
penser ce qu'il en sembla à M\textsuperscript{me} la Duchesse et à elle,
qui avaient tant espéré ce grand mariage pour la même princesse à qui,
en ce cas, Mademoiselle eût porté la mante, et qui se trouvait dans la
nécessité de la lui porter. Ce fut un crève-coeur qu'elles ne purent
supporter, et elles se hasardèrent même assez hautement de s'en faire
entendre, jusque-là qu'il fut jeté en l'air qu'on pouvait bien se passer
de mante quand personne ne la voulait porter, car M\textsuperscript{me}
la Duchesse n'était pas plus docile pour M\textsuperscript{lle} de
Charolais que pour M\textsuperscript{lle} de Bourbon. Il y avait bien
encore les filles de M\textsuperscript{me} la princesse de Conti, mais
la chose eût été trop marquée. La cour était cependant en maligne
attention de voir ce qui arriverait de cette pique qui commençait fort à
grossir, lorsque le roi, qui avait fait le mariage, mais qui ne voulait
ni fâcher Monseigneur ni désespérer M\textsuperscript{me} la Duchesse,
qui avait répandu que c'était uniquement pour lui jouer ce tour que
M\textsuperscript{me} la duchesse d'Orléans venait de mettre ses filles
en religion, le roi, dis-je, demanda à M. le duc d'Orléans s'il ne les
ferait point venir aux noces de leur soeur.

M. le duc d'Orléans, faible, facile, content au delà de toute espérance,
et l'homme le plus éloigné de haine et de malignité, oublia tout ce qui
lui avait été dit là-dessus et tout ce qu'il avait promis à
M\textsuperscript{me} la duchesse d'Orléans. Au lieu de s'en tirer par
la modestie, d'en éviter la dépense, et mieux encore par la crainte de
les dissiper par le spectacle de cette pompe, il consentit à les faire
venir. Je n'oserais dire que la misère de leur en donner le plaisir eut
part à une complaisance si déplacée. M\textsuperscript{me} la duchesse
d'Orléans, au désespoir, imagina de voiler ce retour de ses filles, qui
n'étaient encore qu'ondoyées, par le supplément des cérémonies du
baptême\,; et les fit tenir deux jours avant les fiançailles par
Monseigneur et Madame, et par M\textsuperscript{me} la duchesse de
Bourgogne et M. le duc de Berry. Ainsi M\textsuperscript{lle} de
Chartres, qui a depuis été abbesse de Chelles, porta la mante aux
fiançailles, où les deux fils de M. du Maine signèrent pour la première
fois au contrat de mariage en conséquence de leur nouveau rang.

Le lendemain dimanche, 6 juillet, le mariage fut célébré sur le midi
dans la chapelle par le cardinal de Janson, grand aumônier. Deux
aumôniers du roi tinrent le poêle\,; le roi, les personnes royales, les
princes et les princesses du sang et bâtards présents\,; beaucoup de
duchesses sur leur carreaux, tout de suite des princesses du sang et les
ducs de La Trémoille, de Chevreuse, de Luynes, son petit-fils de
dix-sept ans, Beauvilliers, Aumont, Charost, le duc de Rohan et
plusieurs autres sur les leurs\,; aucun des princes étrangers, mais des
princesses étrangères sur leurs carreaux, parmi les duchesses\,; les
tribunes toutes magnifiquement remplies, où je me mis pour plonger à mon
aise sur la cérémonie, en bas beaucoup de dames derrière les carreaux,
et d'hommes derrière les dames. Après la messe le curé apporta son
registre sur le prie-Dieu du roi, où il signa et les seules personnes
royales, mais aucun prince ni princesse du sang, sinon les enfants de M.
le duc d'Orléans. Ce fut alors que M\textsuperscript{me} de Saint-Simon
partit de dessus son carreau, qui était à gauche au bord des marches du
sanctuaire, et se vint ranger derrière M\textsuperscript{me} la duchesse
de Berry qui allait signer. La signature finie, on se mit en marche pour
sortir de la chapelle. Il y eut forcé gentillesses entre Madame et
M\textsuperscript{me} la duchesse de Berry, qui fit ses façons d'assez
bonne grâce, et que Madame prit enfin par les épaules et la fit passer
devant elle. Chacun de là fut dîner chez soi, le roi à son petit
couvert, et les mariés chez M\textsuperscript{me} la duchesse de
Bourgogne, qui tint après jusqu'au soir un grand jeu dans le salon qui
joint la galerie à son appartement, où toute la cour abonda.

Le roi, qui tint conseil d'État le matin et l'après-dînée, et qui
travailla le soir à l'ordinaire chez M\textsuperscript{me} de Maintenon,
vint sur l'heure du souper chez M\textsuperscript{me} la duchesse de
Bourgogne, où il trouva tout ce qui devait être du festin, préparé dans
la pièce qui a un oeil de boeuf, joignant sa chambre, sur une table à
fer à cheval, où ils allèrent se mettre quelques moments après. Ils
étaient vingt-huit rangés en leurs rangs à droite et à gauche, le roi
seul au milieu, dans son fauteuil, avec son cadenas. Les conviés qui y
soupèrent, et il n'en manqua aucun, furent Monseigneur, Mgr {[}le duc{]}
et M\textsuperscript{me} la duchesse de Bourgogne, M. {[}le duc{]} et
M\textsuperscript{me} la duchesse de Berry, Madame, M. {[}le duc{]} et
M\textsuperscript{me} la duchesse d'Orléans, le duc de Chartres,
M\textsuperscript{me} la Princesse, le comte de Charolais, car M. le Duc
était à l'armée de Flandre, les deux princesses de Conti,
M\textsuperscript{lle}s de Chartres et de Valois, depuis duchesse de
Modène, M\textsuperscript{lle}s de Bourbon et de Charolais,
M\textsuperscript{me}s du Maine et de Vendôme, M. le prince de Conti,
que je devais mettre plus tôt, et ses deux soeurs, le duc du Maine, ses
deux fils, et le comte de Toulouse, M\textsuperscript{me} la
grande-duchesse, que j'ai oublié à mettre après M\textsuperscript{me} la
duchesse d'Orléans. Aucune femme assise n'entra dans le lieu du festin
et fort peu d'autres y parurent, nuls ducs ni princes étrangers,
quelques autres hommes de la cour.

Au sortir de table, le roi alla dans l'aile neuve à l'appartement des
mariés. Toute la cour, hommes et femmes, l'attendait en haie dans la
galerie et l'y suivit avec tout ce qui avait été du souper. Le cardinal
de Janson fit la bénédiction du lit. Le coucher ne fut pas long. Le roi
donna la chemise à M. le duc de Berry. M. de Bouillon avait prétendu la
présenter comme grand chambellan\,; M. de Beauvilliers, comme
gouverneur, eut la décision du roi pour lui, et la présenta. J'y tenais
le bougeoir, et je fus surpris que M. de Bouillon ne s'en allât pas et
vît donner cette chemise. M\textsuperscript{me} la duchesse de Bourgogne
la donna à la mariée, présentée par M\textsuperscript{me} de
Saint-Simon, à qui le roi fit les honnêtetés les plus distinguées. Les
mariés couchés, M. de Beauvilliers et M\textsuperscript{me} de
Saint-Simon tirèrent le rideau chacun de leur côté, non sans rire un peu
d'une telle fonction ensemble. Le lendemain matin, le roi fut en sortant
de la messe chez M\textsuperscript{me} la duchesse de Berry. En se
mettant à sa toilette, M\textsuperscript{me} de Saint-Simon lui présenta
et lui nomma toute la cour comme à une étrangère, et lui fit baiser les
hommes et les femmes titrés, après quoi les personnes royales et les
princes et princesses du sang vinrent à cette toilette. Après le dîner,
comme la veille, même jeu dans le même salon, où le roi avait ordonné
que toutes les dames se trouvassent parées comme la veille pour recevoir
la reine et la princesse d'Angleterre\,; car le roi d'Angleterre était à
l'armée de Flandre, comme l'année précédente.

La reine et la princesse sa fille allèrent d'abord voir Monseigneur qui
jouait chez M\textsuperscript{me} la princesse de Conti, puis chez
M\textsuperscript{me} de Maintenon où était le roi. Elle vint après dans
ce salon voir Mgr {[}le duc{]} et M\textsuperscript{me} la duchesse de
Bourgogne, et finit par aller chez les mariés, d'où elle retourna à
Chaillot, après quoi il ne fut plus du tout mention de noces. La reine
et la princesse d'Angleterre, qui s'étaient toujours flattées de ce
mariage qui même s'était pensé faire, comme je crois l'avoir dit, ne se
faisaient aucune justice sur la situation des affaires. Elles étaient
désolées. Cela fit que le roi voulut leur épargner la noce, et même
toute la cérémonie de la visite que pour cela il régla comme il vient
d'être rapporté.

Le grand deuil de M\textsuperscript{me} la Duchesse lui épargna aussi
tout ce spectacle. Monseigneur dit à M\textsuperscript{me} de
Saint-Simon qu'il lui ferait plaisir de faciliter à
M\textsuperscript{me} la duchesse, encore dans son premier deuil, un
moment de voir M\textsuperscript{me} la duchesse de Berry en
particulier, ce qui fut promptement exécuté. La visite fut courte.
M\textsuperscript{me} de Saint-Simon en fut accablée de compliments et
d'excuses de ce que son état de veuve l'avait empêchée d'aller chez
elle. Le mercredi suivant on alla à Marly. Le roi, qui avait fait un
présent de pierreries fort médiocre à M\textsuperscript{me} la duchesse
de Berry, ne donna rien à M. le duc de Berry. Il avait si peu d'argent
qu'il ne put jouer les premiers jours du voyage. M\textsuperscript{me}
la duchesse de Bourgogne le dit au roi qui sentant l'état où il était
lui-même, la consulta sur ce qu'il n'avait pas plus de cinq cents
pistoles à lui donner, et qu'il lui donna avec excuse sur le malheur des
temps, parce que M\textsuperscript{me} la duchesse de Bourgogne trouva
avec raison que ce peu valait mieux que rien et ne pouvoir jouer.

Ce voyage de Marly fut l'époque du retour des deux soeurs de
M\textsuperscript{me} la duchesse de Berry à Chelles, et de la liberté
de M\textsuperscript{me} de Maré. Elle avait été gouvernante des enfants
de Monsieur en survivance de la maréchale de Grancey sa mère, puis en
chef après elle, et l'était demeurée de ceux de M. le duc d'Orléans avec
beaucoup de considération. Le roi et M\textsuperscript{me} de Maintenon
comptaient qu'elle serait dame d'atours de M\textsuperscript{me} la
duchesse de Berry qu'elle avait élevée, et à qui elle paraissait fort
attachée, et Mademoiselle à elle. Madame et M. {[}le duc{]} et
M\textsuperscript{me} la duchesse d'Orléans le voulaient. Jamais on ne
l'y put résoudre, quelque pressantes et longues que fussent les
instances que tous, jusqu'à M\textsuperscript{me} de Maintenon, lui en
firent. Il faut savoir que la maréchale de Grancey était soeur de
Villarceau, chez qui M\textsuperscript{me} de Maintenon avait tant passé
d'étés, et puis à Montchevreuil avec lui, et qui toute sa vie en
conserva un souvenir si cher, comme je l'ai dit ailleurs. Ce ne fut
qu'aux refus opiniâtres et réitérés de M\textsuperscript{me} de Maré
qu'on nomma une dame d'atours. Elle prétexta son âge, sa santé, son
repos, sa liberté. Elle se retira donc avec les regrets de tout le
monde, les nôtres surtout. Elle était ma parente et de tout temps
intimement mon amie, et elle avait beaucoup d'amis considérables, et
plus de sens et de conduite encore que d'esprit. Elle eut des présents,
deux mille écus de pension du roi, un logement au Luxembourg, et
conserva le sien au Palais-Royal, ses établissements de Saint-Cloud et
les douze mille livres d'appointements de M. le duc d'Orléans, avec le
titre de gouvernante de ses filles, dont elle ne s'embarrassa plus des
fonctions.

Nous ne fûmes pas longtemps sans découvrir la cause de son opiniâtre
résistance à demeurer auprès de M\textsuperscript{me} la duchesse de
Berry. Plus cette princesse se laissa connaître, et elle ne s'en
contraignit guère, plus nous trouvâmes que M\textsuperscript{me} de Maré
avait raison\,; plus nous admirâmes par quel miracle de soins et de
prudence rien n'avait percé, plus nous sentîmes à quel point on agit en
aveugle dans ce qu'on désire avec le plus de passion, et dont le succès
cause plus de peines, de travaux et de joie\,; plus nous gémîmes du
malheur d'avoir réussi dans une affaire que, bien loin d'avoir
entreprise et suivie au point que je le fis\,; j'aurais traversée avec
encore plus d'activité, quand même M\textsuperscript{lle} de Bourbon en
eût dû profiter et l'ignorer, si j'avais su le demi-quart, que dis-je la
millième partie de ce dont nous fûmes si malheureusement témoins. Je
n'en dirai pas davantage pour le présent, et je n'en dirai dans la suite
que ce qui ne s'en pourra taire\,; et je n'en parle sitôt que parce que
ce qui arriva depuis en tant d'étranges sortes commença à pointer, et à
se développer même un peu dès ce premier Marly. Il est temps maintenant
de remonter d'où nous sommes partis pour n'interrompre point la suite de
ce mariage.

\hypertarget{chapitre-xvi.}{%
\chapter{CHAPITRE XVI.}\label{chapitre-xvi.}}

1710

~

{\textsc{Dépôt des papiers d'État.}} {\textsc{- Destination des généraux
d'armée pareille à la dernière.}} {\textsc{- Villars se perd auprès du
roi et se relève incontinent.}} {\textsc{- Rare aventure de deux lettres
contradictoires de Montesquiou, qui brouille Villars avec lui.}}
{\textsc{- Douai assiégé\,; Albergotti dedans.}} {\textsc{- Berwick
envoyé examiner ce qui se passait à l'armée de Flandre.}} {\textsc{-
Récompense d'avance.}} {\textsc{- Fortune rapide de Berwick, qui est
fait duc et pair.}} {\textsc{- Clause étrange de ses lettres, et sa
cause.}} {\textsc{- Nom étrange imposé à son duché, et pourquoi.}}
{\textsc{- Usage d'Angleterre.}} {\textsc{- Berwick en Dauphiné\,; reçu
duc et pair à son retour.}} {\textsc{- Étrange absence d'esprit de
Caumartin au repas de cette réception.}} {\textsc{- Chapelle de
Versailles bénite par le cardinal de Noailles, archevêque de Paris, qui
l'emporte sur la prétendue exemption.}} {\textsc{- Mort de la duchesse
de La Vallière, carmélite, etc., dont la princesse de Conti drape.}}
{\textsc{- Mort de Sablé.}} {\textsc{- Mort et caractère du maréchal de
Joyeuse.}} {\textsc{- Villars gouverneur de Metz.}} {\textsc{- Mort de
Renti et de sa soeur la maréchale de Choiseul.}} {\textsc{- État de
l'armée et de la frontière de Flandre, et du siége de Douai.}}
{\textsc{- Entreprise manquée sur Ypres.}} {\textsc{- Bagatelle à
Liège.}} {\textsc{- Douai rendu.}} {\textsc{- Albergotti chevalier de
l'ordre, etc.}} {\textsc{- Béthune assiégé\,; Puy-Vauban gouverneur
dedans.}} {\textsc{- Béthune rendu.}} {\textsc{- Récompenses.}}
{\textsc{- Entreprise manquée sur Menin.}} {\textsc{- Retour de nos
plénipotentiaires.}} {\textsc{- Ridicule aventure du maréchal de Villars
et d'Heudicourt.}} {\textsc{- Villars veut aller aux eaux.}} {\textsc{-
Harcourt sur le Rhin mandé à la cour\,; est reçu duc et pair au
parlement.}} {\textsc{- Va commander l'armée de Flandre.}} {\textsc{-
Aire et Saint-Venant assiégés.}} {\textsc{- Goesbriant dans Aire.}}
{\textsc{- Force combats.}} {\textsc{- Ravignan bat un convoi.}}
{\textsc{- Listenais et Béranger tués, le chevalier de Rothelin fort
blessé.}} {\textsc{- Aire et Saint-Venant rendus.}} {\textsc{-
Goesbriant chevalier de l'ordre.}} {\textsc{- Campagnes finies en
Flandre, sur le Rhin et en Dauphiné, sans qu'il se passe rien aux deux
dernières.}}

~

Jusque fort avant dans le règne de Louis XIV, on n'avait eu soin sous
aucun roi de ramasser les papiers qui concernaient l'État, à l'exception
de la partie en ce genre la moins importante à tenir secrète, qui est
les finances, laquelle, ayant des formes juridiques, avait par
conséquent des greffes et des dépôts publics à la chambre des comptes.
Louvois fut le premier qui sentit le danger que les dépêches et les
instructions qui, du roi et de ses ministres, étaient adressées aux
généraux des armées, aux gouverneurs, et aux autres chefs de guerre, et
même aux intendants des frontières, et de ceux-là au roi et aux
ministres, restassent entre les mains de ces particuliers, et après eux
de leurs héritiers et souvent de leurs valets, qui en pouvaient faire de
dangereux usages, et quelquefois jusqu'aux beurrières dont il est arrivé
à des curieux d'en retirer de très-importants d'entre leurs mains.
Quoique alors les guerres dont il s'agissait dans ces papiers fussent
finies, et quelquefois depuis fort longtemps, ceux contre qui la France
les avait soutenues y pouvaient trouver l'explication dangereuse de bien
des énigmes et l'éclaircissement de beaucoup de ténèbres importantes à
n'être pas mises au jour, et peut-être des trahisons achetées, encore
plus fatales à découvrir pour les familles intéressées, et pour donner
lieu à s'en mieux garantir.

Ces considérations, qu'on ne comprend pas qui n'aient plus tôt frappé
nos rois et leurs ministres, saisirent M. de Louvois. Il rechercha tout
ce qu'il put retirer d'ancien en ce genre, se fit rendre à mesure ces
sortes de papiers, et les fit ranger par année dans un dépôt aux
Invalides, où cet ordre a continué depuis à être soigneusement observé,
tellement qu'outre la conservation du secret on a encore par là des
instructions sûres où on peut puiser utilement. Ce même défaut était
encore plus périlleux dans la partie de la négociation, et la chose est
si évidente qu'elle n'a pas besoin d'explication. Croissy, chargé des
affaires étrangères, fut réveillé par l'exemple que lui donna Louvois.
Il l'imita pour les recherches du passé, et pour se faire rendre les
papiers qui regardaient son département à mesure, mais il en demeura là.

Torcy, son fils, proposa au roi en mars de cette année de faire un dépôt
public de ces papiers, qui le trouva fort à propos. Torcy prit pour le
roi un pavillon des Petits-Pères, près la place des Victoires, parce
qu'il entrait de son jardin dans le leur à l'autre bout duquel est ce
pavillon très-détaché et éloigné du couvent, isolé de tout, et où on
peut entrer tout droit de la rue. Il y fit mettre en bel ordre tout ce
curieux et important dépôt, où les ministres et les ambassadeurs
trouvent tant de quoi s'instruire, et qui est soigneusement continué
jusqu'à présent, en sorte que les héritiers mêmes des ministres de ces
départements et de leurs principaux commis et secrétaires sont obligés
d'y mettre tout ce qui se trouve dans les bureaux et dans les cabinets
des secrétaires d'État, lorsque par la mort ou autrement ils perdent
leur charge. Un commis principal et de confiance particulière est chargé
de ce dépôt par département sous le secrétaire d'État en charge et y
répond de tout. Pontchartrain ensuite en a fait autant pour le sien de
la marine et de la maison du roi. On peut dire que cet établissement
n'est pas un des moindres ni des moins importants qui aient été faits du
règne de Louis XIV, mais il serait à désirer que ces autres dépôts
fussent placés aussi sûrement et aussi immuablement que l'est celui de
la guerre.

Le roi, qui avait fait une nombreuse promotion militaire, destina les
mêmes généraux aux mêmes armées. Le duc de Noailles partit de bonne
heure pour le Roussillon\,: le duc d'Harcourt avait pris les eaux de
Bourbonne et y devait retourner au mois de mai, pour se rendre de là à
l'armée du Rhin. En attendant il était au Pallier, château du comte de
Tavannes pour éviter le voyage, où Besons eut ordre d'aller conférer
avec lui, et de prendre après le commandement de l'armée en l'y
attendant pour y demeurer sous lui après.

Villars, choisi pour la Flandre où le maréchal de Montesquiou avait
commandé tout l'hiver, et le devait seconder pendant la campagne,
considéra avec peine le fardeau dont il s'allait charger. Monté au plus
prodigieux comble de faveurs et de privances, de richesses, d'honneurs
et de grandeurs, {[}il{]} crut pouvoir hasarder pour la première fois de
sa vie quelques vérités, parce que, n'ayant plus où atteindre, ces
vérités qui déplairaient allaient à sa décharge. Il en dit donc beaucoup
à Desmarets et à Voysin sur le triste état des places, des magasins, des
garnisons, des fournitures pour la campagne, les manquements de toute
espèce, l'état pitoyable des troupes et des officiers, leur paye et la
solde. Peu content de l'effet de ses représentations, il osa les porter
dans toute leur crudité à M\textsuperscript{me} de Maintenon et au roi
même. Il leur parla papier sur table, par preuves et par faits qui ne se
pouvaient contester.

À la levée de ce fatal rideau, l'aspect leur parut si hideux, et tout si
fort embarrassant, qu'ils eurent plus court de se fâcher que de répondre
à un langage si nouveau dans la bouche de Villars, qui n'avait fait tout
ce qu'il avait voulu qu'à force de leur dire et de leur répéter que tout
était en bon état et allait à merveilles. C'était la fréquence et la
hardiesse de ses mensonges qui le leur avaient fait regarder comme leur
seule ressource, et lui donner et lui passer tout, parce que lui seul
trouvait tout bien, et se chargeait de tout sans jamais dire rien de
désagréable, et faisant au contraire tout espérer comme trouvant tout
facile. Le voyant alors parler le langage des autres et de tous les
autres, l'espérance en ces prodiges s'évanouit avec tous les appâts dont
il les avait bercés si utilement pour lui. Alors ils commencèrent à le
regarder avec d'autres yeux, à le voir comme le monde l'avait toujours
vu, à le trouver ridicule, fou, impudent, menteur, insupportable, à se
reprocher une élévation de rien si rapide et si énorme, à l'éviter, à
l'écarter, à lui faire sentir ce qu'ils en pensaient, à le laisser
apercevoir aux autres.

À son tour Villars fut effrayé. Son dessein était bien d'essayer à
l'ombre de sa blessure et de tant de manquements à suppléer qui
demandaient une pleine santé, de jouir en repos de toute sa fortune, et
d'éviter les épines sans nombre et toute la pesanteur d'un emploi qui,
au point où il était parvenu, ne pouvait plus lui présenter de degrés à
escalader\,; mais il voulait en même temps conserver entiers son crédit,
sa faveur, sa considération, ses privances et une confiance qui le fît
consulter, et lui donnât influence sur les partis à prendre, les ordres
à envoyer aux différentes armées, se rendre juge des coups et de la
conduite des généraux, et augmenter son estime auprès du roi par ses
propos avantageux sur la guerre, de l'exécution desquels il ne serait
pas chargé. Quand il sentit un si grand changement à son égard sur
lequel l'ivresse de son orgueil et de son bonheur n'avait pas compté, il
vit avec frayeur à quoi il s'était exposé, et ce qu'il pourrait devenir
hors d'emploi, de faveur et de crédit, sans parents et sans amis qui
pussent le protéger contre tant d'ennemis et d'envieux, ou plutôt contre
tout un public qu'il avait sans cesse bravé et insulté, et que sa
fortune avait irrité. Il prit brusquement son parti, et comme la honte
ne l'avait jamais arrêté sur rien, il n'en eut point de changer tout à
coup de langage, et de reprendre celui dont il s'était si bien trouvé
pour sa fortune. Il saisit les moments d'incertitude à qui donner le dur
emploi de commander en Flandre qui lui était destiné, et qu'on lui
voulait ôter sur le point de l'aller prendre. Il recourut, avec cette
effronterie qui lui était naturelle, à la flatterie, à l'artifice, au
mensonge, à braver les inconvénients, à se moquer des dangers, à
présenter en soi des ressources à tout, à faire toute facile.

La grossièreté de la variation sautait aux yeux, mais l'embarras de
choisir un autre général sautait à la gorge, et l'heureux Villars se
débourba. Ce ne fut pas tout\,: raffermi sur ses étriers après une si
violente secousse, il osa se donner publiquement pour un Romain qui, au
comble de tout, abandonnait repos et santé et tout ce qui peut flatter,
qui n'a plus rien à prétendre, et qui, malgré une blessure qui à
grand'peine lui permettait de monter à cheval, courait au secours de
l'État et du roi, qui le conjurait de se prêter à la nécessité et au
péril de la conjoncture présente\,: À ces bravades, il ajouta qu'il
faisait à la patrie le sacrifice des eaux, qui l'auraient empêché de
demeurer estropié, et il tint là-dessus tant de scandaleux propos, que
le duc de Guiche, qui allait aux eaux pour une blessure au pied, reçue
aussi à Malplaquet, mais bien moins considérable que celle de Villars,
prit tous ces discours pour soi, et ne le lui pardonna pas.

Le maréchal, moyennant sa blessure, partit pour la frontière dans son
carrosse, à petites journées. Pendant son voyage, il arriva une aventure
qui eût été fort plaisante si elle n'eût pas été telle aux dépens de
l'État. Le maréchal de Montesquiou, qui assemblait l'armée sous Cambrai,
qui, comme je l'ai dit, avait passé l'hiver en Flandre, et qui n'en
avait pas déguisé les désordres au maréchal de Villars, destiné dès lors
à y faire la campagne avec lui, écrivit au roi des merveilles du bon
état de toutes choses. Le roi fut si aise de ces bonnes nouvelles, qu'il
envoya à Villars cette dépêche de Montesquiou. Le hasard fit que ce
courrier atteignit Villars en chemin, deux heures après qu'il eut reçu
une longue lettre de Montesquiou, remplie d'amertume et de détails les
plus inquiétants sur tout ce qui manquait aux places, aux magasins, aux
troupes, en un mot de tous côtés.

Villars, bien moins surpris de l'une que de l'autre, n'en fit point à
deux fois\,; sur-le-champ il renvoya au roi le courrier qu'il venait
d'en recevoir, et le chargea de la lettre dont je viens de parler et de
celle qui lui avait été envoyée, et, avec ces deux contradictoires de
même date et du même homme, il ne fit que joindre un billet au roi et un
autre à Voysin, par lesquels il les priait de juger à laquelle des deux
lettres ils devaient ajouter le plus de foi, et continua son voyage,
ravi du bonheur de présenter, aux dépens d'un autre et si naturellement,
les mêmes vérités qui l'avaient conduit si près de la disgrâce et de la
chute, et de montrer tout le poids du fardeau dont il allait se charger.
Les suites n'ont point montré dans le roi l'effet de ce rare
contraste\,; mais il devint public tout aussitôt par Villars même, qui
se garda bien de s'en taire, et l'éclat en fut épouvantable. Les deux
maréchaux ne s'en parlèrent point, mais on peut juger de l'union que
cette aventure dut mettre entre eux, et quel spectacle pour l'armée, qui
n'avait d'ailleurs ni estime ni affection pour eux, qui aussi ne
s'étaient pas mis en soin de se concilier ni l'un ni l'autre.

Le prince Eugène et le duc de Marlborough, qui ne voulaient point de
paix, et dont le but était de percer en France, l'un par vengeance
personnelle contre le roi et {[}pour{]} se faire de plus en plus un
grand nom, l'autre pour gagner des trésors, {[}ce{]} qui était à chacun
leur passion dominante, avait résolu de profiter de l'extrême faiblesse
et du délabrement de nos troupes et de nos places pour pousser pendant
cette campagne leurs conquêtes le plus avant qu'ils pourraient.
Albergotti, lieutenant général, et Dreux, maréchal de camp, avaient eu
ordre d'aller à Douai, où ils eurent à peine le temps de donner ordre
aux choses les plus pressées, qu'ils furent investis et la tranchée
ouverte du 4 au 5 mai. Pomereu, frère du feu conseiller d'État et ancien
capitaine aux gardes, avait eu ce gouvernement en se retirant. Il y
avait diligemment pourvu à tout ce qu'il avait pu. Il compta pour moins
le dégoût de se voir commandé dans sa place, que la démarche d'en sortir
au moment d'un siége. Il passa donc sur toute autre considération, et
fut d'un grand et utile secours à Albergotti pendant tout ce siége. La
garnison y était nombreuse et choisie, les munitions de guerre et de
bouche abondantes\,; tout s'y prépara à une belle défense. M. le Duc
était déjà à l'armée, le roi d'Angleterre y arriva sous le nom et
l'incognito ordinaire de chevalier de Saint-Georges, comme le maréchal
de Villars était en situation de pouvoir combattre les ennemis.

Le roi, piqué de ses pertes continuelles, désirait passionnément une
victoire qui ralentît les desseins des ennemis, et qui pût changer
l'État de la triste et honteuse négociation qui se traitait à
Gertruydemberg. Cependant les ennemis étaient bien postés. Villars avait
perdu en arrivant sur eux une belle occasion de les battre. Toute son
armée avait remarqué cette faute, il en avait été averti à temps par
plusieurs officiers généraux et par le maréchal de Montesquiou, sans les
avoir voulu croire, et il n'osait chercher à les attaquer après les
dispositions qu'il leur avait laissé le loisir de faire. L'armée cria
beaucoup d'une faute si capitale. Villars, empêtré de sentir que ce
n'était pas à tort, paya d'effronterie et ne parlait que de manger
l'armée ennemie avec ses rodomontades usées, tandis qu'il ne savait plus
en effet par où la rapprocher. Dans cette crise que la division des deux
maréchaux et le manque d'estime et d'affection des troupes rendait
très-fâcheuse, le roi jugea à propos d'envoyer en Flandre Berwick, comme
modérateur des conseils et un peu comme dictateur de l'armée, mais sans
autre commandement que celui de son ancienneté de maréchal de France, et
encore dans une armée où il n'était qu'en passant. La bataille livrée,
ou jugée ne la devoir pas être, il avait ordre de revenir aussitôt
rendre compte de toutes choses, pour passer ensuite à la tête de l'armée
de Dauphiné, où la campagne s'ouvrait plus tard qu'ailleurs, à cause des
neiges et des montagnes.

Mais ce n'était plus guère la coutume de rien faire sans une récompense
qui devançât l'entreprise et qui mît en sûreté le succès personnel de
celui qui en était chargé. Usage nouveau, pernicieux à l'État et au roi,
qui, de cette façon, avait de rien formé plusieurs géants de grandeur,
et des pygmées d'action dont on n'avait pas daigné se servir depuis,
sinon de quelques-uns, encore par reprise et à défaut d'autres
très-senti. Nous étions en l'âge d'or des bâtards. Berwick n'avait que
dix-huit ans lorsqu'il arriva en France en 1688 avec le roi Jacques II,
à la révolution d'Angleterre. Il fut fait lieutenant général à
vingt-deux ans tout d'un coup, et en servit en 1692 à l'armée de
Flandre, sans avoir passé auparavant par aucun grade, et n'ayant servi
que de volontaire. À trente-trois ans il commanda en chef l'armée de
France et d'Espagne en Espagne avec une patente de général d'armée, et,
à trente-quatre ans, mérita, par sa victoire d'Almanza, d'être fait
grand d'Espagne et chevalier de la Toison d'or. Il commanda toujours
depuis des armées en chef ou dans de grandes provinces, jusqu'en février
1706, qu'il fut fait maréchal de France seul, lorsqu'il n'avait pas
encore trente-six ans. Il était duc d'Angleterre, et quoiqu'ils n'aient
point de rang en France, le roi l'avait accordé à ceux qui avaient suivi
le roi Jacques, qui avait donné la Jarretière à Berwick sur le point de
la révolution. C'était bien et rapidement pousser la fortune sous un roi
qui regardait les gens de cet âge comme des enfants, mais qui, pour les
bâtards, ne leur trouvait non plus d'âge qu'aux dieux.

Il y avait déjà un an que Berwick, qui voulait tout accumuler sur sa
tête et le partager à ses enfants, avait demandé d'être fait duc et
pair. Le roi à qui de fois à autre il prenait des flux de cette dignité,
qu'il avait tant avilie, en avait aussi des temps de chicheté. Berwick
donna dans un de ceux-là, et n'avait pu réussir. En l'occasion dont je
parle, il sentit qu'il était cru nécessaire, il en saisit le moment\,;
il fit entendre qu'il ne pouvait partir mécontent, et se fit faire duc
et pair. Berwick n'avait qu'un fils de sa première femme, et il avait de
la seconde plusieurs fils et filles. Il était sur l'Angleterre comme les
juifs qui attendent toujours le Messie. Il se flattait toujours aussi
d'une révolution qui remettrait les Stuarts sur le trône, et lui par
conséquent en ses biens et honneurs. Il était fils de la soeur du duc de
Marlborough dont il était fort aimé, et avec lequel, du gré du roi et du
roi d'Angleterre, il entretenait un commerce secret, dont tous trois
furent les dupes, mais qui servait à Berwick à entretenir d'autres en
Angleterre, et à y dresser ses batteries, en sorte qu'il espéra son
rétablissement particulier, même sous le gouvernement établi. C'est dans
ce principe qu'il obtint la grâce inouïe du choix de ses enfants, et
encore de le pouvoir changer tant qu'il voudrait, pour succéder à sa
grandesse. Par la même raison il osa proposer, et on eut la honteuse
faiblesse de la lui accorder, l'exclusion formelle de son fils aîné dans
ses lettres de duc et pair, dans lesquelles il fit appeler tous ceux du
second lit.

Son projet était de revêtir l'aîné de la dignité de duc de Berwick et de
tous ses biens d'Angleterre\,; de faire le second duc et pair, et le
troisième grand d'Espagne, où son dessein était de chercher à le marier
et à l'attacher. Trois fils héréditairement élevés aux trois premières
dignités des trois premiers royaumes de l'Europe, il faut convenir que
ce n'était pas mal cheminer à quarante ans avec tout ce qu'il avait
d'ailleurs\,; mais l'Angleterre lui manqua. Il eut beau la ménager toute
sa vie outre mesure, en courtiser le ministère, recueillir tous les
Anglais considérables qui passaient en France, lier un commerce d'amitié
étroite avec ses ambassadeurs en France, jamais il ne put obtenir de
rétablissement, tellement que, n'y ayant plus de ressource en France
pour l'aîné, après son exclusion de la dignité de duc et pair, il se
rejeta pour lui sur la grandesse, l'attacha en Espagne, l'y maria à une
soeur du duc de Veragua, lequel mourut après sans enfants, et laissa à
cette soeur et à ses enfants plus de cent mille écus de rente, avec des
palais, des meubles et des pierreries en quantité, et les plus grandes
terres. J'aurai lieu d'en parler plus amplement. Le scandale fut grand
de la complaisance qu'eut le roi pour cet arrangement de famille qui
mettait sur la tête d'un cadet la première dignité du royaume après son
père, et qui réservait l'aîné à l'espérance de celle d'Angleterre\,;
mais le temps des monstres était arrivé. Berwick acheta Warties,
médiocre terre sous Clermont en Beauvoisis, qu'il fit ériger sous le
barbare et le honteux nom de \emph{Fitz-James}\,: autre faiblesse qu'on
eut encore pour lui. Le roi qui passa la chose fut choqué du nom, lequel
en ma présence en demanda la raison au duc de Berwick qui la lui
expliqua sans aucun embarras, et que voici\,:

Les rois d'Angleterre en légitimant leurs enfants leur donnent un nom et
des armes qui passent au parlement d'Angleterre et à leur postérité. Les
armes, qui sont toujours celles d'Angleterre, ont des sortes de brisures
distinctes\,; le nom varie. Ainsi le duc de Richemont, bâtard de Charles
II, a eu le nom de Lenox\,; les ducs de Cleveland et Grafton du même
roi, celui de Fitz-Roi, qui veut dire fils de roi\,; le duc de
Saint-Albans aussi du même roi, celui de Beauclerc\,; enfin le duc de
Berwick de Jacques II, duc d'York quand il l'eut, mais roi quand il le
légitima et le fit duc, celui de Fitz-James, qui signifie fils de
Jacques, en sorte que son nom de maison pour sa postérité est celui-là,
et son duché-pairie en France, le duché de fils de Jacques en français,
et les ducs en même langue, les ducs et pairs fils de Jacques. On ne
saurait s'empêcher de rire du ridicule de ce nom s'il se portait en
français, ni de s'étonner du scandale de l'imposer en anglais en France,
Le parlement n'osa ou ne daigna souffler. Tout y fut enregistré sans la
moindre difficulté sur le nom ni sur la clause\,; Berwick ne quitta
point que cela ne fût fait et consommé, et aussitôt après il s'en alla
en Flandre. Il y trouva l'armée des ennemis si avantageusement postée et
retranchée qu'il n'eut pas de peine à se rendre au sentiment commun des
généraux et officiers généraux de celle du roi, qu'il n'était plus temps
de songer à l'attaquer. Il recueillit sagement et séparément leurs
{[}avis{]} sur ce qui s'était passé jusqu'alors, et les trouva uniformes
dans celui que Villars avait manqué la plus belle occasion du monde de
les attaquer. Berwick, n'ayant rien de plus dans sa mission que de se
bien instruire de toutes choses, ne fut pas trois semaines absent. Son
rapport consterna fort le roi et ceux qui le pénétrèrent. Bientôt après,
les lettres de l'armée mirent tout le monde dans le secret, qui révolta
fort contre ce matamore en paroles.

Le duc de Berwick ne fut guère plus de vingt-quatre heures de retour à
la cour qu'il partit pour le Dauphiné, et ne put être reçu duc et pair
au parlement que le 11 décembre suivant. Cet événement est si peu
important à intervertir que je raconterai ici une aventure qui arriva à
cette occasion, et dont le court intermède mérite de ne pas être oublié.
Nous assistâmes en nombre à cette réception, avec la singularité d'y
avoir eu en notre tête bâtards et bâtardeaux, et à notre queue à tous un
bâtard d'Angleterre\,; ce fut matière à réflexions sur le maintien des
lois dans cette île, et par quelle protection ferme, solide et
constante, et l'interversion de tous les nôtres \emph{ad nutum}. Le duc
de Tresmes, ami de Berwick, et accoutumé aux fêtes comme gouverneur de
Paris, donna le festin au sortir du parlement, où la plupart des ducs se
trouvèrent avec plusieurs autres personnes de considération, entre
autres Caumartin, conseiller d'État et intendant des finances, qui était
fort répandu à la cour et dans le plus beau monde, fort ami du duc de
Tresmes, et oncle de sa belle-fille.

Il savait beaucoup et agréablement jusqu'à être un répertoire fort
curieux\,; il était beau parleur et avec de l'esprit, un air de fatuité
imposante par de grands airs, et une belle figure, quoique au fond il
fut bon homme, et même à sa façon respectueux. Je ne sais par quelle
étrange absence d'esprit il s'engagea à table au récit d'un procès
bizarre d'un bâtard dont il avait été autrefois l'un des juges, et
s'étendit sur les difficultés qui roulaient toutes sur cette sorte de
naissance et sur la sévérité des lois à leur égard, qu'il déploya avec
emphase et avec approbation. Chacun baissa les yeux, poussa son voisin
{[}avec{]} un silence profond que Caumartin prit pour attention à la
singularité du fait et aux grâces de son débit. Le duc de Tresmes voulut
rompre les chiens plus d'une fois\,; à toutes Caumartin l'arrêtait,
haussait le ton et continuait. Ce récit dura bien trois bons quarts
d'heure. On s'étouffait de manger ou de mâcher, personne n'osa boire de
peur d'un éclat de rire involontaire\,; on en mourait, et dans la même
crainte on n'osait se regarder. Jamais Caumartin, engoué de son histoire
et du plaisir de tenir le dé, ne s'aperçut d'une si énorme disparate.
Berwick à qui, comme à l'homme du jour, il adressa souvent la parole,
comprit bien qu'il avait totalement oublié qui il était, et ne s'en
offensa jamais, mais le pauvre Tresmes en était que la sueur lui en
tombait du visage. Il est vrai que l'extrême ridicule d'une scène si
entière et si longue me divertit extrêmement, et par les yeux, et par
les oreilles, et par les réflexions sur ce contraste du matin et du
festin même de ce triomphe des bâtards, et de l'énergique étalage de
toute leur infamie et de leur néant.

La nouvelle chapelle étant enfin entièrement achevée, et admirée du roi
et de tous les courtisans, il s'éleva une grande dispute à qui la
consacrerait. Le cardinal de Janson, grand aumônier, avec tout ce qui
est sous sa charge, la pré-tendait exempte de la juridiction de
l'ordinaire, en alléguait beaucoup de titres et de preuves, et
prétendait que c'était à lui à. faire cette cérémonie. Le cardinal de
Noailles, archevêque diocésain, s'en tenait au droit commun, alléguait
qu'il avait officié avec sa croix devant le roi dans la chapelle, et
qu'à tout ce qui s'était fait en présence du roi, de mariages, de
baptêmes, etc., le curé de Versailles y avait toujours été présent en
étole, ainsi qu'aux convois qui en étaient partis\,; et il réclamait la
justice et la piété du roi, et son amour de l'ordre et des règles. Il
l'emporta, parce qu'il était encore bien avec lui et
M\textsuperscript{me} de Maintenon, et dans la vénération de l'un et de
l'autre\,; et il fit la cérémonie le jeudi matin 5 juin en présence de
Mgr le duc de Bourgogne. La chapelle s'était assez échauffée là-dessus,
mais entre les deux cardinaux la dispute se passa avec politesse et
modestie. On détruisit incontinent après l'ancienne chapelle, et on ne
se servit plus que de celle-là. Nonobstant ce jugement, la chapelle
s'est maintenue dans toute sa prétention, le curé dans son usage
d'assister en étole comme il fit depuis au mariage de M. le duc de Berry
et à tous les autres, et aux baptêmes comme auparavant. Mais il est vrai
que depuis aucun archevêque de Paris n'a officié à la chapelle à cause
de la difficulté de sa croix, malgré l'exemple antérieur du cardinal de
Noailles\,; et la seule fois que son successeur y a officié, étant nommé
à Paris à une fête de l'ordre, il n'avait pas encore ses bulles, ainsi
il était sans croix et sans prétention de l'y faire porter devant lui.

M\textsuperscript{me} de La Vallière mourut en ce temps-ci aux
Carmélites de la rue Saint-Jacques, où elle avait fait profession le 3
juin 1675, sous le nom de soeur Marie de la Miséricorde, à trente et un
ans. Sa fortune, et la honte\,; la modestie, la bonté dont elle en
usa\,; la bonne foi de son coeur sans aucun autre mélange\,; tout ce
qu'elle employa pour empêcher le roi d'éterniser la mémoire de sa
faiblesse et de son péché en reconnaissant et légitimant les enfants
qu'il eut d'elle\,; ce qu'elle souffrit du roi et de
M\textsuperscript{me} de Montespan\,; ses deux fuites de la cour, la
première aux Bénédictines de Saint-Cloud, où le roi alla en personne se
la faire rendre, prêt à commander de brûler le couvent, l'autre aux
Filles de Sainte-Marie de Chaillot, où le roi envoya M. de Lauzun, son
capitaine des gardes, avec main-forte pour enfoncer le couvent, qui la
ramena\,; cet adieu public si touchant à la reine, qu'elle avait
toujours respectée et ménagée, et ce pardon si humble qu'elle lui
demanda prosternée à ses pieds devant toute la cour, en partant pour les
Carmélites\,; la pénitence si soutenue tous les jours de sa vie, fort
au-dessus des austérités de sa règle\,; cette fuite exacte des emplois
de la maison, ce souvenir si continuel de son péché, cet éloignement
constant de tout commerce, et de se mêler de quoi que ce fût, ce sont
des choses qui pour la plupart ne sont pas de mon temps, ou qui sont peu
de mon sujet, non plus que la foi, la force et l'humilité qu'elle fit
paraître à la mort du comte de Vermandois, son fils.

M\textsuperscript{me} la princesse de Conti lui rendit toujours de
grands devoirs et de grands soins, qu'elle éloignait et qu'elle
abrégeait autant qu'il lui était possible. Sa délicatesse naturelle
avait infiniment souffert de la sincère âpreté de sa pénitence de corps
et d'esprit, et d'un coeur fort sensible dont elle cachait tout ce
qu'elle pouvait. Mais on découvrit qu'elle l'avait portée jusqu'à s'être
entièrement abstenue de boire pendant toute une année, dont elle tomba
malade à la dernière extrémité. Ses infirmités s'augmentèrent, elle
mourut enfin d'une descente, dans de grandes douleurs, avec toutes les
marques d'une grande sainteté, au milieu des religieuses dont sa douceur
et ses vertus l'avaient rendue les délices, et dont elle se croyait et
se disait sans cesse être la dernière, indigne de vivre parmi des
vierges. M\textsuperscript{me} la princesse de Conti ne fut avertie de
sa maladie, qui fut fort prompte, qu'à l'extrémité. Elle y courut et
n'arriva que pour la voir mourir. Elle parut d'abord fort affligée, mais
elle se consola bientôt. Elle reçut sur cette perte les visites de toute
la cour. Elle s'attendait a celle du roi, et il fut fort remarqué qu'il
n'alla point chez elle.

Il avait conservé pour M\textsuperscript{me} de La Vallière une estime
et une considération sèche dont il s'expliquait même rarement et
courtement. Il voulut pourtant que la reine et les deux dauphines
l'allassent voir et qu'elles la fissent asseoir, elle et
M\textsuperscript{me} d'Épernon, quoique religieuses, comme duchesses
qu'elles avaient été, ce que je crois avoir remarqué ailleurs. Il parut
peu touché de sa mort, il en dit même la raison\,: c'est qu'elle était
morte pour lui du jour de son entrée aux Carmélites. Les enfants de
M\textsuperscript{me} de Montespan furent très-mortifiés de ces visites
publiques reçues à cette occasion, eux qui en pareille circonstance n'en
avaient osé recevoir de marquée. Ils le furent bien autrement quand ils
virent M\textsuperscript{me} la princesse de Conti draper, contre tout
usage, pour une simple religieuse, quoique mère\,; eux qui n'en avaient
point, et qui, pour cette raison, n'avaient osé jusque sur eux-mêmes
porter la plus petite marque de deuil à la mort de M\textsuperscript{me}
de Montespan. Le roi ne put refuser cette grâce à M\textsuperscript{me}
la princesse de Conti, qui le lui demanda instamment, et qui ne fut
guère de son goût. Les autres bâtards essuyèrent ainsi cette sorte
d'insulte que le simple adultère fit au double dont ils étaient sortis,
et qui rendit sensible à la vue de tout le monde la monstrueuse horreur
de leur plus que ténébreuse naissance, dont ils furent cruellement
piqués.

Une autre mort arrivée en même temps parut moins précieuse devant Dieu,
et fit moins de bruit dans le monde. Ce fut celle de Sablé, fils de
Servien, surintendant des finances, qui avait amassé tant de trésors, et
qui en avait tant dépensé à embellir Meudon, dont il enterra le village
et le rebâtit auprès, pour faire cette admirable terrasse, si
prodigieuse en étendue et en hauteur. Il avait marié sa fille au duc de
Sully, frère de la duchesse du Lude, et laissé ses deux fils, Sablé et
l'abbé Servien, si connus tous deux par leurs étranges débauches avec
beaucoup d'esprit et fort aimable et orné. Sablé vendit Meudon à M. de
Louvois, sur les fins Sablé à M. de Torcy, mangea tout, vécut obscur, et
ne fut connu que par des aventures de débauche, et par s'être fait
estropier lui, et rompre le cou à l'arrière-ban d'Anjou qu'il menait au
maréchal de Créqui. Ainsi périssaient promptement les races des
ministres, avant qu'ils eussent trouvé l'art d'établir leurs enfants aux
dépens des seigneurs dans les premières charges de la cour, après les
grandes.

Le maréchal de Joyeuse mourut aussi à plus de quatre-vingts ans, sans
enfants d'une fille de sa maison qu'il avait épousée, dont il était
veuf, et qui ne fut pas heureuse. Il ressemblait tout à fait à un roi
des Huns. Il avait de l'esprit, de la noblesse, de la hauteur et une
grande valeur. Excellent officier général, surtout de cavalerie,
très-bon à mener une aile, mais pour une armée, dont il ne commanda
jamais aucune en chef, qu'en passant et par accident, la tête lui en
tournait et aux autres aussi, par son embarras et sa brutalité qui le
rendait inabordable. Il était assez pauvre et cadet d'un aîné ruiné,
excellent lieutenant général, qu'on appelait le comte de Grandpré,
chevalier de l'ordre en 1661, mort il y avait longtemps, qui traînait
d'ordinaire son cordon bleu à pied, faute de voiture, et qui ne laissa
point d'enfants. Ce maréchal de Joyeuse était une manière de sacre et de
brigand, qui pillait tant qu'il pouvait pour le manger avec
magnificence. Il avait eu le gouvernement de Metz et du pays Messin à la
mort du duc de La Ferté. Il fut donné deux jours après au maréchal de
Villars, en lui conservant les quinze mille livres d'appointements,
comme ayant perdu le gouvernement de Fribourg.

Le marquis de Renti le suivit de près dans une grande piété, et depuis
quelque temps dans une grande retraite. Il était fils de ce marquis de
Renti qui a vécu et est mort en réputation de sainteté, et il était
frère de la maréchale de Choiseul, qui ne le survécut que de quelques
mois. C'était un très-brave, honnête et galant homme, d'un esprit
médiocre et assez difficile, quoique très-bon homme\,; mais impétueux,
médiocre à la guerre pour la capacité, mais honorable et tout à fait
désintéressé. Il était lieutenant général, et lieutenant général de
Franche-Comté, où on ne le laissa guère commander assez mal à propos\,;
mais le titre en est devenu un d'exclusion. Il n'était pas riche, et a
laissé un fils très-brave et honnête homme aussi, mais que l'extrême
incommodité de sa vue a retiré fort tôt du service et presque du monde.

Le maréchal de Villars trouva l'armée assemblée sous Cambrai. Elle était
de cinquante-sept bataillons et de deux cent soixante-deux escadrons\,;
toutes les places outre cela garnies. Mais ces troupes n'étaient pas
bien complètes, même d'officiers. Depuis un mois le prêt leur était payé
et on leur donnait du pain passable et quelque viande. Albergotti se
défendait bien dans Douai. Le duc de Mortemart y commanda une sortie qui
fit un grand désordre dans les tranchées, tua beaucoup de monde et n'en
perdit presque point. L'attaque aussi fut vigoureuse, et de part et
d'autre on travailla fort sous terre pour faire des mines et pour les
éventer. Outre ce qui faisait le siége, l'armée des ennemis était aussi
forte que celle du roi, et tenta une entreprise sur Ypres. Ils crurent
avoir gagné un partisan de la garnison, et par son moyen surprendre la
place. Le partisan en avertit Chevilly qui y commandait, et par son
ordre suivit l'entreprise. Les ennemis, pleins de confiance en leur
marché, détachèrent deux mille chevaux ou dragons de leur armée, portant
chacun un fantassin en croupe, sous prétexte de renforcer leurs
garnisons de Lille et de Menin\,; et le partisan marchait assez près, à
la tête, avec douze ou quinze hommes. Il se présenta à la barrière qu'on
lui ouvrit, en même temps ses douze ou quinze hommes furent pris. Le
détachement arrivait\,; mais il fut averti à temps par le hasard d'un
fusil d'un soldat de milice qui était dans les dehors, qui tira. À ce
bruit, le détachement se crut découvert et s'arrêta. Il se retira
aussitôt après. On leur tua ou blessa une cinquantaine d'hommes du feu
que la place fit sur eux de tous côtés. Le partisan en eut une petite
pension et une commission de lieutenant colonel. Un autre de nos
partisans sortit quelques jours après de Namur, trouva moyen de se
glisser dans Liège, se rendit maître du corps de garde qui était à la
porte, marcha à la place, tua celui qui y commandait, prit toute la
garde, pilla la maison du ministre de l'empereur et celle d'un
Hollandais qui commandait dans la ville, et s'en revint avec un assez
gros butin et cinquante prisonniers, sans y avoir laissé qu'un homme.

Cependant le siége de Douai s'avançait. Il s'y était passé, le 20 juin,
une action considérable. Les ennemis s'étaient rendus maîtres d'une
demi-lune. Dreux et le duc de Mortemart les en chassèrent. Ils revinrent
et s'établirent sur la berme\footnote{La berme était un espace de trois
  ou quatre pieds entre le rempart et le fossé\,; elle servait à
  recevoir les terres du rempart qui s'éboulaient, afin que le fossé
  n'en fût pas comblé.}, où un fourneau qui joua à propos les fit tous
sauter. Ils perdirent environ deux mille hommes\,; mais ils revinrent
une troisième fois et gagnèrent l'angle de cet ouvrage.

Deux jours après ils se rendirent maîtres de deux demi-lunes\,; et comme
la brèche était fort grande, Albergotti fit battre la chamade le 25. Le
duc de Mortemart apporta la capitulation au roi, qui fut toute telle
qu'Albergotti la voulut. La brèche était capable pour deux bataillons de
front. Le roi, content de cette belle défense, et accoutumé à prostituer
le collier du Saint-Esprit en récompenses militaires, fit Albergotti
chevalier de l'ordre\,; Dreux, blessé le dernier jour du siége,
lieutenant général\,; et donna à la garnison d'autres récompenses.
Albergotti eut aussi en même temps le gouvernement de Sarrelouis, vacant
déjà depuis quelque temps par la mort de Choisy\,; et le duc de
Mortemart fut maréchal de camp.

Les ennemis, après avoir réparé et pourvu leur nouvelle conquête, ne
perdirent pas de temps à en faire d'autres, dans l'impuissance où
Villars leur paraissait de les en empêcher. Ils marchèrent à Béthune et
y ouvrirent la tranchée le 24. Du Puy-Vauban, gouverneur de la place, y
commandait avec quatre mille hommes de garnison. Il n'en avait pas voulu
davantage, et il était suffisamment muni et approvisionné. Il fit faire
une sortie cette nuit même de l'ouverture de la tranchée, leur tua huit
cents hommes et y perdit fort peu. Il y eut force coups de main\,; mais,
après une belle défense, du Puy-Vauban battit la chamade le 28 août, et
eut la capitulation telle qu'il la voulut. Il avait le cordon rouge, le
roi y ajouta la grand'croix, et les mille écus de plus, en attendant la
première vacance, qui fut une chose tout à fait contre son usage, et
donna des récompenses aux principaux de la garnison. Tout à la fin de ce
siége, on tenta une entreprise sur Menin. Les troupes détachées furent
mal conduites par les guides. Au lieu d'arriver la nuit, elles furent
surprises par le jour et s'en revinrent comme elles étaient allées.

Tout au commencement de ce même siége, nos plénipotentiaires arrivèrent
de Gertruydemberg, plus que fort fraîchement ensemble. Ils vinrent un
matin à Marly, où le roi les entretint assez longtemps dans son cabinet
avec Torcy. Ce qui se trouvera là-dessus dans les Pièces m'empêche d'en
dire ici d'avantage.

Il arriva au maréchal de Villars une aventure fort ridicule qui fit
grand bruit à l'armée et à la cour. Sa blessure, ou les airs qu'il en
prenait, lui faisait souvent tenir la jambe sur le cou de son cheval à
peu près comme les dames. Il lui échappa un jour, dans l'ennui où il se
trouvait dans son armée, qu'il était bien las de monter à cheval comme
ces p\ldots{} de la suite de M\textsuperscript{me} la duchesse de
Bourgogne, qui, par parenthèse, étaient toutes les jeunes dames de la
cour et les filles de M\textsuperscript{me} la Duchesse. Un tel propos,
tenu en pleine promenade par un général d'armée peu aimé, courut bientôt
d'un bout à l'autre du camp, et ne tarda guère à voler à la cour et à
Paris. Les dames cavalières s'offensèrent, les autres prirent parti pour
elles\,; M\textsuperscript{me} la duchesse de Bourgogne ne put leur
refuser de s'en montrer irritée et de s'en plaindre. Villars en fut tôt
averti, et fort en peine d'un surcroît d'ennemis si redoutables, dont sa
campagne n'avait pas besoin. Il se mit dans la tête de découvrir qui
l'avait décelé. Il fit si bien qu'il sut à n'en pas douter que c'était
Heudicourt qui l'avait mandé\,; et il en fut d'autant plus piqué que,
pour faire sa cour à sa mère, ce mauvais ange de M\textsuperscript{me}
de Maintenon, et à M\textsuperscript{me} de Montgon sa soeur, il l'avait
adomestiqué, protégé, et, chose fort étrange pour le maréchal, lui avait
souvent, non pas prêté, mais donné de l'argent, dont il était toujours
fort dépourvu par sa mauvaise conduite et l'avarice de son père qui
mangeait tout à son âge avec des créatures.

La vieille Heudicourt et sa fille étaient mortes, mais Heudicourt, fort
protégé du roi par M\textsuperscript{me} de Maintenon, à cause de sa
défunte mère, était demeuré comme l'enfant de la maison partout où était
le maréchal de Villars. C'était un drôle de beaucoup d'esprit, qui
excellait à donner des ridicules, à la plaisanterie la plus salée, aux
chansons les plus immortelles, et qui, gâté par la faveur qui l'avait
toujours soutenu, ne s'était contraint pour personne, et par cette même
faveur et par l'audace et le tranchant de sa langue s'était rendu
redoutable. Il n'avait point d'âme, grand ivrogne et débauché, point du
tout poltron, et une figure hideuse de vilain satyre. Il se faisait
justice là-dessus\,; mais hors d'état d'espérer de bonnes fortunes, il
les facilitait volontiers, était sûr dans cet honnête commerce, et
s'était acquis par là beaucoup d'amis de la fleur de la cour, et encore
plus d'amies. Par contraste à sa méchanceté, on ne l'appelait que le
\emph{petit bon}, et le petit bon était de toutes les intrigues, en
menait quantité, et en était un répertoire. C'était parmi les dames à la
cour à qui l'aurait, dont pas une n'eût osé se brouiller avec lui, à
commencer par les plus hautes. Cette protection que personne n'ignorait
le rendait encore plus hardi, tellement que le maréchal de Villars se
trouva dans le dernier embarras. Toutefois, après y avoir bien pensé, il
eut recours à l'effronterie qui toujours l'avait si utilement servi.

Pour cela il envoya chercher une quinzaine d'officiers généraux, tous
considérables par leur poids à l'armée, ou par leurs entours à la cour,
et Heudicourt avec eux. Quand il les sut tous arrivés, il sortit de sa
chambre, et alla où ils étaient, avec ce que le hasard y avait conduit
d'autres gens, comme il en fourmille toujours de toute espèce chez le
général qui voulait faire une scène publique. Là, il demanda tout haut à
chacun de ceux qu'il avait mandés, et l'un après l'autre, s'ils se
souvenaient qu'il eût dit telle chose qu'il répéta. Albergotti, revenu à
l'armée après avoir fait, au sortir de Douai, un tour de huit ou dix
jours à Paris et à la cour, prit en matois la parole le premier,
répondit qu'il se souvenait qu'il avait parlé ainsi des vivandières et
des créatures du camp, et jamais d'autres. Nangis, le prince de Rohan,
le prince Charles, fils de M. le Grand, et tous les autres, ravis d'une
si belle ouverture, la suivirent l'un après l'autre, et la confirmèrent
jusqu'au-dernier. Alors Villars, dans le soulagement qu'on peut juger,
insista pour faire mieux confirmer et consolider la chose, puis,
éclatant contre l'inventeur d'une si affreuse calomnie, contre
l'imposteur qui l'avait écrite à la cour, adressa la parole à Heudicourt
qu'il traita de la plus cruelle façon du monde. Le petit bon, qui
n'avait pas prévu qu'il serait découvert ni la scène où il se trouvait,
fut étrangement interdit, et se voulut défendre\,; mais Villars
produisit des preuves qui ne purent être contredites. Alors le vilain,
acculé, avoua sa turpitude, et eut l'audace de s'approcher de Villars
pour lui parler bas\,; mais le maréchal, se reculant et le repoussant
avec un air d'indignation, lui dit de parler tout haut, parce que, avec
des fripons de sa sorte, il ne voulait rien de particulier. Alors
Heudicourt, reprenant ses esprits, se livra à toute son impudence. Il
soutint qu'aucuns de tout ce qui était là et que Villars avait
interrogé, n'osoient lui déplaire en face, mais {[}qu'ils{]} savaient
fort bien tous la vérité du fait, telle qu'il l'avait écrite\,; qu'il
pouvait avoir tort de l'avoir mandée, mais qu'il n'avait pas imaginé que
dite en si nombreuse compagnie et en lieu si public, elle pût demeurer
secrète, et qu'il fît plus mal de la mander que tant d'autres qui en
avaient pu faire autant.

Le maréchal, outré de colère d'entendre une réponse si hardie, et au
moins si vraisemblable, lui reprocha ses bienfaits et sa scélératesse.
Il ajouta que, quand la chose serait vraie, il n'y aurait pas moins de
crime à lui de la publier qu'à l'inventer, {[}après{]} toutes les
obligations qu'il lui avait\,; le chassa de sa présence\,; et quelques
moments après le fit arrêter et conduire au château de Calais. Cette
violente scène fit à l'armée et à la cour autant de bruit que ce qui
l'avait causée. La conduite suivie et publique du maréchal fut
approuvée. Le roi déclara qu'il le laissait maître du sort
d'Heudicourt\,; M\textsuperscript{me} de Maintenon et
M\textsuperscript{me} la duchesse de Bourgogne, qu'elles
l'abandonnaient\,; et ses amis avouaient que sa faute était inexcusable.
Mais la chance tourna bientôt. Après le premier étourdissement, l'excuse
du petit bon parut valable aux dames qui avaient leurs raisons pour
l'aimer et pour craindre de l'irriter\,; elle la parut aussi dans
l'armée, où le maréchal n'était pas aimé. Plusieurs de ceux qu'il avait
si publiquement interrogés se laissèrent entendre que, dans la surprise
où ils s'étaient trouvés, ils n'avaient pas voulu se commettre. On en
vint bassement à cette discussion que cette allure du maréchal, et son
prétendu propos ne pouvait aller aux vivandières et aux autres femmes
des armées, qui allaient toutes à cheval jambe deçà, jambe delà, au
contraire des dames, surtout de celles qui montaient à cheval avec
M\textsuperscript{me} la duchesse de Bourgogne. On contesta jusqu'au
pouvoir des généraux d'armée de se faire justice à eux-mêmes de leurs
inférieurs, pour des choses personnelles et où le service n'entrait pour
rien\,: en un mot, Heudicourt, au sortir de Calais, où il ne fut pas
longtemps, demeura le petit bon à la mode, en dépit du maréchal. Tant de
choses lui tournèrent mal cette campagne qu'il prit la résolution de
s'en aller aux eaux. Il fit tant qu'il l'obtint. Harcourt, qui ne
faisait qu'arriver à Strasbourg après les avoir prises tout à son aise,
eut ordre de revenir, et la permission de faire le voyage à petites
journées dans son carrosse. Peu de jours après être arrivé, il se fit
recevoir duc et pair au parlement. Il demeura plus d'un mois à Paris, et
s'en alla après dans son carrosse à petites journées à Dourlens, où il
avait rendez-vous avec le maréchal de Villars\,; et de là l'un à l'armée
de Flandre, l'autre droit à Bourbonne sans passer à Paris ni à la cour,
ce qui parut assez extraordinaire et peu agréable. Ainsi un boiteux en
remplaça un autre, et un général aussi peu en état de fatiguer que celui
à qui il succédait. L'un commença, l'autre finit par Bourbonne\,; et
Harcourt par la Flandre qu'il avait évitée d'abord.

Il y trouva une grande désertion dans l'armée, et les ennemis devant
Aire et Saint-Venant à la fois. Chevilly, qui commandait à Ypres,
informé que les ennemis faisaient partir un grand convoi de Gand, fît
sortir de sa place Ravignan, maréchal de camp, la nuit, avec deux mille
cinq cents hommes. Ravignan trouva le convoi à Vive-Saint-Éloi\,; il y
avait quarante-cinq balandres\footnote{Bateaux plats.} chargées de
munitions de guerre et de bouche, escortées au bord de l'eau par treize
cents hommes, dont huit cents Anglais et six cents chevaux. Ravignan les
attaqua brusquement\,; les treize cents hommes furent tous tués, noyés
ou pris, et la cavalerie qui prit la fuite de bonne heure, perdit au
moins moitié. Le fils du comte d'Athlone et presque tous les principaux
officiers furent pris. Après cette expédition, Ravignan éloigna ses
troupes, brûla les quarante-cinq balandres, et fit sauter treize cents
milliers de poudre qui détruisirent le village de Vive-Saint-Éloi. On
crut que cette affaire coûta près de trois millions aux ennemis.

Aire et Saint-Venant se défendaient toujours. Il y eut de grosses
actions aux deux siéges. La tranchée avait été ouverte à Aire en deux
endroits à la fois, le 12 septembre. Goesbriant, gendre de Desmarets, y
commandait, et y faisait de grandes sorties. Le chevalier de Selve en
fit aussi à Saint-Venant, dans une desquelles Listenais fut tué. Le
chevalier de Rothelin eut les deux cuisses percées à Aire, et à
Saint-Venant. Béranger, colonel de Bugey, fort estimé, fut tué. Ce
régiment fut donné à son frère, et celui de Listenais au sien,
Goesbriant fit abandonner aux ennemis l'attaque du côté du château, et
par deux fois les fours à chaux qui étaient à la tête des ouvrages de la
place, mais que lui-même abandonna à la troisième attaque. Il les
repoussa aussi du chemin couvert qu'ils voulaient emporter, où le second
fils du comte de La Mothe fut tué. Ils le furent encore jusqu'à trois
fois le 2 novembre à une grande attaque qu'ils firent, mais enfin
Goesbriant capitula le 8 novembre, et obtint toutes les conditions qu'il
demanda. Il rendit en même temps le fort Saint-François, faute de vivres
à y mettre. Saint-Venant s'était rendu quelque temps auparavant. Ainsi
finit la campagne en Flandre, qui fut la dernière du duc de Marlborough.
Les armées entrèrent en quartiers de fourrage, et incontinent après en
quartiers d'hiver. M. d'Harcourt avait eu pendant ce siége quelque petit
soupçon d'apoplexie, qui ne fut rien. La fin de la campagne lui vint à
propos\,; le maréchal de Montesquiou demeura pour tout l'hiver à
commander en Flandre, d'où tous les officiers généraux non employés
l'hiver, et les particuliers, ne tardèrent pas à revenir. Goesbriant,
comme Albergotti, fut chevalier de l'ordre\,; et force récompenses à sa
garnison.

Sur le Rhin la campagne se passa toute à chercher tranquillement à
subsister, et finit en même temps que celle de Flandre. Le duc de
Berwick passa la sienne en chicanes et en observations. M. de Savoie ne
la fit point. Il était mal content de l'empereur, qu'il menaça même de
songer à ses intérêts particuliers. La récompense d'un démembrement de
quelque chose du Milanais était un objet qui entretenait la
mésintelligence, et qui, pour le déterminer, l'empêcha de faire cette
année de grands efforts. Il faut maintenant voir ce qui s'est passé
d'ailleurs, dont il n'eût pas été à propos d'interrompre la campagne de
Flandre, par la même raison que celles d'Espagne et de Roussillon, qui
seront rapportées après, demandent à l'être tout de suite.

\hypertarget{chapitre-xvii.}{%
\chapter{CHAPITRE XVII.}\label{chapitre-xvii.}}

1710

~

{\textsc{Situation du cardinal de Bouillon.}} {\textsc{- État de la
famille du cardinal de Bouillon, et ses idées bâties dessus.}}
{\textsc{- Cardinal de Bouillon, furieux de la perte d'un procès, passe
à Montrouge {[}et{]} à Ormesson.}} {\textsc{- Évasion du cardinal de
Bouillon, que le prince d'Auvergne conduit à l'armée des ennemis, où il
reçoit toutes sortes d'honneurs.}} {\textsc{- Lettre folle du cardinal
de Bouillon au roi.}} {\textsc{- Analyse de cette lettre.}}

~

Le cardinal de Bouillon languissait d'ennui et de rage dans son exil
dont il ne voyait point la fin, quoique l'adoucissement qu'il en avait
obtenu lui eût donné des espérances. Incapable de se donner aucun repos,
il avait passé tout ce loisir forcé dans une guerre monastique. Il avait
voulu soutenir, et même étendre, sa juridiction d'abbé de Cluni sur les
réformés. Ceux-ci, profitant de la disgrâce, n'oublièrent rien pour
secouer ce que la faveur passée leur avait fait subir de joug. Ce ne
furent donc que procès de part et d'autre sur ce que les moines
traitaient d'entreprises, et le cardinal de révolte. Il n'était pas
douteux que l'abbé de Cluni ne fût général de cet ordre et le supérieur
immédiat de la congrégation. Il ne l'était pas non plus qu'il ne fallût
être moine pour pouvoir être général et en exercer l'autorité. La
grandeur de cette abbaye en collations immenses l'avait fait usurper par
des séculiers puissants. Les cardinaux, les premiers ministres, les
princes du sang qui l'eurent en commende prétendirent les mêmes droits
que les abbés réguliers\,; et la dispute, avec divers succès, n'avait
point cessé jusqu'au temps que le cardinal de Bouillon eut cette abbaye.

La division qui s'y était mise par la réforme tout à fait séparée en
tout des religieux anciens avait augmenté les différends. Ceux-ci, aussi
peu réformés que leurs abbés, tenaient presque tous pour lui contre les
réformés\,; et la passion de posséder les bénéfices claustraux ou
affectés aux religieux était une pomme de discorde dont l'abbé savait
profiter. De là, duplicité d'offices et de titulaires, de bénéfices, de
la collation de l'abbé et de l'élection des religieux\,; et une hydre de
procès et de procédés entre eux, où l'abbé était toujours compromis et
presque toujours partie. On a vu l'éclat que le cardinal de Bouillon fit
contre Vertamont, premier président du grand conseil, sur un arrêt
très-important qu'il prétendit que ce magistrat avait falsifié. Il
renouvela ses plaintes contre le grand conseil, même sur un procès d'où
dépendait une grande partie de sa juridiction. Il prétendit que ce
tribunal tirait pension de l'ordre de Saint-Benoît dont toutes les
causes lui étaient attribuées, et qu'aucune de leurs parties n'y pouvait
avoir justice. La chose alla si loin qu'elle fut longtemps devant le
roi, et lui en espérance qu'elle serait évoquée pour être jugée au
conseil de dépêches\footnote{Conseil où l'on traitait des affaires de
  l'intérieur du royaume et où s'expédiaient les dépèches pour les
  gouverneurs, intendants et autres magistrats chargés du gouvernement
  des provinces. Voy. à la fin du t. I\^{}er, p.~445, la note sur les
  \emph{Conseils du roi}.}. Le chancelier, trouvant qu'il y allait de
l'honneur de la magistrature d'attirer cette affaire devant le roi, et,
qu'après cet éclat le grand conseil aussi n'en pouvait demeurer juge,
prit un tempérament, et proposa au roi de la renvoyer à la grand'chambre
à Paris. Le cardinal, fort affligé de ce renvoi, ne laissa pas de faire
les derniers efforts de crédit par sa famille, qui sollicita tant
qu'elle put, et se trouva à l'entrée des juges, où je ne crus pas leur
devoir refuser d'aller avec eux. L'affaire dura longtemps, et nonobstant
tous ces soins elle fut perdue. Ce fut la dernière goutte d'eau qui fait
répandre l'eau d'un verre trop plein, et qui consomma la résolution que
le cardinal de Bouillon roulait depuis longtemps dans sa tête et qu'il
exécuta pendant le siège de Douai.

Avant d'entrer dans ce récit, il faut se souvenir de l'état de la
famille du cardinal de Bouillon, pour mieux entendre les idées
auxquelles il se livra. Sa grand'mère, seconde femme du maréchal de
Bouillon, était fille du fameux fondateur de la république des
Provinces-Unies, et soeur des électrices palatines et de Brandebourg. Sa
mère était Berghes\footnote{Éléonore-Catherine-Fébronie de Berg ou
  Berghes, morte le 14 juillet 1657. Il en est souvent question dans les
  \emph{Mémoires} du cardinal de Retz.}, dont la maison toujours bien
alliée a tenu un rang distingué, parmi la première noblesse des
Pays-Bas, quoique directement sortie par mâles de Jean, sire de Glimes,
bâtard de Jean II, duc de Lothier, c'est-à-dire de Brabant, et légitimé
par lettres du 27 août 1344, à Francfort, de l'empereur Louis de
Bavière. La comtesse d'Auvergne, première femme de son frère, et la
seule dont il ait eu des enfants, était héritière du marquisat de
Berg-op-Zoom par une Witthem, sa mère, et le père de cette comtesse
d'Auvergne était fils de Jean-Georges, comte de Hohenzollern, que
l'empereur Ferdinand III fit prince de l'empire. La seconde femme du
même comte d'Auvergne était Wassenaer, de la première noblesse de
Hollande, des mieux alliés et fort souvent dans les grands emplois de la
république. Le prince d'Auvergne, son neveu, après avoir déserté, comme
il a été dit, avait épousé la soeur du duc d'Aremberg à Bruxelles.
C'était là des alliances qui donnaient au cardinal de grandes espérances
du côté des Pays-Bas, et le prince Eugène était fils d'une soeur de la
duchesse de Bouillon, belle-soeur du cardinal. {[}De{]} ses deux soeurs,
l'une avait épousé le duc d'Elboeuf, dont le duc et le prince
d'Elboeuf\,; l'autre un oncle paternel de l'électeur de Bavière et de
M\textsuperscript{me} la Dauphine, qui était mort sans enfants en 1705,
et elle l'année suivante, aussi en Allemagne.

De toutes ces alliances, il espéra assez de crédit dans les
Provinces-Unies, dans les Pays-Bas et à Vienne, pour procurer au prince
d'Auvergne, à qui dans cette chimère il persuada sa désertion, d'assez
grands établissements qui, aidés du service et des grades militaires, et
de ses terres dans ces pays-là, le portassent au stathoudérat comme
sorti du fameux prince d'Orange, dont la mémoire est encore si chère à
la république qu'il a fondée. Dans cette chimère, il avait fait faire à
sa soeur de Bavière, qui était riche, un testament par lequel elle donna
tous ses biens au prince d'Auvergne, au préjudice de M. de Bouillon et
de ses enfants, et au défaut de toute postérité du prince d'Auvergne à
la maison de Bavière. Le dessein du cardinal était d'enrichir ce prince
d'Auvergne et sa branche, et d'intéresser en lui, en ses biens et en sa
branche, la maison de Bavière par cette substitution qui la regardait.
Il se repaissait donc de ces idées et des heureux arrangements qu'il
avait ménagés pour en disposer les succès, tandis qu'il errait d'abbaye
en abbaye, qu'il tuait le temps en voyages à petites journées, et qu'il
guerroyait avec ses moines. En même temps il épargnait, avec un soin qui
pouvait passer pour avarice, les grands revenus dont il jouissait en
bénéfices immenses et en patrimoine, dont il n'avait jamais voulu se
dessaisir\,; et il amassait pour les futures contingents dont l'ennui et
le dépit de sa situation le tentait, et pour lesquels il voulait
toujours être préparé. Dans cet esprit il fit passer beaucoup d'argent
en pays étrangers, et ne garda que le nécessaire, le portatif et des
pierreries, pour être en liberté de faire toutefois et quantes tout ce
qu'il voudrait.

Dans ces pensées, outré de ne voir point de fin à son exil ni aux
entreprises de ses moines, il profitait de l'adoucissement de son exil,
qui lui permettait d'aller et de venir sans s'approcher trop près, pour
aller de ses abbayes de Bourgogne à celle de Saint-Ouen de Rouen\,; et
il obtint dans ce voyage la liberté de s'arrêter quelques jours aux
environs de Paris, sans toutefois entrer dans la ville. Outre le plaisir
d'y voir sa famille et ses amis, il espéra que ce nouvel adoucissement
influerait sur son procès, prêt à juger à la grand'chambre, et lui
donnerait moyen d'y veiller avec plus de succès. Il vint donc s'établir
pour quelques jours dans le village de Montrouge, et ce fut là qu'il
apprit qu'il avait entièrement perdu son procès, et sans retour toute
juridiction sur les moines réformés de la congrégation de Cluni, à
l'égard desquels il ne lui était rien laissé de plus qu'à tous les abbés
commendataires du royaume. À cette nouvelle la rage où il entra ne se
peut exprimer. Les fureurs, les injures, les transports, les cris
épouvantèrent\,; il ne se posséda plus, et se livra tout entier au plus
violent désespoir\,: vingt-quatre heures ne purent apaiser une agitation
si violente. Le Nain, son rapporteur\,; le procureur général, depuis
chancelier, dont l'avis et les conclusions ne lui avaient pas été
favorables\,; le parlement entier étaient l'objet de ses imprécations.
Le lendemain il passa la Seine au bac des Invalides, et s'en alla à
Ormesson chez Coulanges qui lui était fort attaché.

Dans ce même temps il se faisait une tentative pour son retour, il parut
même que le roi n'y résisterait pas longtemps\,; mais le moment n'en
était pas encore venu, et ce délai, qui concourut avec la perte de ce
procès, acheva de lui tourner la tête et de précipiter sa résolution. Il
n'avait vu à Montrouge que ses neveux d'Auvergne et ses gens d'affaires.
Il ne voulut voir personne à Ormesson que les mêmes, deux ou trois amis
particuliers, quelques gros bonnets des jésuites, comme les PP. Gaillard
et de La Rue, qui étaient tout à lui, encore les fit-il attendre
longtemps avant de les voir, par grandeur ou par humeur. Il demeura une
quinzaine à Ormesson, où apparemment il arrangea toutes les mesures de
sa fuite, sans sortir presque de sa chambre. Comme il avait la liberté
de toutes ses abbayes, il changea son voyage de Normandie en celui de
Picardie, séjourna peu à Abbeville, et gagna Arras, où il avait l'abbaye
de Saint-Waast\,; de là, feignant d'aller voir son abbaye de Vigogne, il
partit dans son carrosse, monta à cheval en chemin, et piqua au
rendez-vous qu'il avait pris, qu'il manqua de quelques heures.

On sut assez tôt à Arras qu'il avait pris la fuite pour débander un
détachement après lui\,; il fut au moment d'y tomber, mais à force de
courre çà et là, il donna enfin dans un gros de cavalerie ennemie avec
lequel son neveu le cherchait, bien en peine de ce qu'il était devenu.
Là, il vomit ce qu'il retenait sur son coeur depuis tant d'années, en ce
premier moment de liberté. Dès qu'ils furent assez avancés pour être en
sûreté, il mit avec son neveu pied à terre dans un village où ils
conférèrent ensemble, puis remontèrent à cheval et arrivèrent à l'armée
des ennemis.

Aussitôt le prince Eugène et le duc de Marlborough le vinrent saluer, et
lui présenter l'élite de l'armée. Ils lui demandèrent l'ordre, il le
leur donna, et ils le prirent\,; en un mot, ils lui rendirent et lui
firent rendre les plus grands honneurs. Un pareil changement d'état
parut bien doux à cet esprit si altier et si ulcéré, et lui enfla
merveilleusement le courage. Il paya ses nouveaux hôtes par les discours
qui leur furent les plus agréables sur la misère de la France, que ses
fréquents voyages par les provinces avaient montrée à ses yeux, sur son
impuissance à soutenir la guerre, les fautes qui s'y étaient faites, le
mauvais gouvernement, les mécontentements de tout le monde, l'épuisement
extrême et le désespoir des peuples\,; enfin il ne les entretint que de
ce qui les pouvait flatter, et n'oublia rien de tout ce que peut la
perfidie et l'ingratitude, en qui un si prodigieux amas de bienfaits est
tourné en poison et en espérance de piédestal à une nouvelle et
indépendante grandeur, dans le même esprit de félonie qui anima ses
pères et qui leur a bâti cette prodigieuse fortune dont les
établissements immenses n'ont pu gagner ni satisfaire eux ni leur
postérité. Le roi apprit cette évasion par un paquet adressé à Torcy,
laissé par le cardinal à Arras, sur sa table. C'était une lettre au roi
avec une simple adresse à Torcy de deux mots. Cette lettre est une si
monstrueuse production d'insolence, de folie, de félonie, que sa rareté
mérite d'être insérée ici. Jusqu'au style est extravagant qui, à force
d'entasser tout ce dont ce coeur et cette tête regorgeait, rend cette
lettre à peine intelligible.

«\,Sire,

«\,J'envoie à Votre Majesté par cette lettre que je me donne l'honneur
de lui écrire, après dix ans et plus des plus inouïes, des plus injustes
et des moins méritées souffrances, accompagnées durant tout ce temps-là
de ma part, de la plus constante et peut-être trop outrée, non-seulement
à l'égard du monde, mais à l'égard de Dieu et de son Église, patience,
et du plus profond silence\,; j'envoie, dis-je, à Votre Majesté avec un
très-profond respect la démission volontaire, qui ne peut être regardée
par personne comme l'aveu d'un crime que je n'ai pas commis, de ma
charge de grand aumônier de France et de ma dignité de l'un des neuf
prélats commandeurs de l'ordre du Saint-Esprit qui a l'honneur d'avoir
Votre Majesté pour chef et grand maître, qui a juré sur les saints
Évangiles le jour de son sacre l'exacte observation des statuts dudit
ordre, en conséquence desquels statuts je joins dans cette lettre le
cordon et la croix de l'ordre du Saint-Esprit, que par respect et
soumission pour Votre Majesté j'ai toujours porté sous mes habits depuis
l'arrêt que Votre Majesté rendit contre moi, absent et non entendu dans
son conseil d'en haut\footnote{Il ne faut pas confondre le
  \emph{conseils d'en haut}, avec les \emph{conseils du roi}, dont il a
  été question à la fin du t. I\^{}er p.~445, note. Le conseil d'en haut
  ne se composait que d'un petit nombre de ministres. Les secrétaires
  d'État y rapportaient les affaires, tandis que, dans les conseils du
  roi, cette fonction appartenait aux maîtres des requêtes. À l'époque
  dont parle Saint-Simon, le roi, le Dauphin, le chancelier, le duc de
  Beauvilliers et le secrétaire d'État rapporteur formaient le conseil
  d'en haut.}, le il septembre 1701. En conséquence, de ces deux
démissions que j'envoie aujourd'hui à Votre Majesté, je reprends par ce
moyen la liberté que ma naissance de prince étranger, fils de souverain,
ne dépendant que de Dieu et de ma dignité de cardinal-évêque de la
sainte Église romaine, et doyen du sacré collège, évêque d'Ostie,
premier suffragant de l'Église romaine, me donne naturellement\,:
liberté séculière et ecclésiastique dont je ne me suis privé
volontairement que par les deux serments que je fis entre les mains de
Votre Majesté en 1671, le premier pour la charge de grand aumônier de
France, la première des quatre grandes charges de sa maison et de la
couronne, et le second serment pour la dignité d'un des neuf prélats
commandeurs de l'ordre du Saint-Esprit, desquels serments je me suis
toujours très-fidèlement et très-religieusement acquitté, tant que j'ai
possédé ces deux dignités, desquelles je me dépose aujourd'hui
volontairement, et avec une telle fidélité aux ordres et aux volontés de
Votre Majesté, en tout ce qui n'était pas contraire au service de Dieu
et de son Église, que je désirerais bien en avoir une semblable à
l'égard des ordres de Dieu et de ses volontés, à quoi je tâcherai de
travailler uniquement le reste de mes jours, servant Dieu et son Église
dans la première place après la suprême où la divine providence m'a
établi, quoique très-indigne\,; et en cette qualité qui m'attache
uniquement au saint-siége, j'assure Votre Majesté que je suis et serai
jusqu'au dernier soupir de ma vie avec le respect profond qui est dû à
la majesté royale, «\,Sire,

«\,De votre Majesté

«\,Le très-humble et très-obéissant serviteur, «\,\emph{Signé} le
cardinal de Bouillon,

«\,Doyen du sacré Collège.\,»

Quoique cette lettre contienne autant de sottises, d'impudence et de
folie que de mots, on ne peut s'empêcher d'en faire quelque analyse.
Premièrement, il faut avoir bonne haleine et bonne mémoire pour aller
jusqu'au bout de la première phrase, et travailler pour démêler les
continuels entrelacements de ses parenthèses et de son sens si suspendu.
Dans cette phrase autant de faux et de vent que d'insolence. Il a
souffert des persécutions qu'il ose reprocher au roi comme les plus
injustes, les plus inouïes et les moins méritées. C'est donc lui dire,
parlant à lui, qu'il est un tyran, puisqu'il faut l'être pour faire
souffrir le plus injustement, et d'une manière inouïe quiconque ne l'a
pas mérité. Mais ces souffrances quelles sont-elles\,? Après avoir
longtemps souffert un spectacle de désobéissance publique sur le premier
théâtre de l'Europe, et toutes les menées passibles pour s'y faire
soutenir par le pape et tout le sacré collège qui le blâmèrent et se
moquèrent de lui, le roi lui saisit son temporel, et par famine
l'obligea enfin à exécuter l'ordre qu'il lui avait donné de revenir en
France, où son temporel lui fut rendu, et où, pour tout châtiment, il
fut exilé dans ses abbayes.

Voilà donc ces souffrances si injustes, si inouïes qu'il a souffertes,
ajoute-t-il, avec une patience outrée à l'égard du monde, de Dieu et de
son Église. Mais en quoi Dieu et son Église sont-ils intéressés en cet
exil\,? Où est l'offense à Dieu, où le préjudice à l'Église\,? Quelle
part a-t-elle pu y prendre\,; et à l'égard du monde où est le
scandale\,? Il est entier ainsi que le péché dans la désobéissance et
dans la lutte de désobéissance poussée si loin et avec tant d'éclat et
non dans une punition devenue nécessaire, pour le faire obéir, adoucie
incontinent après par le relâchement de ses revenus, et réduite à un
simple exil chez lui dans ses abbayes, c'est-à-dire pour un laïque dans
ses terres\,; et cette patience outrée à le supporter, comment eût-il
fait pour ne l'avoir pas\,? et quel gré peut-il en prétendre\,? Il
envoie, dit-il, sa démission volontaire, et il prend grand soin de la
préserver de l'opinion de l'aveu d'un crime qu'il n'a point commis. Il
était cardinal de la nomination du roi, et il était chargé de ses
affaires à Rome\,; en même temps il avoue lui-même qu'il avait prêté
serment au roi. Dans cet état il tombe en la défiance et en la disgrâce
du roi qui le rappelle. Malgré beaucoup d'ordres réitérés et les plus
précis, il s'obstine à demeurer à Rome, ose mettre en question si un
cardinal est obligé d'obéir à son roi, n'oublie rien pour engager la
cour de Rome à prendre parti pour la négative, et donne ce spectacle
public de lutte contre le roi.

En quel siècle, en quel pays n'est-ce point là un crime et un crime de
lèse-majesté le plus grave après celui du premier chef\,? Il est égal à
celui de la révolte à main armée, puisqu'il n'a pas tenu à lui de faire
une affaire d'État et de religion de la sienne particulière et d'armer
pour soi la cour de Rome. Avec quel front ose-t-il donc nier ce crime si
long et si public jusqu'à la délicatesse de se précautionner contre
l'opinion d'un aveu tacite par sa démission, et cette démission il a
grand soin de l'inculquer volontaire, et de marquer en même temps la
date où elle lui a été demandée\,: or il conste de cette date, qu'il a
désobéi près de dix ans à la volonté du roi là-dessus, et neuf à l'arrêt
qui l'a dépouillé, puisque, n'osant avec tout son orgueil continuer à
porter l'ordre après que l'ambassadeur du roi eut été chez lui pour lui
déclarer et lui faire exécuter cet arrêt, il a eu l'enfance et la misère
qu'il avoue ici, et qui ne peut avoir d'autre nom de porter en dessous
ce qu'il n'osait plus montrer en dessus, et témoigna ainsi sa petitesse
et sa faiblesse d'une part, et de l'autre son orgueil et son
opiniâtreté. Envoyer après dix ans de cette conduite sa démission, et
s'évadant du royaume, cela peut-il s'appeler une démission volontaire\,?
n'est-ce point plutôt une dérision, et dire au roi en effet qu'elle
n'est volontaire que parce que rien n'a pu la tirer de lui, tant qu'il
n'a pas voulu la donner, et qu'il ne la donne que parce qu'il sort du
royaume, et qu'il la veut bien donner\,? et pour ajouter toute espèce
d'insulte, il met dans sa lettre au roi un vieux cordon bleu, sale et
gras, avec sa croix du Saint-Esprit, car le cordon était tel à la
lettre.

L'enflure des dignités dont il se démet n'est digne que de risées.
Personne n'ignore qu'à l'institution de l'ordre, Henri III voulant
favoriser Jacques Amyot, son précepteur et du feu roi son frère, qui
avait été récompensé de l'évêché d'Auxerre et de la charge de grand
aumônier de France, celle de grand ou de seul aumônier de l'ordre fut
attachée pour toujours à celle de grand aumônier de France, et sans
faire aucunes preuves parce que Amyot n'en pouvait faire\,; que par
conséquent toujours, depuis, être grand aumônier et porter l'ordre est
une seule et même chose, sans rien de séparé ni de distinct\,; et
qu'ainsi le grand aumônier, quelque grand qu'il soit par soi ou par sa
charge, n'est point autre chose qu'un officier de l'ordre, n'en fait
point le neuvième prélat, qui tous huit font preuves et sont partagés
par moitié en cardinaux et en évêques. C'est donc un pathos très-puéril
que fait ici le cardinal de Bouillon, et une cheville très-inutile, que
l'énoncé qu'il fait que le roi est grand maître de l'ordre, et qu'il en
a juré les statuts à son sacre\,; il est selon les statuts de dégrader
un chevalier de l'ordre pour certains crimes, surtout de félonie, de
lèse-majesté, etc., dont il y a de grands exemples et en nombre\,; à
plus forte raison est-il en la disposition du roi de faire défaire un
officier de l'ordre de sa charge, dont il y a aussi maints exemples, et
de lui en demander la démission. Ce dernier cas s'est vu plus de quinze
ans dans M. de Châteauneuf-Phélypeaux, secrétaire d'État, greffier de
l'ordre, et le portant au lieu de Castille\footnote{Il s'agit de Jeannin
  de Castille, et non de la province de Castille, comme on l'a supposé
  dans l'édition Sautelet, où l'on a, en conséquence, corrigé ainsi la
  phrase\,: \emph{ait lieu de Castille}, où \emph{il fut tout ce
  temps-là exilé}.}, qui fut tout ce temps-là exilé et par delà pour
refuser sa démission, et qui toutefois ne portait plus l'ordre, et ne
l'a jamais porté depuis qu'au bout de quinze ou seize ans il donna sa
démission, et c'est le grand-père maternel du prince d'Harcourt, qui a
pris le nom de Guise. Mais l'exemple d'Amyot est bien plus juste encore
au cardinal de Bouillon\,: aussi ingrat que lui, il s'abandonna à la
Ligue. Henri IV, commençant à devenir le maître, lui ôta la charge de
grand aumônier, et conséquemment l'ordre, qu'il donna au fameux Renauld
de Beaune, archevêque de Bourges alors, puis de Sens, qui venait de lui
donner l'absolution et de le communier dans l'église de l'abbaye de
Saint-Denis.

Pour un homme qui a autant vécu à la cour que le cardinal de Bouillon,
il est difficile de comprendre ce qu'il veut dire ici, quand il y donne
sa charge pour la première des quatre grandes de la maison du roi et de
la couronne. Premièrement, on lui niera tout court que la charge de
grand aumônier soit un office de la couronne, sans qu'il puisse, ni
aucun autre, ni le prouver, ni en montrer la moindre trace. Ces offices
ont ce privilège particulier qu'ils ne se peuvent ôter aux titulaires,
malgré eux, que juridiquement et pour crime. Quand Amyot fut dépouillé,
la Ligue était encore assez puissante pour le soutenir et pour
embarrasser Henri IV, s'il avait fallu du juridique. Il n'en fut pas
seulement question, et Amyot demeura dépossédé et exilé dans son diocèse
le reste de ses jours qui durèrent encore quelques années. En second
lieu, que veut dire le cardinal de Bouillon avec ses quatre charges de
la maison du roi et de la couronne, dont la sienne est la première\,?
A-t-il oublié que rien n'est plus distinct qu'office de la couronne et
grandes charges de la maison du roi, dont aucune ne s'est jamais égalée
à ces offices\,? En troisième lieu, où n'en a-t-il pris que quatre, et
qui sont-elles à son compte\,? Le connétable, et par usage moderne le
maréchal général, le chancelier et par tolérance le garde des sceaux\,;
le grand maître, le grand chambellan, les maréchaux de France, l'amiral,
le grand écuyer, quoique plus ancien que l'amiral, qui marche au milieu
des maréchaux de France, le colonel général de l'infanterie et le grand
maître de l'artillerie sont les officiers de la couronne. Ils sont donc
plus de quatre, comme on voit, et je ne pense pas qu'aucun d'eux se
laissât persuader de céder au grand aumônier. Quant aux grandes charges
de la maison du roi, telles que les premiers gentilshommes de la
chambre, les gouverneurs des rois enfants et des fils de France, les
premiers chefs des troupes de la garde {[}du roi{]}, le grand maître de
la garde-robe, en voilà aussi plus de quatre, et qui ne seraient pas
plus dociles que les officiers de la couronne à céder au grand aumônier.
On ne sait donc ce que veut dire le cardinal de Bouillon, ou plutôt
lui-même ne le sait pas, mais sa bouffissure est si générale, qu'il se
loue d'avoir exercé cette charge très-fidèlement et
très-religieusement\,: c'est une absurdité que son extrême orgueil lui a
cachée\,; fidèlement, dans une désobéissance éclatante et
très-criminelle dix ans durant, et, à son sens, il était toujours alors
grand aumônier, puisqu'il n'avait pas donné sa démission\,;
religieusement, ni ses moeurs, ni la cour, ni le monde ne lui rendirent
ce témoignage. Voilà pour le personnel, venons maintenant à la
naissance.

En conséqnence de ces démissions de la charge et de l'ordre qu'il veut
toujours séparer pour amplifier vainement, il reprend, écrit-il au roi,
la liberté que lui donne sa naissance de prince étranger, fils de
souverain, ne dépendant que de Dieu et de sa dignité de cardinal,
etc.\,; c'est-à-dire que c'est un manifeste adressé au roi sous la forme
d'une lettre, par lequel il lui dénonce son indépendance prétendue, et
sa très-parfaite ingratitude\,; i1 attente à la majesté de son souverain
en abdiquant sa qualité innée de sujet, et encourt ainsi le crime de
lèse-majesté en plein. Je ne répéterai point ce qui a été expliqué (t.
V, p.~298 et suiv.) de la nature des fiefs de Bouillon, Sedan, etc., de
l'état, comme seigneurs de ces fiefs, de ceux qui les ont possédés, de
la manière dont ils sont entrés dans la famille du cardinal de Bouillon,
du rang que son grand-père, premier possesseur de ces fiefs, a tenu
devant et depuis qu'il les a possédés, de celui de la branche de la
maison de La Marck qui les possédait avant lui, et de quelle manière
enfin son père obtint ce prodigieux échange de ces fiefs, et le rang de
prince étranger. On y voit clairement la mouvance de ces fiefs de Liège
et de l'abbaye de Mouzon, et la violence, non aucun autre titre qui {[}a
fait que{]} par la protection si indignement reconnue d'Henri IV, ces
fiefs sont demeurés au grand-père du cardinal de Bouillon. D'où il
résulte qu'à ces titres, jamais son grand-père ni son père ne furent
souverains ni princes, conséquemment qu'il n'est ni prince étranger ni
fils de souverain, et qu'il ment à son roi avec la dernière impudence.
Il n'a donc point de liberté de rien reprendre à ce titre par la
démission de sa charge, et il demeure tel qu'il était auparavant,
c'est-à-dire gentilhomme français de la province d'Auvergne, du nom de
La Tour, tel qu'il était auparavant, par conséquent sujet du roi comme
tous les autres gentilshommes de cette province, laquelle appartient à
la couronne\,; que si son père, en faveur d'un échange déjà si
étrangement énorme, que, depuis tant d'années de toute puissance du roi
et de toute faveur de M. de Bouillon, il n'a pu être entièrement passé
au parlement, le père du cardinal a obtenu pour sa postérité et pour son
frère le rang de prince étranger malgré les cris et les oppositions de
la noblesse qui le leur fit ôter et qui leur fut rendu, ce que le
parlement a toujours constamment ignoré, c'est une grâce fort injuste,
mais dont le roi est le maître, et dont le bienfait ne donne pas la
manumission de l'état de sujet, et ne peut changer la naissance. C'est
donc le dernier degré d'égarement que montre ici le cardinal de
Bouillon, duquel se sont toujours bien gardés ceux dont la naissance
issue de souverains véritables et actuels ne pouvait être disputée\,:
tels que les Guise qui, dans le plus formidable éclat de leur puissance,
prête à les porter sur le trône, n'ont jamais balancé à se déclarer
sujets, au temps même où ils osèrent faire considérer Henri III comme
déchu de la couronne, et Henri IV comme incapable d'y succéder. Si
l'idée du cardinal de Bouillon pouvait être véritable, non dans un
gentilhomme français comme lui, mais dans un prince, par exemple, de la
maison de Lorraine, il s'ensuivrait que quelque patrimoine qu'il eût en
France, en renonçant aux charges qu'il posséderait, il reprendrait cette
liberté qu'allègue le cardinal de Bouillon, et une pleine
indépendance\,; d'où il résulterait que jamais les rois ne pourraient
être assurés de ceux de cette naissance qui, par elle, seraient en tout
temps les maîtres de demeurer ou de n'être plus leurs sujets.

Le cardinal de Bouillon ajoute qu'il est volontairement privé de cette
liberté par le serment de grand aumônier, laquelle il reprend par sa
démission de cette charge. Encore une fois, ce n'est pas d'un
gentilhomme français tel que lui que je parle, c'est d'un prince de la
naissance dont il ose se dire, et dont il n'est pas. Si ce qu'il dit là
était véritable, lui qui avait un patrimoine en France, lui et les
siens, et rien ailleurs, les princes de la maison de Lorraine établis en
France et qui y ont tout leur bien ne seraient donc pas sujets du roi,
comme il y en a plusieurs qui n'ont ni charge ni gouvernement, et qui
par conséquent ne sont liés à ce titre par {[}aucun{]} serment\,: ce
paradoxe est aussi nouveau qu'incompréhensible. Mais par qui et à qui
est-il si audacieusement avancé\,? Par un gentilhomme originaire de la
province d'Auvergne, dont les pères n'ont jamais eu ni prétendu aucune
distinction ni supériorité quelconque sur pas une des bonnes maisons de
cette province, jusqu'au grand-père du cardinal de Bouillon lorsqu'il
eut Sedan et Bouillon, et qu'aucun ne lui passa jamais ni devant ni
depuis. Et à qui\,? à un des plus grands rois qui aient régné en France,
son souverain, duquel son père tint deux fois la dignité de duc et pair,
son oncle, la première charge de la milice, un gouvernement de province,
la charge de colonel général de la cavalerie, tous deux après avoir
pensé renverser l'État, tous deux après avoir vécu
d'abolitions\footnote{Les abolitions étaient des lettres du souverain
  obtenues en grande chancellerie, par lesquelles il abolissait et
  effaçait un crime qui, de sa nature, n'était pas rémissible, et en
  vertu de la plénitude de sa puissance, remettait la peine portée par
  la loi.}\,; son frère aîné, la charge de grand chambellan et le
gouvernement de sa propre province, avec les survivances pour son fils
qui, tôt après, s'en montra si ingrat\,; son autre frère, un autre
gouvernement de province, et la charge de colonel général de la
cavalerie\,; eux tous le rang de prince étranger, et lui-même, une
profusion énorme des plus grands et des plus singuliers bénéfices, le
cardinalat en un âge qui l'a porté au décanat et la charge de grand
aumônier, avec la faveur la plus distinguée. C'est de cet amas inouï des
plus grands bienfaits versés sur deux générations de frères, que le
cardinal de Bouillon se fait des armes contre celui-là même dont il les
tient, et en parlant à lui. On s'arrête ici, parce que le comble
d'ingratitude est trop au-dessus de tout ce qui se pourrait dire, ainsi
que de l'insolence.

Peu content d'un si monstrueux orgueil, il revient au dédoublement de
son cardinalat pour en multiplier la grandeur, avec une fatuité la plus
misérable. Doyen du sacré collège n'est-ce pas être cardinal, n'est-ce
pas être évêque d'Ostie, n'est-ce pas être le premier suffragant de
Rome, et rien de tout cela peut-il être distinct ou séparé\,? Mais voici
où l'ivresse excelle\,: c'est la première place après la suprême. Il
parle au roi comme il parlait aux paysans de la Ferté lorsqu'il y passa
deux mois, et qu'après avoir quelquefois dit la messe à la paroisse, il
leur faisait admirer en sortant, non la grandeur du mystère qu'il venait
de célébrer, mais la sienne, de lui qui était prince, et qui avait la
première place après la suprême\,; qu'ils le regardassent bien,
ajoutait-il parce que jamais ils n'avaient vu cela dans leur église, et
qu'après lui cela n'y arriverait jamais. Ce peuple ne le comprenait
pas\,; le curé qui avait de l'esprit, et les honnêtes gens du lieu en
riaient entre eux et en avaient pitié. À quelque point d'élévation que
la dignité de cardinal ait été portée, la distance est demeurée si
grande entre le pape et leur doyen que cette expression favorite du
cardinal de Bouillon, qu'il répétait sans cesse à tout le monde, ne put
imposer à personne, et ne peut montrer que le vide et le dérangement de
sa tête.

Toute la fin de la lettre n'est qu'une insulte diversifiée en plusieurs
façons plus insolentes les unes que les autres. Il s'y récrie sur sa
fidélité aux ordres et aux volontés du roi, et il y ajoute cette honnête
et respectueuse restriction\,: en tout ce qui n'était pas contraire au
service de Dieu et de son Église. C'est donc à dire, et en parlant au
roi même, qu'il était capable de vouloir des choses qui y étaient
contraires, qu'il lui en avait même commandé. Il appuie encore ici sur
sa fidélité\,; mais fut-elle le principe de toutes les brigues qu'il
employa pour se faire élire évêque de Liège contre la volonté et les
défenses du roi si déclarées, qu'il ne le manqua que parce que le roi
s'y opposa d'une manière si formelle, qu'il fit déclarer au chapitre
qu'il préférait tout autre au cardinal de Bouillon qui avait les dix
voix, même le candidat porté par la maison d'Autriche, ce qui fit
changer le chapitre et manquer ce siége au cardinal de Bouillon\,? Sa
fidélité fut-elle le motif qui lui fit employer tant de ruses et de
manéges pour tromper le pape et le roi, et réciproquement, persuader à
l'un et à l'autre de faire nécessairement son neveu cardinal en
contre-poids du duc de Saxe-Zeitz, porté vivement par l'empereur à la
promotion duquel le roi s'opposait plus fortement encore, fourberie dans
laquelle le pape et le roi donnèrent si bien, qu'elle ne fut découverte
que par la déclaration que le roi fit au pape qu'il aimait mieux qu'il
passât outre à la promotion du duc de Saxe-Zeitz seul, que d'y consentir
par celle de l'abbé d'Auvergne\,; et pour lors ni de longtemps après, le
duc de Saxe ne le fut\,? Le cardinal de Bouillon était alors à Rome
chargé des affaires du roi, et abusant de sa confiance, à cet énorme
degré. Enfin, pour se borner à quelque chose, était-ce fidélité, aux
ordres les plus exprès du roi, des affaires duquel il était encore
chargé à Rome, que toute la conduite qu'il y tint sur la coadjutorerie
de Strasbourg et sur l'affaire de M. de Cambrai\,? et après des traits
si étranges et si publics, vanter sa fidélité avec reproche\,!

Non content d'une effronterie si incroyable, cet évêque, ce cardinal, ce
premier suffragant de l'Église romaine, cet homme qui réserve avec tant
de religion ce qui la peut blesser dans les ordres du roi, ne craint pas
d'ajouter le blasphème le plus horrible, par le souhait qu'il fait tout
de suite d'avoir pour les ordres et la volonté de Dieu la pareille
fidélité qu'il a eue pour ceux du roi. La protestation qui suit est de
même nature, avec les desseins et les motifs qui le faisaient s'évader
du royaume\,: il proteste, dis-je, qu'il tâchera le reste de ses jours
de servir uniquement Dieu et son Église dans la place, et c'est là où il
paraphrase et multiplie si follement la grandeur de cette place, où la
Providence, dit-il, l'a établi, quoique indigne. Ce dernier mot est la
seule vérité qui lui soit échappée dans toute cette lettre. Mais c'est
au roi à qui il dit que la Providence l'y a établi, à ce même roi qui
l'a nommé cardinal dans un âge qui l'a porté au décanat, à ce même roi
malgré le rappel duquel, faisant ses affaires à Rome, il s'y est
cramponné avec tant d'artifice, puis de désobéissance publique jusqu'à
ce qu'il l'eût recueilli. Il ajoute après que cette qualité l'attache
uniquement au saint-siége, c'est-à-dire l'affranchit de tout autre
attachement, et de celui du roi qui l'a nommé cardinal, et de qui lui et
les siens tiennent tout\,; mais la fin de sa lettre, où il arrive ainsi,
se signale par deux déclarations qui portent encore plus que tout le
reste le crime sur le front. Il assure le roi qu'il sera jusqu'au
dernier soupir de sa vie, avec le respect le plus profond qui est dû à
la majesté royale, son très-humble et très-obéissant serviteur. Cette
expression du respect qui est dû à la majesté royale avertit bien
clairement le roi par sa singularité et sa netteté, de se ne pas
méprendre au respect qu'il lui porte, et de ne pas prendre pour sa
personne ce qui n'est dû qu'à sa couronne, et pour fin, en supprimant le
nom de sujet, il en dénie la qualité avec encore plus de force qu'il n'a
fait dans tout ce tissu de sa lettre, qui peut passer, quoiqu'un grand
galimatias, pour un chef-d'oeuvre d'ingratitude, d'audace et de folie.

\hypertarget{chapitre-xviii.}{%
\chapter{CHAPITRE XVIII.}\label{chapitre-xviii.}}

1710

~

{\textsc{Réflexion sur le rang de prince étranger\,; son époque.}}
{\textsc{- Temporel du cardinal de Bouillon saisi.}} {\textsc{- Ordre du
roi au parlement de lui faire son procès.}} {\textsc{- Conduite de sa
maison.}} {\textsc{- Lettre du roi au cardinal de La Trémoille.}}
{\textsc{- Réflexions sur cette lettre.}} {\textsc{- Cardinal de
Bouillon, etc., décrétés de prise de corps par le parlement, qui après
s'arrête tout court, et les procédures tombent.}} {\textsc{- Réflexion
sur les cardinaux français.}} {\textsc{- De Bar, faussaire des Bouillon,
se tue à la Bastille.}} {\textsc{- Baluze destitué et chassé.}}
{\textsc{- Arrêt du conseil qui condamne au pilon son \emph{Histoire
généalogique de la maison d'Auvergne} bon à voir.}} {\textsc{-
Collations du cardinal de Bouillon commises aux ordinaires des lieux.}}
{\textsc{- Tout monument de prétendue principauté ôté des registres des
curés de la cour, et des abbayes de Cluni et de Saint-Denis, par ordre
du roi.}} {\textsc{- Nouvelles félonies du cardinal de Bouillon à
Tournai.}} {\textsc{- Duc de Bouillon bien avec le roi\,; sa femme et
ses fils mal, et ses neveux.}} {\textsc{- Duc de Bouillon parle au roi
et au chancelier.}} {\textsc{- Écrivant au roi, {[}il{]} n'avait jamais
signé sujet, et ne put être encore induit à s'avouer l'être.}}
{\textsc{- Articles proposés au roi, à faire porter de sa part au
parlement, sur la maison de Bouillon.}} {\textsc{- Justice et usage de
ces articles.}} {\textsc{- Fausse et criminelle rature dans les
registres du parlement.}} {\textsc{- Le roi ordonne à d'Aguesseau,
procureur général, de procéder sur ces articles au parlement, qui élude
et sauve la maison de Bouillon.}} {\textsc{- Infidélité de Pontchartrain
en faveur du cardinal de Bouillon.}} {\textsc{- Réflexions.}} {\textsc{-
Mort du prince d'Auvergne.}} {\textsc{- Le roi défend à ses parents d'en
porter le deuil, et fait défaire le frère de l'abbé d'Auvergne d'un
canonicat de Liége.}} {\textsc{- Cardinal de Bouillon se fait abbé de
Saint-Amand contre les bulles données, sur la nomination du roi, au
cardinal de La Trémoille.}} {\textsc{- Le roi désire inutilement de
faire tomber la coadjutorerie de Cluni.}} {\textsc{- Extraction, fortune
et mariage du prince de Berghes avec une fille du duc de Rohan.}}
{\textsc{- Perte du duc de Mortemart au jeu.}} {\textsc{- Le secrétaire
du maréchal de Montesquiou passe aux ennemis avec ses chiffres.}}

~

Tel est le danger du rang de prince donné à des gentilshommes français,
inconnu avant la puissance des Guise, même pour ceux de maison
souveraine, et pour des gentilshommes avant le règne de Louis XIV.
Devenus princes, ils deviennent honteux de demeurer sujets. Le vicomte
de Turenne, ainsi que ses pères, était demeuré fidèle et avait très-bien
servi Henri IV jusqu'au moment que ce monarque lui procura Bouillon et
Sedan. Ce fut l'époque de ses félonies\footnote{Crime commis par un
  vassal contre la foi qu'il devait à son seigneur.}, dont le reste de
sa vie et celle de ses deux fils fut un tissu, comme le remarquent
toutes les histoires, et que ses fils n'abandonnèrent que par la
difficulté de les plus soutenir\,; et par les monstrueux avantages que
le cardinal Mazarin leur procura dans ses frayeurs personnelles pour
s'en faire un appui. Tel est aussi le danger de permettre à ceux de ce
dangereux rang des alliances étrangères. Mais ces réflexions, qui
naissent abondamment, ne doivent pas trouver ici plus de place.

Quoique ce fût la morsure d'un moucheron à un éléphant, le roi s'en
sentit horriblement piqué. Il avait en sa main la vengeance. Il reçut
cette lettre le 24 mai\,; la remit le lendemain 25 à d'Aguesseau,
procureur général, lui fit remarquer qu'elle était toute de la main du
cardinal de Bouillon, et lui ordonna de la porter au parlement, et d'y
former sa demande de faire le procès au cardinal de Bouillon comme
coupable de félonie. Le roi rendit en même temps un arrêt dans son
conseil d'en haut, qui, en attendant les procédures du parlement, mit en
la main du roi tout le temporel du cardinal, et dit que sa lettre est
encore plus criminelle que son évasion. Ses neveux, exactement avertis,
vinrent ce même jour 25 à Versailles. Ils n'osèrent d'abord se présenter
devant le roi. Les ministres, qu'ils virent, leur dirent qu'ils le
pouvaient faire. Ils ne furent point mal reçus. Le roi leur dit qu'il
les plaignait d'avoir un oncle si extravagant. M\textsuperscript{me} de
Bouillon, qui était ou faisait la malade à Paris, écrivit au roi des
compliments pleins d'esprit et de tour, et on verra bientôt pourquoi
cette lettre d'une femme qui avait son mari si à portée du roi, que le
roi n'aimait point et qui n'allait pas deux fois l'an lui faire sa
cour\,; mais tout était concerté, et M. de Bouillon se trouva à Évreux,
qu'on envoya avertir et qui trouva tout cela fait en arrivant pour
guider après ses démarches. Le 26 le roi écrivit au cardinal de La
Trémoille, chargé de ses affaires à Rome, en lui envoyant une copie de
{[}la lettre{]} du cardinal de Bouillon pour en rendre compte au pape\,;
il est nécessaire d'insérer ici cette lettre du roi au cardinal de La
Trémoille.

«\,Mon cousin, il y a longtemps que j'aurais pardonné au cardinal de
Bouillon ses désobéissances à mes ordres, s'il m'eût été libre d'agir
comme particulier dans une affaire où la majesté royale était
intéressée. Mais comme elle ne me permettait pas de laisser sans
châtiment le crime d'un sujet qui manque à son principal devoir envers
son maître, et je puis ajouter encore envers son bienfaiteur, tout ce
que j'ai pu faire a été d'adoucir par degrés les peines qu'il avait
méritées. Aussi non-seulement je lui ai laissé la jouissance de ses
revenus lorsqu'il est rentré dans mon royaume, mais depuis je lui ai
permis de changer de séjour, quand il m'a représenté les raisons qu'il
avait pour sortir des lieux où j'avais fixé sa demeure. Enfin je lui
avais accordé, sans même qu'il me l'eût demandé, la liberté d'aller dans
telle province et tel endroit du royaume qu'il lui plairait, pourvu que
ce fut à une distance de trente lieues de Paris\,; et lorsque, pour
abréger sa route, il a passé à l'extrémité de cette ville, il a séjourné
aux environs, je ne m'y suis pas opposé. Il supposait qu'il allait en
Normandie pour régler quelques affaires, qu'ensuite il passerait à Lyon,
mais il crut devoir faire enfin connaître le véritable motif et unique
but de son voyage. Au lieu d'aller à Rouen et de passer à Lyon, comme il
l'avait assuré à sa famille, il a fait un assez long séjour en Picardie,
et passant ensuite à Arras, il s'est rendu à l'armée de mes ennemis,
suivant les mêmes sûretés qu'il avait prises avec celui de ses neveux
qui sert actuellement dans la même armée, et qui dès le commencement de
cette guerre avait donné l'exemple de désertion que son oncle vient de
suivre. Le cardinal de Bouillon l'ayant imité dans sa fuite m'a de plus
écrit une lettre dont je vous envoie la copie. Il me suffirait pour
punir son orgueil, d'abandonner cette lettre aux réflexions du public,
mais il faut un exemple d'une justice plus exacte à l'égard d'un sujet
qui joint la désobéissance à l'oubli de son état et à l'ingratitude des
bienfaits dont j'ai comblé sa personne et sa maison\,; et le rang où je
l'ai élevé ne me dispense pas de m'acquitter à son égard des premiers
devoirs de la royauté. J'ordonne à mon parlement de Paris de procéder
contre lui suivant les lois. Vous communiquerez la lettre qu'il m'a
écrite, et vous informerez Sa Sainteté de la manière dont il a passé à
mes ennemis, car il est nécessaire que le pape connaisse par des preuves
aussi évidentes le caractère d'un homme qui se prétend indépendant. Dieu
veuille que cette ambition sans bornes, soutenue seulement par la haute
idée de doyen des cardinaux, ne cause pas un jour quelque désordre dans
l'Église\,; car que peut-on présumer d'un sujet prévenu de l'opinion
qu'il ne dépend que de lui de se soustraire à l'obéissance de son
souverain\,? Il suffira que la place dont le cardinal de Bouillon est
présentement ébloui, lui paroissant inférieure à sa naissance et à ses
talents, il se croira toutes voies permises pour parvenir à la première
dignité de l'Église, lorsqu'il en aura contemplé la splendeur de plus
près, car il y a lieu de croire que son dessein est de passer à Rome. Je
doute que ce soit de concert avec Sa Sainteté, et s'il avait pris
quelques mesures secrètes avec elle, je suis persuadé qu'elle se
repentirait bientôt du consentement qu'elle aurait donné. Quoi qu'il en
soit, mon intention est que, le cardinal de Bouillon arrivé à Rome, vous
n'ayez aucun commerce avec lui, et que vous le regardiez non-seulement
comme un sujet rebelle, mais comme se glorifiant de son crime. Vous
avertirez aussitôt les François qui sont à Rome, aussi bien que les
Italiens qui sont attachés à mes intérêts, de se conformer aux ordres
que je vous donne à son égard\,; sur quoi, je prie Dieu qu'il vous ait,
mon cousin, en sa sainte et digne garde.\,»

Cette lettre reçut peu d'approbation\,; on trouva bien peu décent qu'à
un manifeste aussi injurieux qu'était la lettre du cardinal de Bouillon
au roi, un si grand monarque et si délicat sur le point de son autorité,
prît de si faibles devants à Rome, et répondît comme par un autre
manifeste, qui descendait dans un si bas détail de justification de
l'exil du cardinal de Bouillon\,; qu'il parût craindre un concert avec
le pape d'aller à Rome, et qu'en le montrant il n'y opposât qu'un
chimérique soupçon sur le pontificat dont il n'était pas possible que le
pape pût s'émouvoir. On ne devait pas espérer, au point où en étaient
les cardinaux, de faire trouver bon à la cour de Rome les procédures
contre un des leurs, et de plus leur doyen. Cette promptitude et cette
manière basse de la prévenir n'était bonne qu'à lui faire sentir ses
forces, au lieu d'agir, et de la laisser courir après. Le cardinal de
Bouillon s'en enorgueillit davantage\,; il écrivit au président de
Maisons, sur les procédures dont on le menaçait, une lettre plus
violente encore que celle qu'il avait écrite au roi\,; et fit faire des
écrits de même style sur l'immunité prétendue des cardinaux de toute
justice séculière en quelque cas que ce puisse être, et même de toute
autre que de celle du pape conjointement avec tout le sacré collége.

Le parlement, saisi du procès, rendit un arrêt de prise de corps contre
le cardinal de Bouillon, le sieur de Certes, gentilhomme, son
domestique, qu'il employait fort dans ses intrigues et qui était allé et
venu avec beaucoup de hardiesse à l'occasion de celle-ci, et un jésuite
qui s'en était fort mêlé\,; mais quand il fallut aller plus loin, il se
trouva arrêté par la difficulté des procédures et cette immunité des
cardinaux, confirmée par tant d'exemples que les rois n'ont pu franchir,
et que ceux qui ont voulu se faire justice ne l'ont pu qu'en ayant
recours aux voies de fait, dont les exemples ne sont pas rares, et dont
Rome s'est prudemment tue, si on excepte l'exécution du cardinal de
Guise, parce que Rome se vit appuyée de la formidable puissance de la
Ligue. Les jésuites, de tout temps aux Bouillon, soutinrent sourdement
ce danger de tout leur crédit\,; la politique et la conscience s'unirent
à ne se pas commettre avec Rome, tellement qu'après tout ce fracas et ce
procès même signifié au pape, comme on vient de le voir, tomba de
faiblesse et s'exhala, pour ainsi dire, par insensible transpiration.
Belle leçon aux plus puissants princes, qui, au lieu de se faire un
parti à Rome, en y donnant leur nomination, et de ceux qui l'obtiennent,
et de ceux qui l'espèrent et de tout ce qui tient à eux, gens toujours
sur les lieux, instruits de tout et agissant pour leur service, et
vigilants à la mort des papes à toutes les intrigues qui la suivent,
élèvent de leurs sujets à une grandeur inutile à leurs intérêts, par
leur absence de Rome où ils n'ont ni parents, ni amis, ni faction, et ne
sont bons qu'à envahir trois ou quatre cent mille livres de rente en
bénéfices, du demi-quart desquelles un Italien se tiendrait plus que
récompensé. {[}Un cardinal français{]} est en France l'homme du pape
contre le roi, l'État et l'Église de France\,; se rend chef et le tyran
du clergé, trop ordinairement du ministère, est étranger de lien
d'intérêt, de protection, est hardi à tout parce qu'il est inviolable,
établit puissamment sa famille, et, quand il a tout obtenu, est libre
après de commettre, tête levée, tous les attentats que bon lui semble
sans jamais pouvoir être puni d'aucun.

Après tant d'éclat, on se rabattit à des mortifications plus sensibles
que n'eussent peut-être été des procédures sans exécution. On se souvint
de celle de la chambre de l'Arsenal contre les faussaires, et de son
arrêt du 11 juillet 1704 contre la fausseté prouvée et avouée du célèbre
cartulaire de Brioude, et contre Jean-Pierre Bar, son fabricateur, qui,
se voyant trompé dans l'espérance de protection et d'impunité que lui
avait donnée le cardinal de Bouillon et sa famille, qui l'avaient mis en
besogne, se cassa la tête contre les murs de sa chambre à la Bastille, à
ce que j'ai su de Maréchal qui fut mandé pour l'aller voir, à qui il ne
cacha pas le désespoir qui le lui avait fait faire, et qui en mourut
deux jours après. On s'indigna contre Baluze et cette magnifique
généalogie bâtie sur cette imposture qu'il fit imprimer à Paris avec
privilège sous son nom, avec le titre \emph{d'Histoire généalogique de
la maison d'Auvergne, }de toutes lesquelles choses j'ai parlé en leur
temps. On sentit l'énormité d'une complaisance si contradictoire à la
vérité et à l'arrêt de l'Arsenal, et on essaya d'y remédier par un arrêt
du conseil du 1\^{}er juillet 1710, qu'il n'est pas inutile d'insérer
ici.

«\,Sur ce qu'il a été représenté au roi \emph{étant} en son conseil que
dans le livre intitulé \emph{Histoire généalogique de la maison
d'Auvergne}, imprimé à Paris chez Antoine Dezallier, deux volumes
in-folio, le sieur Baluze, auteur de cette histoire, avait non-seulement
osé avancer différentes propositions sans aucune preuve suffisante, mais
encore que, pour autoriser plusieurs faits avancés contre toute vérité,
il avait inséré dans le volume des preuves plusieurs titres et pièces
qui avaient été déclarées fausses par arrêt de la chambre de l'Arsenal
du 11 juillet 1704, qui est une entreprise d'autant plus condamnable
que, outre le mépris d'un arrêt si authentique et rendu en si grande
connaissance de cause, un pareil ouvrage ne peut être fait que pour
appuyer une usurpation criminelle et ménagée depuis longtemps par les
artifices les plus condamnables, et pour tromper le public dans des
matières aussi importantes que le sont les droits ou les prétentions des
grandes maisons du royaume\,; à quoi étant nécessaire de pourvoir, et
tout considéré, le roi \emph{étant} en son conseil a ordonné et ordonne
que le privilége accordé par Sa Majesté pour l'impression de ladite
\emph{Histoire généalogique de la maison d'Auvergne}, en date du 8
février 1705, sera rapporté pour être cancellé, et qu'il sera fait
recherche exacte de tous les exemplaires dudit ouvrage, qui seront
déchirés et mis au pilon. Enjoint Sa Majesté au sieur d'Argenson,
conseiller d'État et lieutenant général de police à Paris, de tenir la
main à l'exécution du présent arrêt, et d'en certifier M. le chancelier
dans huitaine. Fait au conseil d'État, Sa Majesté y étant, tenu à
Versailles, le premier jour de juillet 1710. \emph{Signé} Phélypeaux.\,»

On imprima qnantité d'exemplaires de cet arrêt, on les distribua à
pleines mains à qui en voulut, pour rendre la chose plus authentique. Le
peu de patrimoine que le cardinal de Bouillon n'avait pu soustraire fut
incontinent confisqué\,; le temporel de ses bénéfices était déjà saisi,
et le 7 juillet il parut une déclaration du roi, qui, privant le
cardinal de Bouillon de toutes ses collations, les attribuait aux
évêques dans le diocèse desquels ces bénéfices se trouveraient situés.
En même temps, Baluze fut privé de sa chaire de professeur au Collège
royal et chassé à l'autre bout du royaume.

Mais tout cela n'allait pas au fait, et montrait seulement en opposition
une indigne complaisance dans un temps, par le privilége donné à ce
livre, au mépris de l'arrêt de l'Arsenal antérieur, et une colère
impuissante dans un autre. Le roi fut excité contre l'injustice, le
désordre et l'abus de ces rangs de princes étrangers donnés à des
gentilshommes français, et il y prêta l'oreille\,; il donna ses ordres
pour la visite de l'abbaye de Cluni, et de tous les monuments d'orgueil
qu'en manière de pierre d'attente, le cardinal de Bouillon y entassait
depuis si longtemps, comme descendant des ducs de Guyenne, suivant la
fausseté du cartulaire de Brioude, fabriqué par ce de Bar, il descendait
masculinement des fondateurs de Cluni. C'était sa chimère de tout temps,
que, faute de preuves et de toute vérité ni vraisemblance, il appuya
enfin de cette insigne fausseté. Il avait en attendant multiplié à Cluni
les actes et les marques de cette fausse descendance dans les temps de
sa faveur et de son autorité, sous prétexte de bienfaits de sa part, et
de reconnaissance des moines\,; il y avait fait conduire les corps de
son père, de sa mère, de plusieurs de ses neveux, et, sous prétexte de
piété, se faisait de leur sépulture des titres et des monuments de
grandeur, avec tout l'art, la hardiesse et la magnificence possible.

Le parlement rendit, le 2 janvier 1711, arrêt portant commission au
lieutenant général de Lyon de visiter cette abbaye, et d'y faire
entièrement biffer et effacer tout ce qui, en quelque façon que ce pût
être, en monuments ou en écritures, était de cette nature, et cela fut
pleinement exécuté.

Le roi fit rapporter de Paris, de Fontainebleau, de Saint-Germain et de
Versailles tous les registres des curés, où la qualité de prince fut
rayée, biffée et annotée en marge, que le cardinal de Bouillon y avait
prise aux baptêmes et aux mariages qu'il avait faits à la cour comme
grand aumônier. Le 15 juillet de cette année 1710, il fut envoyé une
lettre de cachet à l'abbaye de Saint-Denis, accompagnée d'officiers
principaux des bâtiments du roi, pour ôter les armes des Bouillon
partout où ils les avaient mises à la chapelle où M. de Turenne est
enterré, ce qui fut assez légèrement exécuté. Lors de sa mort, et que le
roi fit tant pour sa mémoire, il ne voulut pas que les honneurs
prodigués aux héros tournassent en titres pour sa maison\,: il défendit
très-expressément à Saint-Denis tout ce qui pouvait sentir le moins du
monde le prince, surtout ce titre nulle part, et même que ses armes, ou
entières ou semées, y fussent souffertes nulle part à son tombeau, ni
dans sa chapelle, et c'est ce qui fit que les Bouillon ne voulurent ni
inscriptions sur le cercueil ni épitaphe au dehors\,; mais dans les
suites, à force de caresser les moines, d'ouvrir la bourse, d'être
faciles sur des collations, enfin d'orner un peu cette chapelle, les
armes de la maison, et entières et semées, furent glissées au tombeau et
à l'autel, à la voûte et dans les vitrages, même celles du cardinal de
Bouillon avec le chapeau, comme ayant fait la dépense.

Ces coups furent très-sensibles aux Bouillon\,; mais ce n'était pas le
temps de se plaindre, mais de couler doucement de peur de pis\,; et sous
l'apparente rigueur de l'exécution, de profiter de la faiblesse et du
peu de fidélité des gens des bâtiments pour conserver des vestiges, en
attendant d'autres temps où ils pussent hasarder encore une fois ce
qu'ils y avaient mis une première. Le cardinal de Bouillon éclata sur
toutes ces exécutions avec plus d'emportement que jamais. Il avait dès
auparavant gardé si peu de mesure, qu'il avait officié pontificalement
dans l'église de Tournai au \emph{Te Deum} de la prise de Douai, et que,
de cette ville où il avait fixé sa demeure, il écrivit une grande lettre
à M. de Beauvau, qui en était évêque lorsqu'elle fut prise, et qui ne
voulut ni chanter le \emph{Te Deum}, ni prêter serment, ni demeurer,
quoi que pussent faire les principaux chefs pour l'y engager\,; et par
cette lettre le cardinal de Bouillon l'exhortait à retourner à Tournai
et à s'y soumettre à la domination présente, et n'y ménageait aucun
venin.

Ces recherches des registres des curés de la cour, et dans les abbayes
de Cluni et de Saint-Denis, si promptement suivies des nouveaux éclats
du cardinal de Bouillon, jetèrent le duc son frère en d'étranges
inquiétudes des suites que cela pourrait avoir. Ce fut la matière de
force consultations dans sa famille, et avec ses plus intimes amis. Il
avait auprès du roi le mérite de cinquante années de domesticité et de
familiarité, celui de la plus basse flatterie et d'une grande
assiduité\,; et par-dessus ceux-là si puissants auprès du roi, il en
avait un autre qui les faisait encore plus valoir, c'est qu'il avait
fort peu d'esprit. Il avait ployé avec art et soumission sous les orages
que le cardinal et la duchesse de Bouillon s'étaient attirés, et qui,
sans l'avoir jamais directement regardé, n'avaient pas laissé de
l'entraîner plus d'une fois dans leur exil. Toutes ces choses avaient
touché le roi\,; il disait que c'était un bon homme\,; il ne craignait
rien de lui\,; il le plaignait de ses proches, et il s'était accoutumé à
avoir pour lui de la considération et de l'amitié. Son fils aîné était
mort depuis longtemps dans un reste de disgrâce profonde\,; le duc
d'Albret était un homme que le roi ne voyait jamais, et qu'il n'aimait
point, le chevalier de Bouillon beaucoup moins. Il était d'une débauche
démesurée et d'une audace pareille qui ne se contraignait sur rien, qui
disait du roi que c'était un vieux gentilhomme de campagne dans son
château qui n'avait plus qu'une dent, et qu'il la gardait contre lui. Il
avait été chassé et mis en prison plus d'une fois, et n'en était pas
plus sage. Le comte d'Évreux qui avait fort plu au roi par l'amitié du
comte de Toulouse, et qui avec bien moins d'esprit que ses frères avait
plus de sens et de manége, ne se voit plus depuis la campagne de
Lille\,; il boudait et ne paraissait presque plus à la cour. Il ne
restait du comte d'Auvergne que deux fils en France, tous deux prêtres,
tous deux sans esprit, l'aîné plein d'ambition et de petits manéges,
encore plus d'une débauche qui le bannissait du commerce des honnêtes
gens, et en tout genre fort méprisable et méprisé. Le cadet, qui n'avait
pas ces vices, était une manière d'hébété obscur qui ne voyait personne.
Ainsi M. de Bouillon n'avait point de secours dans sa famille que
soi-même.

Dans cet état pressant, il s'adressa au chancelier, puis un matin au roi
lui-même qu'il prit dans son lit, avec la commodité, le loisir et le
tête-à-tête de cette privance des grandes entrées, où chacun de ce
très-peu qui les ont se retire à l'autre bout de la chambre, ou même en
sort dès qu'on en voit un d'eux qui veut parler au roi. Là, M. de
Bouillon déplora sa condition, les folies de son frère, s'épuisa en
louanges au roi, en actions de grâces de ses bienfaits, surtout en
reconnaissances de sa sujétion, parce que ce n'était qu'en paroles, en
compliments, et encore tête à tête\,; pria, pressa, conjura le roi
d'arrêter les effets de sa colère, et, pour un coupable que sa famille
avait le malheur d'avoir produit, ne pas flétrir sa maison. Le roi,
quelque temps froid et silencieux, puis peu à peu ramené à ses premières
bontés par la soumission de tant de propos affectueux, lui répondit
qu'il ne demandait pas mieux que de continuer à distinguer sa personne
et sa famille de son frère rebelle et criminel, mais que la révolte de
son frère portant coup pour toute sa maison, par le déni fait à lui-même
d'être son sujet, par sa lettre sur le fondement de sa naissance, il ne
pouvait tolérer cette injure sans s'en ressentir, et que c'était au duc
lui-même à voir ce qu'il pouvait faire pour donner lieu à éviter ce que
ce déni méritait. M. de Bouillon, fort soulagé par de si bonnes paroles,
redoubla de protestations et de fatras de compliments, supplia le roi de
trouver bon qu'il en parlât à quelqu'un, et lui nomma le chancelier. Le
roi y consentit, et le duc espéra dès lors de sortir bien de cette
périlleuse affaire.

Il ne tarda pas d'aller chez le chancelier. Le roi l'avait instruit, le
chancelier ne le lui cacha pas\,; et comme il savait très-bien
distinguer les choses d'avec les paroles et les propos, il ne tâta point
de celles-ci, et proposa de celles-là. Le fait était, et ce fait est
inconcevable, qu'avec toutes les injures que le duc de Bouillon disait
de son frère au roi, il ne s'estimait pas plus que lui son sujet\,; et
il avait droit d'avoir cette opinion, parce que jamais, en écrivant au
roi, il n'avait mis le mot de \emph{sujet}, et que cette omission
jusqu'alors lui avait été tolérée sans aucune difficulté. Or c'était là
maintenant de quoi il s'agissait, et à quoi on le voulait réduire, et
c'était pour soutenir cet usage dans cette crise que
M\textsuperscript{me} de Bouillon avait pris occasion d'écrire au roi.
Indépendamment de la nature mouvante et jamais souveraine de Sedan et de
Bouillon, indépendamment de la manière dont ces fiefs étaient venus et
demeurés au grand-père et au père de M. de Bouillon, indépendamment de
toutes les félonies qui les leur avaient fait perdre, et de la manière
dont le roi s'en était saisi, toutes choses bien destructives de
souveraineté dans les ducs de Bouillon, le père de celui-ci en avait
fait avec le roi un échange à un avantage en tout genre si prodigieux,
qu'il n'avait pas à s'en plaindre, et celui-ci encore moins depuis le
temps qu'il en jouissait. Avec le rang de prince étranger, la
souveraineté, quand elle eût existé, ne pouvait lui être demeurée,
puisqu'il était dessaisi et dépouillé volontairement de Bouillon et de
Sedan, que le roi possédait en vertu de l'échange. Le domaine simplement
utile\footnote{C'est-à-dire les revenus des terres séparés des droits de
  souveraineté.}laissé à M. de Bouillon n'opérait rien à cet égard\,;
pas un mot des droits, de l'effet, de l'exception de la souveraineté, ni
d'état personnel de souverain, ni dans le contrat d'échange, ni dans le
brevet de rang de prince étranger. Nulle raison, nul prétexte même le
plus frivole à M. de Bouillon de n'être et ne s'avouer pas sujet du roi,
lui duc et pair, grand chambellan, et qui n'avait pas même un pouce de
terre hors du royaume, ni lui, ni ses enfants.

Le chancelier, avec des raisons si péremptoires, n'en oublia aucune pour
lui persuader qu'il n'avait aucun prétexte pour se soustraire à cette
qualité, ni le roi, avec ce qui se passait, aucun non plus de l'endurer
davantage\,; lui remontra tous les fâcheux inconvénients, et tous en la
main du roi, qui pouvaient lui arriver de sa résistance\,; il essaya de
le porter à se reconnaître sujet du roi par un écrit signé par lui, par
ses enfants et par ses neveux, tout fut inutile. M. de Bouillon ne
connut rien de pis que cet aveu et il espéra tout de sa propre
souplesse, de celle du P. Tellier, de ce mélange de bonté et de
faiblesse du roi pour lui, surtout de son peu de suite dans ces sortes
d'affaires, dont il avait si souvent fait d'heureuses expériences. Sa
famille, néanmoins, qui toute se sentait si personnellement mal chacun
avec le roi, craignit d'irréparables foudres, et le pressa d'accorder au
danger et à l'angoisse des conjonctures l'écrit proposé par le
chancelier\,; mais il résista également à eux et à ses plus intimes
amis, et leur répondit avec indignation qu'il était trop maltraité pour
y consentir. Le mauvais traitement consistait donc à la radiation des
faussetés de Bar et de la qualité de prince aux monuments dont j'ai
parlé, et à ôter à Saint-Denis ce que le roi n'y avait jamais voulu
permettre, et qu'il avait expressément défendu lorsqu'il y fit porter M.
de Turenne, et ce que, contre ses ordres, ils y avaient frauduleusement
mis depuis. En tout autre pays qu'en France cet insolent refus de M. de
Bouillon eût suffi seul pour les accabler et surtout pour leur ôter à
jamais ce rang de prince qui soutenait leur chimère, et que ce refus
impudent réalisait autant qu'il était en eux, et s'il était souffert,
autant qu'il était au pouvoir du roi à l'égard d'une chose à qui tout
fondement de vérité manquait, mais qui n'en devenait pas moins
dangereuse.

C'est ce qui fit que, sans plus s'arrêter à l'écrit proposé et rejeté
par M. de Bouillon avec une fermeté qui découvrait le fond de son coeur,
et qui même donné par lui aurait toujours pu passer pour un effet de sa
peur et d'une espèce de violence, il fut proposé au roi de prendre un
biais plus juridique et plus exempt de tout soupçon, parce qu'il était
selon les lois, les règles et les formes\,; ce fut que, le procureur
général fit assigner M. de Bouillon, ses enfants et ses neveux pour voir
dire\,:

«\,I. Que Sedan est fief de Mouzon et arrière-fief de la couronne, ainsi
qu'il conste par sa nature, par les lettres patentes de Charles VII, en
1454, comme souverain seigneur de Mouzon, d'où Sedan relevait, et par
jugement en conformité de ces lettres, rendu à Mouzon en 1455, et qu'il
n'y a titre ni preuve en aucun temps de l'indépendance de Sedan\,;

«\,II. Que Bouillon est originairement mouvant de Reims, et arrière-fief
de la couronne\,; cette mouvance acquise en 1127 de Renaud, archevêque
de Reims, par Albéron, évêque de Liége, seigneur de Bouillon, et que,
passant des évêques de Liége dans la maison de La Marck, ils n'en ont
jamais cédé la mouvance ni même la propriété territoriale, qui a sans
cesse, jusqu'à ce jour, été réclamée et revendiquée par les évêques de
Liége\,;

«\,III. Que Sedan, Bouillon, ensuite Raucourt, Jamets et Florenville,
ces trois derniers fiefs sans nulle apparence d'indépendance ni
prétention d'eux-mêmes, ont passé par voie d'acquisition de la maison de
Braquemont et des évêques de Liége dans la maison de La Marck\,;

«\,IV. Que la maison de La Marck n'a jamais prétendu à la souveraineté
par ces fiefs, et a fait actes du contraire, si ce n'est le père de
l'héritière, première femme et sans enfants du grand-père de M. de
Bouillon et du cardinal son frère, qui se prétendit indépendant\,; aucun
de cette branche de La Marck-Bouillon n'a eu ni prétendu en France, ni
en aucun lieu de l'Europe, à la qualité ni à aucun rang de prince\,;

«\,V. Que ces fiefs de Bouillon, Sedan et leurs dépendances n'ont été
réputés ni dénommés que simples seigneuries, et leurs possesseurs que
seigneurs jusqu'au père susdit de l'héritière, qui le premier usurpa,
sans titre et sans approbation, le titre de prince de Sedan, et qu'à
l'égard de Bouillon il n'a jamais été et n'est encore duché, mais simple
seigneurie\,;

«\,VI. Que lesdits fiefs ne sont passés de la maison de La Marck dans
celle de La Tour ni par acquisition, ni par succession ni à aucun titre
qu'elle puisse montrer, mais par la seule protection du roi Henri IV\,;

«\,VII. Que lesdits fiefs n'ont pas changé de nature entre les mains de
la maison de La Tour, laquelle à ce titre ne peut plus prétendre que n'a
fait la maison de La Marck\,;

«\,VIII. Que la postérité d'Acfred, duc de Guyenne et comte d'Auvergne,
est depuis longtemps éteinte\,;

«\,IX. Que mal à propos la maison de La Tour a usurpé, adopté et joint à
son nom de La Tour le nom à elle étranger d'Auvergne, puis substitué
seul au sien, sans qu'elle en puisse montrer d'autre titre que ce faux
cartulaire de Brioude, fait par le nommé de Bar, condamné comme
faussaire, et qui en a fait l'aveu, et le cartulaire déclaré faux et
condamné comme tel par l'arrêt de la chambre tenue à l'Arsenal du 11
juillet 1704\,;

«\,X. Que cette innovation de nom n'est pas plus ancienne que le père du
cardinal de Bouillon\,;

«\,XI. Que défenses seront faites à ceux de la maison de La Tour de plus
prendre le nom d'Auvergne seul ni joint avec le leur, et que le nom
d'Auvergne seul ou joint au leur sera rayé ou biffé dans tous les actes
ci-devant passés, contrats et autres pièces où il sera trouvé, et dont
recherches seront faites\,;

«\,XII. Que mêmes défenses et exécutions seront faites à l'égard des
armes d'Auvergne pour qu'il ne reste pas trace de telle chimérique
prétention\,;

«\,XIII. Que les seigneurs de la maison de La Tour sont seigneurs
français, sujets du roi comme toutes les autres maisons nobles du
royaume, se diront tous, s'avoueront, se soussigneront tels\,;

«\,XIV. Que lesdits seigneurs de La Tour n'ont aucune descendance
d'Acfred, duc de Guyenne et comte d'Auvergne, dont la postérité est dès
longtemps éteinte\,; et qu'à titre des fiefs et seigneuries de Bouillon,
Sedan, etc., ne pouvant prétendre à la qualité et titre de ce prince,
ces titres et qualités seront biffés et rayés partout où ils les auront
pris ainsi que dessus, et défenses à eux faites de les prendre ni porter
à l'avenir\,;

«\,XV. Que lesdits seigneurs de la maison de La Tour seront condamnés à
toutes réparations, amendes, dommages et intérêts pour avoir usurpé les
noms, armes, titres, usages et prétentions indues, sans droit ni
apparence de droit, et à eux entièrement étrangers, et destitués de tout
titre à ce faire.\,»

Les preuves de ces quinze articles qui se trouvent légèrement tracées
ci-dessus (t. V, p.~298-326) avaient été solidement examinées avant de
proposer ces articles. Ils allaient tous à l'entière destruction de la
chimère d'indépendance, de souveraineté, de principauté\,; ils allaient
plus directement au coeur du cardinal de Bouillon, que quoi qu'on eût pu
faire contre sa personne, quand bien même on en eût été en possession,
et affranchi du bouclier du cardinalat. Tous ces articles étaient vrais,
justes, conséquents, n'outraient rien, ils se tenaient dans le fond de
la chose dont il s'agissait entre le roi et les Bouillon, et y
procédaient par maximes tirées \emph{ex visceribus causae}, et par leurs
conséquences naturelles. En même temps ils n'attaquaient en rien
l'échange dans aucune de ses parties, ils ne touchaient pas même au rang
de prince étranger, inconnu au parlement, et grâce au roi, qui n'a
besoin d'autre fondement que de sa volonté quand il lui plaît qu'elle
soit plus gracieuse pour quelques-uns que juste pour tous les autres, et
qui pour la maison de Rohan n'a ni la chimère d'un Acfred ni des
prétentions de souveraineté pour prétexte.

Ces articles étaient tous de la plus pure compétence du parlement\,; et
il était parfaitement du ministère du procureur général, l'homme du roi
et le censeur public, d'y en porter sa plainte. Dès le premier pas, MM.
de Bouillon assignés se seraient trouvés dans la nécessité de répondre.
S'ils s'étaient sentis hors de moyen de soutenir juridiquement les
usurpations de leur faveur et de leurs manéges, comme il est sans doute
qu'ils s'en seraient trouvés dans l'entière impuissance et qu'ils
eussent acquiescé, toute leur chimère était anéantie et par leur propre
aveu subsistant à toujours dans les registres du parlement. Si malgré
cette impuissance ils avaient essayé de répondre, il est hors de doute
encore qu'ils auraient été condamnés avec plus de solennité, et leur
chimère, anéantie et proscrite sans retour, aurait servi de châtiment
pour eux et de leçon pour d'autres, sans le moindre soupçon de force ni
de violence\,; et c'était après au roi à voir s'il lui convenait, avec
tout ce qui se passait là-dessus avec eux, de leur laisser le rang de
prince étranger. Il se trouvera dans les Pièces un mémoire qui fut
précipitamment demandé et fait en ce temps-là, et qui aurait été
meilleur si l'on avait eu plus de deux fois vingt-quatre heures à le
faire, sur les maisons de Lorraine, de Rohan et de La Tour. Enfin un
procès entre le roi et MM. de Bouillon, non pour des terres et de
l'argent, comme il en a tous les jours avec ses sujets, mais pour raison
de la qualité de sujet, à raison de l'effet de ses propres grâces, de
l'effet d'une descendance fausse d'un côté, d'une transmission forcée et
sans titre de l'autre, et de plus très-onéreusement échangée pour le roi
et dont la nature est un arrière-fief de sa couronne, eût été un
très-singulier spectacle, et qui aurait mis en parfaite évidence que la
chimère n'était que pour un temps, et que les prétentions réelles sur
des provinces, comme patrimoine de ses pères, se réservaient pour
d'autres temps\,; on laisse à juger de l'importance et du danger de
laisser lieu à ces choses.

Mais si la hardiesse et l'art de MM. de Bouillon a pu, à l'égard du roi,
tout ce qu'on vient de rapporter, et des monuments qu'ils se sont faits
peu à peu dans les registres des curés de la cour, dans l'abbaye de
Cluni, et contre les précautions et les ordres les plus exprès du roi
dans celle de Saint-Denis, en voici un trait bien plus difficile à
pratiquer. On a déjà dit que le rang et le nom de prince étranger sont
inconnus au parlement, qui ne reconnaît de princes que ceux du sang
habiles à la couronne\,; ainsi ce rang accordé par le roi dans sa cour à
MM. de Bouillon n'a pu être enregistré au parlement, et le roi n'a
jamais songé à le vouloir\,; quelque puissant qu'il soit, il n'est
maître ni des noms ni des descendances, il ne l'est ni des titres
antérieurs à lui des terres, ni de la spoliation de sa couronne, ni de
son domaine, moins, s'il se peut encore, de son suprême domaine, ni des
effets que le droit attache à ces choses\,; par conséquent il n'a pu et
ne peut jamais faire don à personne d'aucune de ces choses, ni en faire
vérifier le don au parlement, comme en effet il n'en a enregistré
aucun\,; mais les noms de prétendue souveraineté et principauté de
Sedan, Bouillon, etc., se trouvent dans la partie de l'échange qui est
enregistrée, le mot de \emph{prétendue} y est rayé. Or cette rature, qui
est un attentat, et qui a été soufferte, ne prouve que l'attentat, le
crédit pour la tolérance et une hardiesse inouïe et sans exemple comme
sans effet, parce qu'il ne se fait ni ne se peut jamais faire de
radiation d'un seul mot sur les registres du parlement, qu'en vertu d'un
arrêt du conseil ou du parlement qui l'ordonne, et d'une note marginale
à côté qui exprime la date et l'arrêt qui l'a ordonnée\,; et comme il
n'y a ni note marginale ni arrêt qui ait ordonné la radiation de ce mot
\emph{prétendue}, il résulte qu'elle est un pur attentat, et que cette
radiation est nulle de tout droit.

Ces quinze articles furent donc présentés au roi avec les raisons de
leur usage tel qu'il vient d'être expliqué. Il en sentit l'équité et
l'importance, et il comprit aussi que le traité d'échange vérifié ne
portait que sur les terres données en échange, sur l'érection d'Albret
et de Château-Thierry en duchés-pairies, sur la réservation du simple
domaine utile de Bouillon, sur l'abolition des crimes de félonie et
autres, mais que le rang de prince étranger accordé aussi et jamais
vérifié, ni possible à être présenté au parlement pour l'être, demeurait
toujours en sa main royale à titre de volonté, soit pour l'ôter, soit
pour le laisser, quelque arrêt qui pût intervenir dans cette affaire,
dont ce rang ne pouvait être matière. Ainsi, content sur la jalousie de
son autorité, il manda Pelletier, premier président, et d'Aguesseau,
procureur général, auquel il ordonna de procéder ainsi qu'il vient
d'être expliqué.

Ce procureur général, si éclairé, si estimé, de moeurs si graves, se
trouva l'ami intime du duc d'Albret, dont la vie et les moeurs
répondaient si peu aux siennes, et cette amitié, liée dès leur première
jeunesse, s'était toujours si bien entretenue depuis, que le duc
d'Albret n'avait d'autre conseil dans ses affaires que d'Aguesseau, et
que dans celle de la substitution qu'il eut avec tant d'éclat contre le
duc de Bouillon son père, ce fut d'Aguesseau, lors avocat général, qui,
à visage découvert, y fit tout, au point que M. de Bouillon, hors
d'espérance d'accommodement, n'osa risquer le jugement au parlement de
Paris, et fit, par autorité du roi, qui, pour la première fois de sa
vie, se voulut bien montrer partial, et le dire, renvoyer le procès au
parlement de Dijon. Le procureur général reçut avec grand respect les
ordres du roi et force protestations d'obéissance\,; il fit bientôt
naître des difficultés\,; il reçut de nouveaux ordres\,; ils furent
réitérés\,; il les voulut du roi lui-même. Il ne s'effraya point de la
fermeté que le roi lui témoigna dans sa volonté pour la seconde fois. Il
multiplia les difficultés si bien, qu'il donna de l'ombrage sur son
intention, et le confirma par la même conduite. Celui par qui tout
passait entre le roi et d'Aguesseau, fatigué d'un procédé si bizarre,
détourna deux audiences que ce dernier s'étaient ménagées, et ne pouvant
parer la troisième, il s'y trouva en tiers, répondit à tout, aplanit
tout, et indigné de ce qu'il ne se pouvait plus dissimuler, par ce qu'il
voyait du procureur général, il le mit hors du cabinet du roi presque
par les épaules.

Pour achever de bien entendre tout ceci, il faut savoir qu'il y avait
trois canaux dans toute cette affaire\,: celui que je ne nomme point,
qui, par extraordinaire, donna les ordres du roi pour Cluni\,;
Pontchartrain, comme secrétaire d'État de la maison du roi, qui en fut
naturellement chargé pour Saint-Denis, et qui le fit avec tant d'éclat
et de partialité en même temps pour les Bouillon, dont avec raison il
tenait à grand honneur d'avoir épousé l'issue de germaine, que celui qui
avait donné les ordres pour Cluni le fit remarquer au roi, et lui enleva
ceux dont par sa charge il devait être naturellement chargé pour le
procureur général\,; le chancelier, par son office à l'égard du
parlement, qui en cela comme en toute autre affaire pensait et sentait
tout au contraire de son fils. Le procureur général continuait ses
difficultés, et lorsqu'on croyait l'avoir mis au pied du mur, il en
inventa de nouvelles, non sur la chose et le fond qui n'en était pas
susceptible, mais sur cent bagatelles accessoires dont il composait des
volumes de mémoires en forme de questions raisonnées, dans le dessein
d'ennuyer le roi et de lui faire quitter prise, en homme qui connaissoit
bien le terrain. Enfin, tout étant arrêté et convenu, il donna parole
par écrit à celui qui lui donnait les ordres du roi, et au chancelier
aussi, d'aller en avant sans plus de difficultés, et ils croyaient la
chose certaine, quand, à trois jours de là, il revint avec un nouveau
mémoire pour montrer comme en éloignement, avec aussi peu de fondement
que de bonne foi, la part que les alliés, enflés de leurs succès et
excités par le cardinal de Bouillon, pourraient prendre à propos de
Bouillon et de Sedan. Ce mémoire était encore plein de difficultés,
habilement entortillé, expressément diffus et gros, tellement que le
roi, à qui il fallut le communiquer, fatigué à la fin et excédé, se
dépita, et eut plus tôt fait de céder à une opiniâtreté si soutenue et
si importune, que de lire et de discuter ce vaste mémoire, et qu'il aima
mieux surseoir l'exécution de ses ordres.

Le chancelier, outré de colère, et de la chose, et du manquement du
procureur général à la parole qu'il lui avait donnée si fraîchement par
écrit, le traita en petit procureur du roi de siége subalterne. L'autre
adjoint ne l'épargna pas davantage\,; tous deux lui reprochèrent son
infidélité et sa prévarication. Il fut outré de honte et de désespoir,
mais consolé sans doute d'avoir sauvé son bon ami et sa maison d'un
naufrage si certain. Le premier président, dont l'avis et la volonté
pour procéder fut toujours constante, mais dont la faiblesse d'esprit se
sentait trop de celle du corps, eut à se reprocher de n'avoir pas été
assez ferme, ou plutôt de ne se l'être pas montré autant qu'il l'était
intérieurement là-dessus. Il fut le seul du parlement de ce secret qui
fut su de très-peu de personnes. Celui que je ne nomme pas était mon ami
très-intime, tellement que jour à jour il ne m'en laissa rien ignorer,
ni le chancelier non plus. On espéra y revenir par quelque autre voie\,:
l'occasion s'en offrit bientôt par la prise d'un vaisseau chargé
d'argent, de meubles et de papiers du cardinal de Bouillon\,; mais
Pontchartrain vendu aux Bouillon, qui avait la marine dans son
département, étouffa la prise et fit tout rendre au cardinal. Telle fut
l'issue d'une affaire de cet éclat, où le roi, l'État et tout ce qui le
compose avait un si grand intérêt, et de la colère et des menaces si
publiques et si justes d'un roi si absolu, contre un rebelle, auquel sur
ce point toute sa maison abhéra nettement en effet\,; ainsi sont servis
les rois qui ne parlent à personne, et les royaumes qui sont gouvernés
comme le nôtre.

Le cardinal de Bouillon n'eut pas longtemps à rouler ses grands projets
sur la Hollande\,; il perdit, deux mois après son évasion, le prince
d'Auvergne, ce neveu pour lequel il ne songeait pas à moins qu'au
stathoudérat des Provinces-Unies. Il mourut de la petite vérole les
derniers jours de juillet, et laissa son oncle dans la plus inexprimable
douleur\,; ce fut le commencement de sa chute aux Pays-Bas, d'où il ne
put depuis se relever ni même en Italie. Ce déserteur ne laissa qu'une
fille, qui nous ramènera dans peu au cardinal de Bouillon. Longtemps
depuis, elle épousa le prince palatin de Sultzbach, et de ce mariage qui
dura peu, étant morts tous deux jeunes, est venu le prince de Sultzbach
d'aujourd'hui, qui va succéder à tous les États et à la dignité de
l'électeur palatin. Le roi, intérieurement piqué, défendit à M. de
Bouillon et à tous les parents du prince d'Auvergne d'en porter le
deuil, et lui dit tout crûment qu'il était réputé mort du jour que, par
arrêt du parlement, il avait pour sa désertion été pendu en Grève, en
effigie. On prit la liberté à l'oreille de trouver cela petit, et la
marque d'une colère impuissante. Il fit commander en même temps au frère
de l'abbé d'Auvergne de se défaire d'un canonicat qu'il avait à Liége.
Sur ce point au moins et sur le deuil il fut obéi.

Le roi avait donné depuis quelque temps au cardinal de La Trémoille la
riche abbaye de Saint-Amand en Flandre, lequel en avait obtenu les
bulles. Cette abbaye était depuis tombée au pouvoir des ennemis par les
progrès de leurs conquêtes. Le cardinal de Bouillon, qui ne comptait
plus sur aucune des siennes en France, s'avisa sur la fin de l'année,
pour le dire ici tout de suite, de s'en faire élire abbé par la moindre
partie des moines. Vingt-deux autres protestèrent contre cette élection.
Il ne laissa pas d'être curieux de voir ce premier suffragant de
l'Église romaine, ce doyen des cardinaux, qui ne dépend plus, à ce qu'il
écrit au roi, que de Dieu et de sa dignité, et qui ne veut plus songer
qu'à servir Dieu et son Église, se faire élire contre les bulles du pape
et, malgré lui et le pourvu, jouir à main armée des revenus de l'abbaye
par la protection seule des hérétiques.

Les moines de Cluni furent excités sous main de chercher s'il n'y avait
point de moyens qui pussent leur donner lieu d'attaquer la coadjutorerie
de l'abbé d'Auvergne\,; le roi même voulut bien qu'ils sussent que cela
lui serait agréable, autre marque d'impuissante colère quand on a en
main, avec justice et raison, tout ce qu'il faut pour tirer la vengeance
la plus durable et la plus sensible. L'affaire apparemment se trouva si
bien cimentée qu'on ne put y réussir.

Le prince de Berghes, de la maison duquel j'ai parlé à propos de la mère
du cardinal de Bouillon, revint de l'armée de Flandre, au commencement
de la campagne, épouser une fille du duc de Rohan dont il se voulait
défaire à bon marché. Son père était gouverneur de Mons lorsque le roi
le prit. Celui-ci était un très-laid et vilain petit homme, de corps et
d'esprit, dont il avait fort peu, mais il avait une soeur chanoinesse de
Mons, belle et bien faite et d'un air fort noble, qui s'appelait
M\textsuperscript{lle} de Montigny, qui n'avait rien, et dont l'électeur
de Bavière devint amoureux après qu'il eut quitté M\textsuperscript{me}
d'Arco, mère du comte de Bavière, et l'a été jusqu'à sa mort. Il obtint
pour le frère de sa maîtresse une compagnie des gardes du corps du roi
d'Espagne à Bruxelles, l'ordre de la Toison d'or, et enfin la grandesse.
Il est mort sans enfants plusieurs années après\,; et sa soeur en est
devenue grande dame, de laquelle il n'est pas encore temps de parler.

Avant de quitter la Flandre, il faut dire que le duc de Mortemart était
venu apporter au roi la capitulation de Douai, et lui rendre compte du
siége. On fut étonné qu'un homme si marqué, et par sa charge si fort
approché du roi, eût pris une commission si triste, de laquelle il
s'acquitta même si mal que le roi en fut embarrassé par bonté. J'aurais
dû mettre cet article à la suite de la prise de Douai, c'est un oubli
que je répare.

Retourné à l'armée de Flandre, il se mit à jouer tête à tête avec
d'Isenghien à l'hombre, qui y jouait assez mal et qui n'était rien moins
que joueur. C'est le même qui, longues années depuis, est devenu
maréchal de France. L'amusement grossit bientôt, parce que M. de
Mortemart fut piqué d'éprouver la fortune contraire. Tant fut procédé
qu'à force de multiplier les séances, d'enfermer M. d'Isenghien chez
lui, et d'y grossir les parties, malgré lui, qui gagnait, et qui avec
toute l'honnêteté du monde n'osait le refuser, malgré ses remontrances
et celles des spectateurs, que M. de Mortemart perdit, ce qu'il n'a
jamais voulu dire, dont M. d'Isenghien le racquitta enfin, jusqu'à près
de cent mille francs. Cette perte fit grand bruit dans l'armée. M.
d'Isenghien dont la probité était connue, et qui n'était ni joueur ni
encore moins adroit, avait eu avec la fortune les meilleurs et les plus
honnêtes procédés.

On fut choqué qu'un homme fût capable de faire un tel voyage à un jeu
comme l'hombre. Le roi le fut beaucoup, et la cour ne s'en tut pas. M.
de Beauvilliers fut au désespoir de la chose, et de son effet, et de
tout ce qu'elle lui faisait envisager.

Ce n'était pas le premier chagrin cuisant que lui causa ce gendre, ce ne
fut pas aussi le dernier. Sa fille déjà si malheureuse était grosse\,;
elle s'en blessa de déplaisir, et en fut à la dernière extrémité, M. de
Beauvilliers me parla fort confidemment de toutes ses douleurs. Je
l'avais laissé venir là-dessus à cause de ce qui s'était passé entre son
gendre et moi sur M\textsuperscript{me} de Soubise, que j'ai raconté en
son lieu.

Le payement fit encore beaucoup parler. Les ducs de Chevreuse et de
Beauvilliers s'attachèrent trop littéralement au délabrement des
affaires du duc de Mortemart, et à la raison de conscience de préférer
des dettes de marchands et d'ouvriers qui souffraient, et de gens qui
avaient prêté leur bien, à celle qui venait du jeu et d'une grosse
perte\,; ils en essuyèrent force blâme et force propos du monde, dont M.
d'Isenghien continua de mériter l'approbation et les louanges par la
continuation des meilleurs procédés. Je ne pus m'empêcher d'avertir MM.
de Chevreuse et de Beauvilliers du bruit et de l'effet de cette
conduite, et j'eus grande peine à leur faire entendre combien l'honneur
était intéressé à payer promptement les dettes du jeu, et combien le
monde était inexorable là-dessus. Enfin M. de Mortemart que le siége de
Douai avait fait maréchal de camp céda son régiment à M. d'Isenghien à
vendre, et pour le reste de la somme M. de Beauvilliers prit les délais
tels qu'il voulut, et acheva enfin de tout payer.

Une autre aventure y fut plus fâcheuse\,: le secrétaire du maréchal de
Montesquiou, gagné depuis longtemps par le prince Eugène, craignit enfin
d'être découvert, et, tout à la fin de la campagne, disparut, et s'en
alla à Douai avec tous les chiffres et les papiers de son maître. On
changea tous les chiffres, mais on ne put douter que tout ce qu'on avait
cru de plus secret ne l'avait pas été pour les ennemis.

\hypertarget{chapitre-xix.}{%
\chapter{CHAPITRE XIX.}\label{chapitre-xix.}}

1710

~

{\textsc{Art et manége du P. Tellier sur les bénéfices.}} {\textsc{-
Mailly, archevêque d'Arles, passe à Reims.}} {\textsc{- Janson
archevêque d'Arles.}} {\textsc{- Le Normand évêque d'Évreux, Turgot
évêque de Séez.}} {\textsc{- Dromesnil évêque d'Autun, puis de Verdun.}}
{\textsc{- Abbé de Maulevrier\,; sa famille\,; son caractère.}}
{\textsc{- Mort de l'abbé de Langeron.}} {\textsc{- Cardinal Gualterio
met les armes de France sur la porte de son palais à Rome.}} {\textsc{-
Mort de M\textsuperscript{me} de Caderousse\,; naissance et caractère
d'elle et de son mari.}} {\textsc{- Ducs d'Avignon\,; ce que c'est.}}
{\textsc{- Mort du lieutenant civil Le Camus\,; son caractère.}}
{\textsc{- Argouges lieutenant civil.}} {\textsc{- Mort de Lavienne,
premier valet de chambre du roi.}} {\textsc{- Mort de la marquise de
Laval.}} {\textsc{- Mort de Denonville.}} {\textsc{- Duchesse de Luynes
gagne un grand procès contre Matignon.}} {\textsc{- Mort du marquis de
Bellefonds.}} {\textsc{- Le marquis du Châtelet gouverneur et capitaine
de Vincennes.}} {\textsc{- Souper de Saint-Cloud.}} {\textsc{- Tentative
de la flotte ennemie sur Agde et le port de Cette, sans succès.}}
{\textsc{- Situation de l'Espagne.}} {\textsc{- M\textsuperscript{me}
des Ursins fait un léger semblant de la quitter.}} {\textsc{- M. de
Vendôme de nouveau demandé par l'Espagne.}} {\textsc{- Le roi d'Espagne
en Aragon, à la tête de son armée\,; Villadarias sous lui.}} {\textsc{-
Duc de Medina-Celi arrêté, conduit à Ségovie, puis à Baronne, avec
Flotte.}} {\textsc{- Petits exploits des Espagnols.}} {\textsc{-
Staremberg bat les quartiers de l'armée du roi d'Espagne, qui se retire
sous Saragosse.}} {\textsc{- Vendôme va en Espagne, est froidement reçu
à la cour, et mal par M\textsuperscript{me} la duchesse de Bourgogne.}}

~

Il s'était amassé beaucoup de bénéfices à donner. Le P. Tellier, qui
faisait tout sous terre, et qui n'imitait en rien le P. de La Chaise,
bannit les temps accoutumés de les remplir autant qu'il put, qui étaient
les jours de communion du roi, pour mettre les demandeurs en désarroi,
éviter de trouver le roi prévenu en faveur de quelqu'un pour qui on
aurait parlé à temps, et se rendre plus libre et plus maître des
distributions. Il exclut autant qu'il lui fut possible tout homme connu
et de nom, et ne voulut que des va-nu-pieds et des valets à tout faire,
gens obscurs, à mille lieues d'obtenir ce qu'on leur donnait, et qui se
dévouaient sans réserve aux volontés du confesseur, à l'aveugle, et sans
même les savoir, et gens au reste à n'oser broncher après. Il avait dès
lors ses vues, qu'il commencait à préparer, et pour cela choisit ses
gens le mieux qu'il put.

On sut donc à la mi-juillet plusieurs évêchés et grand nombre d'abbayes
donnés, le tout ensemble de deux cent quarante mille livres\,; mais on
ne le sut que peu à peu, dans le dessein de faire faire les nominations
à son gré, qu'il sentait bien qu'ils ne le seraient pas à celui du
public ni de personne. Il craignit la rumeur qu'exciteraient les listes,
comme on les donnait auparavant\,; il les supprima, tant pour cette
raison que pour n'être pas forcé, par la publicité de la liste et le
remercîment au roi, de donner aux nommés ce qui leur était destiné s'il
n'y trouvait pas son compte, et en ce cas faire naître quelque scrupule
au roi qui changeât la destination. Tellement que ce n'était jamais
qu'en rassemblant les remercîments qu'on voyait faire, ou quelquefois
rarement par les intéressés, à qui le P. Tellier l'avait dit, qu'on
ramassait la distribution, qui était annoncée verbalement ou par écrit
aux nommés, quand il plaisait au révérend père de le leur dire ou
écrire, qui gardait quelquefois telle nomination \emph{in petto} un mois
et six semaines, manége profond que l'impatience de la cour ne put
jamais goûter.

De cette nomination-ci, quelques-uns de ceux qui y eurent part méritent
d'être insérés ici, pour les choses qui s'y verront en leur temps. M. de
Mailly, mon ami, archevêque d'Arles, s'était brouillé, et aux couteaux
tirés, avec le cardinal de Noailles, à une assemblée du clergé. La
fortune des Noailles lui était entrée de travers dans la tête. Sa
belle-soeur n'était que nièce à la mode de Bretagne de
M\textsuperscript{me} de Maintenon, la véritable nièce avait épousé le
duc de Noailles. Les miches et la faveur qui en résultaient pénétraient
l'archevêque d'Arles de jalousie, qui, comme je l'ai dit ailleurs,
visait, quoique avec si peu de moyens et d'apparence, au cardinalat, et
qui était enragé que sa belle-soeur n'eût pas valu un duché et toutes
sortes de fortunes à sa maison. Il avait donc voulu parier\footnote{Aller
  de pair.} dans l'assemblée avec le cardinal de Noailles, et l'avait
picoté, fait contre, rassemblé et soulevé tant qu'il avait pu. Le succès
n'avait pas répondu à ses désirs\,: la faveur du cardinal était encore
entière\,; il était aimé et estimé dans le clergé\,; il y était
considéré et ménagé\,; on ne se le voulait point attirer pour des
bagatelles. Le cardinal, qui vit la mauvaise humeur de l'archevêque,
essaya de le ramener avec douceur, politesse et raison\,; l'archevêque
en fut encore plus piqué, et força le naturel bénin et pacifique du
cardinal de lui répondre avec une fermeté et une autorité qui lui
fermèrent la bouche, mais qui remplirent son coeur de haine à ne lui
pardonner jamais.

Dans ce dessein de vengeance, et dans celui de se faire un épaulement
contre le cardinal, il se jeta plus que jamais aux jésuites, à qui il
avait toute sa vie beaucoup fait sa cour. Il n'oublia pas de leur parler
du cardinal de Noailles, dont la haine commune le lia intimement avec le
P. Tellier. Celui-ci trouva dans l'archevêque d'Arles tout ce qu'il
pouvait désirer d'ailleurs pour en faire un grand usage contre le
cardinal de Noailles\,: un nom illustre, une alliance avec
M\textsuperscript{me} de Maintenon, une belle-soeur dame d'atours de
M\textsuperscript{me} la duchesse de Bourgogne, un archevêque déjà un
peu ancien. Il le fallait mettre en place de s'en pouvoir servir, et
pour cela le tirer de Provence\,; c'est ce qui le détermina à le faire
passer à Reims, dont je ne vis jamais homme si aise que le nouveau duc
et pair par toutes sortes de raisons.

Le cardinal de Janson vivait bien avec les jésuites sans penser en rien
comme eux\,; ils voulurent hasarder quelque chose dans son diocèse, et
mettre le roi de la partie, qui, ne voyant que par leurs yeux en ces
matières, s'y laissa aller\,; mais ils eurent affaire à un homme comblé
et au-dessus de tout par ses moeurs, par sa fortune et par sa conduite à
la cour et dans son diocèse. Il prit l'affaire avec la dernière hauteur,
et quand le roi lui voulut parler, duquel avec raison il avait depuis
longtemps la confiance, il lui répondit si ferme que le roi se tut tout
court, et que les jésuites demeurèrent depuis dans la crainte et le
respect avec lui.

Il avait un neveu à Saint-Sulpice, fort saint prêtre, mais d'une
parfaite bêtise, d'une ignorance crasse, et l'homme le plus incrusté de
toutes les misères de Saint-Sulpice qui y ait jamais été nourri. Un tel
sujet parut propre au P. Tellier pour en faire un archevêque d'Arles, et
pour se bien réconcilier le cardinal de Janson, au moins se faire un
mérite auprès du roi de lui proposer son neveu pour en faire tout d'un
coup un archevêque, et dans son propre pays.

Le roi, qui goûta fort ce choix, le voulut apprendre lui-même au
cardinal de Janson. Celui-ci, qui était droit et vrai, au lieu de
remercier, s'écria, dit au roi qu'il ne connaissoit point l'abbé de
Janson\,; qu'il n'était point fait pour être évêque\,; que ce serait
encore trop pour lui qu'être vicaire d'un curé de campagne, et supplia
le roi de l'en croire, et, s'il voulait lui marquer de la bonté, donner
à son neveu de quoi vivre par quelque abbaye de dix ou douze mille
livres de rente, qui serait un Pérou pour lui, et ne l'engagerait à
rien. Le cardinal eut beau dire et beau faire, même à plusieurs
reprises, le roi le loua fort, mais tint ferme, et l'abbé de Janson fut
archevêque d'Arles.

Nîmes fut donné à l'abbé de La Parisière, qui le paya bien à son
protecteur, et qui se rendit aussi célèbre en forfaits que Fléchier, son
prédécesseur, l'était devenu par son esprit, sa rare éloquence, sa vaste
érudition, et sa vie et ses vertus épiscopales.

Le Normand eut Évreux. C'était un homme fait exprès pour le P. Tellier,
un cuistre de la lie du peuple, qui, à force de répéter, puis régenter,
après professer, était devenu habile en cette science dure de l'école et
dans la chicane ecclésiastique, dont il entendait fort bien les
procédures. Je ne sais qui le produisit au cardinal de Noailles, qui le
fit son official, et qui dix ou douze ans après, le chassa honteusement,
pour des trahisons considérables qu'il découvrit, que les jésuites lui
avaient fait faire, et qui l'en récompensèrent par cet évêché.

L'abbé Turgot, aumônier du roi, eut Séez, et le maréchal de Boufflers
eut Autun pour son parent l'abbé de Dromesnil, qui passa depuis à
Verdun, et y a bâti de fond en comble le plus vaste et le plus superbe
palais épiscopal qu'il y ait en France.

Autun avait été donné à l'abbé de Maulevrier, il y avait plus d'un an,
qui le rendit sans avoir pris de bulles, et à qui on donna l'abbaye de
Moutiers-Saint-Jean, de quatorze mille livres de rente, dans son pays,
en Bourgogne, outre ce qu'il avait déjà. Cet abbé de Maulevrier était un
grand homme décharné, d'une pâleur de mort qu'on va porter en terre, qui
s'appelait Andrault, et qui était frère de M\textsuperscript{lle} de
Langeron, qui était à M\textsuperscript{me} la Princesse, et fort
comptée à l'hôtel de Condé. Il était oncle de Langeron, lieutenant
général des armées navales, et de l'abbé de Langeron, si attaché à M. de
Cambrai, qui fut chassé avec lui, passa le reste de sa vie chez M. de
Cambrai, dans sa plus intime confiance, et qui y mourut à la fin de
cette année. Ces Andrault sont si peu de chose que, encore que tout soit
comme anéanti en France par la plus que facilité partout où il faut des
preuves, je ne sais comment ils ont pu se faire admettre dans le
chapitre de Saint-Jean de Lyon\,; où l'abbé de Maulevrier a été
sacristain presque toute sa vie, qui en est une dignité.

Il était originairement aumônier de M\textsuperscript{me} la dauphine de
Bavière, et fort bien avec elle. À sa mort, il eut une place d'aumônier
du roi. Il n'avait jamais suivi sa profession, et il était tout à fait
ignorant, mais grand maître en manéges et en intrigues. Il fut ami
intime du P. de la Chaise, absolument livré aux jésuites, dans
l'intimité de M. de Cambrai, par conséquent jusqu'à un certain point des
ducs de Chevreuse et de Beauvilliers, mais qu'il ne voyait qu'avec
beaucoup de mesure.

Il était doux, poli, flatteur, respectueux, obséquieux, obligeant\,; il
voulait être bien avec tout le monde et il avait des amis considérables
des deux sexes. Très-bien avec Chamillart, aussi bien après avec Voysin,
il avait entièrement apprivoisé Desmarets\,; des amis de Pontchartrain,
et honnêtement seulement avec le chancelier, qui ne s'y fiait pas\,; à
merveille encore avec tous les Villeroy. Mais avec tout son miel, tout
son désir de s'insinuer, de se mêler, d'être instruit de tout, d'avoir
la confiance des grands et des petits, car il était sur tout cela à la
ville comme à la cour, et dans le clergé encore, c'était un homme à qui
il ne fallait pas marcher sur le pied, pétulant et dangereux, qui ne
pardonnait point, et capable de toute espèce de fougasse.

Ses liaisons intimes avec les jésuites et M. de Cambrai l'avaient
foncièrement éloigné du cardinal de Noailles, encore qu'il lui fît sa
cour, et à tous les Noailles avec de grands ménagements. Il avait eu
deux agences du clergé de suite, et par conséquent été promoteur après
de l'assemblée du clergé. Dans cet emploi, il eut des démêlés avec le
cardinal de Noailles, dont les ennemis, ses amis à lui, profitèrent pour
l'animer, en sorte que les choses allèrent jusqu'à l'audace de sa part,
qui, trop poussée en face, lui attira un traitement fâcheux et qui porta
sur l'honneur. Cette affaire lui fit un extrême tort dans le monde, où
il déchut beaucoup, nonobstant ses appuis. Le P. de La Chaise n'avait
jamais pu résoudre le roi à le faire évêque\,: ses intrigues, sa liaison
avec M. de Cambrai lui avaient déplu, et ce grand nombre d'amis.

Il avait été accusé, il y avait plus d'un an, d'une correspondance
étroite et cachée avec M. de Cambrai, le roi en avait parlé au P.
Tellier avec colère\,; cela fut approfondi. Le P. Tellier, qui le
portait doublement, à cause des jésuites et à cause de M. de Cambrai,
lui obtint une audience du roi où il se lava de tout, et le P. Tellier
tira sur le temps pour le faire évêque.

L'abbé de Maulevrier était vieux et gueux, il aimait la bonne chère et
le jeu\,; il sentait que son temps pour l'épiscopat était passé, qu'il
n'y pourrait rien faire, et qu'il n'aurait qu'à s'ennuyer dans son
diocèse. Il ne voulait plus être évêque que pour l'honneur, et comme,
avant Notre-Seigneur les Juives se mariaient pour ôter l'opprobre de
dessus elles. Il n'eut donc jamais envie que d'être nommé, bien résolu,
comme il fit, de rendre son évêché sans en payer de bulles.

Il demeura brouillé avec le cardinal de Noailles. Hors son affaire avec
lui, je ne l'ai jamais ouï taxer de fausseté ni d'aucun trait
malhonnête, et je ne l'ai vu brouillé ni baissé avec aucun de ses
amis\,; mais pour le gros du monde, il ne revint jamais bien de cette
affaire du cardinal de Noailles. Il fut toujours bien avec le cardinal
de Bouillon, et fort lié avec les cardinaux de Coislin et de Janson, et
avec la plupart des grands prélats.

Les deux grosses abbayes furent données\,: Saint-Remy de Reims au
cardinal Gualterio, qui arbora les armes de France snr la porte de son
palais à Rome\,; et celle de Saint-Étienne de Caen au cardinal de La
Trémoille.

J'ai oublié, sur le commencement de cette année, la mort de
M\textsuperscript{me} de Caderousse, sans enfants, la dernière de la
maison de Rambures. C'était une femme qui n'allait point à la cour, mais
qui, à Paris, était fort du monde et du jeu. Son mari, qui s'appelait
Cadart, et qui voulait se nommer Ancezune, était un gentilhomme du
comtat d'Avignon, qui portait le nom de duc de Caderousse, dont il
n'était pas plus avancé.

Il était duc d'Avignon, et ces ducs d'Avignon, que le pape fait, sont
inconnus partout, même à Rome, où ils n'ont, non plus qu'ailleurs, ni
rang, ni honneur, ni distinction quelconque. À Avignon, ils en ont chez
le vice-légat et dans toute cette légation. C'est chose dont les papes
ne sont pas avares, et qui se donne assez ordinairement pour de
l'argent.

Caderousse était un paresseux, grand, bien fait, de beaucoup d'esprit et
orné, qui n'avait guère servi que les dames, et qui n'avait été qu'un
moment fort de la cour. Une longue maladie de poitrine que les médecins
abandonnèrent par écrit, et dont Caretti, dont j'ai parlé ailleurs, le
guérit, et qui voulut cet écrit des plus fameux médecins de Paris avant
de l'entreprendre, commença à lui donner cette grande vogue qu'il eut
depuis, et que la guérison de M. de La Feuillade couronna.

Caderousse passa sa vie à Paris, assez dans le beau monde, intime de
M\textsuperscript{me} de Bouillon, et fort des amis de M. de La
Rochefoucauld, nonobstant la séparation de lieu. Il aimait à se mêler, à
savoir, surtout à régenter et à dogmatiser, et pour le moins autant à
emprunter de qui il pouvait, et à ne le guère rendre, et tout cela avec
les plus grandes manières du monde. Il se mit fort dans la dévotion, et
c'était merveilles de l'entendre moraliser. Il avait beaucoup perdu au
jeu. Avec tout cela, il était considéré et compté, et avait beaucoup
d'amis. Il a vécu fort vieux et toujours fort pauvre.

Le Camus, lieutenant civil, mourut en ce temps-ci. C'était la plus belle
représentation du monde de magistrat\,; il l'était bon aussi et honnête
homme, obligeant, et avait beaucoup d'amis\,; mais il était glorieux à
un point qu'on en riait et qu'on en avait pitié. Il était frère du
premier président de la cour des aides et du cardinal Le Camus\,; et
quand il disait «\,mon frère le cardinal,\,» il se rengorgeait que
c'était un plaisir. Pelletier, de sa retraite, demanda cette charge pour
d'Argouges qui n'avait que vingt-six ans, et qui était fils de sa fille
et de d'Argouges, conseiller d'État, mort longtemps depuis doyen du
conseil. Le roi, qui ne refusait rien à Pelletier, la lui donna.

Lavienne, premier valet de chambre du roi, mourut aussi à plus de
quatre-vingts ans. J'ai assez fait connaître ailleurs ce personnage de
l'intérieur pour n'en pas dire ici davantage. Chancenay, son fils, avait
sa survivance, et est encore premier valet de chambre.

La vieille marquise de Laval mourut à quatre-vingt-huit ans. Elle était
fille aînée du chancelier Séguier, soeur de la duchesse de Sully, puis
de Verneuil, mère, en premières noces, des duc, cardinal et chevalier de
Coislin, et en secondes de la maréchale de Rochefort. Elle avait
beaucoup d'esprit et méchante. Elle laissa un prodigieux bien à l'évêque
de Metz, son petit-fils. J'ai parlé d'elle et de ses mariages
suffisamment ailleurs.

Denonville mourut aussi, brave et vertueux gentilhomme, qui avait été
gouverneur général de Canada, où il avait très-bien servi, s'était fait
aimer, et avait acquis la confiance de tous les sauvages. Mais à la
cour, où M. de Beauvilliers le fit sous-gouverneur des enfants de
Monseigneur, rien de si plat. Il ne fut heureux en femme ni en enfants.

La duchesse de Luynes gagna un procès de quatorze ou quinze cent mille
livres contre Matignon, sur la succession de M\textsuperscript{me} de
Nemours. Le singulier est que Matignon l'avait gagné tout d'une voix aux
requêtes du palais, et qu'il le perdit tout d'une voix à la
grand'chambre. C'était à qui aurait ces terres. Ainsi Matignon manqua
seulement cette grande portion d'héritage outre ce qu'il en avait eu.

Le marquis de Bellefonds, petit-fils du maréchal, mourut tout jeune,
laissant un fils en maillot, et le gouvernement et capitainerie de
Vincennes vacant, qu'il avait eu de son père, gendre du duc de Mazarin,
qui le lui avait donné. Le roi ne voulut point voir la liste des
demandeurs, qui était illustre et nombreuse, et à la prière de
M\textsuperscript{me} la duchesse de Bourgogne, appuyée de
M\textsuperscript{me} de Maintenon, il le donna au marquis du Châtelet,
qu'il chargea de quelque chose pour l'enfant, et qu'il déchargea par
quelque retranchement du soin et de la nourriture des prisonniers du
donjon. Cela valut encore dix-huit mille livres de rente.

La marquise du Châtelet était fille du maréchal de Bellefonds, dame du
palais de M\textsuperscript{me} la duchesse de Bourgogne, et d'une vertu
de toute sa vie, douce, aimable et généralement reconnue, qui faisait
son service sans se mêler de rien. Elle et son mari qui était un
très-brave homme et très-galant homme, fort vertueux aussi, étaient
très-pauvres. On a remarqué que ce fut la seule des dames du palais, et
la plus retirée de toutes, qui eut une grâce de la cour. La maréchale de
Bellefonds, qui par pauvreté demeurait à Vincennes, eut un brevet qui
lui en assura le logement.

Je passerai légèrement ici sur une aventure qui, entée sur quelques
autres, fît du bruit, quelque soin qu'on prît à l'étouffer.
M\textsuperscript{me} la duchesse de Bourgogne fit un souper à
Saint-Cloud avec M\textsuperscript{me} la duchesse de Berry, dont
M\textsuperscript{me} de Saint-Simon se dispensa. M\textsuperscript{me}
la duchesse de Berry et M. le duc d'Orléans, mais elle bien plus que
lui, s'y enivrèrent au point que M\textsuperscript{me} la duchesse de
Bourgogne, M\textsuperscript{me} la duchesse d'Orléans, et tout ce qui
était là ne surent que devenir. M. le duc de Berry y était, à qui on dit
ce qu'on put, et à la nombreuse compagnie que la grande-duchesse amusa
ailleurs du mieux qu'elle put. L'effet du vin, haut et bas, fut tel
qu'on en fut en peine, et ne la désenivra point, tellement qu'il la
fallut ramener en cet état à Versailles. Tous les gens des équipages le
virent et ne s'en turent pas\,; toutefois on parvint à le cacher au roi,
à Monseigneur et à M\textsuperscript{me} de Maintenon.

La flotte ennemie, qui se promenait sur la fin de juillet sur la côte de
Languedoc, mit seize cents hommes à terre, qui prirent un petit
retranchement qu'on avait fait devant le port de Cette. Roquelaure
envoya un courrier à Perpignan demander secours au duc de Noailles, et
un au roi qui y fit marcher trois bataillons. Roquelaure, qui n'avait
pas voulu retirer les troupes qui contenaient le Vivarois et les
Cévennes, courut à Cette avec Bâville et trente hommes avec eux. Ils
trouvèrent qu'ils s'étaient aussi emparés d'Agde, dont les habitants
pouvaient les en empêcher, seulement en leur fermant leurs portes. Le
duc de Noailles accourut lui-même à temps avec des troupes, qui fort
aisément chassèrent les ennemis du port de Cette, l'épée à la main, en
tuèrent trois ou quatre cents, en prirent une centaine\,; et quantité se
noyèrent en se rembarquant à la hâte. Le duc de Noailles avait amené
mille hommes et huit cents chevaux. Ils avaient débarqué trois mille
hommes à Cette ou à Agde qu'ils abandonnèrent, et sans aucun dommage en
même temps. MM. de Noailles et de Roquelaure n'y perdirent que deux
grenadiers.

Il {[}est{]} temps de venir aux événements d'Espagne. Ils furent si
importants cette année qu'on a cru ne les devoir pas interrompre\,;
ainsi il faut remonter aux premiers mois, pour en voir toute la suite
jusqu'à la fin. Elle s'entendra mieux, si on a vu auparavant dans les
Pièces le triste succès du voyage de Torcy à la Haye, et les prétentions
démesurées et plus que barbares de gens résolus à rompre tout moyen de
paix, et qui se flattaient de tout envahir, sur quoi roula et se rompit
toute l'indigne négociation de Gertruydemberg. On y verra en quel danger
était l'Espagne, livrée à sa propre faiblesse, que celle où la France
était réduite à ne pouvoir secourir, bien en peine de se défendre
elle-même, et qui aimait mieux se laisser une espérance d'obtenir une
paix devenue si pressamment nécessaire, en abandonnant l'Espagne
d'effet, que de laisser subsister l'invincible obstacle que formaient
les alliés à prescrire cette dure condition d'une manière à ne pouvoir
être acceptée.

C'est ce qui engagea le roi, pour ôter jusqu'aux apparences, à montrer
qu'il en retirait jusqu'à M\textsuperscript{me} des Ursins\,; et
M\textsuperscript{me} des Ursins à faire toute la contenance d'une
personne qui va partir et qui ne prend plus qu'un mois ou six semaines
pour régler tout à fait son départ. Elle le manda de la sorte à notre
cour, qui prit soin de le répandre. Je doute toutefois que cette
résolution fût bien prise ici\,; et je pense qu'on peut assurer sans se
méprendre que M\textsuperscript{me} des Ursins n'y pensa jamais
sérieusement, ni Leurs Majestés Catholiques. Cette façon ne fut qu'une
complaisance susceptible d'être différée, puis rompue, comme en effet
après cette annonce il n'en fut plus parlé.

D'autre part on manquait tout à fait de généraux en Espagne. M. de
Vendôme en prit occasion d'en profiter. La situation où il se trouvait,
et qu'il voyait s'approfondir tous les jours, lui devenait de plus en
plus insupportable. Il espéra qu'en se faisant demander par le roi
d'Espagne, le roi se trouverait soulagé de l'y laisser aller pour s'en
défaire. Il le fit sentir à la princesse des Ursins, qui, de son côté,
espérait, en l'obtenant, montrer aux alliés que la France s'intéressait
toujours essentiellement aux événements de delà des Pyrénées. C'est en
effet ce soulagement du roi qui fit l'affaire de M. de Vendôme\,; mais
cette montre aux ennemis qui en résultait, fut ce qui retarda son envoi
jusqu'à ce qu'on eût vu à Gertruydemberg qu'il n'y avait point de paix à
espérer. J'ai déjà parlé de cette demande faite de M. de Vendôme par
l'Espagne. Elle fut renouvelée au mois de mars de cette année\,; et, à
la fin de ce même mois, le roi d'Espagne partit de Madrid pour s'aller
mettre à la tête de son armée en Aragon.

Villadarias fut choisi pour la commander sous lui. C'était un de leurs
meilleurs et plus anciens officiers généraux, qui avait servi longtemps
en Flandre sous le règne précédent, qui défendit fort bien Charleroy,
lorsqu'en 1693 les maréchaux de Luxembourg et de Villeroy le prirent. Il
portait alors le nom de Castille. Il eut depuis le titre de marquis de
Villadarias et le dernier grade militaire de capitaine général. Il avait
été employé au siège de Gibraltar, que le maréchal de Tessé ne put
prendre, et il s'était retiré depuis chez lui en Andalousie. Il était
vieux et fort galant homme.

Fort peu de jours auparavant, le duc de Medina-Celi fut arrêté et
conduit au château de Ségovie. M\textsuperscript{me} des Ursins l'avait
mis dans les affaires après qu'elle en eut chassé tous ceux qui avaient
eu part au testament de Charles II, et d'autres encore avec qui elle
s'était brouillée, pour qu'il ne fut pas dit qu'aucun Espagnol n'y avait
de part, et se couvrir elle-même du bouclier d'un nom révéré en Espagne.
Elle l'avait mis dans plusieurs confidences, et, pour s'ancrer, il
s'était rendu souple à ses volontés. À la fin, il s'en lassa et voulut
pointer de son chef. Je ne sais s'il y eut d'autre crime. Quoi qu'il en
soit, il fut mis dans le château destiné aux criminels d'État, où était
aussi Flotte, avec lequel il fut transféré quelque temps après au
château de Bayonne par trente gardes du corps, lorsque l'archiduc fit
les progrès dont il va être parlé. Dès qu'il fut arrêté, quatre
commissaires, gens de robe, furent chargés d'instruire son procès.

Le roi d'Espagne alla de Saragosse à Lerida, où il fut reçu avec de
grandes acclamations des peuples et de son armée, avec laquelle il passa
la Sègre le 14 mai, et s'avança dans le dessein de faire le siége de
Balaguier. Les grandes pluies, qui emportèrent les ponts et firent
déborder cette rivière, rompirent le projet, et firent retourner l'armée
sous Lerida. Jointe un mois après par les troupes arrivées de Flandre,
elle alla chercher celle des ennemis qu'elle ne put attaquer dans le
poste d'Agramont. On se contenta d'envoyer Mahoni avec un gros
détachement nettoyer le pays de quelques petites villes où l'archiduc
avait établi de grands magasins, qui furent enlevés avec cinq mille
habits qui attendaient leurs troupes d'Italie\,; et Mahoni, après cette
petite expédition, revint joindre le roi d'Espagne à Belpuch. Le marquis
de Bay commandait la petite armée d'Estrémadure. Il fit escalader
Miranda de Duero par Montenegro, qui prit la place, le gouverneur, la
garnison et trois cents prisonniers de guerre qu'ils y gardaient. C'est
une place assez considérable de Portugal, qui ouvrit les provinces de
Tras-os-Montes et Entre-Duero-et-Minho pour la contribution.

Cependant le comte de Staremberg, qui avait eu une maladie dont on avait
profité dans ces commencements, se rétablit plus tôt qu'on ne le
pensait, rassembla promptement ses quartiers, marcha au milieu de ceux
de l'armée du roi d'Espagne, en enleva et en battit, et obligea cette
armée étonnée de se retirer sous Saragosse. Le roi d'Espagne entra dans
la ville, où il demeura indisposé, et dépêcha un courrier pour redoubler
ses instances pour obtenir du roi M. de Vendôme. Ce mauvais succès tomba
tout entier sur Villadarias. Il fut accusé d'imprudence et de
négligence. Il fut renvoyé chez lui, et le marquis de Bay mandé de la
frontière de Portugal pour le remplacer en Aragon.

Le roi apprit par le duc d'Albe, dans les premiers jours d'août, cette
mauvaise nouvelle et la recharge sur le duc de Vendôme. Tout était rompu
à Gertuydemberg. Ainsi il fut accordé sur-le-champ et mandé. De cette
affaire de Catalogne, il n'en avait coûté qu'environ mille hommes tués
ou pris avec quelque bagage. Les ennemis aussi y perdirent quelque
monde, et entre autres un prince de Nassau et le lord Carpentier,
lieutenant général\,: ainsi l'effroi et le désordre firent le plus grand
mal.

Le duc de Vendôme, qui, par la princesse des Ursins en Espagne, et par
M. du Maine ici, ne cessait depuis plusieurs mois ses efforts pour aller
en Espagne, s'y était préparé d'avance sourdement, et se trouva prêt à
partir dès qu'il en eut obtenu la permission. Il fut donc mandé pour ce
voyage. Un peu de goutte et un dernier arrangement domestique l'y retint
quelques jours. Il arriva à Versailles le mardi matin 19 août. M. du
Maine avait négocié avec M\textsuperscript{me} de Maintenon de mener
Vendôme chez M\textsuperscript{me} la duchesse de Bourgogne. La
conjoncture leur en parut favorable. Allant en Espagne, demandé par le
roi et la reine sa soeur, et y aller sans voir M\textsuperscript{me} la
duchesse de Bourgogne était une chose fort désagréable. Le duc du Maine,
suivi de Vendôme, arriva donc ce même jour à la toilette de
M\textsuperscript{me} la duchesse de Bourgogne. La rencontre du mardi,
jour des ministres étrangers, et de la veille qu'on allait à Marly,
rendit la toilette fort nombreuse en hommes et en dames.
M\textsuperscript{me} la duchesse de Bourgogne se leva pour eux, comme
elle faisait toujours pour tous les princes du sang et autres, et pour
tous les ducs et duchesses, se rassit aussitôt comme à l'ordinaire\,;
et, après cette première oeillade qui ne se put refuser, elle, qui était
à sa toilette, comme partout ailleurs, regardante et parlante, et fort
peu occupée de son ajustement et, de son miroir, fixa les yeux dessus,
et ne dit pas un seul mot à personne. M. du Maine et M. de Vendôme,
collé à son côté, demeurèrent très-déconcertés sans que M. du Maine, si
libre et si leste, osât proférer un seul mot. Personne ne les approcha
et ne leur parla. Ils demeurèrent ainsi un bon demi-quart d'heure dans
un silence universel de toute la chambre, qui avait les yeux sur eux.
Ils ne les purent soutenir davantage et se retirèrent à la sourdine.

Cet accueil ne leur fut pas assez agréable pour persuader à Vendôme de
s'exposer à une récidive pour prendre congé, et plus embarrassante,
parce qu'il aurait baisé M\textsuperscript{me} la duchesse de Bourgogne,
comme tous les princes du sang et autres, les ducs et les maréchaux de
France, qui prennent congé ou qui arrivent d'une campagne ou d'un long
voyage. Je ne sais s'il ne craignit point l'affront inouï du refus\,;
quoi qu'il en soit, il s'en tint à l'essai qu'il venait de faire, et
partit sans prendre congé d'elle.

Mgr le duc de Bourgogne le traita assez honnêtement, c'est-à-dire
beaucoup trop bien. Le duc d'Albe, Torcy et Voysin furent chez lui. Il
fit sa cour au roi ce jour-là comme à l'ordinaire, et le lendemain
mercredi, il eut une assez longue audience du roi, dans son cabinet,
après son dîner, y prit congé de lui, et s'en vint à Paris. Depuis son
mariage il n'y avait été que vingt-quatre heures pour voir
M\textsuperscript{me} la Princesse. M\textsuperscript{me} de Vendôme
n'avait point été à Anet, où il s'était toujours tenu, de sorte qu'ils
n'avaient pas eu loisir de faire grande connaissance ensemble.

\hypertarget{chapitre-xx.}{%
\chapter{CHAPITRE XX.}\label{chapitre-xx.}}

1710

~

{\textsc{Bataille de Saragosse, où l'armée d'Espagne est défaite.}}
{\textsc{- Ducs de Vendôme et de Noailles à Bayonne\,; Monteil à
Versailles.}} {\textsc{- Duc de Noailles va avec le duc de Vendôme
trouver le roi d'Espagne à Valladolid.}} {\textsc{- Stanhope emporte
contre Staremberg de marcher à Madrid.}} {\textsc{- La cour fort suivie
se retire de Madrid à Valladolid.}} {\textsc{- Merveilles de la reine et
du peuple.}} {\textsc{- Magnanimité du vieux marquis de Mancera.}}
{\textsc{- Courage de la cour.}} {\textsc{- Prodige des Espagnols.}}
{\textsc{- L'archiduc à Madrid tristement proclamé et reçu.}} {\textsc{-
Mancera refuse de prêter serment et de reconnaître l'archiduc, et de le
voir.}} {\textsc{- Éloge des Espagnols, qui dressent une nouvelle
armée.}} {\textsc{- Insolence de Stanhope à l'égard de Staremberg, qui
se retire à Tolède.}} {\textsc{- Ducs de Vendôme et de Noailles à
Valiadolid en même temps que la cour.}} {\textsc{- Le roi va à la tête
de son armée avec Vendôme\,; la reine à Vittoria\,; le duc de Noailles à
Versailles, et de là en Roussillon.}} {\textsc{- Son armée.}} {\textsc{-
Six nouveaux capitaines généraux d'armée.}} {\textsc{- Paredès et Palma,
grands, passent à l'archiduc, qui de sa personne se retire à
Barcelone.}} {\textsc{- D'autres seigneurs arrêtés.}} {\textsc{-
Staremberg, en quittant Tolède, en brûle le beau palais.}} {\textsc{- Le
roi d'Espagne, pour trois jours à Madrid, y visite le marquis de
Mancera.}} {\textsc{- Piège tendu par Staremberg.}} {\textsc{- Stanhope,
etc., emportés et pris dans Brihuega.}} {\textsc{- Bataille de
Villaviciosa perdue par Staremberg, qui se retire en Catalogne.}}
{\textsc{- Belle action du comte de San-Estevan de Gormaz.}} {\textsc{-
Réflexions sur ces deux actions et sur l'étrange conduite du duc de
Vendôme.}} {\textsc{- Zuniga dépêché au roi.}} {\textsc{- Vains efforts
de la cabale de Vendôme.}} {\textsc{- La cour d'Espagne presque tout
l'hiver à Saragosse.}} {\textsc{- Stanhope perdu et dépouillé de ses
emplois.}} {\textsc{- Duc de Noailles investit Girone.}} {\textsc{-
Misérable flatterie de l'abbé de Polignac sur Marly.}} {\textsc{- Amelot
inutilement redemandé en Espagne, qui ne veut point de l'abbé de
Polignac.}}

~

Stareraberg cependant profita de ses avantages\,: il attaqua l'armée
d'Espagne presque sous Saragosse et la défît totalement. Bay la trouva
dans un tel effroi, lorsqu'il y arriva pour en prendre le commandement,
qu'il en espéra peu de chose\,; aussi toute l'infanterie, qui n'était
presque {[}que{]} milices, jeta les armes dès qu'elle fut attaquée. Les
gardes wallones et le peu d'autres corps de troupes ne purent soutenir
seuls, ils furent défaits, la cavalerie fut enfoncée\,; ce fut elle qui
fit le moins mal. En un mot, artillerie, bagages, tout fut perdu, et la
déroute fut entière. Le duc d'Havrec, colonel des gardes wallones, y fut
tué. Ce malheur arriva le 20 août. Le roi d'Espagne était demeuré
incommodé dans Saragosse, d'où il en fut témoin, qui aussitôt prit
diligemment le chemin de Madrid. Bay rassembla dix-huit mille hommes,
avec lesquels il se retira à Tudela sans inquiétude de la part des
ennemis depuis la bataille.

M. de Vendôme en apprit la nouvelle en chemin, qui prudemment, à son
ordinaire, pour soi, se soucia moins de tâcher à rétablir les affaires
que de se donner le temps de les voir s'éclaircir, avant que d'y prendre
une part personnelle. Il poussa donc à Bayonne le temps avec l'épaule.
Le duc de Noailles avait eu ordre de l'y aller trouver pour prendre des
mesures avec lui pour agir du côté de la Catalogne. Ils envoyèrent de là
Monteil au roi pour recevoir ses ordres sur leur conférence, et gagner
temps en l'attendant. C'était un mestre de camp qui servait de maréchal
des logis de la petite armée du duc de Noailles. Il arriva le 7
septembre à Marly\,; il y fut le même jour assez longtemps dans le
cabinet, conduit par Voysin, où Torcy fut mandé. Monteil repartit le 9,
et trouva MM. de Vendôme et de Noailles encore à Bayonne. À son arrivée,
le duc de Noailles publia qu'il allait trouver le roi d'Espagne avec M.
de Vendôme, et fît en effet le voyage avec lui jusqu'à Valladolid, où
ils le rencontrèrent.

L'archiduc joignit le comte de Staremberg après la bataille, en présence
duquel le parti à prendre fut agité avec beaucoup de chaleur. Staremberg
opina de marcher droit à la petite armée que Bay avait laissée sur la
frontière du Portugal, sous le marquis de Richebourg, de la défaire, ce
qui n'aurait coûté que le chemin, de s'établir pied à pied dans le
centre de l'Espagne, pour avoir le Portugal au derrière et les ports de
mer à côté et à portée, laisser en Aragon un petit corps suffisant à
contenir les pays soumis, et faire tête à l'armée battue, lequel petit
corps aurait derrière soi Barcelone et la Catalogne, si fort à eux\,:
parti solide qui eût en peu achevé de ruiner les affaires du roi
d'Espagne, ne lui eût laissé de libre que le côté de Bayonne, coupait
toute autre communication, et on se saisissait pied à pied de toute
l'Espagne, avec des points d'appui qui n'eussent pu être ébranlés, et
qui n'eussent laissé nulle ressource, et aucun moyen dans l'intérieur du
pays de se mouvoir en faveur du roi d'Espagne.

Stanhope, au contraire, fut d'avis d'aller tout droit à Madrid, d'y
mener l'archiduc, l'y faire proclamer roi d'Espagne, d'épouvanter toute
l'Espagne par en saisir la capitale, et de là comme du centre s'étendre
suivant le besoin et l'occasion.

Staremberg avoua l'éclat de ce parti, mais il le maintint peu utile, et
de plus dangereux. Il allégua le grand éloignement de Madrid des
frontières de Portugal, de Catalogne, de la mer et de leurs magasins\,;
que cette ville ni aucune voisine n'a de fortifications, ni toutes ces
campagnes de la Nouvelle-Castille aucun château fort\,; la stérilité du
pays, où on ne rencontrerait nulle subsistance, qu'ils trouveraient
soustraite ou brûlée\,; l'affection de ces peuples pour Philippe V\,;
enfin l'impossibilité de conserver Madrid et de se maintenir dans ce
centre, et la perte d'un temps si précieux à bien employer.

Ces raisons étaient sans doute décisives, mais Stanhope, qui commandait
en chef les troupes anglaises et hollandaises, sans lesquelles cette
armée n'était rien, déclara que les ordres de sa reine étaient de
marcher à Madrid de préférence à tout, si les événements le rendaient
possible\,; qu'il ne souffrirait pas qu'on prît un autre parti, ou qu'il
se retirerait avec ses auxiliaires. Staremberg, qui ne pouvait s'en
passer, n'ayant pu vaincre l'inflexibilité de Stanhope, protesta contre
un parti si peu sensé, et céda comme plus faible.

Ce fut l'attente de l'archiduc, et cette dispute qui suivit son arrivée,
qui les arrêta sans faire un mouvement depuis la bataille, faute
capitale, et salut du débris de l'armée qu'ils venaient de défaire.

Dès que Staremberg forcé eut consenti, ils firent toutes leurs
dispositions pour l'exécution d'un projet, qui fit grand'-peur, mais qui
sauva le roi d'Espagne.

La consternation déjà grande dans Madrid y devint extrême, dès que l'on
ne put plus douter que l'armée de l'archiduc allait y arriver. Le roi
résolut de se retirer d'un lieu qui ne se pouvait défendre, et d'emmener
la reine, le prince et les conseils. Cette résolution acheva de porter
la désolation au comble. Les grands déclarèrent qu'ils suivraient le roi
et sa fortune partout, et très-peu y manquèrent\,; le départ suivit la
déclaration de vingt-quatre heures. La reine, tenant le prince entre ses
bras, se montra sur un balcon du palais, y parla au peuple accouru de
toutes parts, avec tant de grâce, de force et de courage, qu'il est
incroyable avec quel succès. L'impression que ce peuple reçut se
communiqua partout et gagna incontinent toutes les provinces.

La cour sortit donc pour la seconde fois de Madrid, au milieu des cris
les plus lamentables, poussés du fond du coeur, d'un peuple infini qui
voulait suivre le roi et la reine, et qui accourait de la ville et de
toutes les campagnes\,; et ce ne fut qu'avec toute l'autorité et toute
la douceur qui s'y purent employer qu'il se laissa vaincre, et persuader
par son dévouement même de retourner chacun chez soi.

Le marquis de Mancera, dont j'ai parlé plus d'une fois, qui était le
seigneur le plus respecté d'Espagne par sa vertu et par les grands
emplois qu'il avait remplis, voulut suivre, quoiqu'il eût plus de cent
ans accomplis. Le roi et la reine, qui le surent, le lui envoyèrent
défendre avec force amitiés\,; il paya de respect et de compliments, et
partit en chaise à porteurs ne pouvant soutenir d'autre voiture, au
hasard de la lenteur, des partis, des périls et même de l'abandon. Il
fit ainsi quelques lieues\,; mais le roi et la reine, qui en furent
avertis, envoyèrent lui témoigner combien ils étaient touchés de son
zèle et d'une si rare affection, mais avec des ordres si précis de le
faire retourner qu'il ne put désobéir. Ce fut en protestant de ses
regrets de ce que l'obéissance lui arrachait l'honneur de mourir pour
son roi, qui était le meilleur usage qu'il pût faire de ce reste de vie
pour couronner tant d'années qu'il avait passées au service de ses rois,
et qui maintenant le trahissaient par leur excès et leur durée,
puisqu'elles le rendaient témoin de ce qu'il eût voulu racheter de tout
son sang.

Valladolid fut la retraite de cette triste cour, qui, dans ce trouble,
le plus terrible qu'elle eût encore éprouvé, ne perdit ni le jugement ni
le courage. Elle se banda contre la fortune, et n'oublia rien pour se
procurer tous les secours dont une pareille extrémité se trouva
susceptible. Trente-trois grands signèrent une lettre au roi, qu'ils lui
firent présenter par le duc d'Albe, pour l'assurer de leur fidélité pour
Philippe V, et lui demander un secours de troupes.

En attendant on vit en Espagne le plus rare et le plus grand exemple de
fidélité, d'attachement et de courage, en même temps le plus universel
qui se soit jamais vu ni lu. Prélats et le plus bas clergé, seigneurs et
le plus bas peuple, bénéficiers, bourgeois, communautés ensemble, et
particuliers à part, noblesse, gens de robe et de trafic, artisans, tout
se saigna de soi-même jusqu'à la dernière goutte de sa substance pour
former en diligence de nouvelles troupes, former des magasins, porter
avec abondance toutes sortes de provisions à la cour et à tout ce qui
l'avait suivi. Chacun selon ce qu'il put donna peu ou beaucoup, mais ne
se réserva rien\,; en un mot, jamais corps entier de nation ne fit des
efforts si surprenants, sans taxe et sans demande, avec une unanimité et
un concert qui agit et qui effectua de toutes parts tout à la fois.

La reine vendit tout ce qu'elle put, recevait elle-même quelquefois
jusqu'à dix pistoles pour contenter le zèle, et en remerciait avec la
même affection que ces sommes lui étaient offertes, grandes pour ceux
qui les donnaient parce qu'ils ne se réservaient rien. Elle disait à
tous moments qu'elle voulait monter à cheval, se mettre à la tête des
troupes avec son fils entre ses bras. Avec ces langages et sa conduite
elle se dévoua tous les coeurs, et fut très-utile dans une si étrange
extrémité.

L'archiduc était cependant arrivé à Madrid avec son armée. Il y était
entré en triomphe, il y fut proclamé roi d'Espagne par la violence de
ses troupes qui traînèrent le corrégidor tremblant par les rues, qui se
trouvèrent toutes désertes, la plupart des maisons vides d'habitants, et
le peu qu'il en était demeuré dans la ville avait barricadé les portes
et les fenêtres des maisons, et s'était enfermé sur le derrière au plus
loin des rues, sans que les troupes osassent les enfoncer, de peur de
combler le désespoir visible et général, et dans l'espérance d'attirer
et de gagner par douceur. L'entrée de l'archiduc ne fut pas moins triste
que sa proclamation. À peine y put-on entendre quelques acclamations
faibles, et si forcées que l'archiduc, dans un étonnement sensible, les
fit cesser lui-même. Il n'osa loger dans les palais ni dans le centre de
la ville, mais dans l'extrémité, où il ne coucha même que deux ou trois
nuits.

Il envoya Stanhope inviter le vieux marquis de Mancera de le venir voir,
qui s'en excusa sur son âge plus que centenaire, sur quoi il lui renvoya
le même général avec le serment, et ordre de lui faire prêter. Mais
Mancera répondit avec la plus grande fermeté qu'il savait le respect
qu'il devait à la naissance de l'archiduc, et la fidélité qu'il devait
au roi son maître, à qui rien ne l'en ferait manquer ni en reconnaître
un autre\,; et tout de suite pria civilement Stanhope de se retirer,
parce qu'il avait besoin de repos et de se mettre au lit. Il ne lui en
fut pas parlé davantage, et il ne lui fut fait aucun déplaisir, ni aux
siens. La ville aussi ne souffrit presque aucun dommage. Staremberg fut
soigneux d'une discipline exacte qui sentît la clémence, même l'estime
et l'affection, pour tâcher de s'en concilier. Cependant leur armée
périssait de toutes sortes de misères. Rien du pays n'y était apporté,
aucune subsistance pour hommes ni pour chevaux, et même pour de l'argent
il ne leur était rien fourni. Prières, menaces, exécutions, tout fut
parfaitement inutile\,; pas un Castillan qui ne se crût déshonoré de
leur vendre la moindre chose, ni d'en laisser en état d'être pris.

C'est ainsi que ces peuples magnanimes, sans aucun autre secours
possible que celui de leur courage et de leur fidélité, se soutinrent au
milieu de leurs ennemis, dont ils firent périr l'armée, et par des
prodiges inconcevables en reformèrent en même temps une nouvelle et
parfaitement équipée et fournie, et remirent ainsi, eux seuls, et pour
la seconde fois, la couronne sur la tête de leur roi, avec une gloire à
jamais en exemple à tous les peuples de l'Europe, tant il est vrai que
rien n'approche de la force qui se trouve dans le coeur d'une nation
pour le secours et le rétablissement des rois.

Stanhope, qui n'avait pu méconnaître la solidité de l'avis de Staremberg
dès le premier moment de leur dispute, ne fut pas le moins du monde
embarrassé du succès. Il lui échappa insolemment, au milieu de l'entrée
de l'archiduc à Madrid, que maintenant qu'il se voyait avec lui dans
cette ville il avait fait son affaire, puisqu'il avait exécuté les
ordres de sa reine\,; que c'était maintenant celle de Staremberg et à
son habileté à les tirer d'embarras\,; qu'on verrait comment il s'y
prendrait, dont peu à lui importait. Ce pas leur parut en effet si
glissant qu'au bout de dix ou douze jours ils résolurent de s'éloigner
de Madrid vers Tolède, dont rien ne fut emporté que quelques tapisseries
du roi, que Stanhope n'eut pas honte d'emporter, et qu'il eut celle
encore de ne garder pas longtemps. Ce trait de vilenie fut même blâmé
des siens.

Vendôme et Noailles arrivèrent à Valladolid le 20 septembre, presque en
même temps que la cour. Vendôme s'était amusé à Bayonne, et depuis en
chemin, sous divers prétextes de santé, pour se faire désirer davantage
et voir cependant plus clair au cours que prenaient les affaires. Il fut
étonné de les trouver telles qu'il les vit après un si grand désastre.
La reine, peu de jours après, sachant l'archiduc dans Madrid, se retira
avec le prince et les conseils à Vittoria, pour être à portée de France,
et sûre d'y pouvoir passer quand elle le voudrait. En même temps, elle
envoya toutes ses pierreries à Paris au duc d'Albe, pour lui envoyer
tout ce qu'il pourrait trouver d'argent dessus. Le duc de Noailles,
après deux ou trois jours de séjour et de conférence, reprit le chemin
de Catalogne, et trouva un courrier à Toulouse, qui le fit venir à la
cour rendre compte au roi de l'état des affaires en Espagne, et des
partis pris à Valladolid. Il arriva à Marly le 14 octobre, eut force
longues audiences du roi, et repartit le 28 pour aller attendre à
Perpignan le détachement que le duc de Berwick eut ordre de lui envoyer
de Dauphiné, où les neiges avaient terminé la campagne. L'armée du duc
de Noailles fut en tout de cinquante escadrons et de quarante
bataillons, les places fournies, et cinq lieutenants généraux sous lui.

Le roi d'Espagne fit à Valladolid six capitaines généraux, qui en
Espagne est le dernier grade militaire\,; le marquis d'Ayetone, grand
d'Espagne\,; le duc de Popoli, Italien, grand d'Espagne\,; le comte de
Las Torres et le marquis de Valdecanas, Espagnols\,; le comte d'Aguilar,
grand d'Espagne, de qui j'ai souvent parlé et dont j'aurai lieu de
parler encore, et M. de Thouy, lieutenant général français. Il partit
incontinent après la reine et le duc de Noailles, et marcha à Salamanque
avec le duc de Vendôme et douze mille hommes bien complets, bien armés
et bien payés, tandis que le comte de Staremberg faisait relever de la
terre autour de Tolède, où l'archiduc, en partant de Madrid, ordonna à
toutes les dames qui étaient demeurées, et dont les maris avaient suivi
le roi et la reine d'Espagne, de s'y retirer sous peine de confiscation
de biens et de meubles.

Le marquis de Paredès et le comte de Palma, neveu du feu cardinal
Portocarrero et si continuellement maltraité par M\textsuperscript{me}
des Ursins, tous deux grands d'Espagne, passèrent à l'archiduc. Le fils
aîné du duc de Saint-Pierre fut arrêté\,; et le marquis de Torrecusa,
grand d'Espagne napolitain, le fut aussi, accusé d'avoir voulu livrer
Tortose à l'archiduc. Il partit le 11 novembre d'autour de Madrid, prit
une légère escorte de cavalerie pour aller en Aragon, où il ne fit que
passer, et de là à Barcelone.

Staremberg ne fit pas grand séjour à Tolède, mais en quittant la ville
il brûla le superbe palais que Charles-Quint y avait bâti à la moresque,
qu'on appelait l'Alcazar, qui fut un dommage irréparable. Il prétendit
que cet incendie était arrivé par malheur, et tourna vers l'Aragon.

Rien n'empêchant plus le roi d'Espagne d'aller voir ses fidèles sujets à
Madrid, il quitta l'armée pour quelques jours, et entra dans Madrid le 2
décembre, au milieu d'un peuple infini et d'acclamations incroyables. Il
fut descendre à Notre-Dame d'Atocha, dont je parlerai ailleurs, et qui
est la grande dévotion de la ville, d'où il fut trois heures à arriver
au palais, tant la foule était prodigieuse. La ville lui fit présent de
vingt mille pistoles. Dans les trois jours qu'il y demeura, il fit une
chose presque inouïe en Espagne, et qui y reçut la plus sensible et la
plus générale approbation\,: ce fut d'aller voir le marquis de Mancera
chez lui, qui en pensa mourir de joie. Cette visite fut accompagnée de
toutes les marques d'estime, de reconnaissance et d'amitié si justement
dues à la vertu, au courage et à la fidélité de ce vieillard si
vénérable, et de toutes les distinctions possibles. Le roi l'entretint
seul de sa situation présente, de ses projets et de tout ce qui lui
pouvait marquer toute sa confiance, puis fît entrer les gens distingués,
sans permettre au marquis de se lever de sa chaise. En le quittant il
l'embrassa, et ne voulut jamais qu'il mît le pied hors de sa chambre
pour le conduire. Je ne sais si aucun roi d'Espagne a jamais visité
personne depuis Philippe II, qui alla chez le fameux duc d'Albe qui se
mourait, et qui, le voyant entrer dans sa chambre, lui dit qu'il était
trop tard, et se tourna de l'autre côté sans lui avoir voulu parler
davantage. Le quatrième jour après son arrivée à Madrid, le roi en
repartit et alla rejoindre M. de Vendôme et son armée.

Ce monarque presque radicalement détruit, errant, fugitif, sans argent,
sans troupes, sans subsistance, se voyait presque tout à coup à la tête
de douze ou quinze mille hommes bien armés, bien habillés, bien payés,
avec des vivres et des munitions en abondance, et de l'argent, par la
subite conspiration universelle de l'inébranlable fidélité, et de
l'attachement sans exemple de tous les ordres de ses sujets, par leur
industrie et leurs efforts aussi prodigieux l'une que les autres. Ses
ennemis, au contraire, qui, après avoir triomphé dans Madrid de sa
défaite, qui pour tout autre était sans ressource, périssaient dans la
disette de toutes choses, se retiraient parmi des pays soulevés contre
eux, qui se voyaient brûler plutôt que de leur fournir la moindre chose,
et qui ne donnaient quartier à pas un de leurs traîneurs jusqu'à cinq
cents pas de leurs troupes.

Vendôme, dans la dernière surprise d'un changement si peu espérable,
voulut en profiter, et fit le projet de joindre l'armée d'Estrémadure
que Bay tenait ensemble, trop faible pour se présenter devant celle de
Staremberg, mais en état pourtant de la fatiguer et de percer jusqu'au
roi à la faveur de ses mouvements. Il s'en fit donc quantité de prompts
et de hardis pour exécuter cette fonction, que Staremberg, débarrassé de
la personne de l'archiduc, ne songeait qu'à empêcher.

Il connaissoit bien le duc de Vendôme, pour, à son retour du Tyrol, lui
avoir gagné force marches, passé cinq rivières devant lui, et malgré lui
joint le duc de Savoie, comme je l'ai raconté en son lieu. Tout occupé à
lui tendre des piéges avec adresse et vigilance, il chercha à l'attirer
au milieu de son armée, et de l'y mettre en telle posture qu'il lui pût
subitement rompre le cou sans qu'il pût échapper. Dans cette vue il mit
son armée en des quartiers dont tous les accès étaient faciles, qui
étaient proches les uns des autres, et qui se pouvaient mutuellement
secourir avec promptitude et facilité, donna bien ses ordres partout, et
mit dans Brihuega Stanhope avec tous ses Anglais et Hollandais. Brihuega
est une petite ville fortifiée, dont le château de plus était bon, et où
l'art avait ajouté tout ce que le temps avait pu permettre. Elle était à
la tête de tous les quartiers de son armée, et à l'entrée d'un pays
plain, et nécessaire à traverser pour la jonction du roi avec Bay. En
même temps Staremberg était à portée d'être joint d'un moment à l'autre
par son armée d'Estrémadure, qui s'était ébranlée en même temps que Bay
avait fait marcher la sienne, et qui n'avait ni la distance ni pas une
des difficultés que celle de Bay rencontrait pour sa jonction avec celle
du roi d'Espagne.

Vendôme, cependant, avec une armée bien fournie, qui croissait tous les
jours par les renforts que chaque seigneur, chaque prélat, chaque ville
envoyait à mesure qu'ils étaient prêts, marchait toujours sur
Staremberg, n'ayant que sa jonction pour objet, et malgré la rigueur de
la saison trouvant partout ses logements bien fournis comme dans les
meilleurs temps, par les prodiges de soins et de zèle de ces
incomparables Espagnols. Il fut informé de la situation où était
Staremberg, mais en la manière que Staremberg désirait qu'il le fût,
c'est-à-dire qu'il crût Stanhope aventuré mal à propos, en état d'être
enlevé et trop éloigné de l'armée de Staremberg pour en être secouru à
temps, par conséquent tenté de se commettre à un exploit facile qui lui
ouvrirait le passage pour sa jonction avec Bay. En effet les choses
parurent ainsi à Vendôme. Il pressa sa marche, fit ses dispositions, et,
le 8 décembre après midi, il s'approcha de Brihuega, la fit sommer, et,
sur le refus de se rendre, se mit en l'état de l'attaquer.

Incontinent après, sa surprise fut grande lorsqu'il découvrit qu'il y
avait tant de troupes, et qu'en croyant n'avoir affaire qu'à un poste
peu accommodé il se trouvait engagé devant une place. Il ne voulut pas
reculer, et ne l'eût peut-être pas fait bien impunément. Il se mit donc
à tempêter avec ses expressions accoutumées, aussi peu honnêtes
qu'injurieuses, à payer d'audace, et à faire tout ce qui était en lui
pour exciter ses troupes à diligenter une conquête si différente de ce
qu'il se l'était figurée, et avec cela si dangereuse à laisser languir.

Cependant le poids de la bévue, s'appesantissant à mesure que les heures
s'écoulaient et qu'il venait des nouvelles des ennemis, Vendôme, à qui
deux assauts avaient déjà mal réussi, joua à quitte ou à double, et
ordonna un troisième assaut. Comme la disposition s'en faisait, le 9
décembre, on apprit que Staremberg marchait au roi d'Espagne avec quatre
ou cinq mille hommes, c'est-à-dire avec la franche moitié moins qu'il
n'en amenait en effet. Dans cette angoisse Vendôme ne balança pas à
jouer la couronne d'Espagne à trois dés\,: il hâta tout pour l'assaut,
et lui cependant avec le roi d'Espagne prit toute sa cavalerie, marcha
sur des hauteurs par où venait l'armée ennemie.

Durant cette marche toute l'infanterie attaqua Brihuega de toutes ses
forces et toute à la fois. Chacun des assaillants, connaissant
l'extrémité du danger de la conjoncture, s'y porta avec tant de
valliance et d'impétuosité que la ville fut emportée malgré son
opiniâtre résistance, avec une perte fort considérable des attaquants.
Les assiégés, retirés dans le château, capitulèrent incontinent,
c'est-à-dire que la garnison, composée de huit bataillons et de huit
escadrons, se rendit prisonnière de guerre, et avec elle Stanhope leur
général, Garpenter etWilz, lieutenants généraux, et deux brigadiers,
toute leur artillerte, armes, munitions et bagages\,; et ce fut là où
Stanhope, si triomphant dans Madrid, revomit les tapisseries du roi
d'Espagne qu'il avait prises dans son palais.

Tandis qu'on faisait cette capitulation avec les otages envoyés du
château, il vint divers avis de la marche du comte de Staremberg, qu'il
fallut avoir une attention extrême à cacher à ces otages qui auraient pu
rompre et le château se défendre, s'ils avaient su leur libérateur à une
lieue et demie d'eux, comme il y était déjà, et qu'il continuait sa
marche à l'entrée de la nuit, après s'être un peu reposé avec ses
troupes. La nuit fut pourtant tranquille. Le lendemain matin 11, M. de
Vendôme se trouva dans un autre embarras\,: il s'agissait en même temps
de marcher pour aller recevoir Staremberg déjà fort proche, et de
pourvoir à la sortie de Brihuega de cette nombreuse garnison qui y était
demeurée enfermée durant la nuit, et qu'il fallait acheminer en la
Vieille-Castille. Tout cela se fit pourtant fort heureusement. Les
régiments de gardes espagnoles et wallones restèrent à Brihuega jusqu'à
la parfaite évacuation\,; et lorsque Vendôme, toujours marchant à
Staremberg, vit l'action prochaine, il envoya chercher en diligence son
infanterie à Brihuega, avec ordre de n'y laisser que quatre cents
hommes.

Alors il mit son armée en bataille dans une plaine assez unie, mais
embarrassée par de petites murailles sèches en plusieurs endroits, fort
nuisibles pour la cavalerie. Incontinent après le canon commença à tirer
de part et d'autre, et presque aussitôt les deux lignes du roi d'Espagne
s'ébranlèrent pour charger. Il était alors trois heures et demie
après-midi, et il faut remarquer que les jours d'hiver sont un peu moins
courts en Espagne qu'en ce pays-ci. La bataille commença dans cet
instant par la droite de la cavalerie qui rompit leur gauche, la mit en
déroute, et tomba sur quelques-uns de leurs bataillons, les enfonça et
s'empara d'une batterie que ces bataillons avaient à leur gauche. Un
moment après, la gauche du roi d'Espagne chargea leur droite, fit
plusieurs charges, poussa et fut poussée à diverses reprises, repoussa
enfin, gagna les derrières de leur infanterie, et fut jointe par la
cavalerie de la droite du roi d'Espagne, qui avait battu et enfoncé les
ennemis de son côté, par les derrières de cette infanterie de leur
droite, qui combattait la cavalerie de notre gauche avec beaucoup de
vigueur et la poussait sur la réserve. Cette réserve était des gardes
wallones qui venaient d'arriver de Brihuega. Elles pénétrèrent les deux
lignes des ennemis et leur corps de réserve, et poussèrent ce qui se
trouva devant elles bien au delà du champ de bataille. Néanmoins le
centre espagnol pliait, et la gauche de sa cavalerie n'entamait pas la
droite des ennemis. M. de Vendôme s'en aperçut si fort qu'il crut qu'il
fallait songer à se retirer vers Torija, et qu'il en donna les ordres.
Il s'y achemina avec le roi d'Espagne, et une bonne partie des troupes.
Dans cette retraite il eut nouvelle que le marquis de Valdecanas et
Mahoni avaient chargé l'infanterie ennemie avec la cavalerie qu'ils
avaient à leurs ordres, l'avaient fort maltraitée, et s'étaient rendus
maîtres du champ de bataille, d'un grand nombre de prisonniers, et de
l'artillerie que les ennemis avaient abandonnée. Des avis si agréables
et si peu attendus firent perdre le parti au duc de Vendôme de remarcher
avec le roi d'Espagne et les troupes qui les avaient suivis, et de
s'avancer, en attendant qu'il fût jour, sur les hauteurs de Brihuega,
pour rentrer au champ de bataille, et y joindre les deux vainqueurs. Ils
y avaient formé, fort près des ennemis, un corps de cavalerie, et ces
ennemis étaient cinq ou six bataillons et autant d'escadrons, qui
étaient demeurés sur le champ de bataille ne sachant où se retirer, et
qui se firent jour avec précipitation, abandonnant vingt pièces de
canon, deux mortiers, leurs blessés et leurs équipages, que la cavalerie
victorieuse avait pillés le soir, et entièrement dispersés sur le champ
de bataille. Aussitôt on détacha après les débris de l'armée. Beaucoup
de fuyards, de traîneurs et d'équipages furent pris\,; mais le comte de
Staremberg se retira en bon ordre avec sept ou huit mille hommes, parce
qu'il avait l'avance de toute la nuit. Ses bagages et la plupart des
charrettes de son armée et de ses munitions furent la proie du
vainqueur.

On ne doit pas oublier une action particulière, dont la piété, la
résolution et la valeur, méritent une louange immortelle. Comme on
allait donner le troisième assaut à Brihuega, le comte de San-Estevan de
Gormaz, grand d'Espagne, officier général et capitaine général
d'Andalousie, vint se mettre avec les grenadiers les plus avancés. Le
capitaine qui les commandait, surpris de voir un homme si distingué
vouloir marcher avec lui, lui représenta combien ce poste était
au-dessous de lui. San-Estevan de Gormaz lui répondit froidement qu'il
savait là-dessus tout ce qu'il pouvait lui dire, mais que le duc
d'Escalona son père, plus ordinairement nommé le marquis de Villena,
était depuis très-longtemps prisonnier des Impériaux, indignement traité
à Pizzighettone, avec les fers aux pieds, sans qu'ils eussent jamais
voulu entendre à aucune rançon\,; qu'il y avait dans Brihuega des
principaux officiers généraux impériaux et anglais\,; qu'il était résolu
à les prendre pour délivrer son père ou de mourir en la peine. I1 donna
dans la place avec ce détachement, fit merveilles, prit de sa main
quelques-uns de ces généraux, et peu de temps après en fit l'échange
avec son père, qui avait été pris à Gaëte, vice-roi de Naples, les armes
à la main, comme je l'ai raconté en son lieu.

J'aurai occasion ailleurs de parler de ce père et de ce fils illustres,
morts tous deux successivement majordomes-majors du roi, chose qui n'a
point d'exemple en Espagne. Le père surtout était la vertu, la valeur,
la modestie et la piété même, le seigneur le plus estimé et respecté
d'Espagne, et, chose bien rare en ce pays-là, fort savant.

En comptant la garnison de Brihuega, il en coûta aux ennemis onze mille
hommes tués ou pris, leurs munitions, artillerie, bagages et grand
nombre de drapeaux et d'étendards. Le roi d'Espagne y perdit deux mille
hommes. Touy, bien que fort blessé à la main, dont il demeura estropié,
à l'attaque de Brihuega, se voulut trouver encore à la bataille qui fut
appelée de Villaviciosa, d'une villette fort proche. Il s'y distingua
fort et y servit très-utilement. Il fut même fait prisonnier, mais
bientôt après relâché quand le désordre commença à se mettre parmi les
ennemis. Il faut dire, pour fixer la position, que Brihuega est entre
Siguenza et Guadalaxara, et plus près de la dernière qui est sur le
chemin de France, à vingt-cinq de nos lieues en deçà de Madrid,
lorsqu'on prend le chemin de Pampelune.

Quand on considère le péril extrême, et pour cette fois, si {[}la
chose{]} eût mal bâté sans ressource, de la fortune du roi d'Espagne
dans cette occasion, on en tremble encore aujourd'hui. Celle qu'il avait
trouvée dans le coeur et le courage des fidèles et magnanimes Espagnols,
après sa défaite à Saragosse, était un prodige inespéré qui, une fois
perdue encore, ne pouvait plus se réparer. Il y en avait encore moins à
espérer de la France dans une seconde catastrophe. Son épuisement et ses
pertes ne lui permettaient pas d'entreprendre de relever de telles
ruines. Flattée par des pensées ténébreuses de paix, dont le besoin
extrême croissait à tous moments par l'impuissance de se défendre
elle-même, elle aurait vu la perte de la couronne d'Espagne comme un
affranchissement des conditions affreuses d'y contribuer, qui lui
étaient imposées, pour obtenir cette honteuse et dure paix après
laquelle elle soupirait avec tant de violence. Au lieu de ménager des
forces comme miraculeusement rassemblées, et rétablir peu à peu les
affaires, sans les commettre toutes à la fois aux derniers hasards,
l'imprudence de M. de Vendôme le fait jeter à corps perdu dans le
panneau qui lui est tendu. Sa négligence ne se donne pas la peine d'être
instruit du lieu qu'il prétend enlever d'emblée. Au lieu d'un poste il
trouve une place lorsqu'il a le nez dessus\,; au lieu de quelque faible
détachement avancé, il rencontre une grosse garnison commandée par la
seconde personne, mais la plus puissante de l'armée ennemie, et cette
armée à portée de venir tomber sur lui pendant son attaque. Alors il
commence à voir où il s'est embarqué, il voit le double péril d'une
double action à soutenir tout à la fois contre Stanhope qu'il faut
emporter de furie, après y avoir été repoussé par deux fois, et
Staremberg qu'il faut aller recevoir, et le défaire\,; et s'il les
manque, leur laisser la couronne d'Espagne sûrement, et peut-être la
personne de Philippe V pour prix de sa folie. Le prodige s'achève,
Brihuega est emportée sans lui, et sans lui la bataille de Villaviciosa
est gagnée. Seconde faute insigne\,: ce coup d'oeil tant vanté par les
siens se trouble, il ne voit pas le succès, il n'aperçait qu'un léger
ébranlement du centre. Ce héros qui se récrie si outrageusement à
Audenarde contre une indispensable retraite, la précipite ici avec ce
qu'il trouve de troupes sous sa main. Et ce même homme qui crut tout
perdu à Cassano, qui se retire seul dans une cassine éloignée du lieu du
combat, qui y pourpense tristement par où se sauver de ce revers, et qui
y apprend par Albergotti, qui l'y découvre enfin, après l'avoir
longtemps cherché, que le combat est gagné, qui y pique des deux à sa
parole, et s'y va montrer en vainqueur, ce même homme apprend dans
Torija même, où il s'était retiré et où il était arrivé, que la bataille
est gagnée, il y retourne avec les troupes qu'il en avait emmenées, et
quand il est jour il aperçait toutes les marques de la victoire. Il
n'est honteux ni de sa lourde méprise, ni de l'étrange contre-temps de
sa retraite, ni d'avoir sauvé Staremberg par l'absence des troupes dont
il s'était fait suivre, sans s'embarrasser de ce que deviendraient les
autres. Il s'écrie qu'il a vaincu, avec une impudence à laquelle il
n'avait pas encore accoutumé l'Espagne comme il avait fait l'Italie et
la France, et qui aussi ne s'en paya pas, tellement qu'après avoir mis
le roi d'Espagne à un cheveu de sa perte radicale, il manqua encore, par
cette aveugle retraite, de finir la guerre d'un seul coup, en détruisant
l'armée de Staremberg, qui ne lui aurait pu échapper, s'il n'avait pas
emmené les troupes, et qui, par cette faute insigne, eut le moyen de se
retirer, et toute la nuit devant soi et longue, pour se mettre en ordre
et ramasser tout ce qu'il put pour se grossir. Tel fut l'exploit de ce
grand homme de guerre, si désiré en Espagne pour la ressusciter, et la
première montre de sa capacité tout en y arrivant.

Du moment que le roi d'Espagne fut ramené sur le champ de bataille avec
ses troupes par Vendôme, et qu'ils ne purent plus douter de leur
bonheur, il fut dépêché un courrier à la reine. Ses mortelles angoisses
furent à l'instant changées en une si grande joie qu'elle sortit à
l'instant à pied par les rues de Vittoria, où tout retentit d'allégresse
ainsi que par toute l'Espagne et surtout à Madrid qui en donna des
marques extraordinaires. Don Gaspard de Zuniga, frère du duc de Bejar,
jeune homme de vingt-deux ans, qui avait fort servi en Flandre pour son
âge, fut dépêché au roi à qui le roi d'Espagne manda qu'il ne pouvait
lui envoyer personne qui lui rendît un meilleur compte de l'action, où
il s'était fort distingué. Il le rendit en effet tel, que le roi et tout
le monde en admirèrent la justesse, l'exactitude, la netteté et la
modestie. J'aurai lieu de parler de lui ailleurs\,; j'eus loisir et
commodité de l'entretenir et de le questionner tout à mon aise chez le
duc de Lauzun, tout en arrivant à Versailles, où je dînai avec lui. Il
ne cacha ni au roi ni au public rien de ce qui vient d'être expliqué sur
le duc de Vendôme, dont la cabale essaya de triompher vainement, pour
cette fois. Il était démasqué, il était disgracié\,: sa cabale ne put se
dissimuler ce que le roi en savait, et pensait de cette dernière
affaire\,; elle n'osa s'élever à la cour ni guère dans le monde. Elle se
contenta de ses manèges accoutumés dans les cafés de Paris et dans les
provinces ignorantes des détails et frappées en gros d'une bataille
gagnée. Bergheyck était venu faire un tour à Versailles, où il apprit
cette grande nouvelle.

Le roi d'Espagne marcha à Siguenza, où il prit quatre ou cinq cents
hommes qui s'étaient sauvés de la bataille, et quelque bagage, parmi
lequel était celui du comte de Staremberg que le roi d'Espagne lui
renvoya civilement. Ce général gagna comme il put la Catalogne\,; le roi
d'Espagne mena son armée en Aragon, et s'établit à Saragosse, où il
passa une partie de l'hiver, et où après un assez long temps la reine le
fut joindre.

Tout tomba sur Stanhope dans le dépit extrême que les alliés conçurent
de cette révolution si merveilleuse\,; les assaillants étaient fort peu
supérieurs à ce qu'il avait dans Brihuega, et il y avait abondance de
munitions de guerre et de bouche, et de l'artillerie à suffisance\,; le
lieu était bon, et il savait le dessein de Staremberg, et pourquoi il
l'y avait mis\,; que sept ou huit heures de résistance de plus faisaient
réussir, et écrasaient tout ce qui restait de troupes et de ressource au
roi d'Espagne. Staremberg, outré d'un succès si différent, et qui
changeait en entier la face des affaires, cria fort contre Stanhope qui
pouvait tenir encore longtemps dans le château. Quelques-uns des
principaux officiers qui y étaient avec lui secondèrent les plaintes de
Staremberg\,; Stanhope même n'osa trop disconvenir de sa faute. Il fut
contraint de demander congé pour s'aller défendre. Il fut mal reçu,
dépouillé de tout grade militaire en Angleterre et en Hollande, et lui
et les autres officiers qui comme lui avaient été d'avis de se rendre,
ne furent pas sans inquiétude pour leur dégradation et pour leur vie.

Le duc de Noailles investit Girone le 15 décembre. Cette expédition qui
est plus de l'année 1711 que de celle-ci y sera remise pour retourner
aux choses qui se sont passées et qui ont été suspendues ici, pour
n'interrompre point la suite importante des événements d'Espagne. On eut
envie d'y envoyer l'abbé de Polignac, ambassadeur. Il brillait cependant
à Marly à son retour de Gertruydemberg. Le roi lui fit voir ses jardins,
comme à un nouveau venu. La pluie surprit la promenade sans
l'interrompre. Le roi en fit une honnêteté à l'abbé de Polignac, qui
était l'hôte de cette journée. Il répondit avec toutes ses grâces que la
pluie de Marly ne mouillait point. Il crut avoir dit merveilles\,; mais
le rire du roi, la contenance du courtisan, et leurs propos au retour
dans le salon, lui montrèrent qu'il n'avait dit qu'une fade et plate
sottise\footnote{Saint-Simon a déjà raconté cette anecdote plus haut (t.
  V, p.~95) avec quelques variantes.}. L'Espagne ne voulut point de lui,
et redemanda instamment Amelot qui y avait si parfaitement réussi\,:
elle n'eut ni l'un ni l'autre.

\hypertarget{note-i.-le-cardinal-de-polignac.}{%
\chapter{NOTE I. LE CARDINAL DE
POLIGNAC.}\label{note-i.-le-cardinal-de-polignac.}}

Le cardinal de Polignac est un des personnages sur lesquels Saint-Simon
a donné carrière à sa causticité\,; il revient souvent à la charge,
répète les mêmes anecdotes, et ne manque aucune occasion de décrier les
talents diplomatiques du cardinal. Sans entreprendre l'apologie de ce
prélat, il est bon de remarquer que d'autres contemporains ont exprimé
sur le cardinal de Polignac une opinion tout opposée. Je citerai, entre
autres, le marquis d'Argenson, qui a été ministre des affaires
étrangères sous Louis XV, et qui ne pèche pas par excès d'indulgence.
Voici dans quels termes il parle du cardinal de Polignac \footnote{\emph{Mémoires
  du marquis d'Argenson}, p.~210 et suiv.}\,:

«\,Je vois quelquefois M. le cardinal de Polignac, et il m'inspire
toujours les mêmes sentiments d'admiration et de respect. Il me semble
que c'est le dernier des grands prélats de l'Église gallicane qui fasse
profession d'éloquence en latin comme en françois, et dont l'érudition
soit très-étendue. Il n'y a plus que lui qui, ayant pris place parmi les
honoraires dans l'Académie des belles-lettres, entende et parle le
langage des savants qui la composent. Il s'exprime sur les matières
d'érudition avec une grâce et une noblesse qui lui sont propres. La
conversation du cardinal est également brillante et instructive. Il sait
de tout, et rend avec clarté et grâce tout ce qu'il sait\,; il parle sur
les sciences et sur les objets d'érudition comme Fontenelle a écrit ses
\emph{Mondes}, en mettant les matières les plus abstraites et les plus
arides à la portée des gens du monde et des femmes, et les rendant dans
des termes avec lesquels la bonne compagnie est accoutumée à traiter les
objets de ses conversations les plus ordinaires.

«\,Personne ne conte avec plus de grâce que lui, et il conte
volontiers\,; mais les histoires les plus simples, ou les traits
d'érudition qui paroîtroient les plus fades dans la bouche d'un autre,
trouvent des grâces dans la sienne, à l'aide des charmes de sa figure et
d'une belle prononciation. L'âge lui a fait perdre quelques-uns de ces
derniers avantages, mais il en conserve assez, surtout quand on se
rappelle dans combien de grandes occasions il a fait briller ses talents
et ses grâces naturelles. Mon oncle, l'évêque de Blois, qui étoit à peu
près son contemporain, m'a souvent parlé de sa jeunesse. Jamais on n'a
fait de cours d'études avec plus d'éclat\,; non-seulement ses thèmes et
ses versions étoient excellents, mais il lui restoit du temps et de la
facilité pour aider ses camarades, ou plutôt faire leurs devoirs à leur
place\,; si bien qu'il est arrivé au collège d'Harcourt\footnote{\emph{Mémoires
  du marquis d'Argenson}, p.~210 et suiv.}, où il étudioit, que les
quatre pièces qui remportèrent les deux prix et les deux \emph{accessit}
étoient également, sou ouvrage. Étant en philosophie au même collège, il
voulut soutenir dans ses thèses publiques le système de Descartes, qui
avoit alors bien de la peine à s'établir. Il s'en tira à merveille, et
confondit tous les partisans des vieilles opinions. Cependant les
anciens docteurs de l'Université ayant trouvé très-mauvais qu'il eût
combattu Aristote, et n'ayant point voulu accorder de degrés à l'ennemi
du précepteur d'Alexandre, il consentit à soutenir une autre thèse, dans
laquelle il chanta la palinodie, et fit triompher à son tour Aristote
des cartésiens mêmes.

«\,À peine fut-il reçu docteur en théologie que le cardinal de Bouillon
le conduisit à Rome, au conclave de 1689, où le pape Alexandre VIII fut
élu. Dès que l'abbé de Polignac fut connu dans cette capitale du monde
chrétien, qui étoit alors le centre de l'érudition la plus profonde et
de la politique la plus raffinée, il y fut généralement aimé et estimé.
Les cardinaux françois et l'ambassadeur de France jugèrent que personne
n'étoit plus propre que lui à faire entendre raison au pape sur les
articles de la fameuse assemblée du clergé de 1682. C'étoit une pilule
difficile à faire avaler à la cour de Rome\,; cependant l'esprit et
l'éloquence de l'abbé de Polignac en vinrent à bout\,; il fut chargé
d'en porter lui-même la nouvelle en France, et eut, à cette occasion,
une audience particulière de Louis XIV, qui dit de lui en françois ce
que le pape Alexandre VIII avoit dit en italien\,: \emph{Ce jeune homme
a l'art de persuader tout ce qu'il veut\,; en paroissant d'abord être de
votre avis, il est d'avis contraire, mais mène à son but avec tant
d'adresse qu'il finit toujours par avoir raison}. Il n'avoit pas encore
mis la dernière main à cette grande affaire lorsque la mort du pape le
rappela à Rome. Il assista encore au conclave où fut élu Innocent XII,
et revint en France l'année suivante, 1692.

«\,Environ deux ans après, le roi le nomma à l'ambassade de Pologne dans
des circonstances fort délicates. Jean Sobieski se mouroit\,; Louis XIV
vouloit non-seulement conserver du crédit en Pologne, mais même donner
pour successeur au roi Jean un prince dévoué à la France. Le prince
s'étoit offert, et Louis XIV avoit chargé très-secrètement l'abbé de
Polignac du soin de le faire élire, malgré la reine
douairière\footnote{Fille du marquis d'Arquien.}, qui étoit Françoise,
mais qui, comme de raison, favorisoit ses enfants, et en dépit de toute
cabale contraire. L'abbé, tenant ses instructions bien secrètes, étoit
arrivé à la cour de Sobieski un an avant sa mort. Il avoit enchanté tous
les Polonois par la facilité avec laquelle il parloit latin. On l'auroit
cru un envoyé de la cour d'Auguste, si on ne l'eût entendu parler
françois avec la reine, qui se laissa séduire par sa figure et son
esprit, mais qui ne pouvoit pas renoncer pour lui à l'intérêt de sa
famille. Sobieski mourut, et la diète générale s'assembla pour lui
choisir un successeur.

\begin{enumerate}
\def\labelenumi{\arabic{enumi}.}
\tightlist
\item
  Maintenant lycée Saint-Louis.
\end{enumerate}

«\,L'éloquence de l'abbé de Polignac, les promesses et les espérances
dont il leurra les Polonois, eurent d'abord tant de succès, qu'une bonne
partie de la nation, ayant à sa tête le primat, proclama le prince de
Conti\,; mais, dans le même moment, les sommes qu'avoit répandues
l'électeur de Saxe furent cause qu'il y eut une double élection, dans
laquelle ce prince allemand fut élu. L'un et l'autre prétendant à la
couronne arrivèrent pour soutenir leur parti, et continuèrent d'employer
les moyens qui leur avoient d'abord réussi\,; mais ceux de l'électeur
étoient plus effectifs et plus solides\,: il avoit de l'argent et même
des troupes. Au contraire, le prince de Conti, après avoir reçu les
honneurs de roi à la cour de France, aborda sur un seul vaisseau
françois à Dantzick, et y séjourna pendant six semaines, mais sans avoir
d'autres moyens pour faire valoir la légitimité de son élection que la
bonne mine et l'éloquence de l'abbé de Polignac. Ces ressources se
trouvèrent bientôt épuisées\,; le prince de Conti et l'abbé même furent
contraints de revenir en France.

«\,Quoique l'on fut trop juste et trop éclairé à la cour de Louis XIV
pour ne pas sentir que ce n'étoit pas la faute de l'ambassadeur si sa
mission n'avoit pas eu un plus glorieux succès, il fut cependant exilé
de la cour pendant quatre ans. Il employa ce temps utilement pour
augmenter la masse de ses connoissances, qui étoit déjà si grande.
Enfin, en 1702, il fut renvoyé à Rome en qualité d'auditeur de rote. Il
y trouva de nouvelles occasions de briller et de se faire admirer, et en
fut récompensé par la nomination du roi Jacques d'Angleterre au
cardinalat.

«\,Il étoit prêt à en jouir lorsqu'il fut rappelé à la cour de France
dans des circonstances très-critiques. En 1710, on l'obligea de se
rendre, avec le maréchal d'Huxelles, à Gertruydemberg, chargé de
proposer aux ennemis de Louis XIV, de la part de ce monarque même, de se
soumettre aux conditions les plus humiliantes pour faire cesser la
guerre. Malheureusement, tout l'esprit et toute l'éloquence du futur
cardinal y échouèrent. Enfin, deux ans après, il fut nommé
plénipotentiaire au fameux congrès d'Utrecht, et il faut remarquer qu'il
étoit dès lors nommé à Rome cardinal \emph{in petto}\,; mais quoique
tout le monde sût en Hollande qui il étoit, il ne portoit ni titre ni
habits ecclésiastiques\,; il étoit vêtu en séculier, et on l'appeloit M.
le comte de Polignac. Ce fut dans cet état, et sous cet
\emph{incognito}, qu'il suivit toutes les négociations d'Utrecht
jusqu'au moment de la signature du traité\,; mais alors il déclara qu'il
ne lui étoit pas possible de signer l'exclusion du trône d'un monarque à
qui il devoit le chapeau de cardinal. Il se retira, et vint jouir à la
cour de France des honneurs du cardinalat.

«\,Lorsque, après la mort de Louis XIV, il fut exilé dans son abbaye
d'Anchin, en Flandre\footnote{Le marquis d'Argenson était intendant du
  Hainaut à l'époque de l'exil du cardinal de Polignac\,; il est
  probable que c'est de ce temps que date sa liaison avec lui.}, ces
bons moines flamands tremblèrent en le voyant arriver dans leur
monastère\,; mais ils pleurèrent et furent au désespoir quand il les
quitta, après la mort du cardinal Dubois et du régent. Ils n'étoient
point capables de juger de son mérite en qualité de bel esprit, ni de
rien entendre à son érudition\,; mais ils l'avoient trouvé doux,
aimable\,; et, loin de les piller, il avoit embelli leur église et
rétabli leur maison.

«\,Il fut obligé de retourner à Rome à la mort de Clément XI, et il
assista aux conclaves où furent élus Innocent XIII, Benoît XIII et
Clément XII. Pendant les deux premiers pontificats, il a été chargé des
affaires de France à Rome. Cette ville a toujours été le plus beau
théâtre de sa gloire\,; l'on eût dit que l'ancienne grandeur romaine
rentroit avec lui dans sa capitale. De son côté, quand il en est revenu,
il a paru chargé des dépouilles de Rome, assujettie par son esprit et
par son éloquence\,; et l'on peut dire au pied de la lettre, qu'à son
dernier voyage, il a transporté une partie de l'ancienne Rome jusque
dans Paris, en plaçant dans son hôtel une collection de statues antiques
et de monuments tirés des ruines du palais des premiers empereurs.

«\,Encore une fois, je ne peux voir le cardinal de Polignac sans me
rappeler tout ce qu'il a fait et appris depuis plus de soixante ans\,;
je reste pour ainsi dire en extase vis-à-vis de lui, et en admiration de
tout ce qu'il dit. On trouve que son ton est vieilli aussi bien que sa
figure. Il est vrai que son ton est passé de mode\,; mais ne seroit-ce
pas à cause que nous avons perdu l'habitude d'entendre parler de science
et d'érudition que M. le cardinal de Polignac commence à nous enuuyer\,?
Car d'ailleurs personne ne traite ces matières avec moins de pédanterie
que lui\,: s'il cite, c'est toujours à propos, parce que, comme il a une
prodigieuse mémoire, elle lui fournit de quoi soutenir la conversation
sur tous les points, quelque matière que l'on traite. Pour moi, qui ai
fait mes études, mais à qui il reste encore bien des choses à apprendre,
j'avoue que je n'ai jamais pris de leçons plus agréables que celles
qu'il donne dans la conversation.\,»

\hypertarget{note-ii.-mademoiselle-de-la-mothe-houdancourt.}{%
\chapter{NOTE II. MADEMOISELLE DE LA
MOTHE-HOUDANCOURT.}\label{note-ii.-mademoiselle-de-la-mothe-houdancourt.}}

M\textsuperscript{lle} de La Mothe-Houdancourt\footnote{Elle se nommait
  Charlotte-Éléonore-Madeleine de La Mothe-Houdancourt\,; elle épousa,
  en 1671, Louis-Charles de Lévi, duc de Ventadour.}, dont Saint-Simon
parle (p.~329 de ce volume), avait vivement excité l'attention de la
cour en 1661, et avait été regardée comme une rivale dangereuse pour
M\textsuperscript{lle} de La Vallière. Les Mémoires du temps sont
remplis de ces détails. M\textsuperscript{me} de Motteville raconte que
le roi, qui était alors à Saint-Germain, avait pris l'habitude d'aller à
l'appartement des filles de la reine. «\,Comme l'entrée de leur chambre,
ajoute-t-elle, lui étoit interdite par la sévérité de la dame d'honneur
\footnote{La dame d'honneur était alors la duchesse de Navailles.}, il
entretenoit souvent M\textsuperscript{lle} de La Mothe-Houdancourt par
un trou qui étoit à une cloison d'ais de sapin, qui pouvoit lui en
donner le moyen.\,»

Bussy-Rabutin parle aussi de M\textsuperscript{lle} de La Mothe dans son
\emph{Histoire amoureuse des Gaules}. Après avoir raconté une scène de
jalousie entre Louis XIV et M\textsuperscript{lle} de La Vallière, il
ajoute\,: «\,Le roi vit, le jour suivant, M\textsuperscript{lle} de La
Mothe, qui est une beauté enjouée, fort agréable et qui a beaucoup
d'esprit. Il lui dit beaucoup de choses fort obligeantes\,; il fut
toujours auprès d'elle, soupira souvent, et en fit assez pour faire dire
dans le monde qu'il en étoit amoureux, et pour le persuader à
M\textsuperscript{me} sa mère, qui grondoit sa fille de ne pas répondre
à la passion d'un si grand monarque. Toutes les amies de la maréchale
s'assemblèrent pour en conférer, et après être convenues que nous
n'étions plus dans la sotte simplicité de nos pères, elles querellèrent
à outrance cette aimable fille. Mais elle avoit dans le coeur une
secrète attache pour le marquis de Richelieu\,; ce qui faisoit qu'elle
voyoit sans plaisir l'amour que le roi lui témoignoit.\,»

Le jeune Brienne donne, dans ses \emph{Mémoires}\footnote{\emph{Mémoires
  de Louis-Henri de Loménie, comte de Brienne}, t. II, p.~173, 174
  (édit. de 1828).}, des détails sur cette passion de
M\textsuperscript{lle} de La Mothe pour le marquis de Richelieu, mais
sans en indiquer les suites. Nous les apprenons par des lettres anonymes
de cette époque\,: elles sont adressées par une femme de la cour à
Pellisson, qui accompagnait Fouquet en Bretagne.

«\,Il ne s'est rien passé de considérable en cette cour depuis que vous
en êtes parti, que le congé donné à M\textsuperscript{lle} de La Mothe
par la reine mère\footnote{Anne d'Autriche.}. Ce fut M. de Guitri qui
eut ordre de le lui dire la veille du départ du roi\footnote{Le roi
  partit pour Nantes le 1\^{}er septembre 1661.}. La reine mère
souhaitoit que la chose se fît sans éclat, et que La Mothe se retirât
sous prétexte de maladie ou quelque autre raison. Mais elle fut chez
M\textsuperscript{me} la Comtesse\footnote{Olympe Mancini, comtesse de
  Soissons. On prétendait que cette nièce de Mazarin avait voulu donner
  M\textsuperscript{lle} de La Mothe pour maîtresse au roi à la place de
  M\textsuperscript{lle} de La Vallière.} le lendemain bon matin, et
après avoir appelé M\textsuperscript{me} de Lyonne au conseil, il fut
résolu qu'on engagerait la reine\footnote{Marie-Thérèse.} à prier la
reine mère en sa faveur. Cette résolution prise, on chercha les moyens
d'engager la reine à faire cette prière. On crut que la voie de
Molina\footnote{La Molina était une des femmes attachées au service de
  Marie-Thérèse. M\textsuperscript{me} de Motteville en parle dans ses
  \emph{Mémoires}.} étoit la meilleure\,; on la prit, et l'abbé de
Gordes fut dépêché vers elle. Molina promit de s'employer de tout son
pouvoir et de faire agir la reine.

«\,En effet, comme la reine mère revenoit de la promenade, elle fut
priée de la part de la reine d'entrer dans son appartement seule, et, y
étant, la reine la pria avec des termes pressants de pardonner à La
Mothe. Elle lui dit qu'elle savoit bien qu'elle n'aimoit pas la
galanterie\,; que si, après ce pardon, La Mothe ne vivoit pas avec la
dernière régularité, et ne servoit pas d'exemple aux filles de la reine
mère et aux siennes, elle seroit la première à prier la reine mère de la
chasser. Voyant que toute cette éloquence étoit inutile, elle fit sortir
La Mothe tout en pleurs de son cabinet, où elle avoit été enfermée toute
l'après-dînée, qui vint se jeter aux pieds de la reine mère, qui,
craignant de s'attendrir, ou, comme elle a dit depuis, ne voulant pas
lui reprocher sa mauvaise conduite, passa dans le grand cabinet de la
reine, et fut entendre une très-mauvaise comédie espagnole.

«\,Depuis, La Mothe a fait prier la reine mère par la reine de souffrir
qu'elle se retirât au Val-de-Grâce\,; ce qui lui a été refusé par la
reine mère, parce qu'elle a dit qu'il y alloit trop de monde. On la mit
à Chaillot.

«\,Le sujet de sa disgrâce est conté diversement. Les uns disent qu'elle
a écrit une lettre où elle traite le marquis de Richelieu de traître et
de perfide, pour l'avoir abandonnée, et que cette lettre a été
interceptée\,; les autres, que le marquis a voulu se rengager dans ce
même commerce avec elle, et qu'on l'a appréhendé\,; qu'il lui a écrit
une lettre plus tendre que toutes celles qu'il lui avoit écrites
autrefois, et qu'on a su qu'il l'avoit écrite. On fait d'étranges contes
d'elle, et c'est ce qui fait qu'elle veut entrer dans un couvent que la
reine mère lui choisira, parce qu'autrement elle ne pourroit se
justifier.\,»

Le 7 septembre, la même personne revenait encore, dans une lettre
adressée à Pellisson, sur la disgrâce de M\textsuperscript{lle} de La
Mothe\,: «\,On a fait quatre vilains vers pour l'aventure de
M\textsuperscript{lle} de La Mothe, que M\textsuperscript{me} de
Beauvais\footnote{M\textsuperscript{me} de Beauvais, fort connue par ses
  aventures galantes, était première femme de chambre de la reine Anne
  d'Autriche.} a fait chasser. C'est le bon M. de La Mothe\footnote{Il
  s'agit ici de La Mothe Le Vayer, dont on parle plusieurs fois dans
  cette lettre, ainsi que de Ménage, de Boisrobert, de Nublé.} qui me
les a dits. Il y a une vilaine parole\,; mais n'importe\,; ce n'est pas
moi qui l'y ai mise\,:

«\,Ami, sais-tu quelque nouvelle

De ce ce qui se passe à la cour\,?

On y dit que la maq\ldots\ldots.

A chassé la fille d'amour.\,»

«\,Tout le monde blâme M. le marquis de Richelieu\footnote{Ces deux
  lettres furent saisies avec les papiers de Pellisson que le roi fit
  arrêter en même temps que Fouquet. Elles sont conservées à la
  Bibliothèque impériale, dans les manuscrits de Baluze.}.\,»

\hypertarget{note-iii.-madame-la-comtesse-de-soissons-olympe-mancini.}{%
\chapter{NOTE III. MADAME LA COMTESSE DE SOISSONS (OLYMPE
MANCINI).}\label{note-iii.-madame-la-comtesse-de-soissons-olympe-mancini.}}

Parmi les personnages que Saint-Simon avoit peu connus, et qu'il a
traités avec une sévérité excessive, on ne doit pas oublier la comtesse
de Soissons (Olympe Mancini)\,; il en parle souvent dans ses Mémoires,
et entre autres à l'occasion de sa mort (t. VI, p.~441-444 de notre
édition). M. Amédée Renée, dans son ingénieux et savant ouvrage sur les
\emph{Nièces de Mazarin}, a relevé plusieurs erreurs de ce dernier
passage. Saint-Simon dit (p.~442) «\,qu'elle (la comtesse de Soissons)
fit sa paix et obtint son rappel par la démission de sa charge, qui fut
donnée à M\textsuperscript{me} de Montespan.\,» Puis il ajoute\,: «\,La
comtesse de Soissons, de retour, se trouva dans un état bien différent
de celui d'où elle étoit tombée.\,» M. Amédée Renée fait remarquer que
ce ne fut que beaucoup plus tard, en 1680, à l'époque où Olympe s'enfuit
hors de France, qu'elle se démit de sa charge, qui fut, non point
\emph{donnée}, mais vendue par elle, moyennant deux cent mille écus, à
M\textsuperscript{me} de Montespan. Celle-ci l'avoit convoitée pendant
tout son règne de favorite, et elle ne l'obtint qu'aux approches de sa
disgrâce.

Une assertion plus grave est relative à l'empoisonnement de Marie-Louise
d'Orléans, reine d'Espagne, dont Saint-Simon accuse la comtesse de
Soissons (t. VI, p.~443). Il prétend que cette princesse mourut peu de
temps après avoir bu du lait glacé que lui apporta Olympe Mancini, et il
ne manque pas de rapprocher le sort de Louise d'Orléans de celui de sa
mère, Henriette d'Angleterre. Nous avons vu (t. III, p.~448) que rien
n'était moins certain que l'empoisonnement de Madame. M. Amédée Renée
prouve que l'on doit également douter des assertions de Saint-Simon
relatives à l'empoisonnement de sa fille\footnote{\emph{Ibid}., p.~225
  et suiv.}. Je me bornerai à résumer ce passage de son livre, et
souvent même je le citerai textuellement.

«\,Saint-Simon, dit-il\footnote{\emph{Ibid}., p.~226.}, avait rapporté
de son voyage d'Espagne cette anecdote de lait empoisonné\,; il y ajouta
foi sans nul doute, sans regarder de près à l'invraisemblance de
l'histoire. Cette reine à qui l'on procure du lait en cachette, comme la
chose la plus introuvable, et qui s'en fait apporter par une princesse
étrangère, au lieu de s'adresser à son maître d'hôtel, cela ne
ressemble-t-il pas à un conte arabe\,? Il n'est guère étonnant
d'ailleurs que les bruits d'empoisonnement qui avaient déjà couru sur la
comtesse de Soissons aient donné lieu en Espagne à de nouveaux soupçons
et à une sorte de légende populaire. Mais dans une sphère plus élevée,
on ne trouve que Saint-Simon qui attribue ce crime à Olympe. Examinons
les témoignages contemporains. La palatine, duchesse d'Orléans, qui
étoit la belle-mère de la reine d'Espagne, croit à l'empoisonnement
comme Saint-Simon, mais il n'est point question de lait à la glace avec
elle\,; elle assure que la jeune reine fut empoisonnée dans des huîtres,
ce qui pourrait bien réduire la chose à un simple accident. Elle dit
encore, et avec peu de vraisemblance, que ce fut le comte de Mansfeld
qui procura le poison à deux femmes de chambre françaises. Quant à la
comtesse de Soissons, il n'est pas question d'elle ici. »

M. Amédée Renée passe en revue tous les auteurs qui ont parlé de cet
événement\,: M\textsuperscript{me} de La Fayette, qui raconte que le
poison fut donné dans une tasse de chocolat\footnote{\emph{Mémoires de
  M\textsuperscript{me} de La Fayette}, collect. Petitot, t. LXIV, p.~75
  et suiv.}\,; Mademoiselle, qui croit aussi à un
empoisonnement\footnote{\emph{Mémoires de Mademoiselle}, collect.
  Petitot, t. XLIII, p.~389 et suiv.}, mais sans parler de la comtesse
de Soissons\,; Dangeau, qui raconte que le roi dit en soupant\,: «\,La
reine d'Espagne est morte empoisonnée dans une tourte d'anguilles\,; la
comtesse de Pernitz, les caméristes Zapata et Nina, qui en ont mangé
après elle, sont mortes du même poison.\,» Enfin, si l'on en croit
Louville, qui, par sa position à Madrid, mérite la plus grande
confiance, «\,il n'est pas douteux que cette intéressante princesse,
morte empoisonnée en 1689, n'ait payé de sa vie l'inutile empire qu'elle
avoit su prendre sur son époux\footnote{}, s Ainsi, de tous les
contemporains qui parlent de cet empoisonnement, Saint-Simon reste le
seul qui ait accusé de ce crime Olympe Mancini, comtesse de Soissons\,;
et comme son récit est d'ailleurs invraisemblable, l'historien des
\emph{Nièces de Mazarin} a rendu un véritable service en le soumettant à
une critique sévère.

\end{document}
